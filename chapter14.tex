% \begin{comment}
\documentclass{book}
\usepackage{master}
\usepackage{changepage}
\newcommand{\rec}{$\text{R\'ecoltes et Semailles}$}
\newcommand{\no}{n$^\circ$}
\hfuzz = 100pt

% NOTE and SUBNOTE FORMATTING
\usepackage{titlesec}
\usepackage[dotinlabels]{titletoc}

% define `note'
\titleclass{\nnote}{straight}[\section]
\newcounter{nnote}
\renewcommand{\thennote}{\thesection.\arabic{nnote}}
\titlespacing*{\nnote}{0pt}{3.5ex plus 1ex minus .2ex}{1ex plus .2ex}
\titleformat{\nnote}[runin]{\bfseries }{\bfseries Note }{0pt}{}[]

% define `subnote'
\titleclass{\subnote}{straight}[\section]
\newcounter{subnote}
\renewcommand{\thesubnote}{\thesection.\arabic{subnote}}
\titlespacing*{\subnote}{0pt}{3.5ex plus 1ex minus .2ex}{1ex plus .2ex}
\titleformat{\subnote}[runin]{\bfseries }{\bfseries Note }{0pt}{}[]

% the first optional arg sets the size of the indentation in the TOC
\titlecontents{nnote}[9em]
{}{Note }{}{\titlerule*[1pc]{.}\contentspage}

% the first optional arg sets the size of the indentation in the TOC
\titlecontents{subnote}[11em]
{}{Note }{}{\titlerule*[1pc]{.}\contentspage}

\begin{document}
% print table of contents with notes and subnotes
% \setcounter{tocdepth}{5}
% \setcounter{secnumdepth}{5}
% \startcontents[chapters]
% \printcontents[chapters]{}{0}{}

\setcounter{chapter}{13}
% \end{comment}

\chapter{B) Pierre and Motives}

\section{IV Motives (burial of a newborn)}

\subsection{Memory from a dream - or the birth of motives}

\nnote{51}\label{note:51} [This note is referenced in note 46 p. 186]

\marginpar{p. 205}(April 19) Since writing the above lines (ending with the note ``My orphans", n$^o 46$) less than a month ago, I have come to realize that they delay the chain of events to a certain extent! I have just received ``Hodge Cycles, Motives and Shimura Varieties" (LN 900) by Pierre Deligne, James S. Milne, Arthur Ogus, and Kuang-Yen Shih, which Deligne kindly transmitted to me along with a list of his publications. This collection of six texts, published in 1982, constitutes an interesting new piece of information since 1970, in that motives are mentioned in the title and present in the text, however modestly, especially through the notion of ``motivic Galois group". Of course, we are still far from a panoramic picture of a theory of motives, which for the past fifteen to twenty years has been awaiting the impetus of a bold mathematician who will be willing to ``paint it, in a vast enough way to serve as a source of inspiration, as a golden thread and a horizon, for one or several generations of arithmetic geometers who will have the privilege of establishing its validity (or in any case of discovering the final word on the reality of motives...) (53).

1982 also seems to mark the year since which\footnote{(*) (May 25) I am delaying the events once again, by one year this time - the turning point took place in June 1981 with the Luminy Colloquium, see the note ``The Inequity - or a feeling of return", n$^o 75$.}(*) the changes in fashion are beginning to slowly turn in favor of derived categories; Zoghman Mebkhout (in a perhaps euphoric rush) already sees them as being on the brink of ``invading all domains of mathematics". If their utility, which was made evident by mere mathematical instinct (for well-informed individuals) since the beginning of the 1960s, is now starting to be recognized, it is (or so it seems to me) thanks to Mebkhout's solitary efforts and his willingness to take on the thankless role of guinea-pig for seven years, with the courage of those who continue to trust their instinct in the face of tyrannical customs...

\marginpar{p. 206}Remarkably, in reading this first publication which marks (twelve years after my departure from the mathematical world) a modest re-entry of the notion of motive into the apparatus of admissible mathematical notions, I could find nothing which would indicate to the uninformed reader that my humble person was involved in any way in the origins of this notion, long considered a taboo. Nor is there any allusion to an authorship of some form ($51_1$) behind the development of a rich and precise ``yoga", which appears in the article (in piecemeal form) as if it came out of the void.

When, just three weeks ago, I laid down my thoughts on the yoga of motives in a page or two, qualifying it as one of the ``orphans" whom I held closer to my heart than any other, I must have been sorely mistaken! Surely was I dreaming, when I seemed to remember years of gestation of a vision, tenuous and elusive at first, and growing richer and more precise over the course of the months and the years, as a result of a persistent effort to grasp the common ``motive", the common quintessence of which the several cohomology theories known at the time (54) were but various incarnations, each telling us in its own language about the nature of the ``motive" of which it was one of the directly tangible manifestations. Surely I was still dreaming when I remembered the strong impression that Serre's intuition had made upon me, regarding his conception of a profinite Galois group, an object which appeared to be of a discrete nature (or at least could be tautologically reduced to simple systems of \textbf{finite} groups) yet gave rise to an immense projective system of $l$-adic \textbf{analytic} or even \textbf{algebraic} groups over $\mathbb{Q}_l$ (by passing to appropriate (algebraic envelopes?)), groups which even had a tendency to be reductive - hence lending themselves to the panoply of intuitions and methods (Lie style) of analytic and algebraic groups. This construction made sense for any prime number $l$, and I felt (or dreamt that I felt...) that there lay a mystery to be probed, regarding the relationship between these algebraic groups associated to different prime numbers; I felt that they must all come from a single projective system of algebraic groups over the only natural common sub-field to all of these base fields, namely the field $\mathbb{Q}$, the ``absolute" field of characteristic zero. And since I like dreaming, I continue dreamed that I remember having gained access to this glimpsed mystery, through work which surely was but a dream since I did not ``prove" anything; and I eventually understood how the notion of motives provided the key to understanding this mystery - that, through the mere presence of a category (here, the category of ``smooth" motives over a given base scheme\marginpar{p. 207}, for instance motives over a given ground field), possessing internal structures similar to those which can be found on the category of linear representations of an algebraic pro-group over a field $k$ (the charm of the notion of algebraic pro-group having been revealed to me by Serre as well at an earlier time), one would be able to reconstruct such a pro-group (given the data of an appropriate ``fiber functor"), and to interpret the ``abstract" category as the category of its linear representations.

This approach towards a ``motivic Galois theory" was suggested to me by the approach which I had found, years earlier, to describe the fundamental group of a topological space or scheme (or even of an arbitrary topos - but here I risk offending delicate ears to whom ``topoi are no fun"...) in terms of the category of \'etale coverings of the ``space" under consideration, and fiber functors on this category. The very language of \textbf{``motivic Galois groups"} (which I could also have called motivic ``fundamental groups", the two intuitions amounting to the same thing in my view since the end of the 1950s...) and of ``fiber functors" (corresponding precisely to the ``manifest incarnations" mentioned earlier, namely the different ``cohomology theories" which may be applied to a given category of motives) is tailor-made to express the profound nature of these groups, and to suggest their intimate ties with ordinary Galois groups and fundamental groups.

I still remember the pleasure and awe which I felt, in playing this game with fiber functors and (torsors under Galois groups?) which send them to one another by ``twisting", upon retrieving in a particularly concrete and fascinating situation the entire panoply of tools from non-commutative cohomology developed in Giraud's book, with the gerbe of fiber functors (here taken over the \'etale topos or, better, the fpqc topos of $\mathbb{Q}$ - interesting and non-trivial topoi if there ever were any!), as well as the (``link"?) (in algebraic groups or pro-groups") connecting this gerbe to the avatars of this link, all being realized by various algebraic groups or pro-groups, corresponding to the different ``sections" of the gerbe, corresponding to the different cohomological functors.
% todo: verify accuracy of translation
The various complex points (for instance) of a scheme over characteristic zero each give rise (via the corresponding Hodge functors) to sections of the gerbe, and to torsors providing a transition from one to the next; and both these torsors and the pro-groups acting on them possess remarkable algebro-geometric properties, expressing specific structures of Hodge cohomology - but now\marginpar{p. 208} I am getting ahead of myself and speaking about another chapter of my dream about motives... This was a time when those who issued the decrees of fashion had not yet declared that topoi, gerbes, and the like did not entertain them an that as such it was stupid to speak of them (not that this would have prevented me from recognizing topoi and gerbes when I saw them...). Now that twelve years have elapsed, these very people are pretending to be discovering and are teaching to others the fact that gerbes (if not topoi for now) are indeed relevant to the study of the cohomology of algebraic varieties, and even to that of periods of abelian integrals...

I could also evoke the dream of another memory (or the memory of another dream...) surrounding the dream about motives, which was also born from a ``strong impression" (I am decidedly in full subjective mode!) which some comments by Serre regarding a certain ``philosophy" behind the Weil conjectures had made upon me. Their translation into cohomological terms, for $l$-adic coefficients with varying $l$, led one to suspect the existence of remarkable structures on the corresponding cohomology theories - the structure of a ``weight filtration"\footnote{(*) (January 24 1985) For a rectification of this distorted memory, see note n$^o 164$ (I4), as well as sub-note n$^o 164_1$, which give more details on the filiation of the ``yoga of weights".}(*). Surely, the common ``motive" to the various $l$-adic cohomology theories had to be the ultimate support for this essential arithmetic structure, which as such took on a geometric aspect, that of the remarkable structure of the geometric object that is a ``motive". It would once again be inaccurate of me to speak of a ``work" when the task was to ``guess" (with the internal coherence of a vision in progress as only guide, using the sparse known or conjectured elements lying here and there...) the specific structure of the various cohomological ``avatars" of a motive, and how the weight filtration was expressed therein
\footnote{(**) (February 28 1985) There was a slight confusion in my mind. I was actually referring to the closely linked filtration by ``levels".}(**), beginning with the Hodge avatar (at a time where Hodge-Deligne theory had not yet been developed, and for good reason...\footnote{(***) This was at a time when the young Deligne had probably not yet heard the word ``scheme" be said in a mathematical context, nor the word ``cohomology". (He learned these notions from me starting in 1965.)}(***)). This allowed me (in a dream) to view within a single vast painting the Tate conjecture on algebraic cycles (a third ``strong impression" which inspired the Dreamer in his dream about motives!) and the Hodge conjecture (55), and to formulate two or three additional conjectures of the same type, about which I spoke to certain people who must have forgotten as I have never heard anyone mention them since, subjected to the same silence as the \marginpar{p. 209}``standard conjectures". In any event, these were only conjectures (unpublished on top of that...). One of them did not concern a specific cohomology theory; rather, it gave a direct interpretation of the weight filtration on the motivic cohomology of a non-singular projective variety over a field in terms of the geometric filtration of this variety by closed subsets of given codimension (with codimension playing the role of ``weight")\footnote{(*) (February 28 1985) The filtration in question here is actually the ``level" filtration (see preceding footnote).}(*).

There was also the work (I should be putting quotation marks around ``work", but I can't find the resolve to do so!) I did towards ``guessing" the behavior of weights with respect to the six operations (completely lost since then...). Here again, I never felt that I was inventing anything, but discovering - or rather listening to what things were telling me whenever I sat down to listen to them with a pen in hand. What they were saying was of a peremptory, and as such unmistakable, precision. 

Then there was a third ``motive-dream", which was in a sense the wedding of the preceding two dreams - regarding the problem of interpreting, in terms of structures imposed on the motivic Galois groups and on the torsors under these groups which can be used to ``twist" a fiber functor to (canonically) obtain any other fiber functor \footnote{(**) Just as the fundamental groups $\pi_1(x), \pi_1(y)$ of some ``space" $X$ at two ``points" $x$ and $y$ can deduced from each other using the torsor $\pi_1(x, y)$ of homotopy classes of paths from $x$ to $y$...}(**), the various additional structures exhibited by the category of motives, with the weight filtration being one of the very first such structures. I seem to remember that this process was less guesswork than at any other point, but rather consisted of accurate mathematical translations. The work involved several new ``exercises" on linear representations of algebraic group, which I spent days and weeks solving with great pleasure and the feeling that I was at last getting closer to a mystery that had been fascinating me for years! Perhaps the most subtle notion that I had to apprehend and formulate in terms of representations was that of ``polarization" of a motive, wherein I drew inspiration from Hodge theory, trying to extract the ideas that still made sense in a motivic context. This reflection must have taken place at the same time as my reflection around formulating the ``standard conjectures", with both events inspired\marginpar{p. 210} by Serre's idea (as always!) of establishing a ``K\"ahler analogue" to the Weil conjectures. In such a situation, in which the things themselves whisper about their secret nature and the ways in which we will be best able to delicately and faithfully express it, even though several essential facts seem to lie outside of the immediate scope of a proof, instinct dictates that we simply write down on paper what the things are insistently whispering, a message which furthermore grows clearer when we take the time to write it down! There is no need to worry about obtaining complete proofs or constructions - to burden oneself with such expectations at this stage of the process is to bar oneself from accessing the most delicate and essential step of a large scale work of discovery: that of the birth of a vision, gaining shape and substance as it emerges from an apparent void. The simple act of \textbf{writing}. \textbf{naming}, and of \textbf{describing} - if only to describe elusive intuitions or mere ``hunches" reticent about taking concrete form - possesses a \textbf{creative power}. There lies the most important instrument in enacting the passion for knowledge, when the latter is invested into things which can be apprehended by the intellect. In the process of discovery for such matters, this work is the most creative step of all; it always precedes proof and enables it - or rather, without it, the question of ``proving" something would not even arise, as nothing pertaining to the heart of the matter would have yet been formulated and seen. Through simple virtue of the effort of formulating, that which was amorphous takes form and lends itself to examination, in process of separating the visibly wrong from the possible, and above all from that which is so perfectly in accordance with the collection of things known, or guessed, that it itself becomes a tangible and reliable component of the vision being born. The latter grows richer and more precise over the course of the formulation process.  

Ten things which are suspected of being true, none of which with certainty (say for instance the Hodge conjecture), through the process of mutually shedding light on one another, completing each other and concurring with a common mysterious harmony, acquire through this harmony the strength of a vision. Even in the event all ten of these things would turn out to be false, the work which resulted in this provisory vision was not done in vain, in that the harmony which it allowed us to glimpse and to ever-so-slightly penetrate is not an illusion, but rather a reality which it is urging us to unravel. Through this work, and only this way, we were able to establish intimate contact with this reality, this hidden and perfect harmony. Once we realize that things are the way they are for a reason\marginpar{p. 211}, and that our vocation is to know them, rather than to dominate them, we are able to see the day when a mistake is highlighted as a day for celebration (56) - just as much as the day when a proof shows us beyond all doubt that such a thing which we were imagining is indeed the true and faithful expression of reality itself.

In either case, such a discovery comes as a reward to labor, and could not have been reached without it. But even though the reward may only come after years at the task, or even in the event that we never reach the final word, an achievement relayed to our successors, the work itself is its own reward, rich in every instant of that which it reveals to us in that instant.

\subnote{51$_1$}\label{note:51.1} (June 5) Zoghman Mebkhout just pointed out to me a reference to ``Grothendieck motives" on page 261 of the volume in question, in a paper of Deligne which ``resumes and completes a letter to Langlands". It reads: ``we will not be working with Grothendieck motives, as he defined them in terms of algebraic cycles, but with \textbf{absolute Hodge motives}, defined analogously in terms of absolute Hodge cycles". ``Grothendieck motives" (not underlined) are thus not mentioned as a source of inspiration, but in order to create a demarcation, insisting that the paper is treating of \textbf{something else} (the latter having been carefully underlined). This distancing is all the more remarkable given that the validity of the Hodge conjecture (a conjecture known to Deligne, I suppose, as well as to any reader of his paper-letter, beginning with its primary addressee Langlands) would imply that these two notions are \textbf{identical}!!

Of course, beginning in 1964 when I formulated the notion of motivic Galois group, I knew full well that a notion of ``Hodge motive" could be developed along the same lines, leading to a corresponding notion of ``motivic Galois-Hodge group", which was introduced independently by Tate (whether at an earlier or later time, I cannot recall) and thereafter has been known as the Hodge-Tate group (associated to a Hodge structure). The crude scam (which doesn't seem to inconvenience anybody, since it comes from such a prestigious character) consists in outright obfuscating the filiation of a novel and profound notion - that of motive - as well as the rich web of intuitions which I have developed surrounding this notion, under the derisory pretext that the technical approach taken to study this notion (via absolute Hodge cycles instead of algebraic cycles) is (maybe, if the Hodge conjecture is false) different from the one which I had (provisionally) adopted. This yoga, which I had developed\marginpar{p. 212} over the course of nearly a ten-year period, was the principal source of inspiration in Deligne's work from his very beginnings in 1968. Its fecundity and power as a tool for discovery were clear well ahead of my departure in 1970, and its identity is independent of the particular technical approach chosen to establish the validity of such or such limited part of this yoga. Deligne deserves credit for finding two such approaches, independently of any conjecture. On the other hand, he did not have the honesty to name his source of inspiration, persisting since 1968 in hiding it from everyone so as to maintain exclusive benefits, awaiting the opportunity to (tacitly) claim the credit for himself in 1982.


\subsection{The burial - or the New Father}

\nnote{52}\label{note:52} Coming back to my dream about motives, I should also mention that I remember dreaming out loud. Granted, a dreamer's work is solitary in nature - yet, the ebbs and flow of this unrelenting journey that took course over the years, in parallel with a vast project of foundations which occupied the majority of my time, had a witness on a daily basis, someone paying more close attention than Serre, who contented himself with observing things from a distance...\footnote{() (May 25) The beginnings of my reflection surrounding motives nonetheless took place before Deligne's appearance. My handwritten notes on motivic Galois theory are dated to the year 1964.}() I wrote about this day-to-day confidant in my reminiscence, saying that he ``had taken on a bit of a student role" around the middle of the 1960s, and that I had taught him ``the little I knew about algebraic geometry". I have also added that I had even told him about what I did not ``know" in the common sense of the word - these mathematical ``dreams" (on the theme of motives as on other topics) which he always welcomed with attentive ears and an alert mind, as eager to understand as I was myself.

It is true that, at the time of writing, when I said that Pierre Deligne had ``taken on a bit of a student role", I was only referring to a wholly subjective impression (57), uncorroborated (as far as I am aware) by any written - or at least by any published - source which would suggest that Deligne may have learned something from me - even though I gladly remember presently that I never once discussed mathematics with him without learning something from the conversation. (And even after I stopped discussing mathematics with him, I continued learning from him about other things which are perhaps more difficult and more important, including on this very day in the writing of these words...)

\marginpar{p. 213}I was told about the existence of a text by Deligne and others regarding the question of motives, or at the very least of ``tannakian categories", by a third party who surmised (I wonder how!) that I could be interested. Upon reaching out to Deligne thereafter, I was met by his sincere surprise that something of the sort could be of interest to me. Reading through the copy which he kindly sent me nonetheless, I realized that his surprise was indeed completely well-founded. Visibly, I had never had anything to do with the subject in question. There is at most an allusion made in passing, in the introduction, regarding the fact that certain ``standard conjectures" (which I had once formulated, heaven knows why) would have consequences for the structure of the category of motives over a field... The reader wishing to know more would be hard-pressed to do so, as no further precision or reference regarding these conjectures is made in the entire book; nor is any mention made to the one and only published paper in which I explain the way in which the category of motives over a field may be constructed in terms of the standard conjectures; nor is the only other text on the topic of motives published pre-1970 cited, an article by Demazure (produced in the context of a S\'eminaire Bourbaki, if I remember correctly) which followed my construction principle ad hoc from a slightly different perspective...\footnote{() Upon verification, I now realize that other than a few pages on the standard conjectures (Algebraic Geometry, Bombay, 1968, Oxford Univ. Press (1969), pp. 193-199), I have never published a mathematical text on the topic of motives. In Demazure's talk (S\'eminaire Bourbaki n$^o 365$, 1969/70) following Manin's talk in Russian, there are references made to a series of talks which I had given at the IHES in 1967 and which were meant (I suppose) to serve as a first broad sketch of a vision on motives. Kleiman also gave a talk on the standard conjectures and their connection to the Weil conjectures, in more details than was given at the Bombay congress announcement (Algebraic Cycles and the Weil conjectures, in Dix expos\'es sur la cohomologie des sch\'emas, Masson-North Holland, 1968, p. 359-386). I am not aware of any reflection on the standard conjectures, notably involving steps taken towards a proof thereof, other than my own pre-1970. Based on the echoes which I have received, it seems to me that the deliberate decision to ignore these key-conjectures (which I considered, as I mentioned in my Bombay sketch, as one of the most important open problems in algebraic geometry, together with the resolution of singularities of (excellent?) schemes) has a lot to do with the impression of stagnation which is currently emanating from the cohomology theory of algebraic varieties.}()

\marginpar{p. 214}Nonetheless, Neantro Saavedra, who was lucky to be one of my ``pre-1970 students", was duly cited. He had written a thesis under my supervision about what I remember calling ``rigid tensor categories", and which he named ``tannakian categories". One may again wonder through what miraculous coincidence Saavedra's thesis had foreseen in advance just the needs of Deligne's theory of motives, which was only to emerge ten years later! In fact, the work that Saavedra does in his thesis is precisely \textbf{the} key step needed for the development of a motivic Galois theory, just as J. L. Verdier's thesis was in principle \textbf{the} key step needed for the development of the formalism of the six operations in cohomology. One difference (among others) to Saavedra's credit is that he actually published his work; granted, he did not have the combined penmanship of Hartshorne, Deligne, and Illusie to exempt him from such a formality. Yet, ten years after the fact, Saavedra's thesis is now reproduced ab ovo and nearly in toto in a remarkable book, written this time around by Deligne and Milne. 

Writing this book was perhaps not strictly necessary, if all that needed to be done was to rectify two particular points in Saavedra's work (58). But everything happens for a raison, and I think I understand why Deligne himself took the trouble to do this\footnote{() See on this topic my reflections in the note ``Clean Slate", n$^o 67$.}(), going against his own extremely stringent criteria when it comes to publications, which he is known to apply with exemplary rigor when it comes to authors other than himself...\footnote{(*) (June 8) All the more so when it comes to publications which bear the sign of my influence - see on this topic the episode ``The note - or the new ethics", section 33.}(*).

Regarding the paternity of the notions involved and of the yoga of motives, the answer goes without saying in the eyes of the uninformed reader (at a time when informed readers are getting rarer by the day and will one day have run their course...) - and this without having to disturb ancestral figures such as Riemann and Hilbert or even the good Lord. If the prestigious author does not say a word about filiation, letting the pretty result on absolute Hodge cycles and abelian varieties appear as a starting point, or even as the birth, of the theory of motives, it is as an honorable act of modesty\marginpar{p. 215}, fully in line with the customs and ethics of the profession, which advise that we let others (if needs be) give credit where credit is visibly due: to the legitimate Father...


\nnote{53}\label{note:53} Affected by the vicissitudes of the orphan at hand, and doubting that someone else will do the work whose need and scope I am apparently the only one to perceive to this day, I presume that the ``bold mathematician" in question will be none other than myself, once I have reached the end of Pursuing Stacks (a project which I expect to last for about another year).

\nnote{54}\label{note:54} Since then, two new cohomology theories for algebraic varieties have appeared (other than Hodge-Deligne theory, which is a natural outgrowth in the ``motivic" spirit of Hodge theory) - namely, Deligne's theory of ``stratified promodules", and most notably the theory of crystals, ``$\mathcal{D}$-modules-style" \'a la Sato-Mebkhout, together with the new light shed on the later by the God-given theorem (alias Mebkhout's theorem) which was discussed earlier. This approach towards constructible discrete coefficients is probably destined to replace Deligne's older approach, due to the fact that it better lends itself to the expression of the connections with de Rham cohomology. These new theories do not produce new fiber functors on the category of smooth motives over a given scheme, but rather (modulo a more extensive work on foundations than has been done as of yet) they provide a way of grasping more precisely the ``Hodge" incarnation of a (not necessarily smooth) motive on a scheme of finite type over the complex numbers, or the ``de Rham" incarnation on a scheme of finite type over a field of characteristic zero. It is possible that the theory (apparently still unwritten) of Hodge-Deligne coefficients on a scheme of finite type over $\mathbb{C}$ will eventually appear as embedded into the (equally unwritten) theory of crystalline coefficients \'a la Sato-Mebkhout (with the key added data of a filtration), or, put more precisely, as the intersection of Hodge-Deligne theory with the theory of constructible discrete coefficients in $\mathbb{Q}$-vector spaces... There also remains the elucidation of the relationships between crystalline theory \'a la Mebkhout and the theory developed in positive characteristic by Berthelot and others, a task perceived by Mebkhout before 1978, in the midst of a completely disinterested environment, and which appears to me to be one of the most fascinating questions currently posed in the endeavor of understanding ``the" (unique and indivisible, i.e. motivic!) cohomology of algebraic varieties.

\nnote{55}\label{note:55}\marginpar{p. 216}Even though I was only dreaming, my dream about the relationship between motives and Hodge structures led me to inadvertently notice an incoherence in the ``generalized" Hodge conjecture, such as it was initially formulated by Hodge, and to replace it by a rectified version which itself should be (or so I would wager) no more nor less false than the ``usual" Hodge conjecture about algebraic cycles.

\subsection{Prelude to a massacre}

\nnote{56}\label{note:56} 
I am notably thinking about Griffith's discovery, in the context of the cohomology of
algebraic varieties, regarding the falsity 
of a tantalizing idea that 
was circulating for a long time concerning algebraic cycles, namely that a 
cycle homologous to zero admitted a multiple 
which is algebraically equivalent to zero. 
This discover of a brand new phenomenon was so striking that I spent a whole week 
trying to wrap my head around Griffith's example, by transposing his construction (which
was transcendental over the field $\CC$) into a ``maximally general'' construction, making
sense over base fields of arbitrary characteristic. This extension was not entirely
obvious, involving (if I remember correctly) Leray spectral sequences and the Lefschetz
theorem. 

(June 16 ) This reflection lead me to develop
the cohomology theory of ``Lefschetz pencils'' in the \'etale context.
My notes on this topic were developed during the 
SGA 7 II seminar (by P. Deligne and N. Katz) as well as in the expose\'es 
XVII, XVIII, XX by N. Katz (who took the care to reference these notes,
which he closely followed).
On the other hand, in the volume's introduction by P. Deligne, where it is said that the
key results of the volume are expos\'e XV (the Picard-Lefschetz formula for \'etale
cohomology) and XVIII (the theory of Lefschetz pencils), 
the author abstains from indicating that I had anything to do with 
this ``key theory'' of Lefschetz pencils. 
In reading the introduction, one gets the impression that I played no part in the
development of the volume's themes.

The long seminar SGA 7, which took place in the years 1967-69,
continued the seminars SGA 1 through SGA 6 which were
developed under my impetus between 1960 and 1967, 
was organized by Deligne and myself, after I kicked off a development of a systematic
theory of groups of vanishing cycles.\todo{\'evanescents}
Since the write-up of the talks by various volunteers dragged for some time, 
the two volumes of the seminar (SGA 7 I and SGA 7 II) were only published in 1973, by
\marginpar{p. 217}Deligne. Even though it was agreed during the seminar that we would be
presenting it as a common endeavor, Deligne made the (strange) request after my departure
that we \textbf{cut the seminar in half}, with a part I presented as 
being directed by me, and the other half by himself and Katz. 
I now view this event as part of an ``operation'' foreshadowing the 
operation ``SGA $4\frac{1}{2}$", which aims 
(among other things) to present the entirety of the foundational series SGA 1
through SGA 7 - which in his mind and point of view were inseparable from my person, 
as well as the series EGA, i.e. El\'ements de G\'eom\'etrie Alg\'ebrique, 
as a compendium of texts with a variety of authors, where I myself only play an
episodic, if not superfluous, role.
This tendency appears very clearly, if not brutally, in the volume SGA $4\frac{1}{2}$
and most notably in the seminar SGA 5, which is inextricably linked to that volume.
See the note ``the clean slate'', and ``the massacre'', \no s 67 and 87, and most of all
``the remains\ldots'' (\no 88), among others.

(June 17) I was responsible for the overall structure of the seminar
SGA 7 (for which I did not see a need for a separation into parts ``I'' and ``II'', and still
do not to this day) while Deligne made important contributions 
(mentioned in my report on Deligne's work written in
1969, see \no 13, 14 of this report), 
the most crucial for the needs of the seminar being the Picard-Lefschetz formula, proven
via a specialization argument starting from the already known case in the transcendental
setting.
The cutting of the seminar in two parts was unjustified mathematically, as well as
regarding our respective contributions - both Deligne and I brought substantial
contributions to each of the two ``pieces'' of SGA 7.

I would of course have been delighted if Deligne had continued the foundational series
SGA, which I had started and was far from reaching its end!
However, this ``operations SGA 7'' is not at all a continuation but rather a sort of
brutal ``saw cut'' (or chainsaw cut\ldots) \textbf{putting an end} to the SGA series, 
with a volume which ostentatiously is distinct from my person, even though it is linked to
my work and bears its mark as much as any other. 
Even though my person is obscured as much as it can be, the tone taken with respect to my
work is not yet the barely disguised tone of disdain that characterizes the ``operation
SGA $4\frac{1}{2}$'' - 
the latter represents an even more brutal saw cut, affecting the unity of
the seminars SGA 4 and 5, and serving as a means and pretext to the lawful plunder of the
unpublished part of SGA 5, whose stolen pieces were equitably shared by
Deligne and Verdier\ldots

\nnote{57}\label{note:57}
\marginpar{p. 218}I should quickly mention that the same remark applied to the other
gifted mathematician about whom I hazard to say (in note \no 19) that he took
on a bit of a student's role, ten years after Deligne.

\nnote{58}\label{note:58} This reminds me that the (publication?) Notes (which had already published six or seven ``pre-1970" PhD theses produced under my supervision) never accepted to publich Yves Ladegaillerie's thesis from ``post-1970" (stated reason: they do not publish theses!). It should be said that they did on the other hand publish Saavedra's thesis for a second time... I had also told Deligne about Ladegaillerie's beautiful result on isotopy which was refused by every journal (with the secret hope that he would lend a hand to help him publish it) - but it unfortunately did not interest him (stated reason: his incompetency in subject of the topology of surfaces...).

End scene...

\subsection{The new ethics (2) - or the free-for-all fair}

\nnote{59}\label{note:59} (April 20) During the few weeks separating me from the writing of the above lines, which identify a contradiction and its cost, I learned with surprise that the person in question had already found a most simple way to ``resolve" said contradiction two years ago - one only had to think about it! This solution could be called ``the method of the preemptive burial" (about which the reader can learn in the double note (50), (51) written yesterday, while still freshly affected by the discovery). I apologize in advance if the unexpected reappearance of the preemptively deceased individual in the famous ``mathematical world" (which sometimes takes on the airs of a free-for-all fair...) risks introducing technical complications to the flawless execution of this brilliant method! In an earlier note (``deontological consensus - and control of information", n$^o 6$) I felt (still confusedly) that the most universally admitted deontological rule in the scientific profession ``went unheeded" as long as the individuals who have control over the circulation of scientific information did not respect the right that any scientist has to make their ideas and results known. Around this stage of the reflection, I also took the time to thoroughly describe a case study in which the disregard shown for this right was flagrant in my eyes - so much so that I felt that the disregard displayed was bordering on disdain for the number one rule, which itself is admitted by general consensus. (See ``The note - or the new ethics", section 30).

\marginpar{p. 219}This wasn't the only time that I felt this very particular sense of unease, witnessing the \textbf{spirit} of the number one rule being disregarded while the very perpetrator was displaying a ``thumbs up" through his position (above all suspicion!), his means, and the casualness of the execution. I attempt to pin down this uneasiness in the note ``The snobbery of the youth - or the defenders of purity", in connection with the aforementioned section. Once we begin disregarding the ``obvious" things about which I am speaking, as well as (I should add) the (possibly deep) things which are neither proven nor patented as ``conjectures" published and known by all, we may as well (given what little there is!) consider them to be public property (trivial, of course)\footnote{(*) Such was the fate of the ``God-given theorem" (aka Mebkhout's theorem),

(June 8) And this, as for the yoga of motives, while also deftly creating an impression of filiation, without ever saying so explicitly! See on this topic (as a case study) the note ``The Prestidigitator" n$^o 75$, and for the brilliant general method or style, the note ``Thumbs up!" n$^o 77$, as well as the note to follow ``Appropriation and disdain", n$^o 59'$.}(*), and as such, when the time comes, as ``one's own", with the greatest nonchalance and the most unaffected conscience - because it goes without saying that we would never claim ownership over a difficult proof of ten or a hundred pages (or even ten lines) which establishes a result that ``we would not have been able to prove" (59'). I did not think I was being so sensitive or accurate (regarding the ``unheeded rule")
\todo{missing sentence}
.\footnote{(**)}(**)

I fortunately have the ability to defend myself - being able to express with some accuracy what I feel and want to say, having acquired (rightly or wrongly) a certain credibility, and having the chance of being heard when I have something to say, or having the possibility to publish if I feel the need to do so. On the other hand, I now vividly realize the ``feeling of injustice and powerlessness" to which aggrieved individuals\marginpar{p. 220} without my privilege are subjected, feeling like their hands are tied in the face while ``those who own everything" may dispose of them arbitrarily - a power which they use however they please.

It is true that I have at times displayed such condemnable behavior in my own life as a mathematician, in equally good conscience - and I have had the occasion in my reflection to speak about some of these instances as my conscience brought them out of my subconscious, where they had been buried together with their unexamined ambiguity. In probing these events I finally understood that I had no reason to be surprised at the fact that today (and for quite some time) the student has surpassed the master, and that I shouldn't repudiate anyone towards whom I feel sympathy or affection. It is nonetheless healthy for me and for others to call a cat a cat, whether that cat be from our home or someone else's home. 

\subsection{Appropriation and contempt}

\nnote{$!59'$}\label{note:!59'} (June 8) I am no longer convinced of the above, concerning my friend Pierre Deligne, as I have witnessed that he eventually gave in to the game of ``tacit filiation" vis-\'a-vis the tool of $l$-adic cohomology, i.e. what I call the ``mastery" of \'etale cohomology. A remarkable evolution occurred between the ``operation SGA 4 $\frac{1}{2}$" (where my name is still spoken, albeit with an attitude of flippant disdain towards this central component of my work, from which his own draws its origins) and the ``The Funeral Service" in which any reference to even the word ``cohomology" is scratched in relation to my name. (See the notes ``Clean slate" and ``One of a kind" for the initial phase, and the notes ``The Funeral Service (1), (2)" for the final phase.)

Among the intermediary phases in this escalation, there was the ``memorable paper" of 1981 on so-called ``perverse" sheaves (see on this topic the notes ``The inequity - or a feeling of return" and ``Thumbs up!", n$^o 75$ and $77$), and the exhumation of motives in LN 900 the following year (the Funeral Service taking place the year after that, in 1983). In all of these cases and others of lesser scale, I was able to realize that the internal attitude and ``method" which allowed Deligne to claim credit for others' ideas with flawless good conscience was that of \textbf{disdain} (one which remains partially tacit, all the while being deftly suggested) towards the ``little" which we are about to appropriate - so ``little" in fact that there is no need to even speak about it, especially given that we are about to use it right away to do truly powerful things - think Weil conjectures, theory of so-called ``perverse" sheaves, ... After the operation is finished, and the appropriation is complete and accepted by all, there is always time to rectify the situation and to modestly show off that which has been appropriated. The same contribution is treated with offhand disdain while it remains attached to the name of one of those\marginpar{p. 221} who are to be buried, only to be highlighted once it has been appropriated by himself ($l$-adic cohomology, motives, Mebkhout's yoga) or by a good friend (yoga of derived categories and yoga of duality, appropriated by Verdier under Deligne's active encouragement).

\section{V My friend Pierre}

\subsection{The child}

\nnote{60}
\marginpar{p. 223} (April 21) Coming back to my dream about a memory, which concerns more than the
birth of a vision... I remember (even though I have forgotten so many things!) the
ever-renewed pleasure which I took in discussing
with the person who had quickly become my confidant on everything which captivated me,
as well as each step forward and encthanting discovering in my day-to-day
love story with mathematics - and as such he never really was a ``student''.
His perennial receptiveness and the ease with which he learned about each thing
(``as if he had always known it\ldots") acted as a constant source of enchantment.
He was an ideal listener, moved by the same thirst for understanding as my own - he has an
extremely sharp ear, signaling a communion between us.
His comments always ran ahead of my own intuitions or restraint,
or shed some new light on the reality I was painstakingly trying to grasp through the mist
that still surrounded it. As I have said elsewhere, he often has answers to the questions
which I was asking, sometimes even on the spot, while other times he would reach the
answer in the days or weeks that followed.
The role of the listener was reciprocated when he took his turn in sharing the answers
which he had found, which he presented as no more than the nature of things, always
appearing with perfect naturality, and presented with the same ease which had tantalized
me and certain of my elders such as Schwartz and Serre (as well as Cartier).
This was the same simplicity, the same ``evidence'' which I had always pursued in my
quest to understand mathematical things.
Without needing to mention it, it was clear through our shared approach and standards
we both belonged to ``the same family''.

Ever since we first met, I had felt that his ``abilities'', as we say, were of a very rare
quality, and far exceeded the modest abilities which I myself possessed, even though we
were of the same breed when it came to our passion for understanding and out exigency
regarding the comprehension of mathematical things.
I also sensed, dimly, without yet being able to put my finger on it that this ``strength''
which I noticed in him (and which I also noticed in myself, although present to a lesser
degree) of ``seeing'' the obvious things which nobody else could see was the faculty of a
child as well as the innocence of a child's eyes.
He held within him something of a child, much more visibly than other mathematicians whom
I have known, and this was surely not an accident.
He once told me that one day, while he was still in high school I believe,
he independently took the time to verify the multiplication table
(as well as the addition table along the way) \marginpar{p. 224}
for the numbers $1$ through $9$ from the definitions.
Not that he expected to find anything surprising - other than possibly the (pleasant as
always\ldots) surprise that the proof could be accomplished so beautifully and completely
in a matter of a few pages, and in just about half an hour.
I could sense, while he cheerily related this anecdote, that this had been a half hour
well spent, and that is something I understand today even better than I did then.
This story of his had marked me, even impressed me (even though I don't recall making that
known to him) - because I saw it as the sign of a \textbf{self autonomy},
and of a certain freedom with regards to received knowledge, both of which had accompanied
my relationship to mathematics since childhood, at the very first contact.
(69)\footnote{() In fact, I believe that this freedom has never entirely disappeared over the course of my mathematical life, and that it is now as present as it had been during my childhood. Two or three years ago, I reminded my friend about the multiplication table episode. He seemed embarrassed by this evocation of a childhood memory, which visibly no longer corresponded to the image he had of himself. I wasn't really surprised by this embarrassment, but it pained me to see once again a confirmation of something which I had by then understood but struggled to admit...}().

This status of privileged interlocutor which we shared with one another,
at a time when we saw each other nearly every day\footnote{(*) Such was the case while I was in Bures, and he was housed in an IHES accommodation. Starting from 1967 (when I moved to Massy), I think that we still saw each other once or twice a week, at least during the time when I was still invested in mathematics.}(*),
continued over a period of 5 years - from 1965 (if I remember correctly) to 1969 included.
I still remember the pleasure I took, during that year, in writing a comprehensive report
of his works as part of my proposal to appoint him as professor at the
institution at which I had been working since its creation (in 1958), and where I produced
the largest part of my mathematical work.
I no longer possess a copy of that report (64), in which I had reviewed around a dozen
papers by my friend. Almost all of them are still unpublished (in fact, many remain
unpublished),
and most if not all of them individually contained, in my opinion, enough substance to
constitute a solid Ph.D. thesis. It made me prouder and happier to present this eloquent report on his behalf than if I had been presenting a report on my own works (something which I have only done twice in my life, and even then only by forcing myself...). Many of his papers brought answers to questions which I had asked (the only published one being the aforementioned paper on the degeneracy of the Leray spectral sequence for a smooth and proper morphism of schemes (63)). Nonetheless, the two most important\marginpar{p. 225} papers constituted answers to questions that Deligne had asked himself, and in this case it was clear that their scope far surpassed that of a ``solid PhD thesis". I am referring to his work on the Ramanujan conjecture (which appeared in the Bourbaki seminar), as well as his work on mixed Hodge structures, also known as ``Hodge-Deligne theory".


At the time of writing of this shining report, the thought that I would be leaving the institution at which I was about to appoint my young and impressive friend, and where I was hoping to dwell for the foreseeable future, would have appeared to me as strange and very much unsuspected.  Likewise (as I am now able to connected these two pairs of double-events), it seems strange, and probably not ``coincidental", that this same (no longer young!) friend of mind has communicated to me a month or two ago his own departure from that very institution, just about a year after I had resumed a regular mathematical activity in the form of an unexpected ``return" to the mathematical world (if not to its ``high spheres"...).

I expressed myself about this departure - this ``salutary disconnection" - more than once
in R\'ecoltes et Semailles, and I wrote even more often about the ``awakening" that
happened soon thereafter, rendering this episode a crucial turning point in my life.
During the intense years that followed, the mathematical world, as well as the members
whom I had loved and the components of mathematics itself which had most fascinated me,
became very distant from my mind - as if drowned into the mist, memories of another ``me"
who had been dead for ages\ldots

Well before this episode, as well as during the years that followed this first 
great turning point, I knew that he who had been my students (a little\footnote{(*) For more on this hesitation to consider the (overly!) brilliant Deligne as one of my students, see the note ``One of a kind" (n$^o$67).}(*))
as well as a confidant and a friend (a lot) needed only follow discontinuous 
impulse of the child in him, at play and in the pursuit of knowledge, to discover nd bring
forth new and unexpected worlds, probing them and understanding their intimate nature - 
and by so doing revealing them to his peers and to himself. 
As such, if I could envision a ``bold mathematician'' busy laying out a rough outline (to begin
with\ldots) after my departure (with no intention to return!) of the vast scenery 
which I had glimpsed, and of which I had only made a few cursory and provisional sketches.
It would definitely have to be him - 
as he had everything in place to do so!
To paint this far reaching tableau, a ``master plan'' uniting in a common vision the
essential points of what was known and already guessed about the cohomology
\marginpar{p. 226}
of algebraic varieties, required the work of a few months rather than years,
for a person within whom this vision was already in place and ready to take form out of
the mist of the yet unwritten.
(Even if it involves further developing it over the course of years or generations if
needed - until the end word of the reality of motives has been fully understood and
established.)
At that time, I had no doubts that this work, which felt ``burning hot'' would be
accomplished at any moment, at the very least during the two or three following years.
After my departure, there remained a single person who was called, by his very impetus
towards knowledge, to pursue this pressing and fascinating work,
even if it meant letting others further pursue this work.
Once the master plan has been
written and proofread, one can always launch into other adventures within the world of
mathematical things, where every twist of the road brings the promise of a new and
limitless world, for those who come with fresh and open eyes\ldots

At the time when my life was still unraveling in the warm scientific cocoon, isolated from
the noises of the world, and when Deligne was developing his extension of Hodge theory
(the year must have been 1968 or 69) it went without saying that this work was a first
step that needed to be taken in order to realize, test, and make more precise a certain
\textbf{part} of this ``tableau of motives", which had not been written out in its
entirety.\footnote{(*) The fact that this Hodge-Deligne theory never (as far as I am aware) developed beyond the stage it reached through this first jet, to grow into a theory of ``Hodge-Deligne coefficients" (and of the associated ``six functor formalism") on schemes of finite type over the complex numbers is indissociable from another strange fact: namely, that this vast ``tableau of motives" has never been painted, its very existence having been carefully silenced to this day...}(*)
In the years following my departure from the cocoon, at a time when mathematics seemed
very distant from  me, I was not surprised upon learning that the Weil conjectures had
been finally proven. (If anything surprised me it was that the ``standard conjectures''
had not been proven in the same stride, even though they had been formulated as a possible
approach towards the Weil conjectures, as well as a way to establish, at the very least, a
theory of semisimple motives over  field.\footnote{(**) It is only in the past few years that I began to vaguely realize (albeit with more precision as of late!) that the ``standard conjectures", as well as the very notion of motives to which they provided a ``constructive" approach, had been \textbf{buried}, for reasons that now appear to me as particularly clear. (Compare also with the preceding footnote).}(**).)
I knew that neither this first 
\marginpar{p. 227}
jet towards a general theory of Hodge-type coefficients, nor
through this proof of certain key conjectures (among many others that were more or less
known) he was still not operating at full measure - he was in fact falling significantly
short of that. And I was impatiently waiting for him to reach his full potential, even
when the bulk of my attention was taken up by other things. (-$>$ 61)

\subsection{The burial}

\nnote{61} 
I had the privilege to witness his first youthful spirit carrying the promise of a vast
expansion to come. During the following fifteen years, I started realizing that this
promise was being repeatedly delayed. There was something delicate within him that I had
been able to sense and recognize (even at a time when I was insensitive to so many things!)
something altogether different than shear intellectual horsepower (which is as smothering
as it is penetrating\ldots) - the most essential thing of all for any truly creative work. 
I had sensed this in others at times, but I had never felt it to such a degree in any
other mathematician. I had expected (as is natural) that this force would continue
blossoming and developing within him, to later effortlessly express itself in the form of
a unique work of which I would have been the modest precursor.
Equally as strangely (and there surely must be a deep and simple link connecting all of
these ``strange phenomena''), I saw this delicate thing, this ``force'', which pertains
neither to muscle nor to brain go progressively dimmer over the course of the years as if
\textbf{buried} under successive layers, ever thicker layers of \textbf{something else}
that I know too well - 
the most common thing in the world! 
Said thing doesn't necessarily collide with intellectual horsepower, nor would it impede a
consummate experience or a skilled flair in a particular field, and both of these skills
would force the admiration of some or fear of others or both at once
through the accumulation of accomplishments that are potentially brilliant and would
surely project force and beauty. Yet it is not \textbf{this} which I had in mind when I
spoke of ``expansion'' or of ``blossoming''. 
The  blossoming I had in mind was to be the
fruit of innocence, hungry for knowledge, and always ready to rejoice at the beauty of
things both small and large in this inexhaustible world of ours, or in some particular
region of this world (such as the vast universe universe of mathematical things\ldots).
Only a blossoming of this kind holds the potential for a profound renewal, being the
renewal of oneself or that of ones understanding of things in the world. This potential
was fully realized, I believe, in the humble man Riemann\footnote{(*)}(*).
This true blossoming removes one from disdain\marginpar{p. 228}; be it disdain towards
others (whom one feels are far below oneself\ldots) or disdain towards things that are too
``small'' or too obvious to warrant attention, or which are deemed below ones legitimate
expectations; or even disdain for a given \textbf{dream}, which is insistently
communicating things which we claim to love\ldots
Disdain thereby becomes as foreign as the self-satisfaction which feeds it.

Granted, because of his impressive ``means'' as well as this delicate things which
impresses no one yet \textbf{creates}, the ``student'' was destined to far surpass the
``master''. I did not doubt that right after my departure from this milieu where I had
witnessed such a beautiful ascent, Deligne would give his full measure 
to the development of a vast and profound work which I will have helped initiate.
The echoes of such work would surely reach me over the course of the years, 
and as I will be engaged in separate quests far from mathematics,
I will only be able to partially grasp the full scope and beauty of the worlds that he was
about to discover. 

However, the student cannot surpass the master while simultaneously inwardly
\textbf{disavowing} him, secretly laboring towards the suppression of any trace of what he
had contributed (whether that contribution had been for the better or the worse\ldots),
both from his own eyes and the eyes of others - in the same way that the son cannot truly
surpass the father while simultaneously disavowing him. 
This is something that I have learned mostly through my relationship with my own children,
but also (later) through my relationships with certain past students of mine; 
mostly the one student, among others, whom I had always refrained from calling a
``student'', having felt since the day we met
that I was about to learn from him as much as he would be learning from
me\footnote{(*)}(*).
Only ten years after that day, post 1975, and especially since I began meditating upon the
meaning of my experiences, have I begun sending this \textbf{hindrance} within 
this person who nonetheless remained dear to me.
I also sensed, obscurely, that this secret disavowal of my person and of the role which I
had played during these crucial years of his life, were also, on a deeper
level, a disavowal \textbf{of himself},
(so it goes, godlessly, every time that we disavow and wish to erase \marginpar{p. 229}
something which has
happened to us, when it is our role to harvest the fruit\ldots).

Having failed to remain ``plugged into'' ``what has been happening in mathematics", 
including what he had been doing himself\footnote{(*)}(*), I had never realized,
until I stopped to reflect a couple of weeks ago, the extent to which this hindrance
had weighed equally on that which he had given his whole: his mathematical work.
Admittedly, more than once in the last eight or nine years, I have seen his mathematical
instinct and simple common sense be erased through a deliberate manifestation of disdain 
(towards me) or of contempt (towards other whom he had the power to discourage) (66).
He also wasn't the only one of my ex-students - with or without quotation marks - 
in whom I witnessed similar attitudes towards people whom I held dear 
(or towards others). None of these other occurrences affected me as painfully.
More than once over the course of the past two months I have alluded to this experience in
my reflection, ``the most bitter experience of my life as a mathematician'' 
- and I have also shared what it taught me as part of my reflection.
This pain was so acute, it taught me something profound about a person whom I had
always held dear (while I was also busy excavating what he taught me about myself and my
past\ldots),
that the question of whether or not this event had an impact on mathematical creativity
whether within him or within those who were discouraged or humiliated, became entirely
accessory not to say derisory.

The note ``refusal of a heritage - or the premise of contradiction'' is the first written
reflection in which I present a summary of what had occurred to me bit by bit, over
the course of the years, both regarding the ``state of the art'' as well as regarding the
work of the person whom I had known so well yet so little.
This marks the first time that I finally saw, at a glance, the \textbf{``price''}, or the
weight within his mathematical work, of this refusal that he must have been carrying within him
for over fifteen years.
Even then, in writing this note, I was ``delaying'' the chain of events, 
in that for more than two years already (without anyone deeming it useful to update me),
motives had become public knowledge and removed from the secretive state in which they had
been kept for twelve years\ldots
Today, as I write the ultimate section (I think) of my reflection on my mathematical past,
and just two days after having learned the broad\marginpar{p. 230}
strokes of the memorable volume 
consecrating this furtive ``introduction", I am stricken by the 
crushing weight that transpires.
By this I mean the weight that is carried around, 
day after day and through circuitous paths,
by a being born to fly - lightly and delicately, with joy and confidence towards the
unknown, out of shear joy for oneself and for the carrying wind\ldots
\footnote{(*)}(*)

If he does not fly, instead contenting himself with being admired and feared by others, by
accumulating proofs of his superiority, I need not worry. He must surely find satisfaction
in carrying around this weight - just as I in turn carried 
some weights and continue to this day to carry those which I did not know how to leave behind. 
He took what he liked from what I had to offer, both good and bad, and left the rest. 
I need not worry about such choices which pertain only to himself; nor need I hereby
decree which choices are better than others (62). What is ``better'' for one is worth for
the other, or sometimes even for the same person (who may have changed over time, albeit
this is a rare occurrence\ldots).

The choices I do make and the acts that follow (even when our words deny them), 
are made at my own risk. If they often bring us the expected rewards (which we receive as
``the goods''), such rewards often end up having repercussions 
(which we recuse as ``bad'' and often react with outrage).
Once we understand that the repercussions cannot be recused, we start considering them as
a price to pay, and we so oblige with reluctance. 
Yet we sometimes understand that these repercussions need not be seen as cashiers 
who must be paid for the good time we had.
Rather, they can be seen as patient and persistent messengers who tirelessly 
come back to us with the same message; an unwelcome message constantly recused
- because this humble message, even moreso than the repercussion itself, is what appears
to us as ``the worst''; worse than a thousand repercussions, worse than a thousand deaths,
or the destruction of the entire universe, about which we no longer give an f\ldots

\marginpar{p. 231}The day we finally decide to welcome this message is the day when our
eyes suddenly open to see that what was feared as ``the worst'' is really a
\textbf{liberation}, an immense salvation - and that the crushing weight from which we are
suddenly relieved is the very thing to which we were holding on just yesterday as ``the
goods''.

\subsection{The event}

\nnote{62} (April 21) 
If there is no reason for me to worry, one may ask why I am elaborating for page after
page about a personal relationship,
when this relationship only concerns me and the
person in questions!

If I felt the need to launch into this retrospective reflection on certain important
aspects of this relationship, it is due to the impact of a specific event which hit close
to home (even though I am only learning about it two years after the fact). 
This event is furthermore public information, even more so than the behaviors and routine
actions of renowned mathematicians (such as Deligne and myself) towards lesser known 
or beginner mathematicians (even though the impact of these acts on people's lives 
is often far greater than in the case at hand). 
The even in question (namely the publication of the ``memorable volume'' that is the
Lecture Notes LN 900, aka ``the burial volume'') as well as the circumstances surrounding
it appeared to me as \textbf{toxic}, at least in my eyes. 
As such, it seemed healthy for everybody, starting with the ``person of interest''
himself, to give a more circumstantial testimony regarding certain
ins and outs of the matter, so as to lay things down as I perceive them today. 

Through this testimony and this reflection, I am not trying to convince anybody of
anything (something far too tiring, not to mention hopeless!)\footnote{(*)}(*)
But rather to understand the events and situations in which I found myself involved. 
If this leads others to a true reflection, beyond the usual formalities, then this
testimony will not have been published in vain. 

\subsection{The eviction}

\nnote{63}
\marginpar{p. 232}
This article\footnote{(*)}(*) in the Publications Math\'ematiques in 1968, 
two years after my departure from the mathematical world. Its starting point had been a
conjecture which I had shared with Deligne, regarding a degeneracy property of certain
spectral sequences, which at the time could have seemed rather incredible, yet  which
nonetheless became plausible through the ``arithmetic'' viewpoint, as a consequence of the
Weil conjectures. This motivation was of great interest in itself, as it showed just how
much one could gain from using the ``yoga of weights'' implicitly contained in the Weil
conjectures (a yoga that was first glimpsed by Serre, in certain important aspects of it).
Already then, I currently applied it to all kinds of analogous situations, 
in order to extract conclusions about the ``geometric'' nature 
(for the cohomology of algebraic varieties) using ``arithmetic'' arguments. 
This happened on a heuristic level for as long as the Weil conjectures had not yet been
established, but it nonetheless had a 
great power as a proving mechanism, representing a \textbf{tool of discovery} of the
highest order. 
Deligne's ``geometric'' proof for the particular conjecture in question, using the
Lefschetz theorem (which had at the time only been established in the characteristic zero
setting) was interesting for a different reason, 
in addition to being independent of any conjecture.
The link indicated by these two approaches between two things which could at first sight
seem to be unrelated, namely the Weil conjectures on the one hand (and the associated yoga
of weights, which for me constituted the most fascinating aspect), and the Lefschetz
theorem on the other hand, was extremely instructive in itself. 

Most interesting for the present purposes, is something 
which only fully appeared to me today, namely that the reader of this article would be
given little reason to think that I had something to do with the initial motivation of the
main result, and no reason at all to learn what exactly this motivation was. 
(See also the beginning of note (49.))
% \todo{cite}
The \textbf{spontaneous} approach
(including, I am sure, for the author himself)
in introducing such a result would have been to begin with the (striking) conjecture,
then to indicate the first piece of motivation, equally striking, which would have
constituted a good occasion to finally ``sell'' this famous yoga of weights, a much
farther reaching accomplishment than the \marginpar{p. 233}principal result of the article
\footnote{(*)}(*) then to follow up with the 
``Lefschetz theorem'' viewpoint, allowing one to prove the initial conjecture under
slightly more general conditions, over an arbitrary base scheme, not necessarily smooth
and proper over a field, but only in zero characteristic. 
On the other hand, the chosen approach begins with homological algebra generalities
(very pretty of course, and presented with the customary elegance of the author), 
generalities which he has probably since forgotten, like everyone else, 
as a sort of axiomatization of the Lefschetz theorem. The main result (the only one of
course which everybody remembers) appears around corollary X in the middle of the paper, 
while the word ``weight'' and my name are only pronounced in ``remark 3.9'' somewhere near
the end (without the reader really knowing why)\ldots

\subnote{$63_1$}

\subsection{The ascension}

\nnote{63'}

\subsection{The ambiguity}

\nnote{63''}

\subsection{The accomplice}

\nnote{63'''}

\subsection{The investiture}

\nnote{64}

\subsection{The knot}

\nnote{65}

\subsection{Two turning points}

\nnote{66}

\subsection{Clean slate}

\nnote{67}

\subnote{$67_1$}

\subsection{One of a kind}

\nnote{67'}

\subsection{The green light}

\nnote{68}

\subsection{The reversal}

\nnote{$!68'$}

\subsection{Squaring the circle}

\nnote{69}

\subsection{The funeral}

\nnote{70}

\subsection{The coffin}

\nnote{71}

\end{document}
