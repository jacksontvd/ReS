\begin{comment}
\documentclass{book}
\usepackage{master}
\usepackage{changepage}
\newcommand{\rec}{$\text{R\'ecoltes et Semailles}$}
\newcommand{\no}{n$^\circ$}
\hfuzz = 100pt

% NOTE and SUBNOTE FORMATTING
\usepackage{titlesec}
\usepackage[dotinlabels]{titletoc}

% define `note'
\titleclass{\nnote}{straight}[\section]
\newcounter{nnote}
\renewcommand{\thennote}{\thesection.\arabic{nnote}}
\titlespacing*{\nnote}{0pt}{3.5ex plus 1ex minus .2ex}{1ex plus .2ex}
\titleformat{\nnote}[runin]{\bfseries }{\bfseries Note }{0pt}{}[]

% define `subnote'
\titleclass{\subnote}{straight}[\section]
\newcounter{subnote}
\renewcommand{\thesubnote}{\thesection.\arabic{subnote}}
\titlespacing*{\subnote}{0pt}{3.5ex plus 1ex minus .2ex}{1ex plus .2ex}
\titleformat{\subnote}[runin]{\bfseries }{\bfseries Note }{0pt}{}[]

% the first optional arg sets the size of the indentation in the TOC
\titlecontents{nnote}[9em]
{}{Note }{}{\titlerule*[1pc]{.}\contentspage}

% the first optional arg sets the size of the indentation in the TOC
\titlecontents{subnote}[11em]
{}{Note }{}{\titlerule*[1pc]{.}\contentspage}

\begin{document}
% print table of contents with notes and subnotes
%\setcounter{tocdepth}{5}
%\setcounter{secnumdepth}{5}
%\startcontents[chapters]
%\printcontents[chapters]{}{0}{}

\end{comment}

\setcounter{chapter}{12}
\chapter{A) Heritage and heir}

\section{Posthumous student}

\subsection{Failure of an instruction (II) - or creation and fatuity}

\nnote{44'}
[This note was mentioned in section 50 of
\textbf{VIII The solitary journey} of part \textbf{(I)
Fatuity and renewal} p. 227]

\marginpar{p. 173}
This passage ``clicked'' for the friend who read the previous section ``the weight of a
past''\footnote{(*) (May 10) This friend is none other than Zoghman Mebkhout, who
authorized me to reveal his identity, after I thought I should keep it secret upon first
writing this letter (on April 2nd 1984).}(*) He wrote: ``for many of your old students,
the aspect, as you put it, of an invasive and borderline destructive ``boss'' remains
strong. Whence the impression you hold.'' (Namely, I presume, the 
``impression'' which is expressed in certain passages of this section as well as in the
Notes
\no \nameref{note:46}, \nameref{note:47}, \nameref{note:50} which complete it.)
Earlier, he writes: ``first of all I think that you did well to leave mathematics for an
instant [!]. Because there was a kind of incomprehension between you and your students,
except of course for Deligne. They were left a bit dumbfounded\ldots''. 

This is the first time that I hears about the impression I made in my role as ``boss''
pre 1970, beyond customary compliments!
Even earlier in the same letter: ``\ldots I have come to realize that your old students 
[namely: those from ``before 1970''] do not really know what a mathematical
\textbf{creation} is, perhaps in part because of you\ldots it must be said that in their
time, the problems were clear-cut\ldots'' \footnote{(**) (May 10) The preceding citation
was heavily modified, in order to respect the anonymity of my correspondent. See the
following note for a complete citation of the relevant passage, as well as for a
commentary on its real meaning, which I had missed at first due to a lack of further
contextual information.}(**).

My correspondent surely meant that \textbf{I} was the one who formulated the ``problems''
and, with them, the notions that needed to be developed instead of 
leaving both tasks to my students; and that in so doing I may have 
prevented them from becoming acquainted with what becomes the essential part of a work of
mathematical creation. This also aligns with an 
\marginpar{p. 174}
impression which I formed after talking
to two of my students from \textbf{after} 1970, about which I wrote in an earlier note
(note (23iv)).
It is true that I was looking first and foremost, in the students that approached me, 
for \textbf{collaborators} 
with whom to develop intuitions and ideas which had already formed within me, 
to ``push along'', in sum, a carriage that was already there, which they did not have to
summon from some kind of void, ``something which my correspondent had to do''.
This summoning - the act of bringing into being a tangible, supple, 
intense body of work from the intangible mist - had indeed always been, for me, the most
fascinating aspect of mathematical work, as well as the part in which I most strongly felt
a process of ``creation'' the ``spirit
of something more delicate and essential than a mere result''.

If I see certain ex-students of mine treating this valuable thing with disdain, 
letting grow within them
this ``snobbery'' which J.H.C. Whitehead talked about (consisting of disparaging 
what is ``immediately provable'')\footnote{(*)See the 
note (the snobbery of the youth - or the defenders of purity), \no 27 p. 247.
}(*), 
I am at least party to blame, for various reasons. 


However, I would not go as far as saying that the work which I suggested to my students, or which they produced with me, was of a purely technical nature, strictly a matter of routine, or inept to using their creative faculties. I offered them some starting points which were tangible and sound, among which they were free to choose, and from which they could launch further, just as I had done before them. I do not think I ever suggested a topic to a student which I would not have been happy to work on myself; nor was any of the journeys which they underwent with me more arid than what I have weathered over the course of my mathematical life, without loosing hope or kicking over the traces, when it was clear that the work had to be done and that there was no way around it. 

\marginpar{p. 175}Thus, it seems to me that the failure that I am today confronting rests on subtler causes than the kind of themes which I suggested, or the extent to which said themes remained nebulous or were clearly delineated. My role in this failure seems due rather to attitudes of fatuity within me, in the way I interacted with mathematics; attitudes which I have examined in the course of this reflection. These attitudes were bound to more or less strongly influence, if not the work itself with a given student, at least the atmosphere surrounding my person. Fatuity, even when expressed in the most ``discreet" way possible, always points towards close-mindedness, towards insensibility to the delicate essence of things and to their inherent beauty - whether these be ``mathematical things", or breathing individuals whom we can welcome and encourage, but also towards whom we can look down from our lofty seat, oblivious to the aura that surrounds us and to the destructive impact it can have on others and on ourselves.

\subsection{A sentiment of injustice and powerlessness}

\nnote{44''} [The appearance of this note does not align with the chronological order of writing]

(May 10) Following my friend's authorization to freely cite excerpts from his work which I may deem useful, I hereby include a more thorough citation\footnote{(*) See second footnote of the preceding note - ``Failure of an instruction - or creation and fatuity", n$^o$44'.}(*), which situates the earlier truncated citation in its proper context:

\begin{adjustwidth}{1cm}{}
``It is true that I underwent a period of isolation between the years 1975 and 1980, except for rare questions to Verdier. But I don't blame your old students for that period, because nobody then really understood the importance of this connection [read: between discrete coefficients and continuous coefficients]. Everything changed in October 1980, when the first highly important application of this connection was found to the theory of semisimple groups, namely the discovery of the Kazhdan-Lusztig multiplicity formula, which used in an essential way the equivalence of categories in question. This equivalence took on the name of ``Riemann-Hilbert correspondence" without further comment - after all, it is so natural! This is when I understood that your old students do not really know what a mathematical \textbf{creation} is, and that perhaps you shared some of the responsibility for this. I still to this day feel a sentiment of injustice and powerlessness. It is true that at the time the problems were already set in stone. The number of applications of this theorem is impressive, in the context of \'etale topology as well as in the transcendantal context, where it still carries the name of Riemann-Hilbert! I am under the impression that my name is unworthy of this result for\marginpar{p. 176} many people, including your old students. But as you can see clearly in the introduction to my work, it is your ``duality" formalism which leads naturally to the result. Like you, I am not worried about the future relevance of this connection between ``discrete constructible coefficients" and crystalline coefficients (or holonomic $\mathcal{D}$-modules). It is clearly applicable to several domains, in the cohomology of spaces as well as in analysis."
\end{adjustwidth}

The above segment from my friend's letter inspired (in addition to the present note) the
later note ``The anonymous worker and the God-given theorem". Based on the letter's
language, I had not realized (what I am now explaining in his stead) that this ``sentiment
of injustice and powerlessness" felt by my friend were a reaction, not only to an attitude
of disdain which systematically \textbf{minimized} his contributions (an attitude that
eventually became familiar in some of my old students), but also to a full-fledged
operation of embezzlement, consisting in outright \textbf{retracting} the authorship of a
key theorem. This situation only became clear to me eight days ago - see regarding this
subject the note ``Unfairness - or a feeling of return" and the subsequent Notes (n$^o$'s
% \nameref{note:75} to \nameref{note:80}),
\todo{ref}
collected under the title "The Colloquium - of Mebkhout's sheaves and Perversity".

\nnote{45} As a result of the changes in my environment and lifestyle, occasions to meet with or otherwise contact my old friends have become rare. The fact remains that many signs of an attitude of "distancing away" have appeared, more or less pronounced depending on the person. However, some people such as Dieudonn\'e, Cartan, or Schwartz - in fact, all of the ``elders" who had warmly welcomed me in my first years, have conveyed nothing of the sort. Other than them, I sometimes feel that there are very few people among my old friends or students in the mathematical community with whom my relationship (whether or not it finds the occasion to be expressed) has not become divided, ``ambivalent", following my departure from what was once a shared milieu, a common world.

\section{13.2. II The orphans}

\subsection{13.2.1. My orphans}

\nnote{46}\label{note:46}
[This note was mentioned in section 50 of chapter \textbf{VIII The solitary journey} of
part \textbf{I (Fatuity and Renewal)}
\todo{ref}
]

\marginpar{p. 177}
I would like to take the time to say a few words concerning the mathematical notions and
ideas, among those which I have brought to life, which seem (by far) to be the farthest
reaching.(\nameref{note:46.1})\footnote{(*)Notes \no \nameref{note:46.1} through 
\nameref{note:46.9} contain more technical
commentaries on the notions reviewed in the present note. In addition, independently from
the particular \textbf{notions} which I have introduced, the reader will also find reflections
regarding what I consider to be the ``core'' of my work (within the collection of work
which I have ``entirely finalized'') in note \no $88$ ``The remains''.}(*)
I will be mostly speaking about five closely linked key-notions, 
which I will briefly review in increasing order of specificity, richness, and depth.

The first idea in question is that of \textbf{derived categories} in homological algebra
(cf. note 48 p. 274), and of their use as a ``catch-all'' formalism called the 
``\textbf{six operations formalism}''
(namely $\tp^L$, $Lf^*$ , $Rf_!$ , $R\under{\Hom}$, $Rf_*$, $Lf^!$)
($46_2$) on the cohomology of the most important kinds of ``spaces'' introduced to this
day in geometry: ``algebraic'' spaces (such as schemes, schematic
multiplicities\todo{fix}, etc \ldots), analytic spaces (i.e. complex analytic as well as
rigid analytic, and assimilated), topological spaces (``tempered spaces'', pending the context
of tempered spaces of all kinds and surely many others, such as that of the category
($\Cat$) of small categories, serving as homotopical models\ldots).
this formalism accommodates both discrete and ``continuous'' coefficients.

The progressive discovery of this duality formalism 
and of its ubiquitousness 
happened through a solitary, persistent, and exacting reflection which took place between
the years 1956 and 1963.
It was during the course of this reflection that the notion of derived category slowly
appeared, and with it an understanding of the role which it played in homological algebra. 

What was still missing from my vision of the cohomological formalism of ``spaces'' was an
understanding of the link which one could conjecture between discrete and continuous
coefficients, beyond the familiar case of local systems 
\marginpar{p. 178}
and their interpretation as
modules with a flat connection, or as modules of crystals. 
This profound link, first formulated in the context of complex analytic spaces was
discovered and established (almost 20 years later) by Zoghman Mebkhout,
in terms of derived categories obtained on the one hand using ``constructible''
coefficients, and on the other hand the notion of ``$\cD$-modules''
of ``complexes of differential operators'' (cf. note $46_3$ p.).
\todo{ref}

For almost $10$ years, in the absence of the encouragement of those among my old students
who were best positioned to offer it, and to support him through their interest and their
experience which they had gained through their work with me. 
Zoghman Mebkhout produced his remarkable work in a near total state of isolation. 
This did not prevent him from discovering and proving two key theorems\footnote{(*) (June
7th) Mebkhout mentioned to me that in addition to these two theorems, I should be
mentioning a third, also expressed in the language of derived categories, namely what he
has called (perhaps a bit improperly) the theorem of biduality for $\cD$-modules, which
was the hardest of the three. For a sketch of the of Mebkhout's ideas and results, and of
their applications, see
Le Dung Trang et Zoghman Mebkhout, Introduction to linear differential Systems, Proc. of
Symposia in Pure Mathematics, vol.40 (1983) part.2, p. 31-63.
}(*)
in the context of a
new crystalline theory which was slowly coming into being 
in the midst of a general indifference.
Both theorems
were expressed in the language of derived categories
(decidedly not a crowd-pleasing topic!): one provided the equivalence of categories
mentioned earlier between ``discrete constructible'' coefficients and crystalline coefficients
(subject to certain conditions of ``holonomicity'' and ``regularity'')
and the other was ``\textbf{the}'' theorem of global crystalline duality for the constant morphism
from a smooth complex analytic space (not necessarily compact, thus involving significant
additional technical difficulties) to a point.
Both are profound theorems,\footnote{(**)(May 30th) The proof of the second theorem
required dealing with the usual technical difficulties of the transcendental context,
involving the recourse to ``\'ev\`etesque'' techniques whence my guess that it ranks among
``difficult'' demonstrations. The proof of the first theorem is ``evident'' and profound,
using the full force of
Hironaka's theorem for the resolution of singularities. As I mention in the penultimate
paragraph of the note ``solidarity'' (\no 85), once the theorem is formulated, 
``anybody'' in the loop would be able to prove it. Compare also with J.H.C. Whitehead's
observation quoted in the
note ``The snobbery of the youth - or the defenders of purity'', (\no 27).
I wrote the latter note as if under the silent dictation of a secret prescience as of yet
not realizing the extent to which the reality was going to surpass my shy and fumbling
suggestions!}(**)
which provide a renewed 
\marginpar{p. 179}
understanding of the cohomology of both analytic as well as
(in characteristic $0$ for now) algebraic spaces, and as such they carry the promise of a
far-reaching renewal of the cohomological theory of these spaces.
They finally earned the author, following two consecutive denials of job application at
the CNRS, a post of research fellow
(equivalent to a post of assistant or master-assistant at a university).

Nobody during these ten years cared to tell Mebkhout, while
he was wrestling
with the significant technical difficulties involved with the transcendental context, 
about the ``formalism of the six variances'', well known by my students\footnote{(*) They
learned it first-hand from the seminars SGA 4 and SGA 5, as well as through the
intervening text ``Residues and Duality'' of R. Hartshorne.}(*), but nowhere to be
found ``written up''.
He finally learned about its existence from me last year 
(in the form of a note, which I was apparently the only one to know about\ldots),
when he kindly and patiently took the time to explain his work to me, even thought I was
out of practice with cohomology\ldots
Neither did anybody think to suggest to him that it may be more ``profitable'' to first
to first focus on the context of schemes in characteristic $0$, where the difficulties
inherent to the transcendental context disappear, while on the other hand the conceptual
questions fundamental to the theory appear just as clearly. 
Nobody thought to mention 
(or even perceived
what I have known ever since I introduced crystals\footnote{(**) (May 30) Something
which I have since forgotten - only to remember it during my second meeting with Mebkhout
last year (see the note ``Meeting from the grave'', \no 78).}(**))
that ``$\cD$-modules'' on smooth (analytic or algebraic) spaces are precisely the same
thing as ``\textbf{modules of crystals}'' (once we put aside matters of ``coherence'' for
either of these notions), and that the latter is a versatile notion which works just as
well for ``spaces'' with arbitrary singularities, as it does for smooth spaces ($46_4$).

In view of the aptitudes (and the rare courage) displayed by Mebkhout it is clear to me
that had he evolved in a sympathetic atmosphere, he would have painlessly and even 
with pleasure established the complete formalism of ``the six variances'' in the context
of crystalline cohomology of schemes in characteristic zero, at a time where all of the
essential ideas for a program of such scope
(including his own, and those of Sato's school and my own)
were already in place, or so it seems to me.
For someone of his caliber, this could have been done in the span of a few years, 
just like the development of the catch-all formalism 
\marginpar{p. 180}
of \'etale cohomology a few years
earlier (1962-1965), given that the guiding framework of the six-operations was already
known (in addition to the two key theorems of base change).
It is true that these years were marked by a flow of enthusiasm and sympathy 
from participants and witnesses, as opposed to a work going upstream relative to the
haughty self-importance of those in charge\ldots

I now come to the second pair of notions, namely that of 
\textbf{schemes} and the tightly related notion of \textbf{topoi}.
The latter is a more intrinsic version of the notion of \textbf{site}, which I introduced in order
to formalize the topological intuition of ``localization''.
(The term ``site'' was introduced later Jean Giraud, who greatly contributed by
providing the notions of site and topos with the necessary flexibility.) 
I was led to introduce the notions of scheme and topos one after another in response to
the glaring needs of algebraic geometry. 
This pair of concepts carried within them the potential for a profound renewal of both algebraic and
arithmetic geometry and of topology, through a \textbf{synthesis} of these ``worlds'',
kept apart for too long, within a common geometric intuition.

The renewal of algebraic and arithmetic geometry through the viewpoint of schemes and the
language of sites (or of ``descent''), carried over the course of twelve years of
foundational work (in addition to the work of my students and other participants of good
faith) has been well-established for twenty years; 
the notion of scheme, and that of \'etale cohomology of schemes (if not that of \'etale
topos and \'etale multiplicity) have finally become customary, and have entered 
the common patrimony. 

On the other hand, this vast synthesis that would also encompass topology
is still biding its time, even though
the essential ideas and principal technical tools
appear to have been in place\footnote{(*) (May 15) The aforementioned ``essential ideas
and principal technical tools" were assembled in the vast fresco of seminaries SGA 4 and
SGA 5 between 1963 and 1965. The strange vicissitudes that affected the writing and
publication of the SGA 5 component of this fresco, which only appeared (unrecognizable,
ravaged) eleven years later (in 1977) illustrated what happened to the enveloping vision
at the hands of a ``certain trend" - or rather, at the hands of certain of my students who
were first to instaure it (see following footnote). These vicissitudes and their meaning
have been progressively revealed over the course of the past four weeks of reflection,
continued in the notes ``The accomplice", ``Clean slate", ``The singular being", ``The
signal". ``The reversal", ``Silence", ``Solidarity", ``Mystification", ``The deceased",
``The massacre", ``The remains", n$^o$s 63'", 67, 67', 68, 68', 84-88.}(*) for twenty years. 
During the\marginpar{p. 181} fifteen years that followed my departure from the world of mathematics, the
fertile unifying idea and powerful tool for discovery that is the notion of topoi has been
maintained by some customary decree\footnote{(*) (May 14) The continuation of my reflection during the six weeks that followed
the writing of these lines (in late March) revealed this ``trend" which was established in
the first place by certain of my students - the very students who were best positioned to
make theirs a certain vision, as well as a range of ideas and technical tools, and who
chose to appropriate certain work instruments, while simultaneously disavowing both the
vision that had given rise to these instruments and the person within whom the vision was
first born.}(*) outside of the range of notions deemed
serious. To this day, few topologists are even aware of the existence of this
potentially considerable enlargement of their science, and of the novel resources which it
offers. 

Within this renewed framework, topological, smooth, and other type of spaces
fit together with schemes (about which they may have heard) 
as well as topological, differential, \textbf{and} scheme-theoretic
(seldom-mentioned) multiplicities
as various incarnation of 
a single class of geometric objects, name \textbf{ringed topoi} 
(46$_5$) which play the role of ``spaces'', and within 
which intuition coming from topology, algebraic geometry, and arithmetic come
into a single geometric vision.
The ``modular'' multiplicities,
which one encounters all over the place
(provided one's eyes are open), provide several striking examples of this structure (46$_6$).
The comprehensive study of ringed topoi constitutes a primary guiding thread for the
purpose of gaining a deeper understanding of the essential properties of geometric
objects (or other objects, if one can find objects which aren't geometric in
nature\ldots).
In this context, modular multiplicities describe the modalities of variation,
degeneration, and generization.
This wealth of ideas remains ignored to this day, 
due to the fact that the notion which allows us to precisely describe it does not fit into 
the range of currently admitted concepts.

Another unexpected aspect of this recused synthesis\footnote{(**)}(**)
is the fact that familiar\marginpar{p. 182} homotopical invariants 
of some of the most common spaces
(46$_7$)
(or rather invariants of their profinite compactifications) 
come equipped with unsuspected arithmetic structures, such as actions of certain profinite
Galois groups\ldots

Nonetheless, for the past fifteen years, it has been customary within ``high society'' to 
look down on those who fancy the word ``topos'', unless in the context of a
joke or if the person happens to be a logician.
(For these people are known to be different, and one must forgive some of their
eccentricities\ldots)
Neither has the yoga of derived categories, serving to express to homology and cohomology
of topological spaces, entered the lingo of topologists for whom K\"unneth's formula
(with coefficients in a ring which is not a field)
continues to be interpreted as a system of two spectral sequences (or at best a pile of
short exact sequences), 
rather than a unique canonical isomorphism within an appropriate category;
just as they continue to ignore the base change theorems (for smooth or proper morphisms
for instance) which (in the neighboring context of \'etale cohomology) constituted the
crucial pivot for the ``kickoff'' of said cohomology (cf note 46$_8$ p. 470).
This comes as no surprise when I realize that the very people who contributed to developing
this yoga have long forgotten about it; and that they will not hesitate to 
strike down anyone who has the misfortune to want to use it!\footnote{(*)(May 13) It appeared upon later reflection that the situation has started to change since the Luminy Colloquium of June 1981: there, some of those who had once ``forgotten" (or rather, buried...) these notions were now parading them around,  continuing nonetheless to strike down the ``poor fellow" without whom this brilliant Colloquium would not have existed. (See notes n$^o$s 75 and 81 for more on this memorable Colloquium.) }(*).

The fifth notion which is close to my heart, perhaps more than any other, is that of
\textbf{``motives"}.
It is distinct from the preceding four ideas in that
\textbf{``the''} correct notion of motive (be it only over a base field, without even
mentioning the case of an arbitrary base scheme) has not been given a 
satisfactory definition to this day, even if we are to accept all ``reasonable'' conjectures
which one may need to this end. Or rather, visibly, 
\marginpar{p. 183}
\textbf{the} ``reasonable conjecture'' to be made in the first place, would be that of the
\textbf{existence} of such a theory, pertaining to certain data and satisfying certain
properties.
It would not be hard (and entirely fascinating!) for somebody in the
know\footnote{(*) (May 13) I eventually understood that the only person (other than myself) who to this day meets the ``reasonably in the know" criterion is Pierre Deligne, who benefitted for four years, at the same time as he was learning from me ``the little I knew about algebraic geometry", from being my day-to-day confidant in the course of my motivic reflections. I did speak about these things to many other colleagues here and there, but it seemed none of them was sufficiently ``tuned in" to assimilate the holistic view which had emerged within me over the course of many years, or to take my indications as a starting point for their own development of a vision or program (as I had myself done beginning with two or three ``strong impressions" effected upon me by some of Serre's ideas). Although I could be mistaken, it seems to me that the people interested in the cohomology of algebraic varieties where not psychologically disposed to ``take motives seriously" for as long as Deligne, who was a figure of authority in cohomology while also being the only one supposed to fully know what these motives were all about, was letting them go unmentioned.}(*),
to explicitly write such a conjecture down.
I was about to do so, shortly before I ``left math''.

In some ways, the situation resembles that of the quest for the ``infinitesimally small''
during the heroic era of differential and integral calculus, with two caveats.
First, we currently possess an experience in the elaboration of sophisticated mathematical
theories, together with an efficient conceptual background, which our predecessors lacked. 
Second, despite the tools which we have at our disposal,
and the twenty years which have elapsed since this visibly essential notion appeared,
nobody has cared (or dared in spite of those who didn't care\ldots) to get their hands dirty,
and to extract the rough features of a theory of motives, the way our ancestors had done
for infinitesimal calculus, without beating around the bush.
It is just as clear today for motives as it once was for the infinitesimally small, that
such beasts exist, and that they manifest themselves in every corner of algebraic
geometry, as long as one is interested in the cohomology of algebraic varieties and 
families of such varieties, and more specifically in the ``arithmetic'' properties of
such objects. 
Even more so perhaps than in the case of the four other notions which I have mentioned, the
idea of motives which is the most specific and richest of all, naturally associates to a
range of intuitions of various kinds, not at all vague and in fact often 
\marginpar{p. 184}
expressible with a perfect precision (provided one is willing, if needed, to admit certain motivic
premises). For me, the most fascinating of these ``motivic intuitions'' was that of
a ``motivic Galois group'', which in a way allows us to ``put a motivic structure'' 
on the profinite Galois groups of fields and schemes of finite type (in the absolute
sense).
(The technical work required to precisely formulate this notion, 
having admitted the ``premises'' giving a temporary foundation for the notion of motive,
was accomplished in the thesis of Neantro Saavedra on ``Tannakian categories''.)

The current consensus surrounding the notion of motive is slightly more nuanced than 
that of its three brothers (or sisters)
of misfortune (derived categories, duality formalism of the so-called
``six-operations'', topoi), 
in the sense that there hasn't been a case of ``swindling''
\footnote{(*) (May 13) As I mentioned in an earlier footnote, derived categories were the subject of an exhumation with great fanfare three years ago (without speaking my name). Topoi and the six operations are still waiting for their turn, as well as motives, except for the small piece thereof which was exhumed two years ago, with a substitute parenthood (see notes n$^o$s 51, 52, 59).}(*). 
Practically speaking, the end-result is nonetheless the same: as long as there hasn't been
a proper ``definition'' of motives and associated ``proofs'', serious people can only
abstain from speaking about them
(naturally with the utmost regret, but such is protocol among serious people\ldots).
Of course we may never arrive to a theory of motives and ``prove'' anything regarding them,
for as long as it is declared that it isn't serious to even speak about them! 

Nonetheless, the few people in the know (and who follow the trends)
know that beginning with the premises, which remain secret, one can prove many things. 
It should be said that as of today, 
in fact since the notion appeared in the wake of the Weil conjectures
(proven by Deligne, which proves a point I guess!), the \textbf{yoga of motives} very much
exists. 
But it has the status of a \textbf{secret science} with very few 
initiates
\footnote{(**) (May 13) I now understand that these ``few initiates" amounted until 1982 to Deligne and him only. It is true that he revealed the aspects of this ``secret science" which are reflected in certain important results included in the yoga, revealing them when he was able to prove then so as to be able to claim credit for them while hiding his source of inspiration, which remained secret. If for the past fifteen years no one has undertaken the development of a wide-ranging theory of motives, it must be because the current times are far from the bold dynamism of the heroic era of infinitesimal calculus!}(**).
Even though it is ``not serious'', it nonetheless\
\marginpar{p. 185}
allows these few initiates to declare in a range of cohomological situations 
``what one should expect".
It thus gives rise to a multitude of intuitions and partial conjectures, which are
sometimes accessible after the fact using tools at hand, 
in light of the understanding provided by the ``yoga". 
Several works of Deligne are inspired by this 
yoga, 
\footnote{(*)(May 13) Having become somewhat familiar with said bibliography, I now realize that Deligne's entire line of work is rooted in this yoga. Furthermore, my bibliographical sampling (together with some cross-checking) give me the impression that in Deligne's entire work, the only reference to this source is to be found in one swift line (referencing me and Serre in the same breath) in ``Th\'eorie de Hodge I" in 1970. (See notes n$^o$s $78_1'$ and $78_2'$.)}(*).
notably (if I am not mistaken) his first published work establishing the degeneracy
of the Leray spectral sequence for a smooth a projective morphism of algebraic
varieties (in characteristic $0$ for the needs of the demonstration).
This result was suggested by ``weight" considerations of an arithmetic nature. 
This is typical of ``Motivic" considerations, which can be formulated in terms of the
``geometry" of motives.
Deligne proved this statement using the theory of Lefschetz-Hodge and (if I remember
correctly) did not say a word concerning the motivation, without which nobody could have
guessed such an improbable result. 

The yoga of motives was in fact born, in the first place, out of this ``yoga of weights"
which I learned from Serre
\footnote{(**) What I learned from Serre (in the early 60s?) was an idea or starting intuition, allowing me to understand that there was something to be understood! This contact provided an initial impulse, triggering a reflection which continued into the following years, first around a ``yoga" of weight and later around a vaster yoga of motives.}(**).
It is him who showed me all the charm of the ``Weil conjectures'' (which have become
a ``theorem of Deligne''). 
He had explained to me how (modulo a resolution of singularities hypothesis in the
characteristic under consideration)
one could, using the yoga of weights, associate to every algebraic variety (not
necessarily smooth or proper) over an arbitrary base field 
the so-called ``virtual Betti-numbers" - something 
which I found extremely striking (\ref{note:46.9}).
I believe it was this idea which started my reflection on weights, a reflection which
continued (in parallel with my project of writing foundational texts)
throughout the following years. 
(This was also the reflection which I resumed in the 70's, through the notion of a ``virtual
motive'' over an arbitrary
\marginpar{p. 186}
base
scheme, which the intention of establishing a ``six operations'' formalism for (at the
very least) virtual motives.) If I discussed this yoga of motives for all these years with
Deligne (figuring as privileged interlocutor) and to whoever else was
interested
\footnote{(*) (April 10) It seems to me that Deligne was the only one who ``listened" - and he took care to keep that privilege to himself. It should also be said that in writing these final lines, I was ``delaying" the chain of events: a partial exhumation of the yoga of motives occurred two years ago, without any allusion made to the role I had played! See notes n$^o$s 50, 51, 59 on this subject, resulting from an unexpected discovery which shed a surprising light (at least in my eyes) on the meaning of the funeral that had been taking place for twelve years. Before then, I had been vaguely aware that some kind of funeral was going on, without taking the time to look closer...}(*),
it wasn't so that he and others would keep this subject under the status of secret
science, reserved to them, and them alone.($\implies$ note \ref{note:47} p. 271)

\subnote{46$_1$}\label{note:46.1}

I will make at most the exception of the ideas and viewpoints introduced alongside the
formulation which I gave to the Riemann-Roch theorem (together with the two proofs which I
discovered), as well as its various variants. If I remember correctly, such variants
appeared in the last expos\'e of the seminar SGA 5 from 1965/66, which was lost along with
several other expos\'es from the same seminar. 
The most interesting such variant in my eyes regards discrete constructible coefficients, 
and I ignore if it has since been made explicit in the literature\footnote{(**) (June 6) I since found it (in a similar form, under the flattering name of ``Deligne-Grothendieck conjecture") in an article of MacPherson which appeared in 1974. See note n$^o$ $87_1$ for details.}(**).
I should observe that there also exists a ``motivic''
variant which boils down to the statement that the ``characteristic classes"
(in the Chow ring of a regular scheme $Y$) 
associated to constructible $\chi$-adic sheaves for different prime numbers $\chi$
(prime to the residual characteristics), in the situation where these sheaves come from a
single ``motive'' (for instance when they are all of the form
$R^if_!\left(\under{Z}_\chi\right)$ for a given $f \colon X\to Y$) are all equal

\subnote{46$_2$}\label{note:46.2}

This formalism can be viewed as a sort of epitome of a cohomological formalism of
``\textbf{global duality}"; in its most ``efficient" form, freed from any superfluous hypotheses (notably of
smoothness for the ``spaces" and morphisms under consideration, or properness of morphisms),
it can be completed by a formalism of \textbf{local duality} in which one distinguishes,
among admissible ``coefficients", the
\marginpar{p. 187}
so-called ``\textbf{dualizing}" objects or complexes (a notion which is stable under the operation
$Lf^!$), i.e. those which give rise to a ``\textbf{biduality theorem}''
(in terms of the operation $R\under{\Hom}$) 
with coefficients satisfying appropriate finiteness conditions (on the degrees, together
with coherence or ``constructibility" conditions on the objects of local cohomology).
When I speak of the ``formalism of the six variances'', I am 
hinting at this complete duality formalism, including both its ``local" and ``global" aspects. 

The first step towards a thorough understanding of duality in cohomology was
the progressive discovery of the formalism of the six variances in an important
special case, namely that of Noetherian schemes and complexes of modules with coherent
cohomology. 
The second step was the discover (in the context of the \'etale cohomology of schemes)
that this formalism also applied to the case of discrete coefficients. 
These two extreme cases were sufficient to 
persuasively suggest the \textbf{ubiquity} of this formalism in all of the geometric situations
giving rise to a Poincar\'e type ``duality'' - a conviction which was later confirmed by
the works of Verdier, Ramis, and Ruget (among others).
It will also surely be confirmed for other types of coefficients when the \textbf{block}
which for fifteen years has been put in place against the development and large scale
use of this formalism will have eroded. 

This ubiquity appears to me to be a \textbf{fact} of considerable importance. It rendered the feeling of a profound unity between Poincar\'e duality and Serre duality unescapable; this unity was eventually demonstrated with the required generality by Mebkhout. This ubiquity positions the ``formalism of the six variances" as a fundamental structure in homological algebra, serving towards an understanding of ``all kinds"\footnote{(*) The interested reader will find a sketch of this formalism in the Appendix to this volume.}(*) of phenomena relating to cohomological duality. That this relatively sophisticated structure has not been made explicit in the past (akin to the absence of a ``good" notion of ``triangulated category", with Verdier's formulation being only temporary and insufficient) does not change this reality; nor does the fact that topologists, and even algebraic geometers who claim to be interested in cohomology, continue as best they can to ignore the very existence of this duality formalism, as well as the language of derived categories upon which it rests.

\subnote{46$_3$}\label{note:46.3}

The framework of $\mathcal{D}$-Modules and of complexes of differential operators was introduced by Sato and first developed by himself and his school, with a perspective quite different\marginpar{p. 188} (or so it seemed to me) from Mebkhout's, which is closer to my own. 

I believe I was the first to formulate the various notions of \textbf{``constructibility"} for ``discrete" coefficients (in complex analytic, real analytic, or piecewise linear contexts) towards the end of the 1950s (and I later reemployed them in the context of \'etale cohomology). I had then asked whether this notion was stable under higher direct images with respect to a proper morphism between real or complex analytic spaces, and I do not know if this stability has been established in the complex analytic case\footnote{(*) (May 25) It was established by J. L. Verdier, see ``The right references", note n$^o$ 82.}(*). In the real analytic case, the notion which I had considered was not the right one, as I was lacking Hironaka's notion of real sub-analytic set, which has the essential preliminary property of stability under direct images. As for operations of a local nature such as \textbf{R\underline{Hom}}, it was clear that the argument which established the stability of constructible coefficients in the context of excellent schemes in characteristic zero 
\todo{clarify: excellent schemes?}
(using Hironaka's resolution of singularities) worked just as in the complex analytic case; ditto for the theorem of biduality (see SGA 5 I). In the piecewise linear context, the natural stability theorems and the biduality theorem are ``easy exercises", which I happily did in the spirit of verifying the ``ubiquity" of the duality formalism, at the beginning of the development of \'etale cohomology (during which period one of the main surprises was the discovery of this very ubiquity).

Coming back to the semi-analytique case, the ``right" context in this direction for establishing stability theorems (for constructible coefficients by the six operations) was visibly that of ``tame spaces" (see Esquisse d'un Programme, par. 5,6).
\todo{cite Esquisse d'un Programme}

\subnote{46$_4$}\label{note:46.4}

Naturally, the framework of $\mathcal{D}$-modules, together with the fact that $\mathcal{D}$ itself is a coherent sheaf of rings, highlights a more hidden notion of ``coherence" for crystals of modules than the one which I am used to working with, and which makes sense for possibly singular (analytic or scheme-theoretic) spaces. It would only be fair to call this notion \textbf{``M-coherence"} (M as in Mebkhout). It then becomes relatively clear, for somebody in the know (in full possession of a healthy mathematical instinct), that the ``right category of coefficients" which generalizes the complexes of ``differential operators" from the smooth settings is nothing but the ``M-coherent" derived category of crystals of modules (where a complex of crystals is called \textbf{M-coherent} if it has M-coherent cohomology objects). This category makes sense independently of the smoothness hypothesis, and it should encompass at once the ordinary theory of ``continuous" coefficients and the theory of ``constructible" discrete coefficients (introducing in the latter case appropriate holonomy and regularity conditions). If my perspective is correct, then the two new conceptual ingredients of this Sato-Mebkhout theory, with respect to the previously known crystalline context, are the notions of M-coherence for crystals of modules as well as the conditions of holonomy and regularity (of a deeper nature) relating to M-coherent complexes of crystals. With these notions in place, one of the first essential tasks would be to develop a formalism of the six variances in the crystalline context, in such a way as to generalize the two special cases (ordinary coherent and discrete) which I have developed over twenty years ago (and which some of my ex-students in cohomology have long ago forgotten in favor of other tasks, surely more important...).

Mebkhout had eventually learned about the existence of a notion of ``crystal" through my writings, and he had felt that his viewpoint should provide an appropriate approach for this notion (at least in characteristic zero) - but this idea fell into deaf ears
\todo{verify: appropriate phrase?}
. Psychologically speaking, it was unconceivable to launch into the vast foundational work ahead in his position, surrounded by a climate of haughty indifference on behalf of the very people who acted as figures of authority in cohomology, and who were as such best positioned to encourage him - or discourage him...

\subnote{46$_5$}\label{note:46.5}

(May 13) The following mostly concerns the notion of ringed topoi associated to a \textbf{commutative local ring}. The idea of describing the structure of a ``variety" in terms of the data of such a sheaf of rings on a topological space was first introduced by H. Cartan, and was taken up again by Serre in his classical paper FAC (Faisceaux alg\'ebriques coh\'erents). This work was the initial impulsion for a reflection which led me to the notion of ``scheme". What was still missing from Cartan's approach reprised by Serre, so as to encompass all types of ``spaces" or ``varieties" which have appeared to this day, was the notion of topos (meaning ``something" on which the notion of ``sheaf of sets" makes sense, and has all of the familiar properties).  

\subnote{46$_6$}\label{note:46.6}

\marginpar{p. 190}Among notable examples of topoi which are not ordinary spaces, and for which there does not seem to exist a satisfactory substitute in terms of commonly ``admitted" notions, are the following: the quotient topos of a topological space under a local equivalence relation (for instance foliations of varieties, in which case the quotient topos is even a ``multiplicity" - i.e. is locally a variety); ``classifying" topoi associated to essentially any kind of mathematical structure (at least those ``expressible in terms of finite projective limits and arbitrary inductive limits"). Given a structure of ``variety" (topological, differentiable, real or complex analytic, or Nash, etc ..., or even smooth scheme-theoretic over a given base field), one obtains a particularly interesting topos, which deserves the name of ``universal variety" (of the space under consideration). Its homotopical invariants (and notably its cohomology, which deserves the name of ``classifying cohomology" for the kind of variety under consideration) should have been studied and determined a long time ago, but as of now no work of the sort seems to be underway...

\subnote{46$_7$}\label{note:46.7}

I am referring to spaces $X$ whose homotopy type can be described ``in a natural way", such as the homotopy type of a complex algebraic variety. The latter may then be defined over a subfield $\mathbb{K}$ of the field of complex numbers, such that $\mathbb{K}$ is an extension of finite type of the prime field $\mathbb{Q}$. The profinite Galois group $\text{Gal}(\bar{\mathbb{K}}/\mathbb{K})$ then acts naturally on the profinite homotopical invariants of $X$. Often (e.g. in the case where $X$ is a homotopy sphere of odd dimension), one can take $\mathbb{K}$ to equal the prime field $\mathbb{Q}$.

\subnote{46$_8$}\label{note:46.8}

(May 13) At the time when I took my first steps in algebraic geometry through Serre's FAC article (which was about to ``launch" me towards schemes), the very notion of base change was practically unknown in algebraic geometry, except for the particular case of changing the base field. With the introduction of the language of schemes, this operation has surely become the most commonly used in algebraic geometry, where it is now present at all times. The fact that this operation remains practically unknown in topology, with the exception of a few very particular cases, appears to me to be a typical sign (among others) of the isolation of topology from ideas and techniques coming from algebraic geometry, as well as of a tenacious legacy of inadequate foundations from ``geometric" topology. 

\subnote{46$_9$}\label{note:46.9}

\marginpar{p. 191}(June 5) Serre's idea was that one should be able to associate to any scheme $X$ of finite type over a field $k$ integers
$$ h^i(X) (i \in \mathbb{N})$$
which he called the ``virtual Betti numbers" of $X$, in such a way that the following properties hold: \\
\textbf{a)} given a closed subscheme $Y$ with complementary open $U$,
$$ h^i(X) = h^i(Y) + h^i(U). $$
\todo{verify that it is indeed i and not ``j" as in the manuscript}
\textbf{b)} for smooth projective $X$,
$$ h^i(X) = i^{th} \text{ Betti number of } X $$
(where the RHS is defined for instance via $\chi$-adic cohomology, for $\chi$ prime to the characteristic of the base field $k$).

If one takes for given resolution of singularities for algebraic schemes over $\bar{k}$, then it is immediate that the $h^i(X)$'s are uniquely determined by the above properties. The \textbf{existence} of such a function $X \mapsto (h^i(X))_{i \in \mathbb{N}}$ for fixed $k$ can be essentially reduced to the case where $k$ is a finite field using the formalism of cohomology with proper support. Upon passing to the ``Grothendieck group" of finite rank vector bundles over $\mathbb{Q}_\chi$, over which $\text{Gal}(\bar{k}/k)$ acts continuously, and taking the $\chi$-adic Euler-Poincar\'e characteristic (with proper support) $\text{EP}(X, \mathbb{Q}_\chi)$ of $X$ in this group, $h^i(X)$ denotes the virtual rank of the ``constituent of weight $i$" of $\text{EP}(X, \mathbb{Q}_\chi)$, where the notion of weight is deduced from the Weil conjectures together with a weak form of resolution of singularities. Even without resolution of singularities, Serre's idea can be realized using the strong form of the Weil conjectures (established by Deligne in ``Conjectures de Weil II").

I have pursued heuristic reflections in this direction, which led me to a formalism of six operations for ``virtual relative schemes", where the base field $k$ is replaced by a more or less arbitrary base scheme $S$, as well as to various notions of ``characteristic classes" for such virtual schemes (of finite presentation) over $S$. I was thus led (coming back for simplicity to the situation over a base field) to considering finer integral numerical invariants than Serre's, denoted $h^{p,q}(X)$, satisfying analogous properties to a) and b) above, and recovering Serre's virtual Betti numbers via the usual formula:
$$ h^i(X) = \sum_{p+q = i} h^{p,q}(X). $$

\subsection{Refusal of an inheritance - or price of a contradiction}

\nnote{47}\label{note:47} [This note is the direct continuation of note 46 from section 13.1.1]

\marginpar{p. 192}One may note that four out of the five notions which I have just reviewed (corresponding to the ones which are deemed ``not serious") concern cohomology, and most prominently, the \textbf{cohomology of schemes and algebraic varieties}. In any event, all for of these notions came to me through the needs felt for a cohomology theory of algebraic varieties, first for continuous coefficients, and later for discrete coefficients. That is to say, one of the principal motivations and constant Leitmotiv in my work during the fifteen year period 1955-1970 has been the cohomology of algebraic varieties.

Remarkably, this is also the theme which Deligne regards today as being his principal source of inspiration, based on what is said on this subject in the IHES booklet from last year\footnote{}(*) I became aware of this fact with some surprise. Indeed, I was still ``at the scene" and in touch with all the trends when Deligne (following his beautiful work on Ramanujan's conjecture) developed his remarkable extension of Hodge theory. This was mostly, for him as for myself, a first step taken towards a construction of the notion of motives over the complex numbers - to begin with! During the first years that followed my ``turning point" in 1970, I of course received echoes regarding Deligne's proof of the Weil conjectures (which also implied the Ramanujan conjecture as a result), and, in the same stride, of the ``hard Lefschetz theorem"\footnote{Translator's note: In French, the theorem is referred to as the ``th\'eor\`eme de Lefschetz vache".}
\addtocounter{footnote}{-1} in positive characteristic. I expected no less from him! I was even sure that he must have proven at the same time the \textbf{``standard conjectures"} which I had formulated towards the end of the 1960s as a first step towards obtaining (at the very least) the notion of ``semisimple" motive over a field, and towards translating certain of the expected properties of such motives in terms of $\chi$-adic cohomological properties of groups of algebraic cycles. Deligne later told me that his proof of the Weil conjectures would definitely not imply the standard conjectures (which are stronger), and that he actually had no idea as to how to approach them. This must have happened around ten years ago. Since then, I have heard nothing in the way of decisive progress towards the understanding of\marginpar{p. 193} ``motivic" (or ``arithmetic") aspects of the cohomology of algebraic varieties. Knowing Deligne's abilities, I had tacitly concluded that his principal interest must have turned towards other subjects - whence my surprise upon reading that the reality was all to the contrary. 

What seems certain to me, it is that for the past twenty years it has no longer been possible to work towards a large scale renewal of our understanding of the cohomology of algebraic varieties without appearing in some sense as a ``continuator of Grothendieck". Zoghman Mebkhout learned the lesson the hard way, and (to some extent) the same applied to Carlos Conto-Carr\`ere, who quickly understood that it was in his best interest to change topics (see $47_1$). Among the very first things that need doing is the development of the famous ``formalism of the six variances" in the context of various coefficients, as close as possible to the context of motives (which currently play the role of a sort of ideal ``horizon"): crystalline coefficients in characteristic zero (in the lineage of the Sato school and of Mebkhout, Grothendieck style) or $p$ (mostly studied by Berthelot, Katz, Messing, as well as a group of clearly enthusiastic younger researchers), ``stratified promodules" \'a la Deligne (which appear as a dual or ``pro" variant of the ``ind"-notion of coherent $\mathcal{D}$-modules or $\mathcal{D}$-coherent crystal), and finally ``Hodge-Deligne" coefficients (which appear to be as good as motives, except for the fact that their definition is transcendental and only applies to base schemes of finite type over the complex numbers)... At the other extreme lies the task of extracting the very notion of motives from the mist surrounding it (and for good reason...), as well as, if possible, to attack questions as precise as the ``standard conjectures". (For the latter, I had considered, among other things, developing a theory of ``intermediary Jacobians" for smooth projective varieties over a field, as a way of maybe obtaining the positivity formula for traces, which was one of the essential ingredients for the standard conjectures.)

The above were burning tasks and questions up until the moment when I ``left math" - burning and juicy topics, none of which ever appeared to me as reaching a ``wall", or coming to a halt\footnote{}(*). They represented a source of inspiration as well as an inexhaustible substance\marginpar{p. 194}; I needed only pull wherever something was sticking out (and things were ``sticking out" all over!) for progress to come, both expected and unexpected. With my limited means, but without being scattered in my work, I know full well what can be achieved once I put my mind to it, in a single day, a year, or ten years. I also know about Deligne's means, having seen him at work at a time when he wasn't scattered in his work, and I am aware of what he can accomplish in a day, a week, or a month, provided he is willing to put his mind to it. But nobody, not even Deligne, can produce fertile work in the long run, work of profound renewal, while towering over the very objects which one is supposed to probe, as well as the language and the assortment of tools which have been developed to this end by one's predecessor (developed moreover with his assistance, among many other people who offered their contribution...) (59).

I am also thinking about the ``Deligne-Mumford" compactifications of the moduli multiplicities $M_{g, \nu}$ (over Spec $\mathbb{Z}$) for smooth connected algebraic curves of genus $g$ with $\nu$ marked points. These were introduced\footnote{(*) In Pub. Math. 36, 1969, pp. 75-110. For comments, see note n$^o 63_1$.}(*) in order to prove the connectedness of the moduli spaces $M_{g, \nu}$ in every characteristic via a specialization argument starting from characteristic zero. The spaces $M_{g, \nu}$ are in my eyes (together with the group $\text{SL}(2)$)  the most beautiful and most fascinating objects that I have encountered in mathematics ($47_2$).
\todo{ref}
Their very existence, with such perfect properties, appear to me as a sort of miracle (which is furthermore perfectly understood) whose scope is immeasurably larger than the connectedness property which was to be established. In my opinion, they quintessentially contain within them that which is most essential in algebraic geometry, namely the totality (more or less) of all algebraic curves (over all possible base fields), these being the ultimate building blocks for all of the other algebraic varieties. Yet, the kind of objects at hand, namely ``smooth and proper multiplicities over Spec $\mathbb{Z}$", still falls outside of the ``accepted" categories, namely those which the mathematical community is \textbf{disposed} (for reasons which are exempt from scrutiny) to kindly ``admit". The commonplace etiquette is for people to speak of these things through allusions at most, while taking the sorry air as they engage in yet more ``general nonsense", and having taken the precaution to say ``stack" or ``champ", so as not to utter the taboo word of ``topos" or ``multiplicity". This is doubtlessly the reason why these unique jewels have not been studied and used (as far as I am aware) since they were first introduced more than ten years ago, other than by myself in seminar notes\marginpar{p. 195} which have remained unpublished. Instead, people persist in working either with ``coarse" moduli spaces or with finite coverings of moduli multiplicities which fortuitously happen to be real schemes - even though both of the above are but relatively pale and awkward shadows of the perfect jewels from which they are issued, and which themselves remain practically banned...

Deligne's work on the four topics that are the Ramanujan conjecture, mixed Hodge structures, the compactification of moduli multiplicities (in collaboration with Mumford), and the Weil conjectures, each constitute a renewal in our understanding of algebraic varieties, and with it, the establishment of a new starting point. These fundamental contributions occurred in the space of only a few years (1968-73). Yet for the past ten years, these great milestones did not serve as launching pads for a new expedition into the unknown, nor as means towards a renewal of vaster scope. They instead ended in a situation of morose stagnation ($47_3$). This is surely not due to the fact that the ``means" which were available ten years ago, possessed by various people, have disappeared as if by magic; nor is it because the beauty of the things within our reach has suddenly vanished. But it is not enough for the world to be beautiful - one still has to rejoice in this fact... 

\subnote{47$_1$}\label{note:47.1} This note concerns Contou-Carr\`ere's promising start, five or six years ago, on a theory of relative local Jacobians, as well as their link with global Jacobians (so called ``generalized Jacobians") for smooth and not necessarily proper (schemes in curves?) over an arbitrary base, and with Cartier's theory of commutative formal groups and (typical curves?).
\todo{clarify this}
Except for an encouraging reaction by Cartier, the reception of Contou-Carr\`ere's first note, by those who were best positioned to be able to appreciate it, was so cold that the author refrained from ever publishing the second note which he had in reserve, and to thereafter change topics (without quite safeguarding himself from later misadventures)\footnote{(*) (June 8) See the sub-note $95_1$ to the note ``Coffin 3 - or the slightly-too-relative Jacobians", n$^o 95$.}(*).
\todo{ref}
I had suggested the theme of local and global Jacobians to him as a first step towards a program that traces back to the late 1950s, aiming notably for a theory of an ``adelic" dualizing complex in arbitrary dimensions, constructed via local Jacobians (for local\marginpar{p. 196} rings of arbitrary dimensions), in analogy with the residual complex of a noetherian scheme (constructed via the dualizing modules of all of its local rings). This part of my cohomological duality program (along with some others) was put on the back burner during the 1960s, as a result of the surge of other tasks which then seemed to be more urgent.

\subnote{47$_2$}\label{note:47.2} To be more precise, it is the ``Teichm\"uller tower" into which all of the aforementioned multiplicities fit, and the discrete or profinite paradigm of this tower in terms of fundamental groupoids, which constitutes the richest and most fascinating object that I have encountered in mathematics. The group $S_\chi(2, \mathbb{Z})$, together with the ``arithmetic" structure of the profinite compactification of $S_\chi(2, \mathbb{Z})$ (consisting of the Gal$(\bar{\mathbb{Q}}/\mathbb{Q})$ action over the latter), can be considered as the main building block leading to the ``profinite version" of said tower.
\todo{I think this should actually be SL(2,Z)?}
For more on this subject, see the indications in ``Sketch of a program"\footnote{``Esquisse d'un Programme" in the French text.}
\addtocounter{footnote}{-1} (awaiting the one or more volumes of Mathematical Reflections\footnote{``R\'eflexions Math\'ematiques" in the French text.}
\addtocounter{footnote}{-1} which will be devoted to this theme).

\subnote{47$_3$}\label{note:47.3} The observation of a ``morose stagnation" is not coming from a carefully weighed opinion from someone in the know about the main events during the past ten years touching on the cohomology of schemes and algebraic varieties. Rather, it is the mere overall \textbf{impression} of an ``outsider", which I have gleaned from conversations and correspondences with Illusie, Verdier, and Mebkhout in 1982 and 1983, among other things. There are surely several ways to further qualify this impression. For instance, Deligne's work in ``Conjectures de Weil II", which appeared in 1980, represents substantial new progress, if not a surprise at the level of the main result. There also seems to have been progress made on crystalline cohomology in characteristic $p > 0$, as well as a ``rush" surrounding intersection cohomology, which led some to (reluctantly) return to the language of derived categories, or even to recall some long renounced affiliations.

\section{III Fashion - or the Life of Illustrious Men}

\subsection{Instinct and fashion - or the law of the strongest}

\nnote{48}\label{note:48} [This note is referenced by note 46 from p. 265]
\todo{ref}

\marginpar{p. 197}As is well known, the theory of derived categories is due to J. L. Verdier. Before he undertook the foundational work which I suggested to him, I had confined myself to working with derived categories in a heuristic way, using a provisory definition (which later turned out to be the right one) and an equally provisory intuition about the essential internal structure (an intuition which turned out to be technically wrong in the expected context, in that the ``mapping cone" does \textbf{not} depend functorially on the morphism in the derived category within which it is defined, and that it is only defined up to non-unique isomorphism). The theory of duality of coherent sheaves (i.e. the formalism of the ``six variances" in the coherent context) which I had developed towards the end of the 1950s\footnote{}(*) only made full sense once foundational work on the notion of derived category had been laid down, something which Verdier did at a later time. 

Verdier's thesis (submitted only in 1967), consisting of about twenty pages, appears to me to be the best introduction to the language of derived categories written to this day, in that it situates the language in the context of its key applications (many of which are due to Verdier himself). This text served only as an introduction to a work in progress, which was completed later. I am proud of being, if not the only, at least one of very few people who can testify to having held this work in their hands, a text which warranted the title of doctor es Sciences to its author by means of an introduction alone! This work is (or was - I do not know if a copy can still be found somewhere...) the only text, to this day, which develops systematic foundations for homological algebra from the viewpoint of derived categories.

\marginpar{p. 198}Perhaps I am the only one to regret that neither the introductory text not the foundations per se have been published\footnote{}(*), so that the essential technical prerequisites to using the language of derived categories find themselves scattered across three different sources in the literature\footnote{}(**). This absence of a systematic reference of a caliber comparable to Cartan-Eilenberg's classical book appears to me as both \textbf{a cause and a sign}, typical of the disaffection which took hold of the formalism of derived categories following my departure from the mathematical world in 1970.

Admittedly, it was already clear in 1968 (in light of the needs of a cohomological theory of traces, developed in SGA 5) that the notion of derived category in its primitive form, as well as the corresponding notion of triangulated category, were insufficient to satisfy certain needs, and that foundational work remained to be done. A useful but still modest step in this direction was taken by Illusie (mostly for the needs of the theory of traces), through the introduction he gave in his thesis to ``filtered derived categories". It seems that my 1970 departure signaled a sudden and definitive stop to all reflections touching to the foundations of homological algebra, along with the intimately linked reflections touching to a theory of motives ($48_1$).
\todo{ref}
Yet, regarding the first of the above themes, all of the essential ideas needed for a large scale foundational work seemed to have been established during the years leading to my departure ($48_2$).
\todo{ref}
(Including the key idea of ``derivator", or ``machine producing derived categories", which seems to be the richest object in common to the triangulated categories encountered to this day; this idea was eventually developed to some extent in the non-additive context, nearly twenty years later, in a chapter of volume 2 of Pursuing Stacks\footnote{``Poursuite des Champs" in the French text.}
\addtocounter{footnote}{-1}.) 
Moreover, the foundational work at hand was in large part already done by Verdier, Hartshorne, Deligne, and Illusie, and their work could be used as is for a synthesis bringing the acquired ideas to the vaster framework of derivators.

\marginpar{p. 199}It is true that the disaffection for the very notion of derived category during the past fifteen years\footnote{}(*), which for some was connected to a disavowal of a time past, is in line with a certain fashion which suggests that we look down upon any reflection touching to foundations, however urgent it may be\footnote{}(**). On the other hand, it is also clear to me that the development of \'etale cohomology, which ``everybody" uses nowadays without thinking twice (if only implicitly via the now proven Weil conjectures...) would not have been possible without the conceptual background consisting of derived categories, the six operation formalism, and the language of sites and topoi (developed in the first place to this very end), without including SGA 1 and SGA 2. And it is just as clear that the stagnation which is to be found today in the cohomological theory of algebraic varieties would not have appeared, let alone settled, if some of my ex-students had known to follow their sane instinct as mathematicians during these past years, instead of following the fashion which they were among the first to establish, and which through their support has gained force of law. 

\subnote{48$_1$}\label{note:48.1} The same thing can be said (modulo some reserves) about the entirety of my foundational program in algebraic geometry, of which only a small part has been realized: it stopped immediately following my departure. This halt stroke me the most concerning the duality program, which I considered to be particularly juicy. Zoghman Mebkhout's work, which he carried in counter-current, nonetheless fall in line with this program (and renew it through the incorporation of unexpected ideas). The same applies to Carlos Contou-Carr\`ere's work in 1976 (about which I spoke in note $47_1$, p. 273) - a work which he carefully paused sine die. There was also work done on duality in the context of the fppf cohomology of surfaces (by Milne). This is all the work of which I have been made aware.

It is true that I never considered writing the sketch of the long term program whose contours had appeared to me during the years 1955 through 1970, just as I did for the past twelve years in Sketch of a Program. The reason, I think, is simply that there was never any particular occasion (such as presently my candidacy for entrance into the CNRS) to\marginpar{p. 200} motivate such a work of exposition. One may find some indications regarding certain theories (notably duality theories) relevant to my pre-1970 agenda, and which still await ground work in order to enter into the common patrimony, in letters to Larry Breen (from 1975), which are reproduced in appendices to Chapter I of the History of Models (Mathematical Reflections 2)
\footnote{``Histoire de Mod\`eles" in the French text.}
\addtocounter{footnote}{-1}.

\subnote{48$_2$}\label{note:48.2} The same is true for the theory of motives as well, except for the fact that the latter will most likely remain conjectural for quite some time.

\subsection{The anonymous worker and the God-given theorem"}

\nnote{48'}\label{note:48'} [This note is referenced by note 46 p. 178]

Although it is customary to name the key theorems of a theory after the people who have done the work to extract and established them, it seems that Zoghman Mebkhout's name has been deemed unfit to be attached to his fundamental theorem, the culmination of four years of obstinate and solitary work (1975-79), in counter-current to the fashion of the day and despite his elders' disdain. The same elders, when there came the day that the scope of the theorem could no longer be ignored, took fancy in calling it ``Riemann-Hilbert theorem", and I trusted (even though neither Riemann nor Hilbert would have demanded as much...) that they had excellent reasons to do so. After all (once a need has been felt - the need for an understanding of the precise relationship between general discrete coefficients and continuous coefficients, which appeared in spite of a widespread indifference; after this impression was refined and made precise through a delicate and patient process, so that the right statement was finally extracted after successive stages; after it was stated clearly and proven; and, at last, after this theorem, the fruit of solitude, had proven its worth in places where it was least expected - only then) this theorem appeared to be so evident (not to say ``trivial", for those who ``would have been in a position to prove it"...) that there really was no need to burden ourselves with the name of some vague anonymous worker!

Encouraged by this precedent, I suggest that we henceforth begin calling ``theorem of Adam and Eve" any theorem that is truly natural and fundamental to the development of a theory, or even to trace things back even further and to attribute recognition where it pertains, by simply calling it a \textbf{``God-given theorem"}\footnote{(*) I have never had in my life as a mathematician the pleasure of inspiring, or even encouraging, a student in producing a thesis containing such a ``God-given theorem" - at least not of a depth or scope comparable to the one in question.}(*).

\marginpar{p. 201}As far as I know, Deligne was the only person other than myself who before Mebkhout sensed the interest in understanding the relationship between discrete coefficients and continuous coefficients in a larger context than that of stratified modules, so as to be able to interpret in ``continuous" terms arbitrary ``constructible" coefficients. The first attempt in this direction formed the main theme of a (yet unpublished) seminar of Deligne at the IHES in 1968 or 1969, during which he introduced the viewpoint of ``stratified promodules" and produced a comparison theorem (over the complex numbers) between transcendental discrete cohomology and the associated de Rham-type cohomology, the latter of which makes sense for schemes of finite type over any base field of characteristic zero. (Apparently, he was not aware at that time of the remarkable result of his distant predecessors Riemann and Hilbert...) Even more so than Verdier\footnote{(*) It seems that Verdier, as official doctoral advisor of Zoghman Mebkhout (and who in this role has even ``granted him a few discussions"), was the principal suspect (except for Mebkhout himself) in the cover-up which took place surrounding the authorship of this fundamental theorem, as well as in the attribution of the credit that his ``student" deserved for the consequent renewal that took place in the cohomological theory of algebraic varieties - through the viewpoint on $\mathcal{D}$-modules developed by Mebkhout. I nonetheless am not under the impression that the situation moved him anymore than it moved Deligne.}(*) or Berthelot\footnote{(**) (May 25) In writing these lines, I have refrained (with some hesitation) from including the name of my friend Luc Illusie in this list of students who were ``best positioned" to provide Zoghman Mebkhout with the encouragements which were naturally in order. I wasn't mindful at the time of a certain uneasiness within me, which would have indicated to me that I was favoring someone for whom I had affection, in appearing to discharge him from a responsibility which falls on him just as it falls on my other ``cohomologist students".}(**), Deligne was therefore particularly well positioned to appreciate all the interest underlying the direction which Mebkhout's research took in 1975, as well as the value of Mebkhout's subsequent results, notably the ``God-given theorem" which provides a more delicate and deeper apprehension of discrete coefficients in terms of continuous coefficients than that which he had achieved himself. None of this takes away from the fact that Mebkhout had to pursue his work in trying moral isolation, and that the credit is his (all the more so, I would say) for his pioneering work which remains hidden to this day, five years after the fact\footnote{(***) (May 25) In fact, this cover-up is first and foremost the act of Deligne and Verdier themselves. For more on this subject, see the note ``The Inequity - or a feeling of return", n$^o 75$.
%todo{ref}
}(***).

\subsection{Canned weights and twelve years of secrecy}

\nnote{49}\label{note:49} [This note is referenced in note 46 p. 185]

\marginpar{p. 202}After cross-checking (in Publications Math\'ematiques 35, 1968), I observe that towards the end of the article ``Th'eor\`eme de Lefschetz et crti\`eres de d\'eg\'enerescence de suites spectrales", there is an allusion made in three lines concerning ``weight considerations" which had led me to conjecture (in a slightly less general form) the principal result of the article. I doubt that this sibylline allusion could have been of use to anybody, nor understood at the time by anyone other than Serre or myself, to either of whom it would have come as no news\footnote{(*) (April 29) For a more careful study of this paper, instructive in more than one way, see the note ``The eviction" (n$^o 63$).}(*).
\todo{ref}

On this topic, I should also mention that I was well aware (and hence so was Deligne) of a very precise ``yoga of weights", including the behavior of weights with respect to operations such as $R^if_*$ and $R^if_!$, starting from the end of the 1960s, in the wake of the Weil conjectures. Part of this yoga was finally established (in the context of $l$-adic coefficients, as an intermediary to the more natural context of motives) in Deligne's paper ``Conjectures de Weil II" (Publications Math\'ematiques 1980). If I am not mistaken, for the twelve years that have elapsed between these two periods\footnote{(**) (April 19) I realize upon looking at a list of Deligne's papers which I have just received and read with interest, that the notion of ``weights" already appears in 1974 in a communication of Deligne at the Vancouver Congress - this thus brings the twelve years of ``secrecy surrounding weights" down to six. This secret nonetheless appears to me to be inseparable from the similar secret surrounding the notion of motives (during the period 1970-1982). The meaning of this secrete has just come to a new light during today's reflection, while writing the long double-note that follows n$^o 51-52$).}(**), there was not a single trace in the literature of an exposition, however partial or succinct, of the yoga of weights (still completely conjectural), which as such remained within the exclusive reach of a few (two or three?) initiates\footnote{(***) (May 25) In light of all of the pieces of information that have appeared during the reflection, it seems that these ``two or three initiates" may actually boil down to Deligne and him only, as he seems to have carefully kept exclusive privilege of access to this yoga he learned from me until 1974 (see preceding footnote), at which point the time was ripe to present these ideas as his own, without making reference to me or today Serre (see notes n$^o 78_1', 78_2'$).

(April 18 1985) Since the time when these lines were written, I also became aware of Deligne's paper ``Th\'eorie de Hodge I", published in the Congr\`s Int. Math. de Nice (1970) (Actes, t.1, pp. 425-430). Unlike what I had first believed based on the partial information in my possession, this paper exposes as earl as in 1970 a substantial part of the yoga of weights. Regarding the origin of these ideas, Deligne only makes a sibylline and strictly formal reference to a paper of Serre (which doesn't address the question), and to ``Grothendieck's conjectural theory of motives". (Compare with notes n$^o 78_1', 78_2'$.) The crucial question of the behavior of the notion of weights under operations such as $R^if_!$ and $R^if_*$ is not even mentioned, and it won't be mentioned until the aforementioned paper ``La Conjecture de Weil II" from 1980, in which my name is not mentioned in relation to the main theorem of the article; neither is Serre's name or mine mentioned in the communication ``Poids dans la cohomologie des vari\'et\'es alg\'ebriques" (``Weights in the cohomology of algebraic varieties") mentioned in the preceding footnote (published exactly one year ago).}(***). However, this yoga constitutes an essential first key in the endeavor of understanding the ``arithmetic" properties of algebraic varieties, thus acting at once as a \textbf{means} of orienting oneself in\marginpar{p. 203} a given situation and for making predictions with an accuracy that has never failed, as well as on of the most urgent and fascinating \textbf{tasks} which lie ahead in the cohomological theory of algebraic varieties. The fact that this yoga has remained nearly ignored until the time when it was finally established (at least in regards to some of its important aspects), appears to me to be a particularly striking example of the \textbf{information blocking} practiced by some of the very people whose functions and privileged situation call for them to oversee the wide broadcast of said information\footnote{(*) See on this topic sections 32 and 33, ``The ethics of a mathematician" and ``The note - or the new ethics (1)", as well as the two related notes ``Deontological consensus and control of information" and ``The snobbery of the youth - or the defenders of purity", n$^o$s 25, 27.}(*).
\todo{ref}

\subsection{There is no stopping progress!}

\nnote{50}\label{note:50} [This note is referenced in section 50 of chapter \textbf{VIII The solitary adventure} of part \textbf{(I) Fatuity and Renewal} p.]
\todo{add page}

My first experience in this sense was the unexpected outcome of my unsuccessful attempts to have Yves Ladegaillerie's thesis on isotropy theorems for surfaces published - even though his work was as good as that of any of the eleven other works submitted in the context of a doctorat d'\'etat (``pre-1970", admittedly!) under my ``directorship". If I recall correctly, these efforts continued over the course of at least a year, and they involved as protagonists several of my old friends (as well as one of my ex-students, naturally)\footnote{(**) See on this topic the note ``Coffin 2 - or the chain-sawn cuts", n$^o 94$.}(**). 
\todo{ref}
The principal events still appear to me to this day as a handful of vaudeville acts!

This episode also constituted my first encounter with a new spirit and with new customs (which had become customary among my friends of yesteryear), about which I have already alluded here and there over the course of my reflection. It was during that year (namely, 1976) that I learned for the first time, but not the last, that the act of properly demonstrating things which everybody uses and that have been taken for granted by one's predecessors (in this case, the non-existence of wild phenomena in the topology of surfaces)\footnote{(***) See also on this topic the episode ``The note - or the new ethics" (section 33). This famous ``note" had made the very mistake of explicitly presenting notions and statements which had hitherto remained vague, even though I had implicitly used them to establish results which bear my name and which everybody had been using unabashedly for almost twenty-five years (something which the two illustrious colleagues in question knew full well).

(June 8) For more details, see the note ``Coffin 4 - or the unceremonious topoi" (n$^o$96). The ``results which bear my name" are results on the (engendrement?) and finite presentation of certain global and local profinite fundamental groups, ``established" among other things in SGA I via descent techniques that remained heuristic in the absence of a careful theoretical justification, until the latter was produced in the (apparently ``unpublishable") work of Olivier Leroy on Van Kampen-type theorems for fundamental groups of topoi.}(***) 
\todo{ref}
is considered (at least when done by a firstcomer...) a mark of a lack of seriousness. The same goes for the proof of a result which encompasses\marginpar{p. 204} as special cases or corollaries several known deep theorems (which evidently indicates that the new result can only be a special case or an easy consequence of the already established results). Ditto for taking the time to carefully formulate the natural hypotheses (sign of a regrettable rambling bout) relevant to stating a given result or to describing one situation in terms of another, rather than limiting oneself to discussing a special case which suits the high-flying individual voicing their opinion. (Just last year, I saw Contou-Carr\`ere be criticized for not limiting himself in his thesis to work over a base field rather than over a general scheme - while nonetheless conceding the extenuating circumstance that is having me as an advisor, so that he must surely have only acted in compliance to his current boss' request. This occurred even though the person who voiced such criticism was sufficiently in the know to realize that even in limiting oneself to working over the complex numbers, the needs of the proof inevitably called for introducing more general base schemes...)

This fashion of the day goes so far as to hold in contempt not only careful demonstrations (if not demonstrations altogether), but sometimes even proper statements and definitions. Given the cost of paper and with the reader's stamina nearing exhaustion, it will soon be time to part ways with such costly luxury! Extrapolating upon the current tendencies, we should be able to project an era when a publication will no longer be required to explicitly state definitions and statements; instead, we shall henceforth content ourselves with naming things using code words, leaving to the tireless and brilliant reader the task of filling the blanks in accordance to their own wits. The referee's task will become all the easier, as it will suffice for them to consult the ``Who is Who" directory to determine whether the author is recognized as credible (so that in any case nobody could possibly refute the blanks and dotted lines composing their brilliant article), or on the contrary an unspeakable nobody who will be (as is already the case today, and in fact has been the case for quite some time) rejected from the get-go...

%\end{document}
