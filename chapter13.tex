% \begin{comment}
\documentclass{book}
\usepackage{master}
\newcommand{\rec}{$\text{R\'ecoltes et Semailles}$}
\newcommand{\no}{n$^\circ$}
\hfuzz = 100pt
\begin{document}
% \end{comment}

\setcounter{chapter}{12}
\chapter{Heritage and heir}

\section{Posthumous student}

\subsection{Failure of an instruction (II) - or creation and fatuity}

\textbf{Note} 44' [This note was mentioned in section 50 of
\textbf{VIII The solitary journey} of part \textbf{(I)
Fatuity and renewal} p. 227]

\marginpar{p. 173}
This passage ``clicked'' for the friend who read the previous section ``the weight of a
past''\footnote{(*) (May 10th) This friend is none other than Zoghman Mebkhout, who
authorized me to reveal his identity, after I thought I should keep it secret upon first
writing this letter (on April 2nd 1984).}(*) He wrote: ``for many of your old students,
the aspect, as you put it, of an invasive and borderline destructive ``boss'' remains
strong. Whence the impression you hold.'' (Namely, I presume, the 
``impression'' which is expressed in certain passages of this section as well as in the
notes
\no 46, 47, 50 which complete it.)
Earlier, he writes: ``first of all I think that you did well to leave mathematics for an
instant [!]. Because there was a kind of incomprehension between you and your students,
except of course for Deligne. They were left a bit dumbfounded\ldots''. 

This is the first time that I hears about the impression I made in my role as ``boss''
pre 1970, beyond customary compliments!
Even earlier in the same letter: ``\ldots I have come to realize that your old students 
[namely: those from ``before 1970''] do not really know what a mathematical
\textbf{creation} is, perhaps in part because of you\ldots it must be said that in their
time, the problems were clear-cut\ldots'' \footnote{(**) (May 10th) The preceding citation
was heavily modified, in order to respect the anonymity of my correspondent. See the
following note for a complete citation of the relevant passage, as well as for a
commentary on its real meaning, which I had missed at first due to a lack of further
contextual information.}(**).

My correspondent surely meant that \textbf{I} was the one who formulated the ``problems''
and, with them, the notions that needed to be developed instead of 
leaving both tasks to my students; and that in so doing I may have 
prevented them from becoming acquainted with what becomes the essential part of a work of
mathematical creation. This also aligns with an 
\marginpar{p. 174}
impression which I formed after talking
to two of my students from \textbf{after} 1970, about which I wrote in an earlier note
(note (23iv)).
It is true that I was looking first and foremost, in the students that approached me, 
for \textbf{collaborators} 
with whom to develop intuitions and ideas which had already formed within me, 
to ``push along'', in sum, a carriage that was already there, which they did not have to
summon from some kind of void, ``something which my correspondent had to do''.
This summoning - the act of bringing into being a tangible, supple, 
intense body of work from the intangible mist - had indeed always been, for me, the most
fascinating aspect of mathematical work, as well as the part in which I most strongly felt
a process of ``creation'' the ``spirit
of something more delicate and essential than a mere result''.

If I see certain ex-students of mine treating this valuable thing with disdain, 
letting grow within them
this ``snobbery'' which J.H.C. Whitehead talked about (consisting of disparaging 
what is ``immediately provable'')\footnote{(*)See the note (the snobbery of the youth - or
the defenders of purity), \no 27 p. 247.}(*), 
I am at least party to blame, for various reasons. 

\end{document}
