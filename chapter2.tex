\chapter{A walk through a life's work, or the child and the mother}

January 1986

\section{The magic of things}

\marginpar{p. P1}
When I was little I liked going to school. The same teacher taught us reading,
writing, arithmetic, singing (he accompanied us with a small violin), and even
about prehistoric men and the discovery of fire.
I do not ever recall being bored at school, during those days.
There was the magic of numbers, that of words, of signs, and of sounds.
That of \textbf{rhymes} as well, through songs and small poems. 
There seemed to be, within rhymes, a mystery which extended beyond words.
I believed this until the day somebody told me that there was a simple ``trick'' to it;
that rhyme was simply when one ends
two consecutive spoken movements by the same syllable, so that, as if by magic, these
phrases became \textbf{verses}. It was a revelation! 
At home, where I found a good audience, for weeks or months on end, I amused myself by
making verses. At one point, I even started exclusively speaking in rhymes. 
That period has passed, fortunately. 
Yet even to this day, I still sometimes write poems - but without trying to force the
rhyme, if it doesn't seem to come by itself. 

On another occasion an older friend, who was already in high school taught me about
negative numbers. It was also a fun game, although I rapidly exhausted it. 
And then there were crosswords - I spent days and weeks constructing them, evermore
interwoven. Within that game the magic of form, that of signs, and that of words found
themselves combined. But even that passion subsided without leaving a trace. 

During my high school years, which began in Germany during my first year, then later in
France, I was a good student without quite being the ``star student''.
I devoted myself without restraint to the
courses which I cared most about, and tended to neglect the others, without really caring
for the appreciation of my ``prof''. 
During my first year of high school in France, in 1940, I was interned at a concentration
camp with my mother, at Rieucros near Mende. It was wartime, and we were foreigners -
``undesirables'', as they said. But the administration of the camp turned a blind eye
towards the kids, however undesirable they may be. We came and went as we pleased. 
I was the oldest, and the only one to go to high school, which was four or five kilometers
away, in the rain and the wind, in makeshift shoes 
\marginpar{p. P2}
that always got wet. 

I still remember my first ``math examination'', in which the teacher gave me a bad grade,
for my proof of one of the ``three cases of equality of triangles''. 
My proof wasn't exactly that of the book, which he followed religiously. 
Yet, I knew very well that my proof was no less convincing than that of the book which I
followed min spirit, through repeated invocation of the traditional
``we slide this figure in such and such a way onto that figure''. 
Visibly, this teacher did not feel capable of judging things on his own (namely the
validity of the reasoning). He had to report to a higher authority, that of a book in this
case. I must have been stricken by such dispositions, for me to still remember this
incident. Ever since then, I have been presented with more than enough evidence to realize
that such dispositions are far from exceptional, but rather they are the quasi-universal
norm. There is a lot to be said on this subject - one which I approach more than once in
one way or another in 
R\'ecsoltes et Semailles. Yet to this day, I find myself invariably 
taken aback whenever I am confronted with such behavior\ldots

During the last few years of the war, while my mother was still interned at the camp, I
lived in a youth refugee house called ``Secours Suisse'', in Chambon sur Lignon. Most of
us were jewish, and when we were told (by the local police) that there would be raids by
the Gestapo, we went to hide in the woods for a night or two, in small groups of no less
than 3, without quite realizing that our life was on the line.
The region was filled with jews hiding in c\'evenol country, and many of us survived
thanks to the solidarity of the lcoal population.

What struck me most about
``Coll\'ege C\'evenol'' (where I was raised), was the extent to which my peers were
disinterested in learning. As for myself, I devoured our textbooks at the beginning of the
school year, thinking that this time around, we would finally learn \textbf{truly}
interesting things; and for 
the rest of the year I utlized my time as best as I could while classes dragged along
inexorably one trimester at a time. Yet we had some wonderful professors. The natural
history professor teacher, mister Friedel, was a person with remarkable intellectual and
social qualities. However, as he lacked authority, the class was acting out of control, to
the extent that it became impossible to hear what he had to say, as his voice was lost in
the hurly-burly. This may be the reason I haven't become a biologist! 


\marginpar{p. P3}
I spent a fair amount of time, including class time (shh\ldots), solving math problems.
The ones in the book soon became insufficient. Perhaps it was because they tended to
resemble one another after a while; but mostly, I believe, because they seemed
to come out of the blue \'a la queue-leue-leue,
with no indication as to where they came from or where they're going.
These were the books problems, not mine. And yet, natural questions were plentiful. For
instance, once the three side lengths $a$, $b$, and $c$ of a triangle are known, so that the
triangle itself is known (up to its position), there has to be an explicit ``formula''
that expresses the area of the triangle as a function of $a$, $b$, and $c$. Likewise, for
a tetrahedron of which the six side lengths are known - what is the volume? I struggled
through that one for a bit, but I must have gotten there eventually. In any case, when a
problem ``grabbed me'', I did not count the hours or days that I spent working on it, 
even if it meant losing track of everything else! (And such remains the case to this
day\ldots)

What I found least satisfying, in our math textbooks, was the absence of 
a serious definition of the notion of length (of a curve), of area (of a surface), or of
volume (of a solid). I promised myself to make up for this omission as soon as I could. 
This is what I devoted most of energy to between the years of 1945-1948, while I was a
student at the University of Montpellier. University lectures weren't for me. Without ever
quite realizing it, 
I must have been under the impression 
that all my professor did was recite the contents
of the textbooks, just like my first math teacher at the lyc\'ee de Mende.
I barely ever set foot on university grounds, just enough to keep up to date with 
the perennial ``program''. Books sufficed to cover said program, but it was clear that
they offered no answers to the questions I was asking myself. Truly, they did not even
\textbf{see} them, no more than my high-school textbooks did. As long as we were provided
with recipes for all sorts of calculations, such as lengths, areas, volumes, through single,
double, triple integrals (dimensions higher than 3 were carefully avoided\ldots), the
problem of providing an intrinsic definition was omitted by both my professors and
textbook authors. 

From my then limited experience, it seemed that I was the only person in the world to be
gifted with a curiosity for mathematical questions. Such was, in any case, my unexpressed
conviction, during those years spent in complete intellectual solitude (which did not
bother me).\footnote{Between 1945-1948, I lived with my mother in a small hamlet about 10
kilometers away from Montpellier, Mairargues (near Vendargues), lost in the middle of
vineyards. (My father disappeared in Auschwitz in 1942.) We scraped by on my meager
student funding. To make ends meet, I took part in the harvest every year, and after the
harvest season I would sell wine under the table (in contravention of the legislation, or
so I hear\ldots). On top of that, there was a self-regulating garden which 
supplied us with an abundance of
figs, spinach, and
even (towards the end) tomatoes planted by a complacent neighbor,
amidst a sea of splendid poppies.
It was the good life - although occasionally a bit rough along the edges, when we had to
replace a pair of glasses, or a pair of worn-out shoes. Luckily, because my mother was
weak and sick due to her long stay in the camps, we received free medical assistance. We
would never have been able to afford a doctor otherwise\ldots}
To be fair, it never occurred to me at that time to investigate
\marginpar{p. P4}
whether or not I was the
only person in the world to take interest in what I was doing. 
My energy was sufficiently absorbed by the task I set for myself: to develop a fully
satisfactory theory.

I never doubted that I would succeed in reaching the end of the story, as long as I was
committed to scrutinizing these structures, spelling out on paper what they were telling
me. The intuition behind \textbf{volume}, say, was irrecusable. It could only be the
reflection of a \textbf{reality}, momentarily elusive, but perfectly reliable. What had to
be done was simply to seize this reality - a bit, perhaps, the way the magic reality of
the ``rhyme'' had been seized, ``understood'' one day. 

When I began this pursuit, at age 17, freshly out of high-school, I thought it would only
take a few weeks. 
I spent three years on the project. It even caused me to fail an exam at the end of my
second year of university - that of spherical trigonometry (in the ``further astronomy''
module), because of a stupid computational mistake.
(I was never very good at computations, I must say, ever since I left high school\ldots).
That is why I had to spend a third year in Montpellier to complete
my bachelor's instead of going to Paris right away - the only place, I was told, where I
would be able to find people aware of what was considered important in Mathematics.
My informant, Mister Soula, assured me that the last problem left in mathematics had been
resolved twenty or thirty years ago by a so-called Lebesgue. 
He had apparently developed (funny coincidence!) a theory of measure and integration which
brought \emph{point final} to mathematics. 

Mister Soula, my ``diff calc'' teacher, was a benevolent man who took a liking to me. I
was still not convinced by his claim. 
\marginpar{p. P5}
There must have already been, within me, the
prescience that mathematics is a thing which is infinite in scope and depth. Does the sea
have a ``\emph{point final}''? Yet I never thought of looking for that book by Lebesgue
which Mister Soula had told me about, and he probably never held it either. 
In my mind there was nothing in common between anything a book contained and
the work that \textbf{I} had been doing, in my own way, in order to answer questions which 
intrigued me. 

\section{The importance of being alone}

When I finally made contact with the mathematical world in Paris, one or two years later,
I ended up learning, among many other things, that the work which I had been doing
independently, and with the means at hand, was (essentially) what ``everybody'' 
knew as the ``Lebesgue theory of measure and integration''. According to the two or three 
experts to whom I mentioned my work (or even showed a manuscript), I had just wasted my
time redoing something ``already known''. I actually do not recall being disappointed. 
At that moment, the idea of receiving ``credit'', or even simply receiving
approbation for the work that I was doing, must have still been foreign to my mind. 
Furthermore, my energy was completely taken by the process of familiarizing myself with an
entirely different milieu and mostly learning what was considered in Paris to be the basic
toolkit of the mathematician.\footnote{I briefly narrate this rough transition period in
the first part of  
R\'ecoltes et Semailles (ReS I), in the section ``The Welcome Stranger'' (nb. 9).}

Yet, thinking back to those three years, I realized that they were not in any way wasted.
Unknowingly, I learned in solitude what is essential 
to the work of a mathematician - something no master could 
truly teach. Without ever having been told, without ever having to encounter someone with
whom I could share my quest for understanding, I knew ``in my gut'' that I was a
mathematician: somebody who ``does'' math, in its fullest sense - the way one makes
``love''. Mathematics had become, for me, a mistress always 
accommodating my desires. These years of solitude laid the foundation for a trust that has
never been shaken - not by the discovery (upon arrival in Paris at age 20) of the scope of
my ignorance and the vastness of what I had to learn; nor (more than 20 years later) by
the eventful episode of my permanent departure from the mathematical world; nor, in these
last few years, by the often crazy episodes of a ``Funeral'' (anticipated 
\marginpar{p. P6}
and cleanly executed)
of my person and life's work, orchestrated by those who used to be my closest
companions\ldots 

To phrase it differently: I learned in those crucial years to ``be alone''.\footnote{
This formulation is somewhat clumsy. I never had to ``learn to be alone'', for the simple
reason that I never \textbf{unlearned} during the course my childhood, 
this innate skill which I had since birth, just as we all do. Yet these three years of
solitary work, during which I could walk to my own beat, 
following my own exigence criteria, confirmed within me a degree of trust and tranquil confidence 
in my relationship with mathematics which owed nothing to the reining trends and
consensus. I make allusion to these again in the notes ``Roots and Solitude'' (Res IC, nb.
171) notably 
% \todo{1080}.
.
} That is, I learned to approach the things which I want to know with my own eyes, rather
than rely on the expressed or implicit ideas that eminate from the group with which I
identify, or a group to which I attribute authority.
An unspoken consensus told me, both in high school and in university, that there was no
need to question the notion of ``volume'', which was presented as ``well-known'',
``self-evident'', ``unproblematic''.
Naturally I turned a blind eye to this consensus - just as Lebesgue, a few decades
earlier, had to \textbf{turn a blind eye}.
It is in this act of ``turning a blind eye'', of being oneself rather than the mere
expression of the reigning consensus, of not to remain inscribed within the imperative
circle to which they assign us - it is within this solitary act, above all else, that
``\textbf{creation}'' lies. Everything else comes after.

In the following years, within the mathematical world which welcomed me, I had the opportunity 
to meet multiple people, both older and younger, which were clearly more brilliant,
``gifted'' than I was. I admired the facility with which they learned new notions, as if
at play, juggling them as if they had known them their whole life - while I felt
heavy-handed and clumsy, laboriously making my way, akin to a mole, through an amorphous
mountain of important things (or so I was told) which I had to learn, despite having no
sense of their ins and out. Actually, I was far from the brilliant student who aced every prestigious
\emph{concours} and assimilating at once the most prohibitive courses.

Many of my more brilliant peers went on to become competent famous mathematicians. In
hindsight, after 30-35 years, it does not seem yo me that they left a deep imprint
\marginpar{p. P7}
upon the mathematics of today. 


% \marginpar{p. P1}
% R\'ecoltes et Semailles
% p .hi
% c ,
% q "
% pa (
% b \textbb{}
