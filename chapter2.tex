\begin{comment}
\documentclass{book}
\usepackage{master}
\newcommand{\rec}{$\text{R\'ecoltes et Semailles}$}
\newcommand{\no}{n$^\circ$}
\hfuzz = 100pt
\begin{document}
\end{comment}


January 1986

\section{The magic of things}

\marginpar{p. P1}
When I was little I liked going to school. The same teacher taught us reading,
writing, arithmetic, singing (he accompanied us with a small violin), and even
about prehistoric men and the discovery of fire.
I do not ever recall being bored at school, during those days.
There was the magic of numbers, that of words, of signs, and of sounds.
That of \textbf{rhymes} as well, through songs and small poems. 
There seemed to be, within rhymes, a mystery which extended beyond words.
I believed this until the day somebody told me that there was a simple ``trick'' to it;
that rhyme was simply when one ends
two consecutive spoken movements by the same syllable, so that, as if by magic, these
phrases became \textbf{verses}. It was a revelation! 
At home, where I found a good audience, for weeks or months on end, I amused myself by
making verses. At one point, I even started exclusively speaking in rhymes. 
That period has passed, fortunately. 
Yet even to this day, I still sometimes write poems - but without trying to force the
rhyme, if it doesn't seem to come by itself. 

On another occasion an older friend, who was already in high school taught me about
negative numbers. It was also a fun game, although I rapidly exhausted it. 
And then there were crosswords - I spent days and weeks constructing them, evermore
interwoven. Within that game the magic of form, that of signs, and that of words found
themselves combined. But even that passion subsided without leaving a trace. 

During my high school years, which began in Germany during my first year, then later in
France, I was a good student without quite being the ``star student''.
I devoted myself without restraint to the
courses which I cared most about, and tended to neglect the others, without really caring
for the appreciation of my ``prof''. 
During my first year of high school in France, in 1940, I was interned at a concentration
camp with my mother, at Rieucros near Mende. It was wartime, and we were foreigners -
``undesirables'', as they said. But the administration of the camp turned a blind eye
towards the kids, however undesirable they may be. We came and went as we pleased. 
I was the oldest, and the only one to go to high school, which was four or five kilometers
away, in the rain and the wind, in makeshift shoes 
\marginpar{p. P2}
that always got wet. 

I still remember my first ``math examination'', in which the teacher gave me a bad grade,
for my proof of one of the ``three cases of equality of triangles''. 
My proof wasn't exactly that of the book, which he followed religiously. 
Yet, I knew very well that my proof was no less convincing than that of the book which I
followed min spirit, through repeated invocation of the traditional
``we slide this figure in such and such a way onto that figure''. 
Visibly, this teacher did not feel capable of judging things on his own (namely the
validity of the reasoning). He had to report to a higher authority, that of a book in this
case. I must have been stricken by such dispositions, for me to still remember this
incident. Ever since then, I have been presented with more than enough evidence to realize
that such dispositions are far from exceptional, but rather they are the quasi-universal
norm. There is a lot to be said on this subject - one which I approach more than once in
one way or another in 
R\'ecoltes et Semailles. Yet to this day, I find myself invariably 
taken aback whenever I am confronted with such behavior\ldots

During the last few years of the war, while my mother was still interned at the camp, I
lived in a youth refugee house called ``Secours Suisse'', in Chambon sur Lignon. Most of
us were jewish, and when we were told (by the local police) that there would be raids by
the Gestapo, we went to hide in the woods for a night or two, in small groups of no less
than 3, without quite realizing that our life was on the line.
The region was filled with jews hiding in C\'evenol country, and many of us survived
thanks to the solidarity of the lcoal population.

What struck me most about
``Coll\`ege C\'evenol'' (where I was raised), was the extent to which my peers were
disinterested in learning. As for myself, I devoured our textbooks at the beginning of the
school year, thinking that this time around, we would finally learn \textbf{truly}
interesting things; and for 
the rest of the year I utlized my time as best as I could while classes dragged along
inexorably one trimester at a time. Yet we had some wonderful professors. The natural
history professor teacher, mister Friedel, was a person with remarkable intellectual and
social qualities. However, as he lacked authority, the class was acting out of control, to
the extent that it became impossible to hear what he had to say, as his voice was lost in
the hurly-burly. This may be the reason I haven't become a biologist! 


\marginpar{p. P3}
I spent a fair amount of time, including class time (shh\ldots), solving math problems.
The ones in the book soon became insufficient. Perhaps it was because they tended to
resemble one another after a while; but mostly, I believe, because they seemed
to come out of the blue \`a la queue-leue-leue,
with no indication as to where they came from or where they're going.
These were the books problems, not mine. And yet, natural questions were plentiful. For
instance, once the three side lengths $a$, $b$, and $c$ of a triangle are known, so that the
triangle itself is known (up to its position), there has to be an explicit ``formula''
that expresses the area of the triangle as a function of $a$, $b$, and $c$. Likewise, for
a tetrahedron of which the six side lengths are known - what is the volume? I struggled
through that one for a bit, but I must have gotten there eventually. In any case, when a
problem ``grabbed me'', I did not count the hours or days that I spent working on it, 
even if it meant losing track of everything else! (And such remains the case to this
day\ldots)

What I found least satisfying, in our math textbooks, was the absence of 
a serious definition of the notion of length (of a curve), of area (of a surface), or of
volume (of a solid). I promised myself to make up for this omission as soon as I could. 
This is what I devoted most of energy to between the years of 1945-1948, while I was a
student at the University of Montpellier. University lectures weren't for me. Without ever
quite realizing it, 
I must have been under the impression 
that all my professor did was recite the contents
of the textbooks, just like my first math teacher at the lyc\'ee de Mende.
I barely ever set foot on university grounds, just enough to keep up to date with 
the perennial ``program''. Books sufficed to cover said program, but it was clear that
they offered no answers to the questions I was asking myself. Truly, they did not even
\textbf{see} them, no more than my high-school textbooks did. As long as we were provided
with recipes for all sorts of calculations, such as lengths, areas, volumes, through single,
double, triple integrals (dimensions higher than 3 were carefully avoided\ldots), the
problem of providing an intrinsic definition was omitted by both my professors and
textbook authors. 

From my then limited experience, it seemed that I was the only person in the world to be
gifted with a curiosity for mathematical questions. Such was, in any case, my unexpressed
conviction, during those years spent in complete intellectual solitude (which did not
bother me).\footnote{Between 1945-1948, I lived with my mother in a small hamlet about 10
kilometers away from Montpellier, Mairargues (near Vendargues), lost in the middle of
vineyards. (My father disappeared in Auschwitz in 1942.) We scraped by on my meager
student funding. To make ends meet, I took part in the harvest every year, and after the
harvest season I would sell wine under the table (in contravention of the legislation, or
so I hear\ldots). On top of that, there was a self-regulating garden which 
supplied us with an abundance of
figs, spinach, and
even (towards the end) tomatoes planted by a complacent neighbor,
amidst a sea of splendid poppies.
It was the good life - although occasionally a bit rough along the edges, when we had to
replace a pair of glasses, or a pair of worn-out shoes. Luckily, because my mother was
weak and sick due to her long stay in the camps, we received free medical assistance. We
would never have been able to afford a doctor otherwise\ldots}
To be fair, it never occurred to me at that time to investigate
\marginpar{p. P4}
whether or not I was the
only person in the world to take interest in what I was doing. 
My energy was sufficiently absorbed by the task I set for myself: to develop a fully
satisfactory theory.

I never doubted that I would succeed in reaching the end of the story, as long as I was
committed to scrutinizing these structures, spelling out on paper what they were telling
me. The intuition behind \textbf{volume}, say, was irrecusable. It could only be the
reflection of a \textbf{reality}, momentarily elusive, but perfectly reliable. What had to
be done was simply to seize this reality - a bit, perhaps, the way the magic reality of
the ``rhyme'' had been seized, ``understood'' one day. 

When I began this pursuit, at age 17, freshly out of high-school, I thought it would only
take a few weeks. 
I spent three years on the project. It even caused me to fail an exam at the end of my
second year of university - that of spherical trigonometry (in the ``further astronomy''
module), because of a stupid computational mistake.
(I was never very good at computations, I must say, ever since I left high school\ldots).
That is why I had to spend a third year in Montpellier to complete
my bachelor's instead of going to Paris right away - the only place, I was told, where I
would be able to find people aware of what was considered important in Mathematics.
My informant, Mister Soula, assured me that the last problem left in mathematics had been
resolved 20 or 30 years ago by a so-called Lebesgue. 
He had apparently developed (funny coincidence!) a theory of measure and integration which
brought \emph{point final} to mathematics. 

Mister Soula, my ``diff calc'' teacher, was a benevolent man who took a liking to me. I
was still not convinced by his claim. 
\marginpar{p. P5}
There must have already been, within me, the
prescience that mathematics is a thing which is infinite in scope and depth. Does the sea
have a ``\emph{point final}''? Yet I never thought of looking for that book by Lebesgue
which Mister Soula had told me about, and he probably never held it either. 
In my mind there was nothing in common between anything a book contained and
the work that \textbf{I} had been doing, in my own way, in order to answer questions which 
intrigued me. 

\section{The importance of being alone}

When I finally made contact with the mathematical world in Paris, one or two years later,
I ended up learning, among many other things, that the work which I had been doing
independently, and with the means at hand, was (essentially) what ``everybody'' 
knew as the ``Lebesgue theory of measure and integration''. According to the two or three 
experts to whom I mentioned my work (or even showed a manuscript), I had just wasted my
time redoing something ``already known''. I actually do not recall being disappointed. 
At that moment, the idea of receiving ``credit'', or even simply receiving
approbation for the work that I was doing, must have still been foreign to my mind. 
Furthermore, my energy was completely taken by the process of familiarizing myself with an
entirely different milieu and mostly learning what was considered in Paris to be the basic
toolkit of the mathematician.\footnote{I briefly narrate this rough transition period in
the first part of  
R\'ecoltes et Semailles (ReS I), in the section ``The Welcome Stranger'' (nb. 9).}

Yet, thinking back to those three years, I realized that they were not in any way wasted.
Unknowingly, I learned in solitude what is essential 
to the work of a mathematician - something no master could 
truly teach. Without ever having been told, without ever having to encounter someone with
whom I could share my quest for understanding, I knew ``in my gut'' that I was a
mathematician: somebody who ``does'' math, in its fullest sense - the way one makes
``love''. Mathematics had become, for me, a mistress always 
accommodating my desires. These years of solitude laid the foundation for a trust that has
never been shaken - not by the discovery (upon arrival in Paris at age 20) of the scope of
my ignorance and the vastness of what I had to learn; nor (more than 20 years later) by
the eventful episode of my permanent departure from the mathematical world; nor, in these
last few years, by the often crazy episodes of a ``Funeral'' (anticipated 
\marginpar{p. P6}
and cleanly executed)
of my person and life's work, orchestrated by those who used to be my closest
companions\ldots 

To phrase it differently: I learned in those crucial years to ``be alone''.\footnote{
This formulation is somewhat clumsy. I never had to ``learn to be alone'', for the simple
reason that I never \textbf{unlearned} during the course my childhood, 
this innate skill which I had since birth, just as we all do. Yet these three years of
solitary work, during which I could walk to my own beat, 
following my own exigence criteria, confirmed within me a degree of trust and tranquil confidence 
in my relationship with mathematics which owed nothing to the reining trends and
consensus. I make allusion to these again in the note ``Roots and Solitude'' 
(Res IV, \no $171_3$) notably page 
% \todo{1080}.
.
} That is, I learned to approach the things which I want to know with my own eyes, rather
than rely on the expressed or implicit ideas that eminate from the group with which I
identify, or a group to which I attribute authority.
An unspoken consensus told me, both in high school and in university, that there was no
need to question the notion of ``volume'', which was presented as ``well-known'',
``self-evident'', ``unproblematic''.
Naturally I turned a blind eye to this consensus - just as Lebesgue, a few decades
earlier, had to \textbf{turn a blind eye}.
It is in this act of ``turning a blind eye'', of being oneself rather than the mere
expression of the reigning consensus, of not to remain inscribed within the imperative
circle to which they assign us - it is within this solitary act, above all else, that
``\textbf{creation}'' lies. Everything else comes after.

In the following years, within the mathematical world which welcomed me, I had the opportunity 
to meet multiple people, both older and younger, which were clearly more brilliant,
``gifted'' than I was. I admired the facility with which they learned new notions, as if
at play, juggling them as if they had known them their whole life - while I felt
heavy-handed and clumsy, laboriously making my way, akin to a mole, through an amorphous
mountain of important things (or so I was told) which I had to learn, despite having no
sense of their ins and out. Actually, I was far from the brilliant student who aced every prestigious
\emph{concours} and assimilating at once the most prohibitive courses.

Many of my more brilliant peers went on to become competent famous mathematicians. In
hindsight, after 30-35 years, it does not seem to me that they left a deep imprint
\marginpar{p. P7}
upon the mathematics of today. 
They did things, often times beautiful things, in a pre-existing context which they would
never have considered altering.
They unknowingly remained prisoners in their imperious circles, which
delimitate the Universe of 
a given time and milieu. In order to overcome them, they would have had to rediscover
within them the ability which they had since birth, just as I did: the capacity to be
alone. 

The small child has no difficulty being alone. He is solitary by nature, even though he
enjoys the occasional company, and knows when to ask for mom's permission teat. And he
knows, without having ever been told, that the teat is his, and that he \textbf{knows}
how to drink. 
Yet often times we lose touch with out inner child. 
And thus we constantly miss out on the best 
without even seeing it\ldots

If in R\'ecoltes et Semailles I address somebody other than myself, it is not a
``public''. I address myself to you, reader, as I would a \textbf{person}, and a person
\textbf{alone}. 
It is to the person inside of you that knows how to be alone, the child, with whom I would
like to speak, and nobody else. 
I am aware that the child is often far away.
He has gone through all sorts of things for quite some time. 
He went hiding god knows where, and it can be hard, often times, to get to him.
One could swear that he has been dead forever, or rather that he has never existed - and
yet I am sure that he is there somewhere, well alive. 

I know too what the \textbf{sign} 
is that I am being heard. 
It is when, beyond all cultural and
experiential differences, what I share about my personal experiences echos within you and finds 
resonance; when you find within it \textbf{your own life}, your own self-experience, in a
new light which you may never have considered before that. It is not about an
``identification'' with something or someone far from you. 
Rather, perhaps, you will rediscover a bit of your own life, or of what is
\textbf{closest} to you, as you follow my own rediscovery of myself throughout 
R\'ecoltes et Semailles, 
including within these very pages which I am currently writing. 

\section{The inner journey - or myth and testimony}

Before all else, R\'ecoltes et Semailles
is a \textbf{reflection} upon myself and my life. 
Because of this, it is also a \textbf{testimony}, in two distinct ways. It is a testimony
about my \textbf{past}, upon which the principle component of the reflection is concerned
with. But it is also, at the same time, a testimony about the immediate present, about the
very moment at which I am writing, and during which R\'ecoltes et Semailles is born,
\marginpar{p. P8}
in the course of hours, nights, and days. 
These pages serve as faithful witnesses to a long meditation upon my life, such as it was
really carried out (and continues to be carried out at this very moment\ldots).

These pages have no literary pretense 
They only constitute a document about myself. I only allowed myself to modify them within 
very narrow bounds\footnote{Thus, the occasional rectification of mistakes (material
and of viewpoint) does not appear in the first pass but rather in 
footnotes or in later reconsideration.} (notably for occasional stylistic edits). If there
is pretense, it is only that of faithfulness. And that is saying a lot. 

This document is also far from an ``autobiography''. 
You will learn neither my date of birth (which would be of little relevance unless one is
making astrological predictions), nor the names of my mother and father 
or what they did in life, nor the names of the person who was my spouse and other women
who have been very important in my life, nor those of the children that were born from
these unions, nor what these children have made of their lives.
This does not mean that these things were not important in my life. Rather, the way this
reflection on myself was engaged and developed never incited me to give a description of
those things, which I lightly touch on here and there, but never take the time to
consciously flesh them out with names and numbers. It never seemed to me that doing so
would add anything whatsoever to the point which I was making at any given time. (Whereas
in the few pages above I was brought, almost inadvertently, 
to include perhaps more material details on my life than
you will find in the thousand pages to come\ldots)

And if you were to ask me what ``point'' I have attempted to make over the course of
these thousand pages, I would answer: it is to narrate, and thereby \textbf{discover}, 
the \textbf{inner journey} that my life has been and still is. 
This narrative/testimony of a journey
is happening simultaneously at the two level which I have mentioned above. There is the
exploration of a journey past, of its roots and of its origin, tracing all the way back to
my childhood. And there is the continuation and renewal of this ``very'' journey, over the
course of the days during which I am writing  R\'ecoltes et Semailles
in spontaneous response to a violent stimulus coming from the outside
world.\footnote{For more details about this ``violent stimulus'', see ``The Letter'',
notably sections 3 through 8.
% \todo{add links}
} 

\marginpar{p. P9}
External facts come to nourish the reflection, only to the extent they they induce and
provoke new
developments in my inner journey, or help clarify it. 
The burial and the plunder of my mathematical work, of which I will speak at length, has
been such a provocation. It awoke in me
a host of powerful reactions, and at the same time revealed to me the profound, and
hitherto unknown links that continue to tie me to the work I have created.

It is true that my being ``good at math'' is not necessarily a reason (and even less so a
good reason) for you to be interested in my particular
journey - nor is the fact that I have had trouble with my colleagues, after shifting
milieu and lifestyle.
Colleagues, or even friends abound, who find it ridiculous to publicly spread 
one's ``inner moods'' - as they say.
To them, what matters are ``results''. 
The ``soul'', meaning that within us which
\textbf{witnesses} the production of these result, 
as well as apprehends them in various ways
(as much in the life of the ``producer'' as in those of his peers), is looked down upon,
sometimes even targeted with open derision.
This attitude aims to display some form of modesty, but what I see is
a sign of withdrawal or asynchrony promoted by the very air which we breath.
I do not write for he who is stricken by this latent self disgust, which makes him reject
the best I have to offer. A disdain 
for what truly makes his \textbf{own life}, and for what makes mine: the superficial or
profound, course or subtle motions which animate the psyche, that ``soul'' which lives the
experience and reacts to it, which freezes or blossoms, which retreats or learns\ldots

The narrative of an inner journey can only be told 
by the person living it and no one else. 
Even though the narrative is only aimed
towards oneself, it often times inserts itself 
within the construction of a \textbf{myth}, of which the narrator is the hero. 
Such a myth is born, not in the creative imagination of a people and a culture, but rather
from vanity of he who dared not assume a humble reality, 
but instead substitutes a construction for it. 
But a \textbf{true} narrative (if such a thing exists), of a journey such as it was truly
lived, is to be prized. And this is not because of renown
which is (rightly or wrongly) attributed to the narrator, but simply by virtue of its
\textbf{existence}, and of its truthfulness. Such a testimony is precious,
whether it comes from an illustrious person, a small clerk with no future 
and with family responsibilities, or from a common criminal.

If there is value for one in such a narrative, it is first and foremost that of self
confrontation, through this unvarnished testimony of the experience 
of an other. 
\marginpar{p. P10}
But also (to phrase it differently) to erase within oneself
(be it only for the span of a reading) this disdain by which one holds one's \textbf{own
journey}, and that ``soul'' of which one is both the passenger and the captain\ldots

\section{The painting of mores}

In speaking of my past as a mathematician, and later in discovering (almost against my
will) the twists and turns of the intricacies of the gigantic Burial of my work, I was
brought, inadvertently, 
to paint the picture of a particular milieu and era - an era
affected by the decay of certain values
which provided meaning to the work of individuals.
That is what I mean by
``painting of mores'', centered around a 
``fait divers'' which is thoughtlessly unique in the annals of Science
as I have said rather clearly earlier, I believe, you will not find in 
R\'ecoltes et Semailles
a ``folder'' concerning a certain unordinary ``case'', quickly bringing you up to date.
And yet a friend of mine, looking for such a folder, 
blindly passed by nearly
everything constituting the substance and flesh of 
R\'ecoltes et Semailles.

As I explain in much more detail in the Letter, the ``investigation'' (or the ``painting
of mores'') carries on in 
parts \ref{part:II} and 
% \ref{4},
% \todo{ref}
``The Funeral (1)
- or the robe of the Emperor of China'' and 
``The Funeral (3) - or the Four Operation''. 
Page after page I persistently extract one after another, 
a number of juicy facts (to say the least) which I attempt to ``classify''
bit by bit. Slowly, these fact
assemble into a global painting which progressively 
emerges from the fog, taking on brighter colors and 
sharper contours.
In these daily notes, the raw facts
``which just appeared''
are inextricably mixed with personal
reminiscing, as well as with commentaries and reflections of a psychological,
philosophical, or even (occasionally) mathematical nature.
That's how it is, and there is nothing I can do about it!

Starting with work I had already done, 
which occupied me for over a year, producing 
a sort of ``investigation proceedings'' folder
should not have taken longer than a few hours 
or days worth of work, depending
on the curiosity or demands of the interested reader.
I tried at one point to produce such a folder. 
That is how I started writing a note which was to be called ``the four
operations''.\footnote{The note eventually exploded into part 
% \ref{part:IV}
% \todo{ref}
(also named ``The four operations'') of \rec, comprising about seventy notes 
running over more than four-hundred pages.}
But in the end I could not bring myself to do it!
That is decidedly\marginpar{p. P11}
not my style of expression, and in my old age less so than ever.
I now consider, having written \rec, 
that I have done enough for the benefit of the ``mathematical community'', 
and therefore can leave, without remorse, 
the task of producing 
the necessary ``folder'' to others (in particular to any of my
colleagues who would feel concerned). 

\section{The heirs and the builders}

It is now time for 
me to say a few words about my mathematical work, which has 
played an important role in my life, and continues to do so (to my own surprise).
I come back to this work more than once in 
\rec\ - sometimes in a way 
that should be
understandable by all, 
and other times in slightly more technical terms.\footnote{One will also find here and
there, in addition to mathematical notes concerning my
previous work, sections containing new mathematical developments. The longest of these is
``the five pictures (crystals and $\cD$-modules)'' in ReS IV, note \no 171 (ix).
% \todo{reference}
} The latter 
will mostly go ``above the heads'', not only of the ``profane'', but also of the
mathematical colleague who may not be completely ``in'' the field in question.
One can of course feel free to skip the sections which seem too ``involved''. Just as one can
try to go through them, glimpsing as one goes, a shadow of the ``mysterious beauty'' (in
the words of a non-mathematician friend of mine) of the universe of mathematical things,
appearing as a multitude of ``strange inaccessible islands'' in the vast 
moving waters of reflection\ldots

Most mathematicians, as I mentioned earlier, are inclined to constrain themselves to a
conceptual framework, a ``\textbf{universe}'' fixed once and for all - the one,
essentially, which they have found ``ready made'' at the time of their studies.
They are like the heirs of a large and beautiful fully-furnished house, with its lounges,
kitchens, workshops, and its kitchenware and tools left and right, with which there is, I
trust, plenty to cook and tinker.
How this house was built, progressively over the course of multiple generations, 
and how and why these tools (and not others\ldots) were conceived and built, 
why the pieces are disposed and organized in such a way - these are all questions the
heirs would never think of asking themselves.
This is the ``universe'', the ``given'', in which we must live, and that is that!
Something which appears massive (and which most of the time we have only been able to
partially explore), yet at the same time \textbf{familiar}, and mostly:
\textbf{immutable}.
They mostly busy themselves with maintaining or embellishing a patrimony: fixing a faulty
piece of furniture, restoring a facade, sharpening a tool, or even sometimes, for the most
enterprising, building an entire workshop, or a whole new piece of furniture. It even
happens,\marginpar{p. P12}
when they fully commit to the task, that the piece of furniture is truly
beautiful, so that the whole house appears embellished by its addition. 

Even more rarely, one of them will consider 
modifying one of the main tools, or even, 
under repeated and insistent pressure or need, to imagine and build a whole new tool. 
And in so doing, he often feels on the brink of profusely apologizing for what he feels is
infringing on the piety owed to the familial tradition, which he has disturbed through his
brazen innovation. 

In most of the rooms of the house, the windows and shutters are carefully closed, probably
on account of a fear that a foreign wind would blow in.
And when the pretty new furnishings, here and there, together with their progeny, begin to
clutter the rooms and invade the corridors, none of these heirs will agree to face the
fact that his familiar and cozy Universe is beginning to feel cramped.
Rather than come to terms with such a fact, most will prefer to 
awkwardly slither,
and try not to get trapped, some between a Louis XV buffet and a
rocking chair in rattan, others between a boisterous toddler
and an Egyptian sarcophagus, while others, as a last resort, will try to climb over a
heteroclite and crumbling pile of chairs and benches\ldots

The picture I have just sketched is not unique to the world of mathematicians. It
illustrates the deeply engrained and immemorial conditioning which 
one encounters in every milieu and sphere of human activity, regardless, as far as I can
tell, of the society or era in question.
I have mentioned such a phenomenon already, and I do not in any way pretend to fall
outside of its influence. 
As will be clear from my testimony, the contrary is true.
It only happens to be the case that at the relatively limited level of the act of
intellectual creation, I was barely affected\footnote{I believe the main reason for such
immunity is a certain favorable climate which surrounded me until age 5, the note ``The
innocence'' (ReS III,\no 107).
% \todo{reference}
} by this conditioning which may be called 
``cultural blindness'' - the incapacity to see (and to evolve) outside of the ``Universe''
fixed by the surrounding culture.

As for myself, I feel that I belong to the lineage of mathematicians whose spontaneous
vocation and joy was to continuously construct new houses\footnote{This archetypal 
picture of a ``house''
to be built, 
surfaces and is formulated for the first time in ``Yin, the Servant, and the new masters''
(ReS III , \no 135).
% \todo{reference}
} In so doing, they cannot help but invent\marginpar{p. P13}
all of the required tools, utensils, and 
furnishings for both the construction of the house from its foundation, and to fill
the kitchens and workshops of the house in abundance, so that one may live in it
comfortably. Yet, once everything down 
to the last sapling and stool has been taken care of, the builder rarely lingers on the
premises, of which every stone and every piece of wood carries a trace of the hand which
shaped and placed it. The builder's place lies not in the quietude of fully 
finished universes, however welcoming and harmonious they may be, whether they are a
product of his own hands or those of his predecessors. His place is in the open air. 
He is friends with the wind, and does not fear solitude at work for weeks, years or,
if need be, for an entire lifetime if no welcome succession presents itself.
Just like everybody else, the builder only has two hands - but two hands which at each
moment know what they need to do, which refuse neither the largest nor the most delicate
tasks, and which never tire of comprehending, again and again, the multitude of things 
which become them. Two hands might be few, given that the World is infinite. 
They will never exhaust it! And yet, two hands can be a lot\ldots

History is not my strong suit, but if I had to give a list of mathematicians inscribed in
this lineage, names that spontaneously come to my mind are those of Galois, Riemann (from
the past century), and Hilbert (at the beginning of the current century). If I were to
name a candidate among the elders which welcomed me into the mathematical
world,\footnote{I speak of these beginnings in the section ``The welcome stranger'' (ReS I
, \no 9).} the name
of Jean Leray comes to my mind before any other, even though my contact with him has
always been episodic.\footnote{I was nonetheless (following H. Cartan and G. Serre) one of
the first users and promoters of one of the great innovative notions introduced by Leray,
that of a sheaf, which has been an essential tool throughout my work as a geometer. It is
also the notion which has provided me with the key to enlarge the notion of a topological
space into that of topos, about which I will be speaking later.

Leray differs from the portrayal I have given of the ``builder'', I believe, in that he
does not seem drawn to ``construct houses from their foundations to their completion''.
Rather be was compelled to lay out vast foundations, in places where nobody would have
thought to look while leaving to others the care of carrying the construction to its
completion, and once the house is built, to settle into the premises (be it only for a
short time)\ldots}

\marginpar{p. P14}
I have just roughly sketched two pictures: that of the ``homebody'' mathematician, who
is content with maintaining and embellishing a heritage, and that of the
builder-pioneer,\footnote{I have just surreptitiously attached herein two qualifiers with 
male connotation (that of ``builder'' and that of ``pioneer''), which express very
different aspects of the impulse of discovery, one which is of a nature more delicate than
what these qualifiers might evoke. Such a discussion will be carried out later in this
walk-reflection, in the step ``The discovery of the Mother - or the two versants''
(\no 17).}
who is drawn to repeatedly crossing these ``invisible and imperious circles'' which
delimitate a given Universe.\footnote{At the same time, and without really meaning to do
so, the builder-pioneer assigns to the old Universe (if not for himself, at the very least
for his more sedentary colleagues) new boundaries, thereby inscribing 
circled which may be larger, but are just as invisible and imperious as those which they
have come to replace.} These two groups
may also be called, somewhat bluntly but also suggestively, ``conservatives'' and
``innovators''. Both have their raison d'\^etre, in one collective adventure that is
carried out through the generations, through centuries and millennia.
During the fruitful periods of a science or art there is neither opposition, nor is there 
antagonism among these two temperament.\footnote{Such has been the case, notably in the
mathematical world, during the period (1948-1969) of which I was a direct witness, as I
myself belonged to that world. Following my departure in 1970, there seems to have been a
large scale reaction, a sort of ``consensus of disdain'' for the ``ideas'' in general, and
more specifically the great innovative ideas that I have introduced.} 
They are different and mutually complementary,
just as dough and yeast. 

Between these two extremes (not at all opposed by nature), one can find a plethora of
intermediary temperament. There is the ``homebody'' that would never think of leaving a
familiar dwelling, and would be even less willing to take on the task of building another,
god knows where, yet will not hesitate, when the house gets cramped to build a basement,
raise the ceiling, or even, if need be, to build a dependency of modest
proportions.\footnote{Most of my ``elders'' (about whom I speak for instance in ``a
welcome debt'' (Introduction \no 10)) conform to this intermediary temperament. I was
thinking notably about Henri Cartan, Claude Chevalley, Andr\'e Weil, Jean-Pierre Serre,
Laurent Schwartz. With the exception maybe of Weil, they have all turned a ``sympathetic
eye'', without ``concern nor secret reprobation'' towards
the solitary adventures into which they saw me embark.} Without being a builder at heart,
he will often view with a sympathetic eye, or at the very least without concern nor secret
reprobation towards another who had shared the same dwelling, and who is already out and
about assembling beams 
\marginpar{p. P15}
and stones in some impossible boonies, with the confidence of somebody
who already sees a castle\ldots 

\section{Viewpoint and vision}

Allow me to return to myself and my work.

If I excelled in the art of mathematics,
it was not through the ability and perseverance to solve problems left by my predecessors,
but rather through a natural tendency within me to 
discover \textbf{questions},
evidently crucial, yet that nobody had yet seen, or to excavate the \textbf{``right notions''}
that were missing
(often without anyone realizing until the new notion appeared), as well as the
\textbf{``right statements''} of which nobody had thought.
Often, notions and statements mesh in such a perfect way, that there can be no doubt in my
mind as to their validity (give or take small adjustments at most) - so that often, when
it boils down to ``travail sur pi\`eces'' destined for publication, I refrain from going
further, and from taking the time to flesh out a proof that often, once the statement and
its context are well-understood, consists of no more than a matter of ``trade'', not to
say routine.
Things which solicit ones attention are countless, and it is impossible to follow 
them all to their end! 
The fact remains that carefully proved propositions and theorems in my written and
published work appear in the thousands, and almost all of them have entered the 
patrimony of things commonly accepted as ``known'' and frequently used all over
mathematics.

I am led more towards the discovery of
fertile \textbf{viewpoints} 
than towards the discovery of questions, notions, and statements, by my particular type of
genius, which is constantly leading me to introduce, and more or
less develop, entirely novel \textbf{themes}. 
It is this, I reckon, which is my most essential contribution to the mathematics of my time.
In fact, these innumerable questions, notions, and statements which I just mentioned, only
truly make sense for me once they are subjected to such a ``viewpoint'' - or more
precisely they \textbf{arise} spontaneously from it; in the same way that a light (even a
dim one) appearing in a pitch black night seems to invoke from the shadows
contours which it suddenly reveals to us.
Without this light 
uniting them in a common sheaf, the ten, or one-hundred, or one thousand questions,
notions, statements would appear as a heterogeneous amorphous pile of 
``mental widgets'' all isolated from one another - 
\marginpar{p. P16}
rather than as the many parts of a
\textbf{Whole} which, while perhaps remaining invisible, 
escaping within the folds of the night,
is nonetheless clearly felt.

The fertile viewpoint is that which reveals to us, organized as the many living parts of a
common Whole, 
enveloping them and giving them meaning, these pressing questions which no one had asked,
and (as if in response, perhaps, to these questions) these extremely natural notions which
nobody had thought of expressing, and these statements finally which seem to immediately
follow, and which nobody had dared to conjecture, for as long as the questions which
brought them about, and the notions that allowed us to formulate them had remained hidden.
Even more than what we call ``key theorems'' in mathematics, it is the fertile viewpoints
which, in our art,\footnote{Such a phenomenon is not exclusive to ``our art'', but (it
seems to me) it appears in every act of discovery, at the very least when such an act
happens at the level of intellectual reckoning.} constitute the most powerful tools of
discovery - or rather, they are
not tools, but they are the very eyes of the researcher who passionately strives to
understand the nature of mathematical things. 

Thus, the fertile viewpoint provides us with an ``eye'' which at once helps us 
\textbf{discover}, and helps us \textbf{recognize the unity} of 
the multiplicity of what is discovered. And such unity is truly the very life and breath
which connects and animates these discoveries. 

But just as the word itself suggests, a ``viewpoint'' by itself remains fragmentary. 
It reveals to us \textbf{one of the aspects}
of a scenery or panorama, among a multiplicity of others which are equally valuable,
equally ``real''. It is when complementary viewpoints of a common reality 
are conjugated, that is, when our ``eyes'' are multiplied, that the gaze is able to
penetrate further ahead in the reckoning of things. 
The richer and more complex the reality which we desire to know, the more important it is
to be equipped with several ``eyes''\footnote{Every viewpoint leads to the development of
a language which is best suited to expressing it. Having several ``eyes'' or several
``viewpoints'' to apprehend a situation, also means (at least in mathematics) having 
several different languages to tackle the situation.} in order to apprehend it in all its ampleness and
subtlety. 

By virtue of our innate ability to grasp the ``multiple'' as the \textbf{One},
it also happens, sometimes, that a sheaf of viewpoints 
converging to a unique and vast scenery, gives rise to a novel thing; a thing which
transcends each of the partial perspectives, in the same way that a living being
transcends\marginpar{p. P17}
each of its limbs and organs. This new thing could be called a \textbf{vision}.
The vision unites the known viewpoints which constitute it, while also revealing to us
other viewpoints that were previously ignored, just as the fertile viewpoint makes us
discover and apprehend, as part of the same Whole, a multiplicity of new questions,
notions, statements. 

To say this in another way: the vision is to the viewpoints, 
from which it seems to arise and which it unites,
as the clear and warm daylight is to the various components of the solar spectrum.
A vast and profound vision is like an inexhaustible source, made to inspire and guide the
work, not only of the one within whom the vision was once conceived and who made himself
its servant, but that of generations, fascinated perhaps (as he first was) 
by these distant horizons which it lets us glimpse\ldots

\section{The ``great idea'' - or the trees and the forest}

The so-called 
``productive'' period
of my mathematical life, meaning
it was marked by proper publications, 
ran between 1950 and 1969, 20 years that is. 
And for 25 years, between 1945 (when I was seventeen)
and 1969 (when I was in my forty-second year), I invested nearly the totality of my energy
in mathematical research. 
An excessive investment certainly. 
This cost me a long period of spiritual stagnation, an incremental ``thickening'', to
which I will be coming back multiple times in the pages of \rec.
Yet, within the limited scope of a purely intellectual activity, 
and through the burgeoning and maturation of a vision restricted to the world of
mathematical things, these were years of intense creativity. 

During this long period of my life,
the near totality of my time and energy were devoted to what might be called 
``\textbf{travail sur pi\`eces}'': a minute process of  
shaping, assembling, and honing, required for the construction from start to finish of 
houses that an inner voice (or an inner demon\ldots) called for me to build, following a
blueprint that it whispered to me at every step of the way. Occupied by the tasks of the
``trade'': such as those of a sculptor, bricklayer, carpenter, even plumber, woodworker, 
cabinet-maker - I rarely stopped to write down, even in rough sketches, the master-plan,
which was invisible to all (as it only appeared later\ldots) except to me, who over the
course of the days, months, and years, guided my hand with the certainty of a sleepwalker.
\footnote{The metaphor of the ``sleepwalker'' was inspired by the title of the wonderful
book ``the sleepwalkers'' by Koestler (Calman L\'evy), presenting an ``Essay on the
history of the conceptions of the universe'', starting from the origins of scientific
thought, all the way to Newton. One of the facets of this history which struck Koestler,
and which he highlights is the extent to which, often, the path from a given 
state of our understanding of the world, to some other state which (logically and with
hindsight) seems very close, sometimes takes the most astounding detours, which appear to
defy reason; and to think that yet, despite those thousand detours that could conceivably
have lost them forever, and with the ``certainty of sleepwalkers'', men who have gone on
the quest for the ``keys'' of the Universe find, as if unintentionally and without even
realizing it, other ``keys'' that they would never have thought of, and which nonetheless
appear to be ``the right ones''. 

From what I have observed all around me, at the level of mathematical discovery, these
extraordinary detours in the path towards discovery are the actuality of some 
high-caliber researchers, but not of all. This could be due to the fact that 
for the past two or three centuries, research in the natural sciences, and even more so in 
mathematics, has freed itself from the religious presupposition 
or metaphysical imperatives, pertaining to a given culture or era, 
which have been strong barriers to the deployment (for better or for worse) of a
``scientific'' understanding of the Universe.
It is nonetheless true that some of the most fundamental and evident ideas and notions in
mathematics (such as the notions of displacement, group, the number $0$, symbolic
arithmetic, the coordinates of a point in space, the notion of a set, or that of
topological ``shape'' without even mentioning negative numbers and complex numbers) took
millennia without making an appearance. These oversights are 
signs of this ingrained ``block'', deeply embedded in the psyche, against the
conceptualization of entirely novel ideas, even in the cases where these are of a
childlike simplicity and seem to impose themselves with the strength of evidence, over the
course of generations or even millennia\ldots

Returning to my own work, I have the impression that within it, the ``mess-ups''
(perhaps more frequent than in the work of most of my colleagues) pertain exclusively to
matters of detail, generally spotted quickly by my own hand. These are simple 
``potholes'', of purely ``local'' nature, and with no serious implications concerning
the validity of the examined situation. 
At the same time, at the level of ideas and larger guiding intuitions, my work
is free of any ``mistakes'' as incredible as that might seem.
It is this always reliable certainty in apprehending each moment, if not the eventual
conclusions of an argument (which often remain hidden from sight), at the very least the 
most fertile directions which present themselves to take me straight to what is essential
- it is that certainty which has brought to my mind Koestler's metaphor of the
``sleepwalker''.}\marginpar{p. P18} I have to say that the 
``travail sur pi\`eces'' to which I like to devote a loving care, 
was not at all displeasing to me. Moreover, the mode of mathematical expression which was
professed and practiced by my elders, gave preeminence (to say the least) to the technical
aspect of one's work, and in no way encouraged the ``digressions'' that would have idled on
the ``motivations''; or even those which appeared to bring out of the mist some vision
which perhaps was inspiring, but which, because it failed to be presented in the form of
tangible constructions in wood, stone, or hard cement, 
was likened more\marginpar{p. P19}
to dream fragments, than to the craft of a 
conscientious and diligent artisan. 

At the quantitative level, my work during these intense years of productivity 
manifested itself mostly in the form of about twelve-thousand pages of publications, in the form of
articles, monographs, or seminaries,\footnote{Starting from the 1960's, part of these
publications were written in collaboration with colleagues (mostly J. Dieudonn\'e) and
students.} as well as hundreds, if not thousands, of new notions 
which have entered into the mathematical patrimony, with the very names which I had
given them upon first discovering them.\footnote{The most important of these notions are
reviewed in the Thematic Sketch,
% \todo{ref} 
as well as in the Historical Commentary
% \todo{ref} 
which accompanies it - both of these are included in Volume IV of the Reflections. 
% \todo{ref} 
Some of these names were suggested by friends or students, such as the term ``smooth
morphism'' (J. Dieudonn\'e) or the panoply ``site, stack, gerbe, and band'' in the thesis
of Jean Giraud.} In the history of mathematics, I think I may have been the person who
introduced the largest number of new notions into our science, and at the same time, the
person who was brought, as a consequence, to invent the largest number of new names, with
the intention of expressing these notions with delicacy, and in as suggestive a way as I
could.

These indications give no more than a very rough feel for my work, 
ignoring what truly lies at its heart, its life and vigor. 
As I wrote about earlier, what I have brought best to mathematics are the new 
``\textbf{viewpoints}'' that I have been able to first \textbf{glimpse}, and later patiently
\textbf{excavate} and to some extent develop. 
These new viewpoints, like the notions I just mentioned, inserting themselves in a vast
multiplicity of different situations, are also nearly enumerable. 

The fact remains that
certain viewpoints are more vast than others, 
those which by themselves encompass a multitude of partial viewpoints, within a multitude
of particularly different situations. Such a viewpoint can be called, rightly, a
``\textbf{great idea}''.
Through its internal fecundity, such an idea gives rise to a vast 
progenitor of ideas which themselves inherit its fecundity, although most (if not all)
are of a lesser scope than the mother idea. 

The task of \textbf{expressing} a great idea, 
of ``saying'' it, can be as delicate as its conception and slow gestation within the
person who conceived of it. To better put it, the 
laborious process of patiently excavating the idea, day after day, from the veils of mist
which surround it at birth, to slowly succeed
\marginpar{p. P20}
in giving it a tangible form, as a painting
which grows richer, firmer, and finer over the course of weeks, months, years.
To simply \textbf{name} the idea by some striking formula, or by more or less technical
keywords, can fit in the span of a few lines, or a few pages - 
but rare are those who, without knowing it beforehand, will know how to listen to this
``name'' and recognize its face. 
And when the idea reaches full maturity, a hundred pages might suffice to express it to
the full satisfaction of the worker within whom it was born - just as it can happen that
ten-thousand pages, 
crafted and weighted at length, may not suffice.\footnote{Upon leaving the mathematical
world in 1970, the totality of my publications (including several works in collaboration)
on the central theme of schemes amounted to some $10,000$ pages. Yet this only represented
a modest fragment of the vast program which I glimpsed ahead, concerning schemes.
This program was abandoned sine die d\`es my departure despite the fact that almost
everything which had already been published for the use of everyone had already entered
the common patrimony of results and notions considered ``well known''.

The part of my program concerned with schemes together with its extensions and
ramifications, which I had completed at the time of my departure, alone represented 
the most vast foundational work ever accomplished in the history of mathematics,
and surely one of the most vast in the history of Science as well.}

In this case, as in others, 
among those who gained awareness of the project, 
which presented the idea in full bloom,
many people could see the vigorous trees, and used them 
(some to climb, others used them as lumber, some as firewood, \ldots)
yet few could see the forest from the trees\ldots

\section{The vision - or twelve themes for a harmony}

Perhaps one might say that the
``great idea'' is the viewpoint which, not only itself reveals new and fertile ideas, 
but that which introduces a novel and vast \textbf{theme} which embodies it.
Every science, when understood not as a tool to gain power or domination, but rather as a
journey towards the understanding of our species through the ages,
is nothing but the harmony, more or less vast and more or less rich from one era to the
next, which deploys itself through generations and centuries, through the delicate
counterbalancing of all the themes which appeared one by one, as if summoned from the
void, to integrate into the science and be interlaced within it.

\marginpar{p. P21}Among the multiple viewpoints, which I have unearthed in mathematics,
there are \textbf{twelve}, in hindsight, which I would call ``great ideas''.\footnote{Here
are, for the mathematically inclined reader, the twelve key ideas, or ``ma\^itre th\`emes''
(in chronological order of appearance):
\begin{enumerate}
\item Topological tensor products and nuclear spaces.
\item ``Continuous'' and ``discrete''  duality (derived categories, ``six operations'').
\item Riemann-Roch-Grothendieck yoga ($K$-theory, relationship with intersection theory).
\item Schemes.
\item Topos.
\item \'Etale and $\ell$-adic cohomology.
\item Motives, Motivic Galois Groups (Grothendieck $\tp$-categories).
\item Crystals, crystalline cohomology, yoga of ``de Rham coefficients'', ``Hodge
coefficients'', \ldots
\item ``Topological algebra'': $\infty$-stacks, derivators;
cohomological formalism of topoi, serving as inspiration for a new conception of
homotopical algebra.
\item Tame topology.
\item The yoga of nonabelian algebraic geometry, Galois-Teichm\"uller Theory.
\item ``Schematic'' or ``arithmetic'' point of view for regular polyhedra and regular
configurations of all kinds.
\end{enumerate}
Apart from the first theme, a large portion of which appeared in my thesis of 1953 and
was further developed in the period in which I worked in functional analysis between 1950
and 1955, these themes were discovered and developed during my time working as a geometer,
starting in 1955.}
To see my mathematical work, to ``feel'' it, is to
see and ``feel'' 
at least some of these ideas, and the great themes that they introduce and which lie at the
heart of my work.

By the nature of things, some of these ideas are ``greater'' than others (which in turn
are ``smaller''!). 
In other words, among these novel themes, some are vaster than others, and some plunge
deeper into the heart of the mystery of mathematical things.\footnote{Among these
themes, the most \textbf{vast} in its \textbf{reach} 
seems to me to be that of \textbf{topoi}, in that it provides the idea of a synthesis of algebraic
geometry, topology, and arithmetic. The most vast in terms of the \textbf{extent
of the developments} to which it has given birth thus far, is the theme of schemes.
(See 
% \todo{cite}
% de b. de p. (*) page 20.
the note ref.) It is that theme which provides the framework 
``par excellence''
in which eight of the other listed themes are developed (namely all but themes 1,5,10),
and it also provides the central notion for a fundamental renewal of algebraic geometry,
and of the algebraic geometric language. 

On the other extreme, the first and last of these twelve themes seem to me to be of more
modest dimensions than the others.
Yet, concerning the last, introducing a new viewpoint 
on the ancient theme of regular polyhedra and regular configurations,
I doubt that the life of a mathematician
who would devote themselves exclusively to this would suffice to exhaust it.
As for the first of all of these themes, topological tensor products, it played a role of
a new tool ready for use, rather than a source of inspiration for later developments.
The fact remains that I still, to this day, receive sporadic echoes of more or less
recent solutions 
(20 or 30 years later) 
to some of the questions which I had left open.

The deepest (to my eyes) of these twelve themes, are the notion of \textbf{motives}, 
and the closely related \textbf{yoga of nonabelian algebraic geometry, and
Galois-Teichm\"uller theory}.

From the viewpoint of \textbf{powerful tools}, perfectly polished under my care, and
commonly used in various ``cutting edge fields'' in research during the past two decades,
the themes of ``\textbf{schemes}'' and ``\textbf{\'etale and $\ell$-adic cohomology''}
have been the most noteworthy.
For a well-informed mathematician, I believe there is no doubt that the
schematic tools, as well as the theory of $\ell$-adic cohomology which arises from it,
are among some of the most important contemporary acquisitions, having come to nourish and
renew our science over the course of the previous generations.} There are three \marginpar{p. P22} 
(and not the least of them) which, having appeared only after I had left the mathematical world,
remain at the embryonic stage; they ``officially'' do not even exist, as no proper publication can
be pointed to as a birth certificate.\footnote{The only ``semi-official'' text where these
themes are briefly sketched, is the Esquisse d'un Programme written in January 1984, in
the context of a detachment request to the CNRS. This text (which is also mentioned in the
Introduction 3, ``compass and luggage'')
% \todo{ref}
will be included in principle in volume 4 of the Reflexions.} Among the nine themes that had appeared prior to my departure, the latest three, which I had left growing in full swing, remain today at a stage of infancy, due to a lack (following my departure) of caring hands that would provide for these ``orphans" left behind in a hostile world\footnote{Right after my departure, following the unceremonious burial of these three orphans, two of them were exhumed with great fanfare and with no mention of the original craftsman, one in 1981 and the other (given the fussless success of the operation) starting the following year.}. As for the remaining six themes, which reached full maturity during the two decades preceding my departure, one could say (modulo one or two caveats\footnote{The ``one or two caveats" concerns mostly the yoga of grothendieckian duality (derived categories and six functor formalism), and that of topoi. These (among other things) will be discussed in detail in parts II and IV of R�coltes et Semailles (The Funeral (1) and (3)).)}) 
% \todo{ref}
that they had by then already entered the mathematical patrimony: among geometers in particular, ``everybody" nowadays invokes them routinely and without even noticing (the way Monsieur Jourdan wrote prose). They are part of the air we breathe when ``doing geometry", or when doing arithmetic, algebra, or analysis of a ``geometric flavor".

\marginpar{p. P23} The twelve main themes of my work are far from isolated from one another. Rather, they together constitute in my eyes a \textbf{unity} of spirit and aim, manifesting itself in the form of a common and persistent backdrop throughout both my ``written" and ``unwritten" work. And as I wrote these lines, I seemed to once again encounter the same key - a calling of sorts! - which had permeated the three years I spent working "free of charge", intensely and alone, at a time when I did not even bother to inquire as to whether there existed mathematicians other than myself in the outside world, being so completely entranced by the fascination I felt for that which was calling me...

This unity is not only that of a common style among works produced by the same hands. The aforementioned themes are interlinked in countless ways, both subtle and obvious, akin to how the different themes of a singular and vast counterpoint are interlinked, intertwined in their deployment, while each remains clearly recognizable. They come together in a harmony that carries them forward and breathes meaning into each in turn, in the form of a movement and plenitude to which all others contribute. Each of the partial themes seems to be issued from this vaster harmony and to be reborn within it moment after moment, rather than the harmony appearing as the ``sum" or ``result" of pre-existing constituent themes. And to tell the truth, I cannot shake away the feeling (surely unreasonable...) that it is in a way this harmony - hitherto invisible, but which surely already ``existed" in the hidden bosom of things yet to be born - which engendered one by one the themes which were only to take their full meaning through it, and that its muted and pressing voice was what was already calling me during these years of intense solitude, at the outset of my teenage years...

The fact remains that the twelve ``ma\^itre-th\`emes" of my work all contribute to a common symphony, as if by some secret predestination - or, to re-invoke a different picture, they incarnate as many different ``viewpoints" that come together to constitue a single and vast \textbf{vision}.

This vision only began to emerge from the mist, unveiling some of its recognizable contours, around the years 1957-58 - which were years of intense gestation\footnote{The year 1957 was that during which I was led to unearth the \textbf{Riemann-Roch} theme (Grothendieck version) - which turned me almost overnight into a ``superstar". This was also the year of my mother's passing, and through it that of an important break in my life. It was one of the most intensely creative years of my life, mathematically and otherwise. That year was the first time I felt that I had more or less ``gone round" what constitutes mathematical work, and that it might be time for me to begin devoting my time to something else. A need for self-actualization had visibly surfaced in me, for the first time in my life. I considered becoming a writer, and stopped all mathematical activity for several months. I eventually decided that I should at least produce a write-up of the mathematical projects which I was already working on, something that I thought would only take a few months, or up to a year at the longest...

The time was not yet ripe, it seems, for the ``grand saut". The fact is that my mathematical work, once taken back up, completely re-absorbed me, and did not let me go for another twelve years!

The year that followed this interlude (1958) was perhaps the most prolific in my life as a mathematician. It is during that year that the burgeoning of the two central themes of the novel geometry took place, starting with the launch ``en force" of the \textbf{theory of schemes} (which was the topic of my talk at the international congress of mathematicians in Edinburgh the summer of that same year), and followed by the formulation of the notion of ``\textbf{site}", a provisory technical prototype for the crucial notion of \textbf{topos}. With nearly 30 years of hindsight, I can now say that this was the year during which the novel geometry was born, in the wake of the two key tools of this geometry: schemes (which are a metamorphosis of the prior notion of ``algebraic variety"), and topoi (which are an even deeper metamorphosis of the notion of space).}. As strange as it may seem, this vision felt so close, so \marginpar{p P24} ``obvious", that up until just a year ago\footnote{I contemplate naming this vision for the first time in the reflection of December $4^{th}$ 1984, in the sub-note (n$^o 136_1$) to the note ``Yin the servant (2) - or the generosity" (ReS III, page 637).}
%\todo{ref}
I had not thought of giving it a name (even though one of my passions has always been to constantly  \textbf{name} the things that I discovered, as a first method of apprehension...). It is true that I cannot think of a particular moment during which I experienced the appearance of this vision, or one which I could recognize as such in hindsight. A novel vision is something so vast that its appearance really cannot be attributed to a particular moment. Rather, the vision has to penetrate and take possession - over the course of many years, if not generations - of the individual or group whose activity it is to contemplate and scrutinize; as if new eyes had to laboriously come into being, behind the familiar eyes which they have come to replace. The vision is also too vast for there to be any chance to ``capture it", the way we may capture the first notion to come our way. There is therefore nothing surprising, after all, about the fact that the very thought of naming something this vaste, yet so close and diffuse, only came with hindsight, once the vision had reached full maturity.

To tell the truth, until two years ago, my relationship with mathematics was limited (with the exception of teaching) to the act of \textbf{doing} it - following an impulse that ceaselessly moved me \textbf{forward}, into an ``unknown" that \marginpar{p P25}continually attracted me. The idea never occurred to me to stop in my stride and to interrogate myself, to turn around even for an instant and perhaps to see the outline of a path taken, or to situate past work (be it to situate it \textbf{in my life}, as something to which profound links long ignored continue binding me; or to situate it in the collective adventure that is \textbf{``mathematics"}.)

Stranger thing even, in order to bring myself to ``lay out" and to reckon with this half-forgotten work, or to even consider giving a \textbf{name} to the vision that lay at its heart, I would have had to suddenly confront the reality of a Funeral of gigantic proportions: the funeral, through silence and derision, both of the vision and of the craftsman in whom it was borne...

\section{Form and structure - or the way of things}

Without planning to, I ended up writing this ``foreword" as a sort of presentation ``en r\`egle" of my work, intended (mostly) to the non-mathematician reader. Having come too far to turn back, it is time to finish ``the introductions"! I would like to attempt to say a few words about the \textbf{substance} of these marvelous ``great ideas" (or ``ma\^itre-the\`emes") which I have mentioned in the above pages, and about the nature of this so-called ``vision" within which these key ideas supposedly come together. In keeping with the non-technical nature of this foreword, I would undoubtedly only be able to convey an extremely vague picture (provided  indeed that anything at all can even ``go through"...)\footnote{A picture may remain ``vague", but it nonetheless can have the potential of being faithful, and of successfully restoring the essence of what is being considered (in this case, my work). Inversely, an image can be sharp yet distorted, and it can moreover include the accessory while entirely omitting the essential. Thus, if you ``connect" to what I have to say about my work (in which case surely some part of the image will ``go through" successfully), you can flatter yourself to have better understood what is the essence of my work than perhaps any of my wise colleagues!}. 

Traditionally, we draw a distinction between three types or ``quality", or ``aspects" of things in the Universe which lend themselves to mathematical thought: they are \textbf{number}\footnote{By ``number" here I mean the ``natural numbers" 0,1,2,3, etc or (at most) numbers (such as the rational numbers) which can be expressed from them using elementary operations. These numbers do not lend themselves, unlike the ``real numbers", to the measure of quantities subject to continuous variations, such as the distance between two points varying along a line, a plane or in space.}, \textbf{size}, and \textbf{shape}. They could also \marginpar{p. P26} be called respectively the \textbf{``arithmetic"} aspect, \textbf{``metric"} (or analytic) aspect, and the \textbf{``geometric"} aspect of things. In most situations under study in mathematics, these three aspects are present simultaneously and are tightly interacting. However, very often, there is a clear prevalence of one of the three. It seems to me that for most mathematicians, it is rather clear (to those who know them, or who are aware of their work) which is their base temperament, whether they are ``arithmeticians", ``analysts" or ``geometers" - and this is so even for those who have many strings to their bow and who have done work in all registers and diapasons imaginable.

My first and solitary reflections, concerning the theories of measure and integration, are situated without any ambiguity in the ``size", or ``analysis" category. The same goes for the first of the new themes which I have introduced in mathematics (which seems to me to be of lesser proportions than the eleven others). The fact that I have entered the world of mathematics ``through" analysis seems due, not to my particular temperament, but rather to what may be called ``fortuitous circumstances": it was because the most blatant gap in my secondary and university education, to my mind enamored with generality and rigor, happened to concern the ``metric" or ``analytic" aspect of things.

The year 1955 marked a crucial turn in my mathematical work: that of the transition from ``analysis" to ``geometry". I still remember the associated striking impression (entirely subjective of course), as if leaving arid and surly steppes to suddenly find myself in a ``promised land" of sorts, full of lush riches, multiplying themselves ad infinitum wherever I cared to look, to reap or to investigate... And this impression of overwhelming wealth, beyond all measure\footnote{I chose to use the words ``accablant, au del\`a de toute mesure" (in the French version) to express as well as I could the German expression ``\"uberw\"altigend", and its English counterpart ``overwhelming".  In the preceding sentence, the (inadequate) expression ``striking impression" is also to be understood in this sense, namely: when the impressions and sentiments evoked in us after confronting a splendor, a grandeur, or a beauty out of the ordinary suddenly submerge us to such an extent that any attempt at expressing that which we feel seems doomed from the start.}, only confirmed itself and deepened over the course of the years, up to this very day.

\marginpar{p. P27} This is to say that if there is one thing in mathematics which fascinates me more than any other (and undoubtedly always has), it is neither ``number" nor ``size", but invariably \textbf{shape}. And among the thousand and one faces under which shape chooses to reveal itself to us, that which has fascinated me more than any other and continues to do so is the \textbf{structure} hidden in mathematical things.

The structure of a thing is not something which it is possible for us to ``invent". We can only patiently unravel it, humbly get to know it and \textbf{``discover"} it. If there is any ingenuity involved in this line of work, and if we sometimes take up the role of a blacksmith or that of a tireless builder, it is never to ``model" or to ``construct" ``structures" - they didn't have to wait for us to exist, and to be precisely what they are! Rather, it is to \textbf{express}, as faithfully as we can, those things which we are in the process of scanning and discovering, the structure that is reluctant to surrender and which we attempt to grasp, fumblingly, and through a perhaps fledgling language. Thus are we constantly led to \textbf{``invent" the language} best suited to ever more finely express the intimate structure of mathematical things, and to ``construct" by means of this language, slowly and from the ground up, the ``theories" that are supposed to report what has been apprehended and seen. Underlying this process is a continual, uninterrupted back-and-forth motion between the \textbf{apprehension} of things and the \textbf{expression} of that which has been apprehended, through a language that grows finer and is created anew over time, under the constant pressure of immediate needs.

As the reader will have no doubt guessed, these ``theories", ``constructed from the ground up", are but the \textbf{``beautiful houses"} which were discussed earlier: the houses which we inherit from our predecessors and those which we are led to build with our own hands, as we listen and follow the calling of things. Having mentioned earlier the ``ingenuity" (or imagination) of the builder or the blacksmith, I should add that what lies at its heart is not the arrogance of he who asserts ``I want this, and not that!" and who decides according to his whim; it is a pitiful architect who sets off with all of his plans fixed in his mind, before having even seen and felt the terrain, surveying its requirements and possibilities. What characterizes the value of the ingenuity and imagination of a researcher is the \textbf{quality of his attention} as he listens to the voice of things - for the things of the Universe never tire of talking about themselves and to reveal themselves to he who cares to listen. Thus, the most beautiful house, \marginpar{p. P28} that in which the love of the builder is most evident, is not that which is larger or higher than the others. Rather, a house is beautiful if it faithfully reflects the structure and beauty hidden in things.

\section{The novel geometry - or the marriage of number and size}

But I digress once more - my intention was to speak of ma\^itre-th\`emes, united under a common vision-m\`ere, akin to rivers returning to the Sea from which they were born...

This vast and unifying vision can be described as a \textbf{novel geometry}. It appears to have been what Kronecker had dreamed of, in the last century\footnote{I only know about ``Kronecker's dream" from hearsay, from when somebody (it might have been John Tate) told me that I was realizing that very dream. In the education that I received from my elders, historical references were extremely rare. I mostly learned through direct communication with other mathematicians, both orally and in the form of letter correspondences, rather than by reading authored texts - be they ancient or contemporary. The main, and perhaps the only external inspiration that preceded the sudden and energetic beginnings of the theory of schemes in 1958, was Serre's famous article FAC (``Faisceaux Alg\'ebriques Coh\'erents"), which appeared a few years earlier. Apart from that article, much of my inspiration in further developing the theory came on its own, and renewed itself over the course of the years, through mere considerations of internal simplicity and coherence, in an effort to recover in this new context what was ``well known" in classical algebraic geometry (material which I absorbed as it took a new form in my hands), and to follow what this ``well known" material led me to anticipate.}. Yet reality (which a bold dream can sometime lead us to anticipate or glimpse at, thereby encouraging us to pursue its discovery...) always surpasses in wealth and resonance even the deepest and most daring of dreams. Surely, concerning more than one arc of this novel geometry (if not all of them), nobody, even up to the very day preceding its appearance, could have seen it coming - the builder no more than anybody else. 

One may say that ``number" is capable of describing the structure of ``discontinuous", or \textbf{``discrete"} aggregates: systems - often finite - composed of ``elements" or ``objects" isolated from one another, so to speak, without a notion of ``continuous motion" from one object to the next. ``Size", on the other hand, is the quality par excellence when it comes to \textbf{``continuous variation"}; as such, it is capable of describing structures and phenomena of a continuous nature: motion, spaces, ``varieties" of all kinds, force fields, etc... Thus, arithmetic appears (roughly) to be the \textbf{science of discrete structures}, and analysis the \textbf{science of continuous structures}.

\marginpar{p. P29}As for geometry, for the more than two thousand years that it has existed as a science (in the modern sense of the term), it sits ``halfway" between these two types of structures, the ``discrete" and the ``continuous"\footnote{Admittedly, it was traditionally the ``continuous" aspects which were the focus of the geometer, while properties of a ``dicrete" nature, and notably arithmetic and combinatorial properties, were glossed over or hastily treated. It is with awe that I discovered, about a decade ago, the wealth of the combinatorial theory of the icosahedron, a theme which is not even scratched (and which was probably not even seen) in Klein's classical treatise on the icosahedron. I perceive another striking sign of this negligence (twice millennial) of geometers regarding discrete structures which spontaneously appear in geometry: it it the fact that the notion of group (notably group of symmetries) has only appeared in the last century, and that it furthermore was first introduced (by Evariste Galois) in a context which was not yet considered as pertaining to ``geometry". It must be said that even nowadays, many algebraists have yet to understand that Galois theory is really, in essence, a \textbf{``geometric" vision}, coming to renew our understanding of so-called ``arithmetic" phenomena...}. Indeed, for a long time, there wasn't really a \textbf{``divorce"} between \textbf{two} geometries of a fundamentally different kind, the one discrete and the other continuous. Rather, there were two different viewpoints involved in the investigation of the \textbf{same} geometric figures: the first focused on the study of ``discrete" properties (notably, properties of a numerical or combinatoric nature), and the second treating of ``continuous" properties (such as that of position in an ambient space, or that of ``size" measured in terms of the pairwise distance between points, etc...).

It is only at the turn of the last century that a divorce emerged, marked by the apparition and development of what may be called \textbf{``(abstract) algebraic geometry"}. Roughly speaking, this consisted in introducing, for every prime number $p$, a theory of (algebraic) geometry ``in characteristic $p$", modeled after the (continuous) analog of (algebraic) geometry developed over the course of the preceding centuries, yet in a context which appeared to be fundamentally ``discontinuous", ``discrete". These new geometric objects grew in importance since the beginning of the century, notably due to their close connections to arithmetic, the science par excellence of discrete structures. This seems to have been one of the leading ideas in the work of \textbf{Andr\'e Weil}\footnote{Andr\'e Weil, French mathematician who emigrated to the United States, is one of the ``founding members" of the ``Bourbaki group" - which will be discussed further in the first part of R\'ecoltes et Semailles (as will Weil himself, occasionally).}, perhaps even his principal id\'ee-force (although it remained relatively tacit in his written work, as it should) - namely, that ``the" theory of (algebraic) geometry, and particularly the \marginpar{p. P30}``discrete" geometries associated to the various prime numbers, were to provide the key for a vast renewal of the theory of arithmetic. It is in this spirit that the celebrated \textbf{``Weil conjectures"} were formulated in 1949. These were utterly spectacular conjectures which gave a glimpse, for these novel ``varieties" (or ``spaces") of a discrete nature, of the possibility of certain kinds of constructions and arguments\footnote{(Intended for the mathematical reader) The ``construction and arguments" mentioned here are those linked to the cohomology theories of smooth or complex manifolds, notably those leading to the Lefschetz fixed point formula and to Hodge theory.} which up until then seemed only conceivable in the context of usual ``spaces" in the sense understood by analysts - namely, so-called ``topological" spaces (where the notion of continuous variation takes place). 

Before all else, the novel geometry can be viewed as a synthesis between these two worlds, hitherto adjacent and tightly connected, but nonetheless separated: the \textbf{``arithmetic" world}, in which (so-called) ``spaces" with no notion of continuity live, and the \textbf{world of continuous size}, where ``spaces" in the proper sense of the term live and are accessible using the tools of the analyst, who (for this very reason) considers them worthy to dwell in the mathematical city. \textbf{Under the novel vision, these two hitherto separated worlds are merged into one.}

The first embryo of this vision of ``arithmetic geometry" (the term I hereby suggest for this novel geometry) can be found in Weil's conjectures. In the development of some of my principal themes\footnote{I am here referring to the four ``median" themes (n$^o 5$ through $8$), namely those of \textbf{topos}, \textbf{\'etale and $l$-adic cohomology}, \textbf{motives}, and (to a lesser extent) \textbf{crystals}. I unearthed these themes one by one between 1958 and 1966.}, these conjectures served as my main source of inspiration, throughout the years 1958-1969. Before me, \textbf{Oscar Zariski} on the one hand, and later \textbf{Jean-Pierre Serre} on the other, had developed some ``topological" methods tailored for the unruly spaces of ``abstract" algebraic geometry, inspired by the standard methods previously used for the ``nice spaces" everybody knows\footnote{(Intended for the mathematical reader) Zariski's main contributions in this sense seems to me to have been his introduction of the ``Zariski topology" (which was later an essential tool for Serre in FAC), as well as his ``connectedness principle", what he called his ``theory of holomorphic functions" - which became in his hands the theory of formal schemes, and his ``comparison theorems" between the formal and the algebraic (with, as a second source of inspiration, Serre's fundamental GAGA article). As for Serre's contribution alluded to in the above text, it refers of course, before all else, to his insertion of the viewpoint of sheaf theory into abstract algebraic geometry (a viewpoint which was introduced by \textbf{Jean Leray} about a dozen years earlier in an entirely different context), which takes place in his other fundamental article cited earlier: FAC (``Faisceaux alg\'ebriques coh\'erents").}.

\marginpar{p. P31} Of course, their ideas played an important role during my first steps towards the development of the theory of arithmetic geometry; this was the case more so as starting points and as \textbf{tools} (which I had to more or less entirely remodel to cater to the needs of a much more general context), than as a source of inspiration that would have nourished my dreams and projects over the course of months and years. In any case, it was clear from the get-go that these tools, even remodeled, were far from sufficient in light of what was required to take even the very first steps in the direction of Weil's fantastic conjectures.

\section{The magical fan - or the innocence}

The two crucial ``id\'ees-forces" in starting up and developing the novel geometry were those of \textbf{schemes} and \textbf{topoi}. They appeared at roughly the same time and in tight symbiosis\footnote{I speak of these beginnings, which took place in 1958, in the footnote on page 23. The notion of site, or \textbf{``Grothendieck topology"} (provisory prototype of the notion of topos) appeared in the immediate succession of the notion of scheme. It is that notion which in turn led to the new language of ``localization" and ``descent", used at every step of the development of the schematic theme and the schematic tools. The more intrinsic and geometric notion of topos, which remained implicit at first during the succeeding years, was mostly uncovered starting in 1963 with the development of \'etale cohomology, and later slowly imposed itself in my eyes as the more fundamental notion.}, and they acted as one and the same \textbf{motor nerve} in the spectacular expansion of the novel geometry, beginning the very year of the apparition. To complete this tour of my work, all that remains is for me to say a few words about these two ideas.

The notion of scheme is the most natural, the most ``obvious", when it comes to englobe into a unique notion the infinite series of notions of (algebraic) ``variety"\marginpar{p. P32} which we manipulated previously (one such notion for \textbf{each} prime number\footnote{This series should also include the case $p = \infty$ corresponding to algebraic varieties ``in zero characteristic".}...). In fact, a unique ``scheme" (or ``variety" nouveau style) gives rise to a well-defined ``(algebraic) variety in characteristic $p$" for every prime number $p$. The collection of these various varieties in different characteristics can then be viewed as a sort of ``(infinite) fan of varieties" (one for each characteristic). The ``scheme" is precisely this magical fan, which links to one another several different ``branches", its ``avatars" or ``incarnations" in every possible characteristic. Through this very fact, it provides an efficient ``crossing point" between different ``varieties", which had hitherto appeared as more or less isolated, separated from one another. They thus became englobed in a common ``geometry" through which they were linked - what may be called \textbf{schematic geometry}, first sketch of the ``arithmetic geometry" into which it was to grow in the following years.

The very idea of a scheme is of a childlike simplicity - so simple, so humble, that no one before me had even thought to look so low. So ``silly", in fact, that for many years and despite the evidence pointing to the contrary, many of my erudite colleagues found this whole affair ``not serious"! It actually took me months of intense and solitary work, to convince myself that it indeed ``worked" just fine - that this new language, so silly, that I had the incorrigible naivety to persist in testing, was after all adequate in capturing in a new light and with greater finesse, and in a common framework, some of the very first geometric intuitions attached to the prior ``geometries in characteristic $p$". It was the kind of exercise, considered mindless and doomed in advance by any ``well informed" person, which I was without a doubt the only one, among all of my colleagues and friends, to ever dare attempt, and even (under the impulse of some secret demon...) carry to a successful end despite everybody's expectation!

Rather than letting myself be distracted by the consensus which prevailed around me, regarding what is considered ``serious" and what isn't, I simply \textbf{trusted}, as I had before, the humble voice of things, and followed that within me which knew how to listen. The reward was immediate, and beyond all expectations. Within the span of a few months, without even ``meaning to", I had discovered powerful and unexpected new tools. They allowed me\marginpar{p. P33} to not only recover old results, reputedly difficult, in a more telling light, but also to surpass them, as well as to finally tackle and solve problems in ``geometry in characteristic $p$" which to that day seemed out of reach using the methods known at the time\footnote{The summary of these ``energetic beginnings" to the theory of schemes was the subject of my talk at the International Congress of Mathematicians in Edinburgh, in 1958. The text of this expos\'e appears to me to be one of the best introductions to the schematic viewpoint, as a means to (perhaps) motivate the geometer reading it to gain familiarity with the imposing (later) treatise ``El\'ements de G\'eom\'etrie Alg\'ebrique", which exposes in depth (and without omitting any technical detail) the new foundations and new techniques of algebraic geometry.}.

In our probing of things in the Universe (mathematical or otherwise), we dispose of a crucial rehabilitating power: \textbf{innocence}. By this I mean the original innocence which we have all received at birth and which rests within us, often the target of our scorn and of our deepest fears. It alone unites the humility and audacity which allow us to penetrate into the heart of things, while also allowing these things to penetrate into us and impregnate us with their meaning.

This power does not only come as a privilege for the extraordinarily ``gifted" - (say) with an exceptional intellectual power allowing them to absorb and manipulate, with ease and dexterity, an impressing quantity of known facts, ideas, and techniques. Such gifts are admittedly precious, and susceptible to generate the envy of those (like myself) who were not so gifted at birth, ``beyond all measure".

Yet, it is not those gifts, nor even the most burning of ambitions, accompanied with a relentless will, which allow us to cross the ``invisible and imperious circles" which enclose our Universe. Only innocence can cross them, without noticing or even caring to, during the times where we find ourselves alone and listening to the voice of things, intensely absorbed in child's play...

\section{Topology - or surveying through the mist} 

The innovative idea of ``scheme" is, as we just discussed, that which allows us to bring together the various ``geometries" associated to different prime numbers (i.e. to different ``characteristics"). These geometries nonetheless each remained essentially ``discrete" or ``discontinuous" in nature, in contrast with the traditional geometry inherited from centuries past (going back all the way to Euclid). The new ideas introduced by Zariski and Serre restored to some extent a ``dimension" of continuity\marginpar{p. P34} for these geometries, which was instantly inherited by the ``schematic geometry" which had just appeared, with the aim of uniting them. But we were still a long way from approaching the ``fantastic conjectures" (of Weil). These ``Zariski topologies" were, from this viewpoint, so coarse that it was almost as if we had remained at the stage of ``discrete aggregates". What was visibly missing was some novel principle, one which would allow us to connect these geometric objects (``varieties", or ``schemes") to the usual ``nice" (topological) ``spaces"; those spaces, say, in which ``points" are clearly \textbf{separated} from one another, something which failed in Zariski's unruly spaces, where points had the poor tendency to cluster around one another...

Decidedly, it was the appearance of such a ``novel principle", and nothing else, which would allow for this ``marriage of number and size", of the ``geometry of the discontinuous" with that of the ``continuous", of which Weil's conjectures were a first premonition.

The notion of \textbf{``space"} is undoubtedly one of the oldest in mathematics. It is so fundamental to our ``geometric" apprehension of the world around us, that is has remained more or less tacit for over two millennia. Only in the last century did this notion finally begin to progressively detach itself from the tyrannical stranglehold of immediate perception (that of a unique ``space" surrounding us), and from its traditional (``euclidian") theorization, in an effort to acquire an autonomy and dynamic of its own. Nowadays, it is one of the most universal and most commonly used notions in mathematics, one with which every mathematician is almost surely familiar. It is moreover a remarkably multifaceted notion, one of a thousand and one faces, depending on the structure with which we choose to equip these spaces, from the richest of all (such as the venerable ``euclidian" structures, or the ``affine" or ``projective" structures, or the ``algebraic" structures of the eponymous ``varieties"", which generalize and relax the latter) to the most bare: those where all forms of ``quantitative` information seems to have been loss forever, and where alone subsist the quintessence of the qualitative notion of \textbf{``proximity"} or that of \textbf{``limit"}\footnote{When speaking of the notion of ``limit", I mean mostly that of ``passing to the limit", rather than the notion (more familiar to the non-mathematical reader) of ``boundary".}, as well as the most elusive version of the notion of (so called ``topological") \textbf{shape}. The most bare of all such notions, which to this day\marginpar{p. P35} had served as a sort of vast and common conceptual bosom englobing all the others, was that of \textbf{topological space}. The study of such spaces constitutes one of the most fascinating and vigorous branches of geometry: that of \textbf{topology}.

As elusive as this ``purely qualitative" structure incarnated by a (so-called ``topological") ``space" might seem at first sight, in the complete absence of data of a quantitative nature allowing us to hold on to some familiar intuition of ``largeness" or ``smallness" (such as the distance between two points), we have nonetheless succeeded, over the course of the previous century, to finely capture these spaces in the tight and supple meshes of a carefully ``tailor-made" language. Better yet, we have invented and created from the ground up all sorts of ``meters" and ``metersticks" that let us, against all odds, attached ``measurements" of sorts (called ``topological invariants") to these tentacular ``spaces" that previously seemed to escape all attempts to measure them - akin to an elusive mist. It is true that most of these invariants, including the most essential, are of a subtler nature than the simple notions of ``number" or ``size" - they are themselves more or less delicate mathematical structures, attached to the space under consideration (by means of more or less sophisticated constructions). One of the oldest and most crucial of these invariants, introduced already in the last century (by the mathematician \textbf{Betti}), consists of various so-called ``cohomology groups" (or ``spaces") associated to a space\footnote{What Betti actually introduced were so-called \textbf{homological} invariants. \textbf{Cohomology} constitutes a more or less equivalent version of those, a ``dual" notion that was only introduced much later. The latter notion gained prevalence over the initial one of ``homology" due (doubtlessly) to Jean Leray's introduction of the viewpoint of sheaves, about which I write in more detail later. At the technical level, one could say that a large portion of my work as a geometer consisted in unearthing, and in developing to various extents, the cohomology theories that were missing for all kinds of spaces and varieties, particularly for ``algebraic varieties" and for schemes. Along the way, I was brought to reinterpret the traditional homological invariants in cohomological terms, and thereby to reintroduce them in an entirely new light.

Many other ``topological invariants" were introduced by topologists, in their attempts to capture various kinds of properties of topological spaces. Apart from the ``dimension" of a space and (co)homological invariants, the first invariants to have been introduced were the ``homotopy groups". I introduced another one in 1957, the (so-called ``Grothendieck") group $K(X)$, which immediately took off successfully, and whose importance (in topology as well as in arithmetic) continued to grow over time. 

The introduction of a host of new invariants, of a nature more subtle than the currently known and used invariants, yet which I sense are fundamental, is part of my program of ``tame topology" (\textit{``topologie mod\'er\'ee" in French}) (of which a brief sketch can be found in the ``Sketch of a Program" (``Esquisse d'un Programme" in French) to appear in volume 4 of the Reflections). This program is based on the notion of ``tame theory", or ``tame space", which constitutes, in the same way as the topos, a (second) ``metamorphosis of the notion of space". This one is much more obvious (in my eyes) and much less deep than the latter. I expect that its immediate implications in ``classical" topology will nonetheless be felt much more acutely, and that it will fundamentally transform the ``profession" of the topological geometer, via a profound transformation of the contextual framework in which his work takes place (as was the case in algebraic geometry with the introduction of the viewpoint of scheme theory). I have already send my ``Esquisse" to many of my old friends and reputed topologists, but none of them seems to have found it particularly interesting...}. These cohomology groups are involved (mostly \marginpar{p. P36}``between the lines", admittedly) in Weil's conjectures, and they are what constitutes their deeper ``raison d'\^etre" (in my eyes at least, after having been ``brought to speed" by Serre's explanations), what gives them their full meaning. But the very possibility of associating such invariants to the ``abstract" algebraic varieties occurring in these conjectures, in a way that would respond to the very specific desirata imposed by the needs of the situation, was no more than a mere hope. I doubt that anybody other than Serre and myself (including, and especially, Andr\'e Weil himself!\footnote{Paradoxically, Weil had a persistent ``block", seemingly visceral, against the cohomological formalism - even though his famous conjectures were in large part what inspired the development of cohomological theories in algebraic geometry, starting from the year 1955 (with Serre marking the starting point, with his fundamental article FAC, already mentioned in an earlier footnote). It seems that, for Weil, this ``block" fit into a general aversion against ``big machinery", i.e. that which fits into a formalism (which cannot be summed up into a few pages) or takes the form of a relatively nested ``construction". Weil was not a ``builder", admittedly, and it was almost against his own will that he was forced, during the 1930s, to develop the initial foundations of ``abstract" algebraic geometry, foundations which (in view of his propensity) turned out to be a ``Procrustean bed" for the user. 

I do not know whether he held a grudge against me for going beyond, and for engaging in the construction of vast dwellings which later allowed for Kronecker's dream as well as his own to find an incarnation as a language and as a collection of efficient and delicate tools. In any case, he has never communicated a single word to me regarding the work in which he saw me involved, or regarding the work that had already been done. I received no echo concerning R\'ecoltes et Semailles either, of which I had sent him a copy more than three months earlier, accompanied by a hearty handwritten dedication.}) really believed this could work...

A little earlier, our understanding of cohomological invariants found a deep enrichment and renewal through the work of \textbf{Jean Leary} (produced while he was held in captivity in wartime Germany, \marginpar{p. P37} during the first half of the 1940s). The essential innovation was the notion of (abelian) \textbf{sheaf} over a space, to which Leray associated a sequence of ``cohomology groups" (said to ``take values in that sheaf"). It was as if the good old classical ``cohomological meterstick" of which we hitherto disposed to ``survey" a space was suddenly multiplied into an unimaginably large multiplicity of new ``metersticks" of all sizes, shapes, and substance imaginable, each of them intimately linked to the space at hand, and conveying to us information about it with perfect precision, information which it alone was in a position to reveal. The notion of a sheaf was the ``id\`ee ma\^itresse" of a profound transformation in our probing of spaces of all kinds, and surely one of the most crucial ideas to have appeared in the course of this century. Thanks mostly to the later work of Jean-Pierre Serre, Leray's ideas first bore fruit during the decade following their appearance, in the form of an impressive restart of the theory of topological spaces (notably the theory of so-called ``homotopical" invariants, intimately linked to cohomology), as well as in the form of another restart, no less capital, in so-called ``abstract" algebraic geometry (through Serre's fundamental FAC article, which appeared in 1955). My own work in geometry, beginning in 1955, fits in the continuity of Serre's work, and thus also falls in line with Leray's innovative ideas. 

\section{Topoi - or the double bed}

The viewpoint and language of sheaf theory introduced by Leray brought us to view ``spaces" and ``varieties" of all kinds in a new light. They did not touch the notion of space itself; rather, they allowed us to probe more finely, with new eyes, the traditional ``spaces" which were already familiar to all. However, it turned out that this notion of space was inadequate to capture the most essential ``topological invariants" expressing the ``shape" of ``abstract" algebraic varieties (including those which applied to Weil's conjectures), or the shape of general ``schemes" (which generalized the old varieties). For the anticipated ``marriage of number and size", the existing notion of space was a rather cramped bed, where only one of the future spouses (namely, size) could with some effort manage to fit, but certainly not both at once! The ``novel principle" which remained to be found, to arrive at the promised wedding, was nothing but the spacious ``bed" which the future spouses were missing, escaping everybody's notice up to this point...

This ``double bed" appeared (as if by magic) with the idea of the \textbf{topos}. This idea englobes in a common topological\marginpar{p. P38} both the traditional (topological) spaces, incarnating the world of continuous size, and that (so-called) ``spaces" (or ``varieties") of the impertinent abstract algebraic geometers, as well as countless other types of structures which seemed hitherto to be irremediably locked in the ``arithmetic world" of ``discontinuous" or ``discrete" aggregates.

It is the viewpoint of sheaf theory which was the silent and sure guide, the efficient (and not in any way secret) key which led me without delay nor detours towards the bridal chamber and its ample marital bed. A bed so vast indeed (akin to a vast and tranquil deep river), that  
\begin{equation*} \begin{split}
\text{``all of the king's horses} \\
\text{could drink from it at once..."} 
\end{split} \end{equation*}
\footnote{Translator's note: In French, the poem reads:
\begin{equation*} \begin{split}
\text{``tous les chevaux du roi} \\
\text{y pourraient boire ensemble..."}
\end{split} \end{equation*} }
\addtocounter{footnote}{-1}
- as goes an old air which you surely have sung in the past, or at least heard. And the person who first sang it has better felt the secret beauty and serene strength of the topos than any of my wise students and friends of yesteryear...

The key was the same, both in the initial and provisory approach (via the convenient but non-intrinsic notion of ``site") and in that of topos. It is the idea of the topos which I would now like to try describing.

Consider the set of \textbf{all} sheaves on a given (topological) space, which is, in a sense, the prodigious arsenal formed by \textbf{all} the ``metersticks" used to probe this space\footnote{(Intended for the mathematical reader) To be precise, I am here speaking of sheaves of \textbf{sets}, not of the \textbf{abelian} sheaves which were introduced by Leray as the most general type of coefficients that could be used to construct ``cohomology groups". I believe I was the first to have systematically worked with sheaves of sets (starting from 1955, in my article ``A general theory of fibre spaces with structure sheaf" at the University of Kansas).}. We consider this ``set" or ``arsenal" equipped with its most evident structure, which appears - so to speak - ``in front of our nose"; namely, the structure of a ``category". (The non-mathematical reader needs not worry if the term is unfamiliar. Its meaning will not be needed in what follows.) It is this ``probing superstructure" of sorts, called ``category of sheaves" (on the space considered), which will henceforth be considered as the ``incarnation" of what is most essential in the space. Such a perspective is legitimate (in terms of mathematical ``common"-sense), in that we can entirely ``recover" a topological space\footnote{(Intended for the mathematical reader) Strictly speaking, this only holds for so-called ``sober" spaces. These nonetheless accounts for the near-totality of the spaces which one commonly encounters, including all of the ``separated" spaces that are dear to the analysts.} from its associated ``category of sheaves" (or probing\marginpar{p. P39} arsenal). (To verify this fact is a simple exercise - admittedly, the question first needs to be asked...) This suffices to assure us that (if it fits our purposes for one reason or another) we can henceforth ``forget" the initial spaces, and only retain and use the associated ``category" (or arsenal), thought of as the most adequate incarnation of the ``topological (or spatial) structure" which we are trying to express.

As often happens in mathematics, we have succeeded (through the crucial idea of a ``sheaf", or ``cohomological meterstick") to express one notion (that of ``space" in this case) in terms of another (that of ``category"). Each time, the discovery of such a textbf{translation} of one notion (expressing a certain type of situation) in terms of another (corresponding to another type of situations) enriches our understanding of both notions, through the unexpected confluence of specific forms of intuition pertaining to one notion or the other. Thus, a situation of a ``topological" nature (incarnated by a given space) is hereby translated into a situation of an ``algebraic" nature (incarnated by a ``category"); alternatively, the ``continuity" incarnated by the space is ``translated" or ``expressed" by the structure of the associated category, which is of ``algebraic" nature (something which had hitherto been perceived as being of a fundamentally ``discontinuous" or ``discrete" nature).

But on the other side, there is more. The first of these notions, that of a space, seemed to be a ``maximal" notion - so general already, that we could hardly imagine how one could go about finding a ``reasonable" further extension. On the other hand, these ``categories" (or ``arsenals") which we encounter on the other side of the looking glass\footnote{The ``looking glass" in question, as in Alice Through the Looking Glass, is that which returns as ``image" of a given space the associated ``category", considered as a sort of ``double" of the space, ``on the other side of the looking glass".}, starting from topological spaces, are of a very particular form. They enjoy a collection of very specific properties\footnote{(Intended for the mathematical reader) I am here referring mostly to the properties which I have introduced in category theory under the name of ``exactness properties" (at the same time as the modern categorical notion of general inductive and projective ``limits"). See ``Sur quelques points d'alg\`ebre homologique", Tohoku math. journal, 1957 (p. 119-221).}, which liken them to a ``pastiche" of the simplest such category - that obtained as the mirror of a one point space. Thus, a ``space nouveau style" (or topos), generalizing the traditional notion of topological space, can be described simply as a ``category" which, without necessarily coming from a specific topological space, possesses all of the good properties (singled out once and for all) pertaining to such a ``category of sheaves". 
\begin{center} {\fontencoding{U}\fontfamily{mpi001}\fontsize{14}{14}\selectfont \char109} \end{center}

\marginpar{p. P40}This, then, was the novel idea. Its appearance can be viewed as a consequence of the (nearly childlike) observation that what really matters in a topological space isn't its ``points" or its subsets of points\footnote{Indeed, it is possible to construct very ``large" topoi which nonetheless have only one ``point", or even no ``points" at all!} (and the proximity relations between them, etc...); rather, it is the \textbf{sheaves} on that space that matter, and the category which they form. I have only, in effect, carried Leray's initial idea to its ultimate consequence - and, having done this, \textbf{taken the step}.

As with the very idea of sheaves (due to Leray), or that of schemes, indeed as for any ``great idea" which comes to disrupt an ingrained vision of things, the notion of topoi is unsettling due to its natural character, its ``evidence"; due to a simplicity (which is almost, one could say, naive or simplistic, even ``silly") of the particular flavor which makes us so often exclaims: ``Oh, that's all there was to it!", in a tone half disappointed, half jealous; with perhaps an undertone pertaining to the ``zany", the ``unreasonable", which is often reserved to that which is baffling through an excess of unexpected simplicity. A simplicity which reminds us, perhaps, of the long buried and repressed days of our childhood.

\section{Mutation of the notion of space - or the breath and the faith}

The notion of scheme constitutes a vast enlargement of the notion of ``algebraic variety", and as such it has led to a profound renewal of the algebraic geometry inherited from my predecessors. As for the notion of topos, it constitutes an unexpected enlargement, or rather, \textbf{a metamorphosis of the notion of space}. It thereby bears promise for a similar renewal of topology, and beyond it, of geometry. Already, it has played a crucial role in the expansion of the novel geometry (especially through the themes of $l$-adic and crystalline cohomology issued from it, and the resulting proof of the Weil conjectures). Like its older sister (and quasi-twin), it possess the two complementary characteristics that are essential to any fruitful generalization, which are described in what follows.

First, the new notion is not \textbf{too vast}, in that inside the new ``spaces" (called ``topoi" so as not to bother delicate ears\footnote{The name ``topos" was chosen (in association with ``topology", or ``topological") to suggest that this is the ``objet par excellence" to which topological intuition applies. In line with the rich clod of mental images that this name evokes, it is to be considered as essentially the equivalent of the term (topological) ``space", only with a stricter emphasis on the ``topological" specificity of the notion. (Thus, one can speak of ``vector spaces", but not, as far as I am aware, of ``vector topoi"!) It is necessary to keep the two expressions conjointly, each with its particular specificity.}), the most essential\marginpar{p. P41} ``geometric" intuitions and constructions\footnote{Among these ``constructions" is comprised notably that all of the familiar ``topological invariants", including cohomological invariants. Concerning the latter, I had already laid all of the necessary groundwork in the previously mentioned article (``Tohoku", 1955) necessary to express them in an arbitrary ``topos".}, familiar within the traditional spaces of yesteryear, admit a more or less obvious transposition. In other words, when working with the new objects, we have at our disposal the entirety of the rich array of the images, mental associations, notions, and to some extent the techniques which hitherto were restricted to the old fashioned objects.

Second, the new notion is at the same time \textbf{sufficiently vast} to englobe a host of situations which were hitherto not considered to be suitable for intuition of a ``topologico-geometric" nature - an intuition which we had in the past precisely reserved to the study of ordinary topological spaces (and for good reasons...). 

Crucially, from the perspective of the Weil conjectures, this new notion is vast enough to allow us to associate to any ``scheme" such a ``generalized space" or ``topos" (called the ``\'etale topos" associated to the scheme under consideration). Some of the ``cohomological invariants" of this topos (``silly" as they may be!) then seemed to stand a good chance to provide us with ``all that was needed" to make sense of these conjectures, and (who knows) to give us the means of proving them.

It is in these pages which I am currently writing that, for the first time in my life as a mathematician, I take the time to depict (even if only to myself) the collection of ``ma\^itre-th\`emes" and great governing principles underlying my mathematical work. This leads me to better appreciate the place and scope of each of these themes and of the ``viewpoints" which they incarnate, within the grand geometric vision which unites them and from which they are issued. It is through this process that    : the notion of \textbf{schemes}, and that of \textbf{topoi}.

It is the second of these ideas, that of topoi, which presently appears\marginpar{p. P42} to me as the deeper of the two. If, by adventure, I had not taken it upon myself in the late 1950s to diligently develop, day after day, over the course of twelve long years, a ``schematic tool" of perfect accuracy and power, it seems almost unthinkable that within the following ten to twenty years, others would not have eventually introduced (be it even against their will) the notion which was visibly required, putting together at the very least a handful of ``prefabricated" shabby barracks, instead of the spacious and comfortable dwellings which I took to heart to assemble stone by stone and to erect with my own hands. On the other hand, I cannot think of anyone else on the mathematical scene who, in the course of the past three decades, could have had the naivet\'e, or innocence, required to take (in my stead) the most crucial step of all in introducing the childlike idea of the topos (or even that of ``sites"). In fact, even presupposing that the idea had been generously provided, and with it the shy promise it seemed to enclose - I see nobody else, be it among my old friends or among my students, who would have had the breath, and most importantly the faith, to carry this humble idea (so derisory in appearance, while the goal seemed infinitely distant...) to completion\footnote{(Intended for the mathematical reader) When I speak of ``carrying this humble idea to terms", I refer to the idea of \'etale cohomology as an approach to the Weil conjectures. It is under this inspiration that I discovered the notion of site in 1958, and that this notion (or the closely tied notion of topos), as well as the formalism of \'etale cohomology, were developed between the years 1962 and 1966 following my impulsion (and assisted by certain collaborators who will be mentioned at a later point).

When I speak of ``breath" and of ``faith", I am referring to qualities of a ``non-technical" nature, yet which here really appear to be the essential required qualities. At another level, I could also add what could be called ``cohomological flair", meaning the kind of flair which had developed within me while working on the edification of cohomological theories. I was under the impression that I had communicated that flair to my cohomology students. In hindsight, seventeen years after my departure from the mathematical worlds, I realize that it did not persist in any of them.}: from its fledgling beginnings to full maturity and the ``mastery of \'etale cohomology" into which it ended up transforming between my hands, during the years that followed.

\section{All of the king's horses...}

Indeed, deep is the river, and vast and tranquil are the waters of my childhood, in a kingdom I thought I had long since left. All of the king's horses could drink from these waters at once, at ease and to their full satisfaction, without exhausting them! The water comes from glaciers, ardent as the distant snowfalls, and it is soft as the loam of the plains. I just told you about one of these horses, which a child brought to drink at length and to its fill. I also saw another\marginpar{p. P43} horse come to drink for a moment, possibly following the footsteps of the same kid - but not for long. Someone must have chased it away. And that is all, I must say. And yet I see countless herds of parched horses roaming the plains - in fact, just this morning, their neighs got me out of bed, at an undue hour, even though I am in my sixties and cherish tranquility. There was no choice, I had to wake up. It hurt to see them in this state, weak and beaten up, even though there was no shortage of fresh water or green pastures. It was as if a malicious spell had been cast on a region which I once knew to be welcoming, blocking off all access to these generous waters. Or perhaps was it a plot orchestrated by the country's horse dealers to drop prices, who knows? Or was this a country where no children were left to lead the horses to the water, and where the horses suffered from thirst, for lack of a child to show them the way back to the river...

\section{Motives - or the heart within the heart}

The theme of topoi was issued from that of schemes, in the same year that schemes appears - yet it far surpasses the mother theme in its scope. It is the theme of topoi, not that of schemes, which constitutes this ``bed", or ``deep river", in which geometry and algebra, topology and arithmetic, mathematical logic and category theory, the world of the continuous and that of ``discontinuous" or ``discrete" structures come to be wed. If the theme of schemes is to be seen as the \textbf{heart} of the novel geometry, then the theme of topoi is its envelop, or its \textbf{dwelling}. It is the vastest of my conceptions, in its ability to subtly grasp, with a unique language rich in geometric resonance, the common ``essence" of situations far removed from one another, coming from one region or another of the vast universe of mathematical things. 

Yet, the theme of topoi is far from having encountered the success of that of schemes. Now is not the time to speak of the strange vicissitudes that have affected this notion - I touch on this topic on various occasions in R\'ecoltes et Semailles. Two of the ``ma\^itre-th\`emes" of the novel geometry are nonetheless issued from that of topoi, namely two complementary ``cohomology theories", both conceived with the intention of formulating an approach towards the Weil conjectures: the \textbf{\'etale} (or \textbf{``$l$-adic"}) \textbf{theme}, and the \textbf{crystalline theme}. The first concretized in my hands into the $l$-adic cohomology tool, which has already imposed itself as one of the most powerful mathematical tools of the century. As for the crystalline theme, reduced to a quasi-occult status following my departure, it was finally exhumed (under pressure of needs) in June 1981, in the limelight and under a borrowed name, in circumstances stranger even than those surrounding topoi...

\marginpar{p. P44}The $l$-adic cohomology tool was, as predicted, the key tool used to establish the Weil conjectures. I proved several of them myself, and the last step was taken masterfully three years later by Pierre Deligne, the brightest of my ``cohomologist" students.

Around the year 1968, I had also formulated a stronger and more ``geometric" version of the Weil conjectures. The latter remained ``tied" (so to speak!) to an apparently irreducible ``arithmetic" aspect, even though the very spirit of these conjectures is to express and capture the ``arithmetic" (or the ``discrete") by means of the ``geometric" (or the ``continuous")\footnote{(Intended for the mathematical reader) The Weil conjectures are subject to hypotheses of an ``arithmetic" nature, stemming from the fact notably that the varieties under consideration are defined over a finite field. From the perspective of cohomological formalism, this leads to the \textbf{Frobenius endomorphism} playing a special role in the situation. In my approach, the crucial properties (of the kind ``generalized index theorem") concern \textbf{arbitrary} algebraic correspondences, and make no hypothesis of an arithmetic nature on the given base field.}. In this sense, the version of the conjectures which I had formulated seem to me more ``faithful" to ``Weil's philosophy" than Weil's own conjectures - a philosophy left unwritten and rarely told, yet which was perhaps \textbf{the} main tacit motivation underlying the extraordinary expansion of geometry over the course of the past four decades\footnote{However, there was following my departure in 1970 a clear reactionary movement, concretized by a situation of relative stagnation, which I evoke more than once in R\'ecoltes et Semailles.}. My reformulation essentially consisted in extracting a ``quintessence" of sorts of what had to remain true in the context of so-called ``abstract" algebraic varieties, of classical ``Hodge theory", valid for ``ordinary" algebraic varieties\footnote{Here, ``ordinary" means ``defined over the field of complex numbers". Hodge theory (said to be the ``theory of harmonic integrals") was then the most powerful known cohomology theory in the context of complex algebraic varieties.}. I called this new, entirely geometric version of Weil's famous conjectures the \textbf{``standard conjectures"} (for algebraic cycles).

In my mind, this constituted a new step towards these conjectures, following the development of the $l$-adic cohomology tool. But it was also and most importantly a principle through which one could approach what I consider to this day to be the deepest theme which I have introduced\marginpar{p. P45} in mathematics\footnote{This is at least the deepest theme which I introduced during the ``public" period of my mathematical activity, between the years 1950 and 1969, up to my departure from the mathematical scene. I consider the themes of nonabelian algebraic geometry and Galois-Teichm\"uller theory, developed starting from the year 1977, to be of comparable depth.}: that of \textbf{motives} (which itself was born out of the ``$l$-adic cohomology theme").  This theme is akin to the \textbf{heart} or soul, the part most concealed and hidden from view, of the schematic theme, which itself lies at the heart of the novel vision. And the handful of key phenomena unearthed by the standard conjectures\footnote{(Intended for the algebraic geometers)  There eventually came a time when these conjectures had to be reformulated. For a more detailed commentary on this, see ``A tour of the construction sites" (ReS IV note n$^o$ 178, p. 1215-1216) and the footnote on p. 769 of ``Conviction and knowledge" (ReS III, n$^o$ 162).
%todo: ref
} can in turn be conceived of as forming a sort of ultimate quintessence of the motivic theme, akin to the vital \textbf{`breath"} of this most subtle of themes, constituting the \textbf{``heart within the heart"} of the novel geometry.

Here is a rough overview of what these motives are about. We saw earlier, for a given prime number $p$, the importance (notably from the perspective of the Weil conjectures) of knowing how to construct ``cohomology theories" for ``(algebraic) varieties in characteristic $p$". In fact, the ``$l$-adic cohomology tool" provides just such a theory; it even provides an \textbf{infinity of different cohomology theories}, one for each prime number different from the underlying characteristic $p$. There remains visibly a ``missing theory", corresponding to the case where $l$ equals $p$. To provide such a theory, I conceived of yet another cohomology theory (which I already alluded to earlier) called ``crystalline cohomology". Furthermore, in the important case where $p$ is infinite, we have at our disposal three additional cohomology theories\footnote{(Intended for the mathematical reader) These theories correspond, respectively, to Betti cohomology (defined by transcendental methods, via an embedding of the base field into the field of complex numbers), Hodge cohomology (defined by Serre), and De Rham cohomology (defined by myself), the latter two tracing their origins to the 1950s (and Betti cohomology to the last century).} - and there is no reason why we wouldn't be led in the future to introduce other new cohomology theories, sharing analogous formal properties. Contrarily to what happens in ordinary topology, we are thus faced with a disconcerting abundance of different cohomology theories. One had the net impression that, in a sense which had yet to be clarified, all of these theories had to be ``essentially the same", that they ``gave the same results"\footnote{(Intended for the mathematical reader) For instance, if $f$ is an endomorphism of the algebraic variety $X$ inducing an endomorphism of the cohomological space $H^i(X)$, the ``characteristic polynomial" of the latter should have \textbf{integral} coefficients, independently of the specific cohomology theory we have chosen (e.g. $l$-adic cohomology for varying $l$). Ditto for general algebraic correspondences when $X$ is assumed to be proper and smooth. The sad truth (which hints at the state of deplorable abandon of the cohomology theory of algebraic varieties in characteristic $p > 0$ since my departure) is that this has not yet been proven as the time of writing, even in the special case where $X$ is a smooth projective \textbf{surface} and $i=2$. In fact, as far as I am aware, no one after my departure cared to look into this crucial question, typical of the kind of questions which appear as subordinate to the standard conjectures. The decree in vogue is that the only endomorphism worth our attention is the Frobenius endomorphism (which was treated separately by Deligne with the means at hand)}. It is in order to\marginpar{p. P46} express this intuition of ``kinship" between different cohomology theories that I have formulated the notion of \textbf{``motive"} associated to an algebraic variety. In choosing this term, I intended to suggest the interpretation of ``common motive" (or ``common \textbf{reason}") underlying this multitude of different cohomological invariants associated to the given variety, by means of the multitude of a priori available cohomology theories. Under this framework, these different cohomology theories would appear as different thematic developments, each in the most appropriate ``tempo", ``key" or ``mode" (``major" or ``minor"), of a single ``base motive" (called ``\textbf{motivic} cohomology theory"), while the latter would at the same time be the most fundamental, or ``finest", of all of the different thematic ``incarnations" (meaning, of all of the possible cohomology theories). Thus, the motive associated to an algebraic variety would constitute the ``ultimate" cohomological invariant, the invariant ``par excellence", from which all others (associated to the various possible cohomology theories) could be deduced, as various musical ``incarnations", or as different ``realizations". All the essential properties of ``\textbf{the} cohomology" of the variety could be ``read" (or ``heard") on the corresponding motive, so that all of the familiar properties and structures pertaining to particular cohomological invariants (such as $l$-adic or crystalline invariants) would simply appear as the faithful reflections of properties and structures \textbf{internal to the motive}\footnote{(Intended for the mathematical reader) Another way to think about the category of motives on a field $k$ is to visualize it as a sort of ``enveloping abelian category" of the category of separated schemes of finite type over $k$. The motif associated to such a scheme $X$ (also called the ``motivic cohomology of $X$", which I denote by $H^*_{\text{mot}}(X)$) thereby appears as an ``abelianized" avatar of sorts of $X$. Crucial here is the fact that, just as an algebraic variety $X$ can be subject to ``continuous variation" (so that its isomorphism class depends on continuous ``parameters", or ``modules"), the motive associated to $X$, or more generally a ``variable" motive, should also be subject to continuous variation. The latter aspect of motivic cohomology stands in stark contrast to the behavior of all classical cohomological invariants, including $l$-adic invariants, with the only exception of Hodge cohomology for complex algebraic varieties.

This should give an idea of the extent to which ``motivic cohomology" is a finer invariant, capturing much more tightly the ``arithmetic shape" (if I dare use this expression) of $X$ than traditional invariants of a purely topological nature can. In my vision, motives constitute a subtly hidden and delicate ``thread" linking together the algebro-geometric properties of an algebraic variety and the properties of an ``arithmetic" nature incarnated by its associated motive. The latter can be considered as an object of a ``geometric" nature in its very spirit, albeit one where ``arithmetic" properties subordinate to the geometry are, so to speak, ``laid bare". 

Thus, the motive appears to me to be the deepest ``shape invariant" which we have been able to associate to an algebraic variety to this day, with the exception of its ``motivic fundamental group". Both invariants represent in my eyes the ``shadows" of a ``motivic homotopy type" which is yet to be described (and about which I write a few words in passing in the note ``A tour of the construction sites - or tools and vision" (ReS IV, n$^o$178, see construction site 5 (Motives), notably page 1214)). It is the latter object which in my eyes ought to be the most adequate incarnation of the elusive intuition of the ``arithmetic (or motivic) shape" of an arbitrary algebraic variety.}.
%todo: ref. 

\marginpar{p. P47}Such is, expressed in the non-technical language of a musical metaphor, the quintessence of an idea once again of childish simplicity, at once delicate and bold. I developed this idea on the margins of the foundational tasks which I considered more urgent, under the name of ``theory of motives" or ``philosophy (or yoga) of motives" throughout the years 1963-69. It is a theory of fascinating structural richness, about which a great deal remains conjectural to this day\footnote{I have explained my vision of motives to anyone who cared to listen, throughout these years, without going through the trouble of publishing anything about this subject in print (having to cater to several other tasks at the service of all). This later allowed certain of my students to plunder more at ease, under the tender eye of all of my old friends, well informed about the situation. (See the footnote on the following page.)}.

I express myself about this ``yoga of motives", which I hold particularly dear to my heart, at various points in R\'ecoltes et Semailles. Now is not the time to repeat what I have written about it elsewhere. I shall only mention that the ``standard conjectures" ensue in the most natural way from this yoga of motives. At the same time, they also provide an approach to one of the possible constructions of the notion of motive.

These conjectures seemed to me (and they still do to this day) to be one of the two most fundamental standing questions in algebraic geometry. Neither this question, nor the equally crucial other question (concerning the ``resolution of singularities") has been resolved as of today. But while the second question appears today, as it did a hundred years ago, to be prestigious and formidable, the question which I had the honor\marginpar{p. P48} of formulating was relegated by the peremptory decrees of fashion (beginning in the years that followed my departure from the mathematical scene, just as it happened to the motivic theme itself\footnote{In fact, the motivic theme was exhumed in 1982 (a year after the crystalline theme), in its original name this time (and in a narrowed form, accounting only for the characteristic zero case), with no mention of the name of the original craftsman in sight. This is an example among others of a notion or theme buried following my departure as grothendieckian phantasmagories, only to be exhumed one after the other by certain of my students during the ten to fifteen years that followed, with a modest pride and (need I mention it again) without mentioning the original craftsman...}) to the status of benign grothendieckian farce. But once again I am getting ahead of myself...

\section{Toward the discovery of the Mother - or the two sides}

To tell the truth, my reflections upon the Weil conjectures themselves, in the process of establishing them, remained sporadic. The panorama which had begun to open up in front of me, and which captured my attention in my attempt to scrutinize it and render it, far surpassed in amplitude and in depth the hypothetical needs of a proof; in fact, it even surpassed all that these famous conjectures had allowed us to glimpse in the first place. With the sudden appearance of the schematic theme and that of topoi, a new and unsuspected world had suddenly opened up. ``The conjectures" played a central role, to be sure, akin somewhat to the role which the capital city of a vast empire or continent would play in relation to the countless surrounding provinces, with most of the latter nonetheless maintaining only the most distant rapports with the brilliant and prestigious capital. Without ever having to spell it out for myself, I knew that I had from that point onwards become the servant to a great task: that of exploring this immense and unknown world, of apprehending its contours all the way to the farthest frontiers; and at the same time, of roaming in every direction and establishing with tenacious and methodical care the inventory of neighboring and distant provinces alike, drawing maps with scrupulous fidelity and precision, in which the smallest of hamlets and cottages would find their place...

It is mostly the above work which ended up absorbing the greatest part of my energy - a patient and vast foundational work which I alone perceived with clarity and, most importantly, ``felt in my guts". It is that work which by far occupied the most of my time, between 1958 (the year where the schematic theme and that of topoi appeared, one after the other) and 1970 (the year of my departure from the mathematical scene).

I often had to step on the breaks, feeling restrained as if by a persistent and clinging weight by the endless tasks which (once what was essential had been established) felt closer to an act of ``stewardship" than to a thrust into the unknown. I had to constantly resist the urge to plow ahead -\marginpar{p. P49} the urge of the pioneer or explorer, gone at the discovery and exploration of unknown and unnamed worlds, ceaselessly calling him to come to know them and give them a name. That urge, and the energy which I devoted to it (almost covertly!) were constantly present in the background.

Yet, I knew deep down that it was precisely that energy, snatched (so to speak) from what I owed to my ``tasks", which was of the rarest and most unbound essence - that it was \textbf{there} that lay the ``creation" in my work as a mathematician: in this intense attention for apprehending, within the obscure, amorphous and moist folds of a warm and inexhaustible nourishing womb, the first traces of shapes and contours yet to be born and which seemed to be calling me in an effort to take form, to become incarnate and come to life... In the process of discovery, this intense attention and ardent solicitude are an essential force, akin to the sun's heat contributing to the underground gestation of sowings buried in the nourishing earth, carrying them toward their humble and miraculous hatching in broad daylight.

In my work as a mathematician, I mostly see two forces or urges at play, both equally profound, yet of a different nature (or so it seems to me). To evoke one and the other, I have used in turn the image of the \textbf{builder} and that of the \textbf{pioneer} or explorer. Put next to each other, they now both strike me suddenly as very ``yang", very ``masculine", even ``macho"! They bear the haughty resonance of myths, that of ``grand occasions." Surely they were inspired within me by my old ``heroic" vision of creative work, the super-yang vision. As such, they offer a strongly tinted perspective, not to say frozen, ``ten-hut", of a much more fluid reality, humbler and ``simpler" - a \textbf{living} reality.

In this male urge of the ``builder", pushing me ceaselessly toward new construction sites, I can nonetheless discern equally the urge of the \textbf{homebody}: that of a person profoundly attached to \textbf{``the"} home. Before anything else, it is \textbf{``his"} house, that of ``close ones" - the location of an intimate living entity to which he feels a sense of belonging. Only later, as the circle of what is considered ``close" progressively grows, does it become a ``home for all". And in this impulse to ``make houses" (the way we would ``make" love...), there is first and foremost a \textbf{tenderness}. There is the pulse of a \textbf{contact} with the materials we come to shape one by one, with loving care, and which we only really come to know through this affectionate contact. Then, following the building of the walls and the laying of the beams and ceiling, comes the profound\marginpar{p. P50} satisfaction of installing one room after another, and to progressively witness the emergence, amidst these halls, rooms, and nooks, of the harmonious order of a living house - beautiful, welcoming, good for living. Indeed, \textbf{home} is also, first and foremost and secretly for  each of us, \textbf{the mother} who surrounds us and shelters us, at once refuge and solace; and perhaps (on an even deeper level, and despite the fact that we are constructing it from scratch) it is also that from which we ourselves are issued, that which hosted and nourished us during the times forever forgotten that preceded our birth... It is also the \textbf{Bosom}.

And the picture that thus spontaneously appeared, in an effort to go beyond the prestigious title of ``pioneer" and to grasp the hidden reality which it covered, was also free from any ``heroic" emphasis. Once again, it was the archetypal image of the maternal that appeared to me - that of the nourishing ``womb" and of its amorphous and obscure workings...

These two urges which at first seemed to be ``of a different nature" are actually closer to one another than I had thought. They both share the nature of an \textbf{``urge of contact"}, carrying us toward meeting \textbf{``the Mother"}: toward That which \textbf{both} incarnates what is near, ``known", \textbf{and} what is ``unknown". To abandon myself to one urge or to the other, is to ``return to the Mother". It is to renew contact at once with what is near, ``more or less known", and with what is ``unknown" yet at the same time felt, at the brink of making itself known.

The difference herein is one of tonality, of dosage, rather than one of nature. When I am ``building houses", it is the ``known" that dominates, and when I am ``exploring", it is the unknown. These two ``modes" of discovery, or rather, these two aspects of the same process, the same work, are inextricably linked. They are essential and complementary to one another. In my mathematical work, I discern a constant back-and-forth motion between these two modes of approach, or rather, between moments (or periods) during one mode or the other dominates\footnote{What I say here about mathematical work is equally true for ``meditation" work (which is frequently discussed in R\'ecoltes et Semailles). I have no doubt that this is in fact a phenomenon that is featured in any work of discovery, including that of the artist (writer of poet, for instance). The two ``sides" which I describe above can both be seen alternatively as that of \textbf{expression} and the accompanying ``technical" requirements, and that of \textbf{reception} (of all sorts of perceptions and impressions), which become \textbf{inspiration} under the effect of intense attention. Both sides are present at every moment of the process, and there is a constant motion of ``back-and-forth" between the ``times" where one mode predominates and times where the other mode does.}. However, it is also clear that in each moment, both modes\marginpar{p. P51} are nonetheless present. When I am building, adjusting, or when I am clearing, cleaning, organizing, it is the ``yang", or ``masculine" ``mode" or ``side" of work which sets the pace. When I am fumbling through the elusive, the amorphous, the nameless, I am closer to the ``yin", or ``feminine" side of my being.

My point is not that either side of my nature should be minimized or renounced, as they are both essential - the ``masculine" builds and engenders, and the ``feminine" conceives and hosts the slow and obscure gestations. I ``am" one and the other - ``yang" and ``yin", ``man" and ``woman". Yet I also know that the most delicate essence, the one most unrestrained when engaging in creative processes lies on the ``yin" , the ``feminine" side - a side which is humble, obscure, and often of feeble appearance.

It is that side of work which, for as long as I can remember, has exerted upon me the most powerful fascination. And this was so despite the fact that the ruling consensus was encouraging me to invest the largest part of my energy to the other side, through which tangible, not to say finished and completed ``products" are incarnated and established - products whose contours are well defined, a testament to their reality with the evidence of carved stone...

I clearly perceive, in hindsight, how these consensuses have weighted upon me, but also how I managed to ``bear the weight" - seamlessly! Admittedly, the ``conception" or ``exploration" part of my work was kept at a minimum. Yet, looking back at what constituted my work as a mathematician, it is manifest with striking evidence that what constitutes the essence and power of this work is the side which is nowadays overlooked, one which is free from derision and from condescending disdain: that of \textbf{``ideas"}, even of \textbf{``dreams"}, and not in any capacity that of ``results". In trying to capture what is most essential in what I have contributed to the mathematics of today, by means of a sweeping glance that embraces the forest rather than linger on the trees, I have seen, not a track record of ``great theorems", but rather a living fan of fertile ideas\footnote{This is not to say that so-called ``great theorems" are missing from my work, including theorems resolving questions which were asked by others and which no one had known how to solve before I did. (I review some of them in the footnote (***) on page 554, in the note ``The rising sea..." (ReS III, n$^o$ 122).) Rather, as I already mentioned at the early stages of this ``walk"   (at the step ``Viewpoint and vision", n$^o$6), these theorems only gain their full meaning in the nourishing context of a greater theme, one initiated by one of these ``fertile ideas". Their proof then follows, naturally and effortlessly, from the very nature, the ``depth" of the theme which carries them - the way the river's waves seem to gently be born from the depths of its waters, smoothly and effortlessly. I speak in analogous terms, but with a different imagery, in the aforementioned note ``The rising sea...".}, coming together under one common and vast vision.

\section{The child and the Mother}

\marginpar{p. P52}When this ``foreword" started turning into a walk through my work as a mathematician, with my little synopsis on the ``heirs" (dyed-in-the-wool) and the ``builders" (incorrigible), there came to my mind a name for this failed attempt at a foreword: ``The child and the builder". Over the course of the days that followed, it became clear that the ``child" and the ``builder" were actually the same character. Thus, this name turned into ``The builder child". A name, I must say, that has a certain charm, and completely pleases me!

The reflection thus led to the realization that this haughty ``builder", or (more modestly) the child-who-plays-at-building-houses, was but once of the \textbf{two} faces of the famous child-at-play. There is also the child-who-likes-exploring-things, who enjoys rummaging and burrowing away in the sand or the nameless muddy waters, in the most improbable and bizarre places...  Probably in an attempt to put up a front (even if only for myself...), I started introducing him under the flamboyant title of ``pioneer", followed that of ``explorer",  a more down to earth title that nonetheless remained charged with prestige. It begged the question of which title between ``builder" and ``pioneer-explorer" sounded more manly, more appealing! Heads or tails?

And yet, in looking closer, we find that our intrepid ``pioneer" was actually a \textbf{girl} all along (who I had fancied dressing up as a boy) - a sister of the ponds, the rain, the drizzle and the night, silent and almost invisible as a result of her tendency to fade into the shadows, and always forgotten (when we aren't mocking her...). I myself have found a way to forget her, day after day - to doubly forget her I should say: at first, I only cared to see the boy (who plays at building houses...) - and, even when I finally couldn't help but to see the \textbf{other}, I still mistook her for a boy...

As for the nice name for my walk, it can no longer do. It is a strictly yang name, entirely ``macho", a limping-name. To become acceptable in earnest, the name would need to feature \textbf{the other} as well. Yet, strangely, \textbf{``the other" doesn't really have a\marginpar{p. P53} name}. The only one that somewhat does it is ``explorer", but it is decidedly a boy's name, there is nothing to be done about it. In this matter, language is being difficult, it manages to trick us while leaving us unaware, in a way which is visibly influenced by ancestral prejudice.

We could perhaps get away with the title of ``The builder-child and the explorer-child", leaving unspoken the fact that one is a ``boy" and the other is a ``girl", and that it is in fact a single boy-girl child who, at once, builds while exploring and explores while building... However, yet another aspect of things appeared to me yesterday, in addition to the double-sided yin-yang associated to that which contemplates and explores, and that which names and builds.

The Universe, the World, or even the Cosmos, are strange and highly remote things at the core. They don't really concern us. It is not towards \textbf{them} that our deeply seated urge for knowledge carries us. Instead, what attracts us is their most tangible and immediate \textbf{Incarnation}, that which is closest to us, most ``carnal", loaded with profound echoes and rich in mystery - That which blends with the origins of our being of flesh and with those of our species; and That also which at all times awaits us ``at the other end of the tunnel", silently and with arms open. It is \textbf{from her}, the Mother, the One who gave us birth as she gave birth to the World, that the urge emanates, and that the roads of desire soar - and it is towards \textbf{Her} encounter that they lead us, towards \textbf{Her} that they flow, to continually return and surrender to Her.

Thus, on this detour off the path of this unexpected ``walk", I find myself faced with a once familiar parable which I had started to forget - the parable of \textbf{the child and the Mother}. It could also be interpreted as a parable for \textbf{``Life, in the quest for itself}. Or alternatively, at the most humble level of individual existence, as a parable for \textbf{``the being in the quest of things"}. 

It is both a parable and the expression of an ancestral experience, deeply embedded in the psyche - that of the most powerful aboriginal symbol nourishing deep-seated creative layers. I seem to hereby recognize, expressed in the immemorial language of archetypal imagery, the very breath of the creative power which animates man's flesh and mind, as much in in its most humble and ephemeral manifestations as in its most dazzling and durable ones. 

This ``breath", as well as the carnal imagery which embodies it, is the most humble thing in the world. It is also the most fragile, the most ignored and the most scorned upon...

\marginpar{p. P54}The history of the vicissitudes of this breath over the course of your existence is nothing but \textbf{your} adventure, the ``adventure towards knowledge" in \textbf{your} life. And the wordless parable that expresses this adventure is that of the child and the Mother.

You are the child, issued from the Mother, hosted within Her, nourished by Her power. And the child sets off from the Mother, the Near-by, the Well-known - and evolves towards the Mother, the Infinite, forever Unknown and full of mystery...

\begin{flushright}
End of the ``Walk through a life's work". 
\end{flushright}










%\end{document}

















% \marginpar{p. P1}
% p .
% c ,
% q "
% pa (
% b \textbb{}
% ``travail sur pi\'eces''
