January 1986

\section{The magic of things}

\marginpar{p. P1}
When I was little I liked going to school. The same teacher taught us reading,
writing, arithmetic, singing (he accompanied us with a small violin), and even
about prehistoric men and the discovery of fire.
I do not ever recall being bored at school, during those days.
There was the magic of numbers, that of words, of signs, and of sounds.
That of \textbf{rhymes} as well, through songs and small poems. 
There seemed to be, within rhymes, a mystery which extended beyond words.
I believed this until the day somebody told me that there was a simple ``trick'' to it;
that rhyme was simply when one ends
two consecutive spoken movements by the same syllable, so that, as if by magic, these
phrases became \textbf{verses}. It was a revelation! 
At home, where I found a good audience, for weeks or months on end, I amused myself by
making verses. At one point, I even started exclusively speaking in rhymes. 
That period has passed, fortunately. 
Yet even to this day, I still sometimes write poems - but without trying to force the
rhyme, if it doesn't seem to come by itself. 

On another occasion an older friend, who was already in high school taught me about
negative numbers. It was also a fun game, although I rapidly exhausted it. 
And then there were crosswords - I spent days and weeks constructing them, evermore
interwoven. Within that game the magic of form, that of signs, and that of words found
themselves combined. But even that passion subsided without leaving a trace. 

During my high school years, which began in Germany during my first year, then later in
France, I was a good student without quite being the ``star student''.
I devoted myself without restraint to the
courses which I cared most about, and tended to neglect the others, without really caring
for the appreciation of my ``prof''. 
During my first year of high school in France, in 1940, I was interned at a concentration
camp with my mother, at Rieucros near Mende. It was wartime, and we were foreigners -
``undesirables'', as they said. But the administration of the camp turned a blind eye
towards the kids, however undesirable they may be. We came and went as we pleased. 
I was the oldest, and the only one to go to high school, which was four or five kilometers
away, in the rain and the wind, in makeshift shoes 
\marginpar{p. P2}
that always got wet. 

I still remember my first ``math examination'', in which the teacher gave me a bad grade,
for my proof of one of the ``three cases of equality of triangles''. 
My proof wasn't exactly that of the book, which he followed religiously. 
Yet, I knew very well that my proof was no less convincing than that of the book which I
followed min spirit, through repeated invocation of the traditional
``we slide this figure in such and such a way onto that figure''. 
Visibly, this teacher did not feel capable of judging things on his own (namely the
validity of the reasoning). He had to report to a higher authority, that of a book in this
case. I must have been stricken by such dispositions, for me to still remember this
incident. Ever since then, I have been presented with more than enough evidence to realize
that such dispositions are far from exceptional, but rather they are the quasi-universal
norm. There is a lot to be said on this subject - one which I approach more than once in
one way or another in 
R\'ecsoltes et Semailles. Yet to this day, I find myself invariably 
taken aback whenever I am confronted with such behavior\ldots

During the last few years of the war, while my mother was still interned at the camp, I
lived in a youth refugee house called ``Secours Suisse'', in Chambon sur Lignon. Most of
us were jewish, and when we were told (by the local police) that there would be raids by
the Gestapo, we went to hide in the woods for a night or two, in small groups of no less
than 3, without quite realizing that our life was on the line.
The region was filled with jews hiding in c\'evenol country, and many of us survived
thanks to the solidarity of the lcoal population.

What struck me most about
``Coll\'ege C\'evenol'' (where I was raised), was the extent to which my peers were
disinterested in learning. As for myself, I devoured our textbooks at the beginning of the
school year, thinking that this time around, we would finally learn \textbf{truly}
interesting things; and for 
the rest of the year I utlized my time as best as I could while classes dragged along
inexorably one trimester at a time. Yet we had some wonderful professors. The natural
history professor teacher, mister Friedel, was a person with remarkable intellectual and
social qualities. However, as he lacked authority, the class was acting out of control, to
the extent that it became impossible to hear what he had to say, as his voice was lost in
the hurly-burly. This may be the reason I haven't become a biologist! 


\marginpar{p. P3}
I spent a fair amount of time, including class time (shh\ldots), solving math problems.
The ones in the book soon became insufficient. Perhaps it was because they tended to
resemble one another after a while; but mostly, I believe, because they seemed
to come out of the blue \'a la queue-leue-leue,
with no indication as to where they came from or where they're going.
These were the books problems, not mine. And yet, natural questions were plentiful. For
instance, once the three side lengths $a$, $b$, and $c$ of a triangle are known, so that the
triangle itself is known (up to its position), there has to be an explicit ``formula''
that expresses the area of the triangle as a function of $a$, $b$, and $c$. Likewise, for
a tetrahedron of which the six side lengths are known - what is the volume? I struggled
through that one for a bit, but I must have gotten there eventually. In any case, when a
problem ``grabbed me'', I did not count the hours or days that I spent working on it, 
even if it meant losing track of everything else! (And such remains the case to this
day\ldots)

What I found least satisfying, in our math textbooks, was the absence of 
a serious definition of the notion of length (of a curve), of area (of a surface), or of
volume (of a solid). I promised myself to make up for this omission as soon as I could. 
This is what I devoted most of energy to between the years of 1945-1948, while I was a
student at the University of Montpellier. University lectures weren't for me. Without ever
quite realizing it, 
I must have been under the impression 
that all my professor did was recite the contents
of the textbooks, just like my first math teacher at the lyc\'ee de Mende.
I barely ever set foot on university grounds, just enough to keep up to date with 
the perennial ``program''. Books sufficed to cover said program, but it was clear that
they offered no answers to the questions I was asking myself. Truly, they did not even
\textbf{see} them, no more than my high-school textbooks did. As long as we were provided
with recipes for all sorts of calculations, such as lengths, areas, volumes, through single,
double, triple integrals (dimensions higher than 3 were carefully avoided\ldots), the
problem of providing an intrinsic definition was omitted by both my professors and
textbook authors. 

From my then limited experience, it seemed that I was the only person in the world to be
gifted with a curiosity for mathematical questions. Such was, in any case, my unexpressed
conviction, during those years spent in complete intellectual solitude (which did not
bother me).\footnote{Between 1945-1948, I lived with my mother in a small hamlet about 10
kilometers away from Montpellier, Mairargues (near Vendargues), lost in the middle of
vineyards. (My father disappeared in Auschwitz in 1942.) We scraped by on my meager
student funding. To make ends meet, I took part in the harvest every year, and after the
harvest season I would sell wine under the table (in contravention of the legislation, or
so I hear\ldots). On top of that, there was a self-regulating garden which 
supplied us with an abundance of
figs, spinach, and
even (towards the end) tomatoes planted by a complacent neighbor,
amidst a sea of splendid poppies.
It was the good life - although occasionally a bit rough along the edges, when we had to
replace a pair of glasses, or a pair of worn-out shoes. Luckily, because my mother was
weak and sick due to her long stay in the camps, we received free medical assistance. We
would never have been able to afford a doctor otherwise\ldots}
To be fair, it never occurred to me at that time to investigate
\marginpar{p. P4}
whether or not I was the
only person in the world to take interest in what I was doing. 
My energy was sufficiently absorbed by the task I set for myself: to develop a fully
satisfactory theory.

I never doubted that I would succeed in reaching the end of the story, as long as I was
committed to scrutinizing these structures, spelling out on paper what they were telling
me. The intuition behind \textbf{volume}, say, was irrecusable. It could only be the
reflection of a \textbf{reality}, momentarily elusive, but perfectly reliable. What had to
be done was simply to seize this reality - a bit, perhaps, the way the magic reality of
the ``rhyme'' had been seized, ``understood'' one day. 

When I began this pursuit, at age 17, freshly out of high-school, I thought it would only
take a few weeks. 
I spent three years on the project. It even caused me to fail an exam at the end of my
second year of university - that of spherical trigonometry (in the ``further astronomy''
module), because of a stupid computational mistake.
(I was never very good at computations, I must say, ever since I left high school\ldots).
That is why I had to spend a third year in Montpellier to complete
my bachelor's instead of going to Paris right away - the only place, I was told, where I
would be able to find people aware of what was considered important in Mathematics.
My informant, Mister Soula, assured me that the last problem left in mathematics had been
resolved twenty or thirty years ago by a so-called Lebesgue. 
He had apparently developed (funny coincidence!) a theory of measure and integration which
brought \emph{point final} to mathematics. 

Mister Soula, my ``diff calc'' teacher, was a benevolent man who took a liking to me. I
was still not convinced by his claim. 
\marginpar{p. P5}
There must have already been, within me, the
prescience that mathematics is a thing which is infinite in scope and depth. Does the sea
have a ``\emph{point final}''? Yet I never thought of looking for that book by Lebesgue
which Mister Soula had told me about, and he probably never held it either. 
In my mind there was nothing in common between anything a book contained and
the work that \textbf{I} had been doing, in my own way, in order to answer questions which 
intrigued me. 

\section{The importance of being alone}

When I finally made contact with the mathematical world in Paris, one or two years later,
I ended up learning, among many other things, that the work which I had been doing
independently, and with the means at hand, was (essentially) what ``everybody'' 
knew as the ``Lebesgue theory of measure and integration''. According to the two or three 
experts to whom I mentioned my work (or even showed a manuscript), I had just wasted my
time redoing something ``already known''. I actually do not recall being disappointed. 
At that moment, the idea of receiving ``credit'', or even simply receiving
approbation for the work that I was doing, must have still been foreign to my mind. 
Furthermore, my energy was completely taken by the process of familiarizing myself with an
entirely different milieu and mostly learning what was considered in Paris to be the basic
toolkit of the mathematician.\footnote{I briefly narrate this rough transition period in
the first part of  
R\'ecoltes et Semailles (ReS I), in the section ``The Welcome Stranger'' (nb. 9).}

Yet, thinking back to those three years, I realized that they were not in any way wasted.
Unknowingly, I learned in solitude what is essential 
to the work of a mathematician - something no master could 
truly teach. Without ever having been told, without ever having to encounter someone with
whom I could share my quest for understanding, I knew ``in my gut'' that I was a
mathematician: somebody who ``does'' math, in its fullest sense - the way one makes
``love''. Mathematics had become, for me, a mistress always 
accommodating my desires. These years of solitude laid the foundation for a trust that has
never been shaken - not by the discovery (upon arrival in Paris at age 20) of the scope of
my ignorance and the vastness of what I had to learn; nor (more than 20 years later) by
the eventful episode of my permanent departure from the mathematical world; nor, in these
last few years, by the often crazy episodes of a ``Funeral'' (anticipated 
\marginpar{p. P6}
and cleanly executed)
of my person and life's work, orchestrated by those who used to be my closest
companions\ldots 

To phrase it differently: I learned in those crucial years to ``be alone''.\footnote{
This formulation is somewhat clumsy. I never had to ``learn to be alone'', for the simple
reason that I never \textbf{unlearned} during the course my childhood, 
this innate skill which I had since birth, just as we all do. Yet these three years of
solitary work, during which I could walk to my own beat, 
following my own exigence criteria, confirmed within me a degree of trust and tranquil confidence 
in my relationship with mathematics which owed nothing to the reining trends and
consensus. I make allusion to these again in the note ``Roots and Solitude'' 
(Res IV, \no $171_3$) notably page 
% \todo{1080}.
.
} That is, I learned to approach the things which I want to know with my own eyes, rather
than rely on the expressed or implicit ideas that eminate from the group with which I
identify, or a group to which I attribute authority.
An unspoken consensus told me, both in high school and in university, that there was no
need to question the notion of ``volume'', which was presented as ``well-known'',
``self-evident'', ``unproblematic''.
Naturally I turned a blind eye to this consensus - just as Lebesgue, a few decades
earlier, had to \textbf{turn a blind eye}.
It is in this act of ``turning a blind eye'', of being oneself rather than the mere
expression of the reigning consensus, of not to remain inscribed within the imperative
circle to which they assign us - it is within this solitary act, above all else, that
``\textbf{creation}'' lies. Everything else comes after.

In the following years, within the mathematical world which welcomed me, I had the opportunity 
to meet multiple people, both older and younger, which were clearly more brilliant,
``gifted'' than I was. I admired the facility with which they learned new notions, as if
at play, juggling them as if they had known them their whole life - while I felt
heavy-handed and clumsy, laboriously making my way, akin to a mole, through an amorphous
mountain of important things (or so I was told) which I had to learn, despite having no
sense of their ins and out. Actually, I was far from the brilliant student who aced every prestigious
\emph{concours} and assimilating at once the most prohibitive courses.

Many of my more brilliant peers went on to become competent famous mathematicians. In
hindsight, after 30-35 years, it does not seem to me that they left a deep imprint
\marginpar{p. P7}
upon the mathematics of today. 
They did things, often times beautiful things, in a pre-existing context which they would
never have considered altering.
They unknowingly remained prisoners in their imperious circles, which
delimitate the Universe of 
a given time and milieu. In order to overcome them, they would have had to rediscover
within them the ability which they had since birth, just as I did: the capacity to be
alone. 

The small child has no difficulty being alone. He is solitary by nature, even though he
enjoys the occasional company, and knows when to ask for mom's permission teat. And he
knows, without having ever been told, that the teat is his, and that he \textbf{knows}
how to drink. 
Yet often times we lose touch with out inner child. 
And thus we constantly miss out on the best 
without even seeing it\ldots

If in R\'ecoltes et Semailles I address somebody other than myself, it is not a
``public''. I address myself to you, reader, as I would a \textbf{person}, and a person
\textbf{alone}. 
It is to the person inside of you that knows how to be alone, the child, with whom I would
like to speak, and nobody else. 
I am aware that the child is often far away.
He has gone through all sorts of things for quite some time. 
He went hiding god knows where, and it can be hard, often times, to get to him.
One could swear that he has been dead forever, or rather that he has never existed - and
yet I am sure that he is there somewhere, well alive. 

I know too what the \textbf{sign} 
is that I am being heard. 
It is when, beyond all cultural and
experiential differences, what I share about my personal experiences echos within you and finds 
resonance; when you find within it \textbf{your own life}, your own self-experience, in a
new light which you may never have considered before that. It is not about an
``identification'' with something or someone far from you. 
Rather, perhaps, you will rediscover a bit of your own life, or of what is
\textbf{closest} to you, as you follow my own rediscovery of myself throughout 
R\'ecoltes et Semailles, 
including within these very pages which I am currently writing. 

\section{The inner journey - or myth and testimony}

Before all else, R\'ecoltes et Semailles
is a \textbf{reflection} upon myself and my life. 
Because of this, it is also a \textbf{testimony}, in two distinct ways. It is a testimony
about my \textbf{past}, upon which the principle component of the reflection is concerned
with. But it is also, at the same time, a testimony about the immediate present, about the
very moment at which I am writing, and during which R\'ecoltes et Semailles is born,
\marginpar{p. P8}
in the course of hours, nights, and days. 
These pages serve as faithful witnesses to a long meditation upon my life, such as it was
really carried out (and continues to be carried out at this very moment\ldots).

These pages have no literary pretense 
They only constitute a document about myself. I only allowed myself to modify them within 
very narrow bounds\footnote{Thus, the occasional rectification of mistakes (material
and of viewpoint) does not appear in the first pass but rather in 
footnotes or in later reconsideration.} (notably for occasional stylistic edits). If there
is pretense, it is only that of faithfulness. And that is saying a lot. 

This document is also far from an ``autobiography''. 
You will learn neither my date of birth (which would be of little relevance unless one is
making astrological predictions), nor the names of my mother and father 
or what they did in life, nor the names of the person who was my spouse and other women
who have been very important in my life, nor those of the children that were born from
these unions, nor what these children have made of their lives.
This does not mean that these things were not important in my life. Rather, the way this
reflection on myself was engaged and developed never incited me to give a description of
those things, which I lightly touch on here and there, but never take the time to
consciously flesh them out with names and numbers. It never seemed to me that doing so
would add anything whatsoever to the point which I was making at any given time. (Whereas
in the few pages above I was brought, almost inadvertently, 
to include perhaps more material details on my life than
you will find in the thousand pages to come\ldots)

And if you were to ask me what ``point'' I have attempted to make over the course of
these thousand pages, I would answer: it is to narrate, and thereby \textbf{discover}, 
the \textbf{inner journey} that my life has been and still is. 
This narrative/testimony of a journey
is happening simultaneously at the two level which I have mentioned above. There is the
exploration of a journey past, of its roots and of its origin, tracing all the way back to
my childhood. And there is the continuation and renewal of this ``very'' journey, over the
course of the days during which I am writing  R\'ecoltes et Semailles
in spontaneous response to a violent stimulus coming from the outside
world.\footnote{For more details about this ``violent stimulus'', see ``The Letter'',
notably sections 3 through 8.
% \todo{add links}
} 

\marginpar{p. P9}
External facts come to nourish the reflection, only to the extent they they induce and
provoke new
developments in my inner journey, or help clarify it. 
The burial and the plunder of my mathematical work, of which I will speak at length, has
been such a provocation. It awoke in me
a host of powerful reactions, and at the same time revealed to me the profound, and
hitherto unknown links that continue to tie me to the work I have created.

It is true that my being ``good at math'' is not necessarily a reason (and even less so a
good reason) for you to be interested in my particular
journey - nor is the fact that I have had trouble with my colleagues, after shifting
milieu and lifestyle.
Colleagues, or even friends abound, who find it ridiculous to publicly spread 
one's ``inner moods'' - as they say.
To them, what matters are ``results''. 
The ``soul'', meaning that within us which
\textbf{witnesses} the production of these result, 
as well as apprehends them in various ways
(as much in the life of the ``producer'' as in those of his peers), is looked down upon,
sometimes even targeted with open derision.
This attitude aims to display some form of modesty, but what I see is
a sign of withdrawal or asynchrony promoted by the very air which we breath.
I do not write for he who is stricken by this latent self disgust, which makes him reject
the best I have to offer. A disdain 
for what truly makes his \textbf{own life}, and for what makes mine: the superficial or
profound, course or subtle motions which animate the psyche, that ``soul'' which lives the
experience and reacts to it, which freezes or blossoms, which retreats or learns\ldots

The narrative of an inner journey can only be told 
by the person living it and no one else. 
Even though the narrative is only aimed
towards oneself, it often times inserts itself 
within the construction of a \textbf{myth}, of which the narrator is the hero. 
Such a myth is born, not in the creative imagination of a people and a culture, but rather
from vanity of he who dared not assume a humble reality, 
but instead substitutes a construction for it. 
But a \textbf{true} narrative (if such a thing exists), of a journey such as it was truly
lived, is to be prized. And this is not because of renown
which is (rightly or wrongly) attributed to the narrator, but simply by virtue of its
\textbf{existence}, and of its truthfulness. Such a testimony is precious,
whether it comes from an illustrious person, a small clerk with no future 
and with family responsibilities, or from a common criminal.

If there is value for one in such a narrative, it is first and foremost that of self
confrontation, through this unvarnished testimony of the experience 
of an other. 
\marginpar{p. P10}
But also (to phrase it differently) to erase within oneself
(be it only for the span of a reading) this disdain by which one holds one's \textbf{own
journey}, and that ``soul'' of which one is both the passenger and the captain\ldots

\section{The painting of mores}

In speaking of my past as a mathematician, and later in discovering (almost against my
will) the twists and turns of the intricacies of the gigantic Burial of my work, I was
brought, inadvertently, 
to paint the picture of a particular milieu and era - an era
affected by the decay of certain values
which provided meaning to the work of individuals.
That is what I mean by
``painting of mores'', centered around a 
``fait divers'' which is thoughtlessly unique in the annals of Science
as I have said rather clearly earlier, I believe, you will not find in 
R\'ecoltes et Semailles
a ``folder'' concerning a certain unordinary ``case'', quickly bringing you up to date.
And yet a friend of mine, looking for such a folder, 
blindly passed by nearly
everything constituting the substance and flesh of 
R\'ecoltes et Semailles.

As I explain in much more detail in the Letter, the ``investigation'' (or the ``painting
of mores'') carries on in 
parts \ref{part:II} and \ref{part:IV},
``The Funeral (1)
- or the robe of the Emperor of China'' and 
``The Funeral (3) - or the Four Operation''. 
Page after page I persistently extract one after another, 
a number of juicy facts (to say the least) which I attempt to ``classify''
bit by bit. Slowly, these fact
assemble into a global painting which progressively 
emerges from the fog, taking on brighter colors and 
sharper contours.
In these daily notes, the raw facts
``which just appeared''
are inextricably mixed with personal
reminiscing, as well as with commentaries and reflections of a psychological,
philosophical, or even (occasionally) mathematical nature.
That's how it is, and there is nothing I can do about it!

Starting with work I had already done, 
which occupied me for over a year, producing 
a sort of ``investigation proceedings'' folder
should not have taken longer than a few hours 
or days worth of work, depending
on the curiosity or demands of the interested reader.
I tried at one point to produce such a folder. 
That is how I started writing a note which was to be called ``the four
operations''.\footnote{The note eventually exploded into part \ref{part:IV}
(also named ``The four operations'') of \rec, comprising about seventy notes 
running over more than four-hundred pages.}
But in the end I could not bring myself to do it!
That is decidedly\marginpar{p. P11}
not my style of expression, and in my old age less so than ever.
I now consider, having written \rec, 
that I have done enough for the benefit of the ``mathematical community'', 
and therefore can leave, without remorse, 
the task of producing 
the necessary ``folder'' to others (in particular to any of my
colleagues who would feel concerned). 

\section{The heirs and the builders}

It is now time for 
me to say a few words about my mathematical work, which has 
played an important role in my life, and continues to do so (to my own surprise).
I come back to this work more than once in 
\rec\ - sometimes in a way 
that should be
understandable by all, 
and other times in slightly more technical terms.\footnote{One will also find here and
there, in addition to mathematical notes concerning my
previous work, sections containing new mathematical developments. The longest of these is
``the five pictures (crystals and $\cD$-modules)'' in ReS IV, note \no 171 (ix).
% \todo{reference}
} The latter 
will mostly go ``above the heads'', not only of the ``profane'', but also of the
mathematical colleague who may not be completely ``in'' the field in question.
One can of course feel free to skip the sections which seem too ``involved''. Just as one can
try to go through them, glimpsing as one goes, a shadow of the ``mysterious beauty'' (in
the words of a non-mathematician friend of mine) of the universe of mathematical things,
appearing as a multitude of ``strange inaccessible islands'' in the vast 
moving waters of reflection\ldots

Most mathematicians, as I mentioned earlier, are inclined to constrain themselves to a
conceptual framework, a ``\textbf{universe}'' fixed once and for all - the one,
essentially, which they have found ``ready made'' at the time of their studies.
They are like the heirs of a large and beautiful fully-furnished house, with its lounges,
kitchens, workshops, and its kitchenware and tools left and right, with which there is, I
trust, plenty to cook and tinker.
How this house was built, progressively over the course of multiple generations, 
and how and why these tools (and not others\ldots) were conceived and built, 
why the pieces are disposed and organized in such a way - these are all questions the
heirs would never think of asking themselves.
This is the ``universe'', the ``given'', in which we must live, and that is that!
Something which appears massive (and which most of the time we have only been able to
partially explore), yet at the same time \textbf{familiar}, and mostly:
\textbf{immutable}.
They mostly busy themselves with maintaining or embellishing a patrimony: fixing a faulty
piece of furniture, restoring a facade, sharpening a tool, or even sometimes, for the most
enterprising, building an entire workshop, or a whole new piece of furniture. It even
happens,\marginpar{p. P12}
when they fully commit to the task, that the piece of furniture is truly
beautiful, so that the whole house appears embellished by its addition. 

Even more rarely, one of them will consider 
modifying one of the main tools, or even, 
under repeated and insistent pressure or need, to imagine and build a whole new tool. 
And in so doing, he often feels on the brink of profusely apologizing for what he feels is
infringing on the piety owed to the familial tradition, which he has disturbed through his
brazen innovation. 

In most of the rooms of the house, the windows and shutters are carefully closed, probably
on account of a fear that a foreign wind would blow in.
And when the pretty new furnishings, here and there, together with their progeny, begin to
clutter the rooms and invade the corridors, none of these heirs will agree to face the
fact that his familiar and cozy Universe is beginning to feel cramped.
Rather than come to terms with such a fact, most will prefer to 
awkwardly slither,
and try not to get trapped, some between a Louis XV buffet and a
rocking chair in rattan, others between a boisterous toddler
and an Egyptian sarcophagus, while others, as a last resort, will try to climb over a
heteroclite and crumbling pile of chairs and benches\ldots

The picture I have just sketched is not unique to the world of mathematicians. It
illustrates the deeply engrained and immemorial conditioning which 
one encounters in every milieu and sphere of human activity, regardless, as far as I can
tell, of the society or era in question.
I have mentioned such a phenomenon already, and I do not in any way pretend to fall
outside of its influence. 
As will be clear from my testimony, the contrary is true.
It only happens to be the case that at the relatively limited level of the act of
intellectual creation, I was barely affected\footnote{I believe the main reason for such
immunity is a certain favorable climate which surrounded me until age 5, the note ``The
innocence'' (ReS III,\no 107).
% \todo{reference}
} by this conditioning which may be called 
``cultural blindness'' - the incapacity to see (and to evolve) outside of the ``Universe''
fixed by the surrounding culture.

As for myself, I feel that I belong to the lineage of mathematicians whose spontaneous
vocation and joy was to continuously construct new houses\footnote{This archetypal 
picture of a ``house''
to be built, 
surfaces and is formulated for the first time in ``Yin, the Servant, and the new masters''
(ReS III , \no 135).
% \todo{reference}
} In so doing, they cannot help but invent\marginpar{p. P13}
all of the required tools, utensils, and 
furnishings for both the construction of the house from its foundation, and to fill
the kitchens and workshops of the house in abundance, so that one may live in it
comfortably. Yet, once everything down 
to the last sapling and stool has been taken care of, the builder rarely lingers on the
premises, of which every stone and every piece of wood carries a trace of the hand which
shaped and placed it. The builder's place lies not in the quietude of fully 
finished universes, however welcoming and harmonious they may be, whether they are a
product of his own hands or those of his predecessors. His place is in the open air. 
He is friends with the wind, and does not fear solitude at work for weeks, years or,
if need be, for an entire lifetime if no welcome succession presents itself.
Just like everybody else, the builder only has two hands - but two hands which at each
moment know what they need to do, which refuse neither the largest nor the most delicate
tasks, and which never tire of comprehending, again and again, the multitude of things 
which become them. Two hands might be few, given that the World is infinite. 
They will never exhaust it! And yet, two hands can be a lot\ldots

History is not my strong suit, but if I had to give a list of mathematicians inscribed in
this lineage, names that spontaneously come to my mind are those of Galois, Riemann (from
the past century), and Hilbert (at the beginning of the current century). If I were to
name a candidate among the elders which welcomed me into the mathematical
world,\footnote{I speak of these beginnings in the section ``The welcome stranger'' (ReS I
, \no 9).} the name
of Jean Leray comes to my mind before any other, even though my contact with him has
always been episodic.\footnote{I was nonetheless (following H. Cartan and G. Serre) one of
the first users and promoters of one of the great innovative notions introduced by Leray,
that of a sheaf, which has been an essential tool throughout my work as a geometer. It is
also the notion which has provided me with the key to enlarge the notion of a topological
space into that of topos, about which I will be speaking later.

Leray differs from the portrayal I have given of the ``builder'', I believe, in that he
does not seem drawn to ``construct houses from their foundations to their completion''.
Rather be was compelled to lay out vast foundations, in places where nobody would have
thought to look while leaving to others the care of carrying the construction to its
completion, and once the house is built, to settle into the premises (be it only for a
short time)\ldots}

\marginpar{p. P14}
I have just roughly sketched two pictures: that of the ``homebody'' mathematician, who
is content with maintaining and embellishing a heritage, and that of the
builder-pioneer,\footnote{I have just surreptitiously attached herein two qualifiers with 
male connotation (that of ``builder'' and that of ``pioneer''), which express very
different aspects of the impulse of discovery, one which is of a nature more delicate than
what these qualifiers might evoke. Such a discussion will be carried out later in this
walk-reflection, in the step ``The discovery of the Mother - or the two versants''
(\no 17).}
who is drawn to repeatedly crossing these ``invisible and imperious circles'' which
delimitate a given Universe.\footnote{At the same time, and without really meaning to do
so, the builder-pioneer assigns to the old Universe (if not for himself, at the very least
for his more sedentary colleagues) new boundaries, thereby inscribing 
circled which may be larger, but are just as invisible and imperious as those which they
have come to replace.} These two groups
may also be called, somewhat bluntly but also suggestively, ``conservatives'' and
``innovators''. Both have their raison d'\^etre, in one collective adventure that is
carried out through the generations, through centuries and millennia.
During the fruitful periods of a science or art there is neither opposition, nor is there 
antagonism among these two temperament.\footnote{Such has been the case, notably in the
mathematical world, during the period (1948-1969) of which I was a direct witness, as I
myself belonged to that world. Following my departure in 1970, there seems to have been a
large scale reaction, a sort of ``consensus of disdain'' for the ``ideas'' in general, and
more specifically the great innovative ideas that I have introduced.} 
They are different and mutually complementary,
just as dough and yeast. 

Between these two extremes (not at all opposed by nature), one can find a plethora of
intermediary temperament. There is the ``homebody'' that would never think of leaving a
familiar dwelling, and would be even less willing to take on the task of building another,
god knows where, yet will not hesitate, when the house gets cramped to build a basement,
raise the ceiling, or even, if need be, to build a dependency of modest
proportions.\footnote{Most of my ``elders'' (about whom I speak for instance in ``a
welcome debt'' (Introduction \no 10)) conform to this intermediary temperament. I was
thinking notably about Henri Cartan, Claude Chevalley, Andr\'e Weil, Jean-Pierre Serre,
Laurent Schwartz. With the exception maybe of Weil, they have all turned a ``sympathetic
eye'', without ``concern nor secret reprobation'' towards
the solitary adventures into which they saw me embark.} Without being a builder at heart,
he will often view with a sympathetic eye, or at the very least without concern nor secret
reprobation towards another who had shared the same dwelling, and who is already out and
about assembling beams 
\marginpar{p. P15}
and stones in some impossible boonies, with the confidence of somebody
who already sees a castle\ldots 

\section{Viewpoint and vision}

Allow me to return to myself and my work.

If I excelled in the art of mathematics,
it was not through the ability and perseverance to solve problems left by my predecessors,
but rather through a natural tendency within me to 
discover \textbf{questions},
evidently crucial, yet that nobody had yet seen, or to excavate the \textbf{``right notions''}
that were missing
(often without anyone realizing until the new notion appeared), as well as the
\textbf{``right statements''} of which nobody had thought.
Often, notions and statements mesh in such a perfect way, that there can be no doubt in my
mind as to their validity (give or take small adjustments at most) - so that often, when
it boils down to ``travail sur pi\'eces'' destined for publication, I refrain from going
further, and from taking the time to flesh out a proof that often, once the statement and
its context are well-understood, consists of no more than a matter of ``trade'', not to
say routine.
Things which solicit ones attention are countless, and it is impossible to follow 
them all to their end! 
The fact remains that carefully proved propositions and theorems in my written and
published work appear in the thousands, and almost all of them have entered the 
patrimony of things commonly accepted as ``known'' and frequently used all over
mathematics.

I am led more towards the discovery of
fertile \textbf{viewpoints} 
than towards the discovery of questions, notions, and statements, by my particular type of
genius, which is constantly leading me to introduce, and more or
less develop, entirely novel \textbf{themes}. 
It is this, I reckon, which is my most essential contribution to the mathematics of my time.
In fact, these innumerable questions, notions, and statements which I just mentioned, only
truly make sense for me once they are subjected to such a ``viewpoint'' - or more
precisely they \textbf{arise} spontaneously from it; in the same way that a light (even a
dim one) appearing in a pitch black night seems to invoke from the shadows
contours which it suddenly reveals to us.
Without this light 
uniting them in a common sheaf, the ten, or one-hundred, or one thousand questions,
notions, statements would appear as a heterogeneous amorphous pile of 
``mental widgets'' all isolated from one another - 
\marginpar{p. P16}
rather than as the many parts of a
\textbf{Whole} which, while perhaps remaining invisible, 
escaping within the folds of the night,
is nonetheless clearly felt.

The fertile viewpoint is that which reveals to us, organized as the many living parts of a
common Whole, 
enveloping them and giving them meaning, these pressing questions which no one had asked,
and (as if in response, perhaps, to these questions) these extremely natural notions which
nobody had thought of expressing, and these statements finally which seem to immediately
follow, and which nobody had dared to conjecture, for as long as the questions which
brought them about, and the notions that allowed us to formulate them had remained hidden.
Even more than what we call ``key theorems'' in mathematics, it is the fertile viewpoints
which, in our art,\footnote{Such a phenomenon is not exclusive to ``our art'', but (it
seems to me) it appears in every act of discovery, at the very least when such an act
happens at the level of intellectual reckoning.} constitute the most powerful tools of
discovery - or rather, they are
not tools, but they are the very eyes of the researcher who passionately strives to
understand the nature of mathematical things. 

Thus, the fertile viewpoint provides us with an ``eye'' which at once helps us 
\textbf{discover}, and helps us \textbf{recognize the unity} of 
the multiplicity of what is discovered. And such unity is truly the very life and breath
which connects and animates these discoveries. 

But just as the word itself suggests, a ``viewpoint'' by itself remains fragmentary. 
It reveals to us \textbf{one of the aspects}
of a scenery or panorama, among a multiplicity of others which are equally valuable,
equally ``real''. It is when complementary viewpoints of a common reality 
are conjugated, that is, when our ``eyes'' are multiplied, that the gaze is able to
penetrate further ahead in the reckoning of things. 
The richer and more complex the reality which we desire to know, the more important it is
to be equipped with several ``eyes''\footnote{Every viewpoint leads to the development of
a language which is best suited to expressing it. Having several ``eyes'' or several
``viewpoints'' to apprehend a situation, also means (at least in mathematics) having 
several different languages to tackle the situation.} in order to apprehend it in all its ampleness and
subtlety. 

By virtue of our innate ability to grasp the ``multiple'' as the \textbf{One},
it also happens, sometimes, that a sheaf of viewpoints 
converging to a unique and vast scenery, gives rise to a novel thing; a thing which
transcends each of the partial perspectives, in the same way that a living being
transcends\marginpar{p. P17}
each of its limbs and organs. This new thing could be called a \textbf{vision}.
The vision unites the known viewpoints which constitute it, while also revealing to us
other viewpoints that were previously ignored, just as the fertile viewpoint makes us
discover and apprehend, as part of the same Whole, a multiplicity of new questions,
notions, statements. 

To say this in another way: the vision is to the viewpoints, 
from which it seems to arise and which it unites,
as the clear and warm daylight is to the various components of the solar spectrum.
A vast and profound vision is like an inexhaustible source, made to inspire and guide the
work, not only of the one within whom the vision was once conceived and who made himself
its servant, but that of generations, fascinated perhaps (as he first was) 
by these distant horizons which it lets us glimpse\ldots

% \marginpar{p. P1}
% p .
% c ,
% q "
% pa (
% b \textbb{}
