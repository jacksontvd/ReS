\begin{comment}
\documentclass{book}
\usepackage{master}
\usepackage{changepage}
\newcommand{\rec}{$\text{R\'ecoltes et Semailles}$}
\newcommand{\no}{n$^\circ$}
\hfuzz = 100pt

% NOTE and SUBNOTE FORMATTING
\usepackage{titlesec}
\usepackage[dotinlabels]{titletoc}

% define `note'
\titleclass{\nnote}{straight}[\section]
\newcounter{nnote}
\renewcommand{\thennote}{\thesection.\arabic{nnote}}
\titlespacing*{\nnote}{0pt}{3.5ex plus 1ex minus .2ex}{1ex plus .2ex}
\titleformat{\nnote}[runin]{\bfseries }{\bfseries Note }{0pt}{}[]

% define `subnote'
\titleclass{\subnote}{straight}[\section]
\newcounter{subnote}
\renewcommand{\thesubnote}{\thesection.\arabic{subnote}}
\titlespacing*{\subnote}{0pt}{3.5ex plus 1ex minus .2ex}{1ex plus .2ex}
\titleformat{\subnote}[runin]{\bfseries }{\bfseries Note }{0pt}{}[]

% the first optional arg sets the size of the indentation in the TOC
\titlecontents{nnote}[9em]
{}{Note }{}{\titlerule*[1pc]{.}\contentspage}

% the first optional arg sets the size of the indentation in the TOC
\titlecontents{subnote}[11em]
{}{Note }{}{\titlerule*[1pc]{.}\contentspage}

\newtheorem{remark}{Remark}
\newtheorem*{remark*}{Remark}

\begin{document}

% print table of contents with notes and subnotes
 %\setcounter{tocdepth}{5}
 %\setcounter{secnumdepth}{5}
 %\startcontents[chapters]
 %\printcontents[chapters]{}{0}{}

\setcounter{chapter}{16}

\end{comment}

\chapter{The defunct (who still lives...)}
\label{chapter:17}

\section{The incident - or body and mind}

\nnote{98}\label{note:98}\marginpar{p. 421}(September 22) The latest Burial note (with the exception of a few footnotes) dates from May 24 - that is, four months ago. The two weeks that followed, up until June 10, were mostly devoted to re-reading, completing, and adjusting the already written notes here and there, putting aside a visit by Zoghman Mebkhout for a day or two, having come to read the Burial notes in their entirety and to share his comments with me, before I was to entrust him with the typing. I thought that the definitive manuscript would be ready around early June, and that it would be typeset and printed before the summer holidays (although that might have been overly optimistic...). I liked the idea of sending out my ``five hundred page letter" before the commotion of the start of the holidays!

In actuality, the text of the Burial is still not complete as I am writing these lines: today as was the case four months ago, two or three last notes remain to be written - plus an additional one\footnote{(September 23) In fact, it appears that this ``note" eventually split into three distinct ones (n$^o$s 99-101)} whose need started being felt since then: this is the note that I am now writing, and which is intended to serve as a brief summary of what has happened since then. 

On June 10$^{th}$, a new unforeseen event intervened during the writing of R\'ecoltes et Semailles - a process that had already been full of unexpected turns: I fell ill! A stitch suddenly appeared (catching me completely unaware) and peremptorily forced me to lie in bed, leaving me with no other choice. The very act of standing or sitting up became arduous, and only when lying down could I feel relatively at ease. It was very silly, and occurred right at a time when I was about to conclude and file away an urgent project! There was no way for me to use the typewriter while lying down, and handwriting in this position was not much of a sinecure either...

It took me nearly two weeks to face the obvious, all the while trying to continue working against all odds: my body was exhausted and insistently demanded complete rest, but it kept on falling on deaf ears.

I had such a difficult time hearing my body's plea because my mind had remained fresh and alert throughout, eager to carry its momentum forward, acting as if it had a life of its own, entirely independent of the rest of my body. It was so fresh and nimble that it struggled mightily to accept my body's need for sleep, pushing me to the limits of exhaustion by persistently refusing the takeover of sleep, the annihilator of the spinning mind!



\section{The trap - or ease and exhaustion}

\nnote{99}\label{note:99} (September 23) I had to force myself to stop working yesterday night, so as to avoid following my natural course until two or three a.m. and thereby becoming caught in a cycle that I know only too well. I felt fresh and receptive, and had I followed my natural inclination, I could have kept going until sunrise! The trap laid by intellectual work - at least when such work is pursued with passion, in a subject in which we eventually feel like fish in water, as a result of a prolonged familiarity - is that it is so incredibly \textbf{easy}. We pull, and pull, and it keeps coming; at most, we occasionally feel some amount of effort, some friction, indicating a little bit of resistance...

Yet, I remember the persistent feeling of heaviness and gravity which I used to feel during my early years as a mathematician, a feeling which I had to surmount through obstinate effort, only to be left with a sensation of fatigue. This mostly corresponded to a period in my life during which I was working with an incomplete, or even inadequate, toolbox; as well as to a later period during which I had to more or less laboriously acquire new tools ``left and right", under the pressure of an environment (essentially, that of the Bourbaki group) where they were routinely used - and I had to do so even though I did not perceive these tools' ``raison d'\^etre" until a later point, up to several years later in some cases. I have spoken on occasion about these rather arduous years (see ``The welcome stranger" s.9, and ``One hundred irons in the fire - or: there is no use in being stuck!", note n$^o$10) in the first part of R\'ecoltes et Semailles. This period mostly took place between the years 1945 and 1955, coinciding with my\marginpar{p. 429} time as a functional analyst. (It seems to me that the students whom I later supervised, between 1960 and 1970, experienced much less resistance than I did regarding having to learn new things without sufficient motivation and to absorb notions and techniques based on the elders' authority, taken on faith - in fact, I did not perceive any resistance at all.)

Coming back to my original point, it was from the year 1955 onward that I started having the impression that I was ``flying" - in that doing mathematics felt like play, unencumbered by any sensation of effort - the same way my elders did, exhibiting a quasi-miraculous ease for which I had once envied them, considering that such facility was out of reach for my modest and heavy person! Today, I understand that such a ``facility" is not a privilege one acquires as some exceptional gift (as seemed to be the case for certain individuals, at a time when such a ``gift" appeared to be entirely absent in my case); rather, this facility emerges on its own as the fruit of the union of a passionate interest for a given subject (such as mathematics) together with a more or less lengthy familiarization with said subject. If a ``gift" does play a role in the appearance of such ease, it is through the length of time it takes to reach perfect fluency in one's work on a given subject\footnote{(*) I nonetheless know several mathematicians having each made profound contributions, yet having never seemed to produce the impression of ease and ``facility" which I am writing about - they seem to be constantly grappling with a ubiquitous gravity, which they must surmount with effort at every step. For one reason or another, the aforementioned ``natural fruit" did not ``appear on its own" for these eminent figures the way it was supposed to. This goes to show that not every union is destined to bear the fruits that we may have expected...}(*), something which can vary from one person to the next (as well as from one occasion to the next for the same individual, admittedly...).

The fact remains that the more time passes - the more years I have accumulated as a mathematician, the more I experience this feeling of ``ease" when doing mathematics - the sensation that things are pleading to be revealed to us, if we only take the time to look at them and inspect them a little bit. This ease is not a matter of technical virtuosity - it is clear to me that, in this respect, I am in much worse shape than I was in 1970, at the time when I ``left math". Since then, I have mostly had the occasion to unlearn what I had learned, and I only ``do math" sporadically, on my own, in a very different spirit and on themes different from the ones that I used to work on (at least at first glance). I am also not saying that I would need only assign myself a given famous problem (such as, say, Fermat's last theorem, the Riemann hypothesis, or the Poincar\'e conjecture) so as to launch on a geodesic path towards its solution, in a year or two, or even\footnote{(*) I nonetheless know several mathematicians having each made profound contributions, yet having never seemed to produce the impression of ease and ``facility" which I am writing about - they seem to be constantly grappling with a ubiquitous gravity, which they must surmount with effort at every step. For one reason or another, the aforementioned ``natural fruit" did not ``appear on its own" for these eminent figures the way it was supposed to. This goes to show that not every union is destined to bear the fruits that we may have expected...}(*)\todo{The footnote seems to be repeated twice - can we doube check this with any other edition of ReS?}\marginpar{p. 430} three! The ease that I am speaking of does not apply to a process whereby we set out to reach a given \textbf{goal}, fixed ahead of time - such as proving some conjecture or finding a counter-example... Rather, it applies to excursions into the unknown, following a direction which some obscure instinct tells us will be fruitful, and supported by the inner confidence, never misplaced, that each day and each hour of our journey shall bring us its own fresh harvest of new understandings. \textbf{Which} understanding we shall reach in the morrow, or even in the following hour, is something we can indeed anticipate - yet, it is the constant rectification of this ``anticipation", and the suspense that results, which constantly draw us forward, as the very things which we are probing are inviting us to come closer. What is eventually understood always surpasses what had been anticipated in terms of precision, flavor, and richness; at which point the known readily turns into a new starting point and as raw material allowing us to form a renewed anticipation, causing us to launch further into the unknown that is avidly awaiting to become understood. In this game of discovery, the \textbf{direction} which we follow at every moment is known, but the \textbf{goal} is forgotten - supposing we had even started with a goal which we intended to reach. This ``goal" actually constituted a \textbf{starting point}, issued from some combination of our ambition and ignorance; it served its purpose by motivating ``the boss", fixing an initial direction, and setting the game in action; but the game itself has no use for a goal. As long as the journey we undertake lasts for longer than a day or two, so that we are in it for the long haul, the things which will be revealed to us over the course of days and months, and the places we will be reaching at the term of a series of unknown twists and turns, are a complete mystery to us as travelers; in fact, they form such a distant and inaccessible mystery so as to be immaterial! If the traveler does look out at the horizon, it is not to guess what the unpredictable endpoint shall be, and even less so to decide on a chosen endpoint, but rather to take stock of where he currently stands, and to choose, among the directions available to him in the continuation of his journey, the one which he feels to be most burning...

Such is the ``incredible ease" which I alluded to earlier, regarding intellectual work proceeding entirely in an intellectual direction, such as mathematics. It is neither \textbf{held back} by inner \textbf{resistance}\footnote{(*) I nonetheless know a remarkably gifted mathematician whose relationship to mathematics is typically conflictual, impeded at every step by powerful resistances such as the fear that a given expectation (in the form, say, of a conjecture) would turn out to be false. Such resistances can sometimes culminate in a state of full-out intellectual paralysis. Compare this with the preceding footnote.}(*) (as is often the case with meditation in the way that\marginpar{p. 431} I practice it) nor by a \textbf{physical effort} that needs to be supplied and which would generate a feeling of fatigue, eventually culminating in an unequivocal stop signal. \textbf{Intellectual} effort (if we can even speak of an ``effort", once we have reached a stage where the only remaining ``resistance" is the time factor...) does not seem to generate either intellectual or physical tiredness. More precisely, if physical ``tiredness" does occur, it is not experienced as such, other than through occasional soreness resulting from having remained in a fixed sitting position for too long, or other incidental annoyances of this kind. Such soreness can easily be remedied by simply changing position. Horizontal positions have the unfortunate virtue of alleviating the soreness, so as to enable a continuation of the intellectual work instead of some much needed sleep!

There exists nonetheless, as I eventually came to realize, a subtler and more insidious physical ``tiredness" that muscular or nervous tiredness, which manifests itself through an inescapable need for rest and sleep. The word ``exhaustion" (rather than ``tiredness") would be more accurate in this case, although it should be understood that I am not using this word in its common sense, namely that of an extreme feeling of fatigue, manifesting itself notably through a sensation of great effort accompanying one's attempt to even stand up, walk a few steps, etc... Rather, I am talking about an ``exhaustion" of the body's energy for the benefit of the brain, manifesting itself through a gradual degradation of the body's general ``tonicity" and of its level of vital energy. By exhaustion as a result of excessive intellectual work I mean: work that is not compensated by sufficient physical activity, the latter of which generates bodily fatigue and a need for rest. This type of exhaustion is gradual and \textbf{cumulative}. Its effects must depend on both the \textbf{intensity} and the \textbf{duration} of the intellectual activity over the course of a given period. At the level of intensity at which I pursue intellectual work, and at my current age and constitution, it seems that the cumulative exhaustion in question reaches a critical and dangerous threshold after one or two years of uninterrupted activity uncompensated by a regular physical activity

In a sense, the ``facility" about which I am talking is only apparent. Intense intellectual activity clearly requires considerable energy: and this energy has to be drawn from somewhere, so as to be ``spent" on one's work. It appears that this ``somewhere" pertains to the body, which ``endures" (or rather \textbf{forks out}) as much as it can the (sometimes vertiginous) expenses which the head indulges in carelessly. The normal path towards the recovery of the energy provided by the body is sleep. But when the brain's bulimia begins impinging on\marginpar{p. 432} sleep, one begins digging into the energy-capital without renewing it. The trap and the danger of the ``ease" of intellectual work is that it endlessly incites us to cross this threshold, or to remain past it once it has been crossed, and that all the while this crossing is not brought to our attention by the usual and clear-cut signs of tiredness, or even exhaustion. I now realize that a great vigilance is required so as to detect the moment at which one is approaching and crossing the threshold in question, when one's entire being is engaged in the pursuit of a thrilling adventure. To be able to perceive this energy shortage in one's body requires a state of attentiveness which I often lacked and which few people have. I even doubt that such a state of communion between one's conscious activity and one's body could blossom in anybody during a period that is dominated by a purely intellectual activity and which excludes all physical activity.

Many intellectuals by profession instinctively feel the need for such physical activity, and adapt their life in consequence: gardening, handiwork, mountaineering, boating, sports... Those who, like me, neglected this healthy instinct for the benefit of an overly invasive passion (or an overpowering lethargy), eventually have to pay the price. I have had to pay for my neglect three times over in the past three years, and each time I did so without complaining, or rather with indebtedness, realizing with each episode of sickness that I was only harvesting the fruits of my own negligence, and, moreover, that the episode brought with it a lesson that it alone could provide me. Perhaps the main lesson that the latest episode, which just ended, brought with it, is that it is high time for me to take the lead and make such wake-up calls unnecessary - or, more concretely: that it is high time for me to tend to my garden!

 













%\end{document}
