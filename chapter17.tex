\begin{comment}
\documentclass{book}
\usepackage{master}
\usepackage{changepage}
\newcommand{\rec}{$\text{R\'ecoltes et Semailles}$}
\newcommand{\no}{n$^\circ$}
\hfuzz = 100pt

% NOTE and SUBNOTE FORMATTING
\usepackage{titlesec}
\usepackage[dotinlabels]{titletoc}

% define `note'
\titleclass{\nnote}{straight}[\section]
\newcounter{nnote}
\renewcommand{\thennote}{\thesection.\arabic{nnote}}
\titlespacing*{\nnote}{0pt}{3.5ex plus 1ex minus .2ex}{1ex plus .2ex}
\titleformat{\nnote}[runin]{\bfseries }{\bfseries Note }{0pt}{}[]

% define `subnote'
\titleclass{\subnote}{straight}[\section]
\newcounter{subnote}
\renewcommand{\thesubnote}{\thesection.\arabic{subnote}}
\titlespacing*{\subnote}{0pt}{3.5ex plus 1ex minus .2ex}{1ex plus .2ex}
\titleformat{\subnote}[runin]{\bfseries }{\bfseries Note }{0pt}{}[]

% the first optional arg sets the size of the indentation in the TOC
\titlecontents{nnote}[9em]
{}{Note }{}{\titlerule*[1pc]{.}\contentspage}

% the first optional arg sets the size of the indentation in the TOC
\titlecontents{subnote}[11em]
{}{Note }{}{\titlerule*[1pc]{.}\contentspage}

\newtheorem{remark}{Remark}
\newtheorem*{remark*}{Remark}

\begin{document}

% print table of contents with notes and subnotes
 %\setcounter{tocdepth}{5}
 %\setcounter{secnumdepth}{5}
 %\startcontents[chapters]
 %\printcontents[chapters]{}{0}{}

\setcounter{chapter}{16}

\end{comment}

\chapter{The defunct (who still lives...)}
\label{chapter:17}

\section{The incident - or body and mind}

\nnote{98}\label{note:98}\marginpar{p. 421}(September 22) The latest Burial note (with the exception of a few footnotes) dates from May 24 - that is, four months ago. The two weeks that followed, up until June 10, were mostly devoted to re-reading, completing, and adjusting the already written notes here and there, putting aside a visit by Zoghman Mebkhout for a day or two, having come to read the Burial notes in their entirety and to share his comments with me, before I was to entrust him with the typing. I thought that the definitive manuscript would be ready around early June, and that it would be typeset and printed before the summer holidays (although that might have been overly optimistic...). I liked the idea of sending out my ``five hundred page letter" before the commotion of the start of the holidays!

In actuality, the text of the Burial is still not complete as I am writing these lines: today as was the case four months ago, two or three last notes remain to be written - plus an additional one\footnote{(September 23) In fact, it appears that this ``note" eventually split into three distinct ones (n$^o$s 99-101)} whose need started being felt since then: this is the note that I am now writing, and which is intended to serve as a brief summary of what has happened since then. 

On June 10$^{th}$, a new unforeseen event intervened during the writing of R\'ecoltes et Semailles - a process that had already been full of unexpected turns: I fell ill! A stitch suddenly appeared (catching me completely unaware) and peremptorily forced me to lie in bed, leaving me with no other choice. The very act of standing or sitting up became arduous, and only when lying down could I feel relatively at ease. It was very silly, and occurred right at a time when I was about to conclude and file away an urgent project! There was no way for me to use the typewriter while lying down, and handwriting in this position was not much of a sinecure either...

It took me nearly two weeks to face the obvious, all the while trying to continue working against all odds: my body was exhausted and insistently demanded complete rest, but it kept on falling on deaf ears.

I had such a difficult time hearing my body's plea because my mind had remained fresh and alert throughout, eager to carry its momentum forward, acting as if it had a life of its own, entirely independent of the rest of my body. It was so fresh and nimble that it struggled mightily to accept my body's need for sleep, pushing me to the limits of exhaustion by persistently refusing the takeover of sleep, the annihilator of the spinning mind!

All my life long, until three or four years ago, my limitless capacity for recovery through a profound and prolonged sleeping period had been a reliable and salutary counterpart to my sometimes excessive energy expenditures: provided one's sleep is reliable, there is nothing to fear, and one can (somewhat safely) launch into full-blown work orgies until exhaustion follows - with the confidence that these excesses can always be balanced out by restorative sleep! This capacity, which all my life had come to me as naturally as my capacity for work and for discovery (the latter two surely being intimately connected...), eventually started wearing out over the past few years, sometimes even disappearing altogether, for reasons which I cannot yet fully articulate and which I haven't yet begun to explore. Following a long day spent at my typewriter (or on my handwritten notes), upon yielding to my body's injunctions to stop working and go to bed, I increasingly find myself resuming my reflections while lying down (a position which offers some relief to the tension accumulated from sitting). My mind launches anew, for hours on end if not for the entire night (or whatever remains of it...). Even though I realize that this system is not cost-effective (supposing it is even $\textbf{livable}$ in the long run), in that (in my case at least) a prolonged reflection unsupported by writing ends up going in circles and becoming a sort of rumination - the bad habit has been set, and the tendency is only growing worse. It seems to have become the greatest channel of energy dispersion in my life during these past few years, while other dispersion mechanisms have been eliminated, one by one, over the course of the years.

If this mechanism has taken hold in my life with such tenacity, and if I have been disposed to pay its price for all these years, it must be that something within me found it valuable, and would again find it valuable when the time comes. It is high time that I start examining this situation more closely - something I was on the brink of doing several times in the past four months.

\marginpar{p. 423}This was doubtless an urgent task; but I eventually realized that there was an even more urgent task at hand. That is: I needed to re-establish contact with my body, help it recover from the state of exhaustion which I started feeling and admitting to myself, and thereby recover my lost vigor. I understood that in order to do so, I would have to renounce all intellectual activity for an indefinite period of time - including the very act of reflecting on the meaning of what was happening to me. This long and salutary ``parenthesis" in my large intellectual investments, which for a time (starting in February of this year) had converged in the writing of $\textbf{``R\'ecoltes et Semailles"}$, ends today with the writing of these notes. The present note is my very first reflection, or at least a summative account of this four-month long ``parenthesis".

By the time I finally came to terms with my need for complete rest, what was once great fatigue had turned into profound exhaustion. Refusing to heed my body's peremptory warnings, I persisted in producing a few derisory pages of commentaries and edits to the Burial. These small contributions, wrested from a state of physical fatigue during the first two weeks, were made at what in retrospect appears to be an insane energy cost! Following this prowess, I became bedridden for weeks on end, only standing up for a few hours a day to attend to essential practical tasks.

Remarkably, once I had $\textbf{understood}$ the need for complete rest, I did not experience the slightest difficulty in completely giving up all intellectual activity, nor did I feel the slighted desire to ``cheat". Strictly speaking, I did not even have to make any decision - from the very moment I understood, I let go entirely. Tasks which the previous day had held me breathless suddenly seemed very remote, as if they belonged to a distant past...

This does not mean that my days suddenly became empty. Even though sleep remained hard to come by for weeks and months, so that I lay awake for hours on end, seemingly in complete inaction, I did not once find myself bored. I was renewing contact with my body and my immediate environment - with my room, or with the piece of grass or dried herbs, graced by the sun, which had stood just before my eyes during a short (and cautious...) walk around my house. I spent long moments following the rhythmic flight of a fly through the sun rays, or the peregrinations of an ant or some other green or pink minuscule and translucent critter along endless blades of grass\marginpar{p. 424}, lost amongst dense forests of such blades growing ever more entangled before my eyes. In this particular disposition, brought about by silence and a state of great fatigue, I found myself following with solicitude the hesitant peregrinations of the slightest wind through my guts - it is, in short, a disposition wherein one is led to renew contact with elementary and essential things; wherein one is able to fully measure the benefits of restful sleep, or even the marvel that is being able to piss without hindrance! The simple functioning of the body is nothing short of a miracle, of which we only become somewhat aware (sometimes against our own will) when this functioning finds itself perturbed in one way or another.

It was very clear that, ``technically speaking", what lay at the heart of my ``health problem" was a disturbed sleep. The underlying reasons for this perturbation were and remain mysterious to this day. I tentatively attempted to first and foremost reacquaint myself with sleep, good old deep sleep such as I once knew it, and which was slipping away from my grasp at the time when I most needed it! I only regained it a short while ago. Needless to say, I had no intention to resort to sleeping pills, and even though I occasionally experimented with herbal teas and orange flower water (which I discovered at this occasion), I always viewed them as expedients. On a more serious front, I took advantage of this occasion to introduce important changes to my diet: I reduced starchy foods in favor of fruits and green vegetables (raw or cooked), reintroduced (moderate amounts of) meat as a regular part of my diet, and, most importantly, drastically reduced my consumption of fats and sugars, two food groups whose intake had been systematically imbalanced (for me as for many others living in affluent countries) since at least the end of the war. My step-son Ahmed, who is a practitioner of Chinese medicine and has a very good ``feeling" for this sort of things, helped me a great deal in realizing the importance of such a dietary change in recovering from a perturbed life balance. He was also the one who insisted tirelessly on the importance of significant physical exercise, to the order of a few hours a day, in balancing out intense intellectual activity. The latter otherwise tends to exhaust the body by concentrating the available vital energy to the head, thereby creating a strong yang imbalance.

In fact, Ahmed did not content himself with offering me helpful advice, framed in a yin-yang dialectic to which I am rather attuned, having spent the past four or five years familiarizing myself with the delicate dynamics of such things. As soon as I had recovered enough to be able to garden again\marginpar{p. 425}, I started spending some time attending to my sad-looking small garden. Seeing this, Ahmed took the lead in initiating large-scale projects: clearing new strips of land, displacing soil to them, transplanting and sowing, creating terraces and retaining walls, rearranging the composting pile... Over the course of days and weeks, under the impulsion of my untiring friend, I witnessed the deployment of enough landscaping tasks to keep me occupied for years, if not for the rest of my days!

This was exactly what I needed - and what I need in the long run in order to counterbalance my fiery intellectual activity. In this respect, daily walks - which were suggested to me many times in the past - are of no help: my mind continues to race while walking, as it does in bed, refusing to let itself be bothered by the beauty of the scenery, which I practically traverse without seeing anything! On the other hand, while watering the garden, I feel responsible for attending to its well-being. Hoeing strips of vegetables is even more efficient: I can't help but pay attention to the task and let it permeate me to some extent - I start paying attention to the soil's texture and to how it is affected by hoeing, by vegetable plants and ``weeds" growing in it, by compost and by mulching. I also grow more aware over time of the state of the plants which I am supposed to care for - a state which more or less reflects the amount of attention that I have been able to give them. The activity of gardening, and everything surrounding it, meets two of my strongest aspirations or dispositions: that of witnessing $\textbf{something come out of my actions}$ day after day (which is not the case of walking, and even less so of weightlifting, which was once recommended to me by a colleague and friend...); and that of having the occasion to $\textbf{learn}$ through direct contact with the things with which I am engaging. It seems that I learn best in situations where I am actively ``doing" something - ``something" which is taking form and transforming itself in my hands...

Once my initial state of exhaustion proper had passed, my convalescence seems to have been aided by two types of activity; or rather, my day to day activities, both at home and in the garden, involved two types of important and beneficial aspects. There was first $\textbf{physical effort}$: even when I felt tired and sluggish before starting to work, the ``harder" the work - such as, say, \marginpar{p. 426)}handling a heavy pickaxe or moving large stones around - the better I felt after the fact, taken over by a hearty fatigue. Then, there was my contact with $\textbf{living things}$: the plants that I had to care for; the soil which I had to ready to welcome them, then mulch or hoe; the food which I had to cook and which I ate with as much pleasure as I took preparing my meal; the cat asking for its share of food and affection; the various tools and utensils, all the way down to the uneven and often damaged rocks which had to be maneuvered in every possible position so as to fit together into relatively steady retaining walls... 

Physical effort and contact with living things - here are two aspects of life that are missing from intellectual work, an activity which is therefore by nature incomplete, fragmentary, and borderline dangerous or harmful if it is not completed and balanced out by other things. It was the third time in a little over three years that I came to this realization. At this point, it has become quite clear that I am facing a drastic ultimatum: I can either change my lifestyle, seeking to recover a balance wherein the yin side of my being (my body) is not being constantly neglected at the benefit of my yang side (my mind, or rather, my head), or I can ignore this imbalance and by so doing risk losing my life in the very near future. This is what my body has communicated to me, as clearly as can be! I have now reached a point in my life where the need for a certain amount of elementary ``wisdom" has become a matter of $\textbf{survival}$, in the literal sense of the word. This is surely a good thing - when this wasn't the case, said ``wisdom" found itself perpetually relegated to the back-burner at the profit of my bulimic tendencies when it came to intellectual activity, the latter of which has been one of the dominating forces of my entire adult life.

Once I was faced with such a clear ultimatum: ``change or die!", I did not have to think too hard to come to a decision. This is how I have been able to refrain from all intellectual activity, mathematical or not, for nearly four months, without ever feeling like I was doing violence to myself. I knew, without needing to verbalize it, that a living gardener is worth more than a dead mathematician (or a dead ``philosopher" or ``writer", while we're at it!). With some mischief, one may add: worth more than a living mathematician as well! (But this is an entirely different story...)

I do not believe I will ever find myself again in such an ``extremal" situation, wherein I would have to renounce all intellectual activity, be it mathematical or meditative, for a long period of time. Rather, the most immediate and urgent\marginpar{p. 427} practical task that seems to present itself for the years to come is that of arriving at a life balance wherein these two types of activity, those involving the body and those involving the mind, are able to coexist on a daily basis, without either one ever growing all-consuming and squeezing out the other. There is no hiding the fact that it is in the direction of the ``mind" that my most powerful drives since childhood reside, and that the two principal passions that have occupied me during the past few years belong to that realm. Of these two passions, mathematics and meditation, it seems to me that it is the former which almost exclusively acts as a factor of imbalance in my life - in that it has had the distressing tendency to ``devour" everything around it for its only benefit. It is surely not a coincidence if the three ``sickness episodes" in my life since 1981 which were tied to a state of imbalance took place precisely during periods when my passion for mathematics was at the forefront. 

This was arguably not quite the case for this last episode, which occurred during the writing of R\'ecoltes et Semailles, an activity which largely consisted in internal reflection, not to stay outright meditation. But the fact remains that this reflection on my past as a mathematician was constantly fueled by my passion for mathematics. It seems to me that this was especially true of the writing of the second part, the Burial, wherein the egotistic component of this passion was involved to a particularly strong and persistent extent. Nonetheless, even in retrospect, I am not under the impression that the reflection ever took the devouring or demonic rhythm of the previous two episodes at the term of which my body was forced to emit an unarguable ``enough!". Considered separately from the context of the rest of my life, my intellectual activity during the past year and a half (since ``resuming" the writing of Pursuing Stacks, followed by R\'ecoltes et Semailles) appears to have followed a perfectly reasonable rhythm, wherein I always remembered to eat and drink (although I was sometimes slightly careless with my sleep...). If this period culminated in a third ``health episode" (to use a euphemism), it is surely the result of the lifelong imbalance caused by a domineering heard imposing its rhythm and its rules to a robust body which long withstood the blows without saying a word\footnote{(*) Here I should make an exception for the five years between 1974 and 1978, which were not dominated by some great intellectual project, and during which manual occupations absorbed a non-trivial part of my time and energy.}(*).

\marginpar{p. 428}During the past two months, I have had ample occasions to learn about the irreplaceable benefits of physical work and of the intimate contact with humble living things, the latter of which silently speak to me of simple and essential realities that books and pure thought are unable to teach us. Thanks to this process, I was able to reunite with sleep, a life companion even more precious than food or drink - and through it, I experienced renewed vigor and a robustness which I had once thought gone for good. I was also able to realize that, in the stage of life that is now mine, the only way for me to be able to continue the new mathematical journey upon which I launched last year for a few more years, without endangering my health and my life, is with my two feet solidly planted in the soil of my garden.

It is time for me in the coming months to establish a new lifestyle wherein activities of both body and mind find their place and are conciliated on a day-to-day basis. There is work to be done!


\section{The trap - or ease and exhaustion}

\nnote{99}\label{note:99} (September 23) I had to force myself to stop working yesterday night, so as to avoid following my natural course until two or three a.m. and thereby becoming caught in a cycle that I know only too well. I felt fresh and receptive, and had I followed my natural inclination, I could have kept going until sunrise! The trap laid by intellectual work - at least when such work is pursued with passion, in a subject in which we eventually feel like fish in water, as a result of a prolonged familiarity - is that it is so incredibly \textbf{easy}. We pull, and pull, and it keeps coming; at most, we occasionally feel some amount of effort, some friction, indicating a little bit of resistance...

Yet, I remember the persistent feeling of heaviness and gravity which I used to feel during my early years as a mathematician, a feeling which I had to surmount through obstinate effort, only to be left with a sensation of fatigue. This mostly corresponded to a period in my life during which I was working with an incomplete, or even inadequate, toolbox; as well as to a later period during which I had to more or less laboriously acquire new tools ``left and right", under the pressure of an environment (essentially, that of the Bourbaki group) where they were routinely used - and I had to do so even though I did not perceive these tools' ``raison d'\^etre" until a later point, up to several years later in some cases. I have spoken on occasion about these rather arduous years (see ``The welcome stranger" s.9, and ``One hundred irons in the fire - or: there is no use in being stuck!", note n$^o$10) in the first part of R\'ecoltes et Semailles. This period mostly took place between the years 1945 and 1955, coinciding with my\marginpar{p. 429} time as a functional analyst. (It seems to me that the students whom I later supervised, between 1960 and 1970, experienced much less resistance than I did regarding having to learn new things without sufficient motivation and to absorb notions and techniques based on the elders' authority, taken on faith - in fact, I did not perceive any resistance at all.)

Coming back to my original point, it was from the year 1955 onward that I started having the impression that I was ``flying" - in that doing mathematics felt like play, unencumbered by any sensation of effort - the same way my elders did, exhibiting a quasi-miraculous ease for which I had once envied them, considering that such facility was out of reach for my modest and heavy person! Today, I understand that such a ``facility" is not a privilege one acquires as some exceptional gift (as seemed to be the case for certain individuals, at a time when such a ``gift" appeared to be entirely absent in my case); rather, this facility emerges on its own as the fruit of the union of a passionate interest for a given subject (such as mathematics) together with a more or less lengthy familiarization with said subject. If a ``gift" does play a role in the appearance of such ease, it is through the length of time it takes to reach perfect fluency in one's work on a given subject\footnote{(*) I nonetheless know several mathematicians having each made profound contributions, yet having never seemed to produce the impression of ease and ``facility" which I am writing about - they seem to be constantly grappling with a ubiquitous gravity, which they must surmount with effort at every step. For one reason or another, the aforementioned ``natural fruit" did not ``appear on its own" for these eminent figures the way it was supposed to. This goes to show that not every union is destined to bear the fruits that we may have expected...}(*), something which can vary from one person to the next (as well as from one occasion to the next for the same individual, admittedly...).

The fact remains that the more time passes - the more years I have accumulated as a mathematician, the more I experience this feeling of ``ease" when doing mathematics - the sensation that things are pleading to be revealed to us, if we only take the time to look at them and inspect them a little bit. This ease is not a matter of technical virtuosity - it is clear to me that, in this respect, I am in much worse shape than I was in 1970, at the time when I ``left math". Since then, I have mostly had the occasion to unlearn what I had learned, and I only ``do math" sporadically, on my own, in a very different spirit and on themes different from the ones that I used to work on (at least at first glance). I am also not saying that I would need only assign myself a given famous problem (such as, say, Fermat's last theorem, the Riemann hypothesis, or the Poincar\'e conjecture) so as to launch on a geodesic path towards its solution, in a year or two, or even\footnote{(*) I nonetheless know several mathematicians having each made profound contributions, yet having never seemed to produce the impression of ease and ``facility" which I am writing about - they seem to be constantly grappling with a ubiquitous gravity, which they must surmount with effort at every step. For one reason or another, the aforementioned ``natural fruit" did not ``appear on its own" for these eminent figures the way it was supposed to. This goes to show that not every union is destined to bear the fruits that we may have expected...}(*)\todo{The footnote seems to be repeated twice - can we doube check this with any other edition of ReS?}\marginpar{p. 430} three! The ease that I am speaking of does not apply to a process whereby we set out to reach a given \textbf{goal}, fixed ahead of time - such as proving some conjecture or finding a counter-example... Rather, it applies to excursions into the unknown, following a direction which some obscure instinct tells us will be fruitful, and supported by the inner confidence, never misplaced, that each day and each hour of our journey shall bring us its own fresh harvest of new understandings. \textbf{Which} understanding we shall reach in the morrow, or even in the following hour, is something we can indeed anticipate - yet, it is the constant rectification of this ``anticipation", and the suspense that results, which constantly draw us forward, as the very things which we are probing are inviting us to come closer. What is eventually understood always surpasses what had been anticipated in terms of precision, flavor, and richness; at which point the known readily turns into a new starting point and as raw material allowing us to form a renewed anticipation, causing us to launch further into the unknown that is avidly awaiting to become understood. In this game of discovery, the \textbf{direction} which we follow at every moment is known, but the \textbf{goal} is forgotten - supposing we had even started with a goal which we intended to reach. This ``goal" actually constituted a \textbf{starting point}, issued from some combination of our ambition and ignorance; it served its purpose by motivating ``the boss", fixing an initial direction, and setting the game in action; but the game itself has no use for a goal. As long as the journey we undertake lasts for longer than a day or two, so that we are in it for the long haul, the things which will be revealed to us over the course of days and months, and the places we will be reaching at the term of a series of unknown twists and turns, are a complete mystery to us as travelers; in fact, they form such a distant and inaccessible mystery so as to be immaterial! If the traveler does look out at the horizon, it is not to guess what the unpredictable endpoint shall be, and even less so to decide on a chosen endpoint, but rather to take stock of where he currently stands, and to choose, among the directions available to him in the continuation of his journey, the one which he feels to be most burning...

Such is the ``incredible ease" which I alluded to earlier, regarding intellectual work proceeding entirely in an intellectual direction, such as mathematics. It is neither \textbf{held back} by inner \textbf{resistance}\footnote{(*) I nonetheless know a remarkably gifted mathematician whose relationship to mathematics is typically conflictual, impeded at every step by powerful resistances such as the fear that a given expectation (in the form, say, of a conjecture) would turn out to be false. Such resistances can sometimes culminate in a state of full-out intellectual paralysis. Compare this with the preceding footnote.}(*) (as is often the case with meditation in the way that\marginpar{p. 431} I practice it) nor by a \textbf{physical effort} that needs to be supplied and which would generate a feeling of fatigue, eventually culminating in an unequivocal stop signal. \textbf{Intellectual} effort (if we can even speak of an ``effort", once we have reached a stage where the only remaining ``resistance" is the time factor...) does not seem to generate either intellectual or physical tiredness. More precisely, if physical ``tiredness" does occur, it is not experienced as such, other than through occasional soreness resulting from having remained in a fixed sitting position for too long, or other incidental annoyances of this kind. Such soreness can easily be remedied by simply changing position. Horizontal positions have the unfortunate virtue of alleviating the soreness, so as to enable a continuation of the intellectual work instead of some much needed sleep!

There exists nonetheless, as I eventually came to realize, a subtler and more insidious physical ``tiredness" that muscular or nervous tiredness, which manifests itself through an inescapable need for rest and sleep. The word ``exhaustion" (rather than ``tiredness") would be more accurate in this case, although it should be understood that I am not using this word in its common sense, namely that of an extreme feeling of fatigue, manifesting itself notably through a sensation of great effort accompanying one's attempt to even stand up, walk a few steps, etc... Rather, I am talking about an ``exhaustion" of the body's energy for the benefit of the brain, manifesting itself through a gradual degradation of the body's general ``tonicity" and of its level of vital energy. By exhaustion as a result of excessive intellectual work I mean: work that is not compensated by sufficient physical activity, the latter of which generates bodily fatigue and a need for rest. This type of exhaustion is gradual and \textbf{cumulative}. Its effects must depend on both the \textbf{intensity} and the \textbf{duration} of the intellectual activity over the course of a given period. At the level of intensity at which I pursue intellectual work, and at my current age and constitution, it seems that the cumulative exhaustion in question reaches a critical and dangerous threshold after one or two years of uninterrupted activity uncompensated by a regular physical activity

In a sense, the ``facility" about which I am talking is only apparent. Intense intellectual activity clearly requires considerable energy: and this energy has to be drawn from somewhere, so as to be ``spent" on one's work. It appears that this ``somewhere" pertains to the body, which ``endures" (or rather \textbf{forks out}) as much as it can the (sometimes vertiginous) expenses which the head indulges in carelessly. The normal path towards the recovery of the energy provided by the body is sleep. But when the brain's bulimia begins impinging on\marginpar{p. 432} sleep, one begins digging into the energy-capital without renewing it. The trap and the danger of the ``ease" of intellectual work is that it endlessly incites us to cross this threshold, or to remain past it once it has been crossed, and that all the while this crossing is not brought to our attention by the usual and clear-cut signs of tiredness, or even exhaustion. I now realize that a great vigilance is required so as to detect the moment at which one is approaching and crossing the threshold in question, when one's entire being is engaged in the pursuit of a thrilling adventure. To be able to perceive this energy shortage in one's body requires a state of attentiveness which I often lacked and which few people have. I even doubt that such a state of communion between one's conscious activity and one's body could blossom in anybody during a period that is dominated by a purely intellectual activity and which excludes all physical activity.

Many intellectuals by profession instinctively feel the need for such physical activity, and adapt their life in consequence: gardening, handiwork, mountaineering, boating, sports... Those who, like me, neglected this healthy instinct for the benefit of an overly invasive passion (or an overpowering lethargy), eventually have to pay the price. I have had to pay for my neglect three times over in the past three years, and each time I did so without complaining, or rather with indebtedness, realizing with each episode of sickness that I was only harvesting the fruits of my own negligence, and, moreover, that the episode brought with it a lesson that it alone could provide me. Perhaps the main lesson that the latest episode, which just ended, brought with it, is that it is high time for me to take the lead and make such wake-up calls unnecessary - or, more concretely: that it is high time for me to tend to my garden!

 \section{A farewell to Claude Chevalley}
 
 \nnote{97}\label{note:97} In both today's and yesterday's reflection, I voluntarily set aside an event that took place during my sickness-episode - in early July, at a time when I was still bedridden. I am referring to the death of Claude Chevalley.
 
 I learned about it in a vague article from Lib\'eration which was more or less dedicated to the event; the article was passed on to me by a friend who thought I might find it interesting. Almost nothing was said of Chevalley himself, save for some generic sentences about Bourbaki, of which he was a founding member. It had been months since I felt on the brink of wrapping up R\'ecoltes et Semailles, at which point I would be able to bring a freshly made typed, printed, and bound copy to him in Paris\marginpar{p. 433}! He was the one person in the world who I was certain would read my thick volume with real interest, and with regular pleasure - and I wasn't at all sure that there was anyone else who would share his enthusiasm!
 
 Since the start of the reflection, I had realized that Chevalley had contributed something to my process, at a crucial point in my itinerary - that he had sowed a seed during a period of turmoil, which had later germinated in silence. The reason I felt a connection to him was not so much because of a $\textbf{feeling}$, say of gratitude, sympathy, or affection. These feelings were definitely present, as they remain present towards some of the ``elders" who welcomed me in their midst more than twenty years ago. However, what sets my relationship to Chevalley apart from my relationships with my elders and with most of my friends, not to say all of them, is of a different nature. It is the feeling, or rather, the perception, of an essential $\textbf{kinship}$ between us, reaching beyond cultural differences or conditionings of all kinds that have affected us since infancy. I could not say whether or not this ``kinship" comes through in the parts of my reflection devoted to him\footnote{(*) See ``Encounter with Claude Chevalley - or: liberty and good intentions" (section 11), as well as the last paragraph of ``Merit and disdain"}\todo{Cross-check titles, add references}(*). During the period of my life to which these segments refer, Chevalley features as perhaps more of an ``elder", regarding his understanding of the elementary facts of life, than as a ``relative". Such is the distance which my later maturing must have reduced, or even abolished, as had been the case for a long time on the mathematical front regarding my relationship to him and to some of my elders. In attempting to verbalize the meaning behind this kinship, or at least one of its signs, the following comes to mind: we are both ``lone riders" by nature - engaged on our own ``solitary adventure". I write about mine in the last (and eponymous) chapter of ``Fatuity and Renewal"\footnote{(**) See especially, for this purpose, the two sections ``The forbidden fruit" and ``The solitary adventure", n$^o$s 46 and 47.}\todo{add references}(**). For those who knew Chevalley (and perhaps even for others), this part of the reflection may be more adequate in capturing what I am trying to express than the one which explicitly concerns him.
 
Had I been able to meet with him and talk to him again, I could have reached a better understanding than the one I currently have of our friendship and our essential kinship, as well as our differences. Other than Pierre Deligne\marginpar{p. 434}, Claude Chevalley was the person to whom I was most looking forward to giving a copy of R\'ecoltes et Semailles by hand. He was also the one person whose comments, had they been mischievous or sarcastic, would have carried the most weight. On that fated day in early July, I knew at once that I would never have the pleasure of bringing him what I had best to offer, nor would I ever hear the sound of his voice again.

Strangely - and adding to how $\textbf{stupid}$ the news of his death made me feel - Chevalley's health troubles had come to my mind more than once during the preceding months, at times when the prospect of a future encounter with him arose in my mind. I felt a certain anxiety, which I constantly brushed away, at the thought that this meeting may never occur, that my friend might disappear before I could go see him. The idea had of course occurred to me to write or call him, if only to inquire about his health and well-being, to share a few words with him about the work I had been doing, and to communicate my intention to come share this work with him. The fact that I dismissed this idea as being silly and inopportune (thinking that there was no reason for me to... etc.), as one so frequently does in situations of this kind, serves to illustrate the extent to which I, like many others, continue to live ``below my means" - in the sense that I continue to repress the obscure prescience of things conveying knowledge in a whisper that I am too busy and too lazy to hear...
 
 \section{Surface and depth}
 
 
 
 













%\end{document}
