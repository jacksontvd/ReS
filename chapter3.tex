\begin{comment}
\documentclass{book}
\usepackage{master}
\newcommand{\rec}{$\text{R\'ecoltes et Semailles}$}
\newcommand{\no}{n$^\circ$}
\hfuzz = 100pt
\begin{document}
\end{comment}

May 1985

\section{The one-thousand page letter}

\marginpar{p. L1}The text which I am hereby sending you, of which a limited number of copies were typed and printed by my university, is neither an off-print, nor a preprint. Its title, R\'ecoltes et Semailles, makes this much clear. I am sending it to you the way I would send a long letter - including the personal dimension, indeed. If I have decided to send it to you, rather than await for you to learn about it some day (if your curiosity leads you to it) in the form of some publicly available volume in a library (if there even exists an editor crazy enough to engage in such an adventure...), it is because I am addressing this letter to you more than to others. I have thought of you more than once in the course of writing it - I must say that I have now been writing this letter for more than a year, and devoting all my energy to the task. It is a gift I am making you, and I took great care in the process to give out what I had best to offer (at any given moment). I do not know whether or not you will welcome this gift - until your response (or absence thereof) brings me the answer. 

At the same time as I am sending you R\'ecoltes et Semailles, I am also sending it to all of the colleagues, friends, and (ex-)students of the mathematical world with whom I was close at one time or another, as well as to those who appear in my reflection in some form, both named and unnamed. There is a chance that you yourself appear in what follows, and if you make the effort to read it not only with your eyes and head but also with your heart, you will surely recognize yourself even in places where you are not explicitly named. I am also sending R\'ecoltes et Semailles to a handful of other friends, both inside and outside the scientific community.

This ``letter of introduction" which you are currently reading, which announces and introduces a ``one-thousand page letter" (to begin with...), will also serve as a Foreword. The latter has not yet been written at the time of writing these words. Additionally, R\'ecoltes et Semailles consists of five parts (including an introduction ``with drawers"). I am hereby sending you parts I (Fatuity and Renewal), II (The Burial (1) - or the Robe of The Emperor of China), and IV (The Burial (3) - or the Four Operations)\footnote{I am singling out colleagues who appear in my reflection in some way, but who I do not know personally. To those I am only sending ``The Four Operations" (which particularly concerns them), as well as ``booklet O" which consists of the present letter together with the Introduction to R\'ecoltes et Semailles (as well as the detailed table of contents for the first four parts).}. These are the parts which seemed to concern you in particular. Part III (The Burial (2) - or the Key to the Yin and Yang) is the most personal segment of my testimony, and at the same time it is the part which, more so than\marginpar{p. L2} the others, appears to me to hold ``universal" value, beyond the particular circumstances that surrounded its creation. I refer to that part in various places in part IV (The Four Operations), which can nonetheless be read independently, and even (to a large extent) independently of the three parts which precede it\footnote{(*) More generally, you will notice that each ``section" (in Fatuity and Renewal) and each ``note" (in any of the following three parts of R\'ecoltes et Semailles) has its own unity and autonomy. Each can be read independently of the rest, the way one can find interest and stimulation in simply observing a hand, a foot, a finger, a toe, or any part, large or small, of the human body, without forgetting that it is part of a Whole, and that it is only with respect to that Whole (which remains unspoken) that the part takes on its full meaning.}(*) If reading what I have sent you prompts you to respond (as is my wish), and if it makes you want to read the missing part as well, do let me know. It will be my pleasure to send it to you, as long as your response makes it clear that your interest goes beyond superficial curiosity.

\section{Birth of R\'ecoltes et Semailles (a lightning fast retrospective)}

In this pre-letter, I would like to tell you in the span of a few pages (if at all possible) what R\'ecoltes et Semailles is about - to do so in more details than can manage the subtitle: ``Reflections and testimony about the past of a mathematician" (my past, as you will have guessed...). R\'ecoltes et Semailles contains many things, and many will see them as several different things: a \textbf{voyage} of discovery through the past; a \textbf{meditation} on existence; a \textbf{painting of mores} of a milieu and an era (or a painting of the insidious and unstoppable transition from one era to the next...); an \textbf{investigation} (almost police-like at times, and elsewhere approaching the style of a cloak-and-dagger novel taking place in the underbelly of the mathematical megapolis...); a vast \textbf{mathematical divagation} (which will lose more than one reader...); a practical treaty in applied psychoanalysis (or, alternatively, a book of \textbf{``psychoanalytic fiction"}); a eulogy of \textbf{self-knowledge}; \textbf{``My confessions"}; a private \textbf{diary}; a psychological study of the processes of \textbf{discovery} and \textbf{creation}; an \textbf{indictment} (unforgiving, as it must...), or even a \textbf{settling of accounts} in the mathematical ``beau monde" (in which no punches are pulled...). If there is one thing I can guarantee, it is that I never once was bored in the process of writing it, and that from it I have learned and seen a great deal. If you find the time to read it among your other important duties,\marginpar{p. L3} I doubt that you will be bored. Unless you force yourself to read it, who knows...

As such, this work is not only meant to be read by mathematicians. It is true that certain parts of it will cater to mathematicians more than to others. In this pre-letter to R\'ecoltes et Semailles, I would like to summarize and highlight what it is, then, which concerns you in particular as a mathematician. The most natural way to go about this is to simply tell you how, one thing leading to another, I progressively got around to writing the four or five ``tomes" mentioned above.

As you know, I left the ``grand monde" of mathematics in the year 1970, following an issue of military funding at my home institution (the IHES). After a few years devoted to anti-military and ecological activism, in the style of a ``cultural revolution" - about which you probably received echoes here and there - I practically disappeared from the public sphere, and settled at a remote provincial university. Rumor has it that I am spending my days herding sheep and digging wells. The truth is that, in addition to several other occupations, I am bravely continuing to give university lecture, like everybody else (teaching was my first source of income, and it remains so to this day). It even just so happened that I would sometimes spend a few days, or even a few weeks or a few months doing mathematics again - I have filled several boxes with my doodles, which I am probably the only person in a position to decrypt. However - at least at first sight - these projects revolved around very different topics than the ones I used to work on. Between the years 1955 and 1970, my theme of predilection had been cohomology, more specifically the cohomology of varieties of all kinds (and of algebraic varieties in particular). I considered that I had done enough work in that direction for others to carry on without my help, and decided that if I were to continue doing mathematics, I might as well change things up...

In 1976, a new passion appeared in my life, one about which I felt as strongly as I once felt about mathematics, and which is closely related to the latter. It was a passion for what I call ``meditation" (things need to be named after all). This name, like any other, risks leading to countless misunderstandings. As for mathematics, the process of meditation is one of discovery. I express myself on this subject at various points in R\'ecoltes et Semailles. It visibly held enough in store to keep me occupied for the rest of my life. In fact, I have more than once gotten to thinking that mathematics was now a matter of the past, and that it was time for me to orient myself towards more serious matters - time to ``meditate".

\marginpar{p. L4}I nonetheless ended up facing the evidence (four years ago) that my passion for mathematics was still very much alive. In fact, to my own surprise, and despite my long-standing conviction (for almost fifteen years) that I would never publish a single new line of mathematics in my lifetime, I found myself suddenly engaged in the writing of a mathematical project which seemed never-ending and would require producing volume after volume; and while I was at it, I might as well just write out all that I had to say about mathematics in an (infinite?) series of books which would be called ``Mathematical Reflections", and that would be that.

This project began two years ago, during the spring of 1983. I was then already too busy writing (volume 1 of) ``Pursuing Stacks" (``\`A la Poursuite des Champs" in French), which also was to constitute volume 1 of the (mathematical) ``Reflections", to stop and reflect on what was happening to me. Nine months later, as is fit, this first volume was virtually complete, and all that was left for me to do was to write the introduction, re-read everything, add annotations - and off it could be sent to be printed... 

The volume in question is still not finished to this day - it hasn't moved by an inch for the past year and a half. The introduction that I had left to write grew to take over twelve hundred pages (typewritten), and when all will be said and done I estimate it will be some fourteen hundred pages long. You will have guessed that this ``introduction" is nothing but R\'ecoltes et Semailles. Last I checked, it was supposed to constitute volumes 1 and 2, as well as part of volume 3 of the much-feted ``series" in the works. The latter had to undergo a name change and is now called ``Reflections" (i.e. not necessarily mathematical). The remainder of volume 3 will consist mostly of mathematical writings, ones which I deem more pressing than Pursuing Stacks. The latter can wait until next year for me to come around to adding annotations, an index, and, of course, an introduction...

End of the first Act!

\section{The death of the boss - or abandoned construction sites}

I sense that it is time for me to provide some explanations as to why I so abruptly left a world in which I had apparently felt at ease for more than twenty years of my life; why I had the strange idea of ``coming back" (like a ghost...) when everybody seemed to have been doing just fine without me for the past fifteen years; finally, as to why the introduction to a mathematical work of six or seven hundred pages grew in turn to reach the length of twelve (or fourteen) hundred pages. As I cut to the chase, I will doubtlessly sadden you (sorry!), perhaps even upset you. For you, as I once did, surely prefer to see through ``rose-colored glasses" the milieu to which you belong, the one in which you have found your place, your name and so on. I know what this is like... And what follows might cause some teeth grinding...

\marginpar{p. L5}I mention the episode of my departure at various places in R\'ecoltes et Semailles, without lingering on it. This ``departure" serves rather as an important rupture in my life as a mathematician - it is in relation to that ``reference point" that the events of my life as a mathematician arrange themselves, taking place on either side of a ``before" and an ``after". A very strong \textbf{shock} was needed to uproot me from a milieu in which I was firmly entrenched, and from a clearly delineated ``trajectory". This shock came in the form of a confrontation, within a milieu with which I strongly identified myself, with a certain kind of corruption\footnote{I am referring to the open collaboration, ``establishments" at the head, of scientists from all of the world's countries with military institutions, as a convenient source of funding, prestige, and power. This question is barely scratched in passing, once or twice, in R\'ecoltes et Semailles, such as for instance in the note ``Respect" from April 2$^{nd}$ of last year (n$^o$ 179, pages 1221-1223).
%todo: ref
} which I had chosen to ignore up to that point (by simply abstaining from participating in it). In hindsight, I eventually realized that beyond that singular event, there was a deeper force at work within me, signaling an intense \textbf{need for an inner renewal}. Such a renewal could not be accomplished nor pursued in the lukewarm atmosphere that is the scientific vacuum of an institution of high-standing. Behind me lay twenty years of intense mathematical creativity and of mathematical devotion beyond measure - and, at the same time, twenty years of spiritual stagnation, ``in a silo"... Without realizing it, I was suffocating - what I needed was some fresh air! My providential ``departure" marked the sudden end of a long stagnation period, and it also marked the first step taken towards an equalization of the deep forces at play within my being, which were folded and screwed in a state of intense disequilibrium, frozen in place... This departure was, truly, a \textbf{new start} - the first step in a new journey...

As I said earlier, my passion for mathematics nonetheless remains alive. In recent times, it has found expression in the form of reflections that have remained sporadic, and going in directions which are different from those that I had been working on ``before". As to the \textbf{body of work} which I was leaving behind, what I had produced ``before", including both published texts and, perhaps even more importantly, material which hadn't yet reached the stage of writing and publication - it could almost appear as if it had effectively become detached from my person - and it seemed so to me. Up until last year, with the beginning of R\'ecoltes et Semailles, I never thought of ever ``weighing in" on the scattered echoes that reached me here and there.
%todo: think about use of ``weighing in" for ``poser"
I knew that all that I had done in mathematics, and in particular what I had produced during my \marginpar{p. L6}``geometric" period between 1955 and 1970, were things that \textbf{had} to be done - and that the things which I had seen or glimpsed were things that \textbf{had} to appear, that \textbf{must} be brought to light. Additionally, I knew that the work which I had done, as well as the work which was done under my direction, was work well done, and that I had applied myself to it entirely. I had devoted all of my strength and love to it, and (or so it seemed to me) it could henceforth carry on autonomously - as a living and vigorous body which no longer needed to rely on my parental care. On this front, I left with a perfectly clear conscience. There was no doubt in my mind that the written and unwritten things which I was leaving behind were in good hands, put under the care of others who would make sure that they would deploy themselves, grow and multiply following the intrinsic nature of living and vigorous things.

During these fifteen years of intense mathematical work, a vast \textbf{unifying vision} had hatched, matured, and grown within me, taking the form of a handful of very simple \textbf{id\'ees-force}. It was the vision of an ``arithmetic geometry", a synthesis of topology, geometry (both algebraic and analytic), and arithmetic, a first embryo of which I had found in the Weil conjectures. This vision was my principal source of inspiration during those years, a period which I mostly remember as the one during which I managed to formulate the key ideas of this novel geometry, and to develop some of its main tools. Over time, this vision and these ``id\'ees-force" became second nature to me. (And this feeling of ``second nature" persists in me to this day, despite having ceased all contact with these ideas for nearly fifteen years!) I found them to be so simple, so obvious, that it was natural for ``everyone else" to internalize them and to make them their own over time, at the same time as I went through these motions myself. It is only recently, during the past few months, that I realized that neither this vision, nor the ``id\'ees-force" which had been my constant guides, could be found in writing in any existing publication, save perhaps for tacit appearances between the lines. Most importantly, I also realized that this vision which I thought I had imparted to others, and the ``id\'ees forces" which carry it, remained ignored by all to this day, twenty years after having reached their full maturity. I alone, the worker and servant to these things which I had the privilege of discovering, remain the sole vessel in which they have remained alive.

Some tool or other which I have crafted will find itself used in various places to ``break open" a problem reputed for being difficult, the way one would break open a safe. The tool is visibly solid. Yet, I am aware of a  ``force" it has other than that of a crowbar. The tool is part of a Whole, just as a limb\marginpar{p. L7} is part of a body - it is part of a Whole from which it is issued, which gives it its full meaning and infuses it with strength and life energy. Granted, you can use a bone (if it is big enough) to break open a skull. But such is not its true function, its ``raison d'\^etre". Yet, I am witnessing these tools being dispersed, grabbed by one person or another, like bones being carefully butchered and cleaned, after being torn from a body - a living body that they are pretending to ignore...

What I am hereby spelling out in carefully chosen terms, at the term of a long reflection, I probably first noticed progressively and vaguely, over the course of successive years. It first occurred to me at the level of the unformulated which does not yet seek to take the form of a thought or conscious image, nor that of clearly articulated speech. I had decided that the past, after all, no longer concerned me. The echoes that occasionally reached me, although filtered, were nonetheless eloquent. I had considered myself a worker among others, busying myself on five or six ``construction sites"\footnote{I speak about these deserted ``construction sites", and I eventually list them, in the series of notes ``The deserted construction sites" (n$^o$176 through 178) written three months ago. A year prior, before the discovery of the Burial, I had already touched on this, in the first note in which I resume contact with my previous work and its recent course, titled ``My orphans" (n$^o$46).} in full swing - a more experienced worker perhaps, the senior who for many years was the only person working on these sites, waiting for a welcome succession; senior, perhaps, but not fundamentally different from the others. And yet, upon his departure, it was as if a masonry enterprise had gone bankrupt, following the unexpected passing of the boss: from one day to the next, so to speak, the construction sites were deserted. The ``workers" were gone, each of them carrying some small gewgaw which they thought may be of use at a later time. The cash register was gone, and as such there was no longer any reason for them to tire themselves out at work...

The above is once again a formulation that is the result of a slow process of decantation, the end product of a reflection and an investigation that took place over the course of more than a year. Yet, this was surely something I already felt ``on some level" in the first few years following my departure. Putting aside Deligne's work on the absolute values of Frobenius eigenvalues (the ``million dollar question", from what I have recently gathered...) - whenever I happened to come across a close acquaintance from yesteryear, someone with whom I had worked on the same construction sites, and asked them ``so... ?", I was always met with the same eloquent gesture, arms in the air\marginpar{p. L8} as if asking for grace... Visibly, they were all too busy working on other things which were more important than the projects which were close to my heart - and, just as visibly, while they were all chugging along with an occupied and important air, nothing much was actually being done. The essential feature of mathematical work had disappeared - the presence of a \textbf{unity} which gave meaning to each partial task, and also, I believe, the presence of a \textbf{warmth}. What remained was a scattered collection of tasks unattached to a whole, with each worker hiding their little bounty away in a corner, or scrambling for a way to bring it to fruition. 

I couldn't help but to feel sorrow over the fact that everything seemed to have stopped in its tracks; I no longer heard news about motives, topoi, the six functor formalism, De Rham and Hodge coefficients, nor about the ``mysterious functor" which was supposed to unite under one umbrella De Rham and $l$-adic coefficients for all prime numbers, nor about crystals (except to learn that they remained at a standstill), nor about the ``standard conjectures" and other conjectures which I had formulated and which, evidently, represented crucial questions. Even the vast foundational work begun in the El\'ements de G\'eom\'etrie Alg\'ebrique (with the unflagging help of Dieudonn\'e), which one needed only push along a set track, was left behind: everybody was content to simply settle between the four walls and amidst the furniture that someone else had patiently assembled, built, and polished. With the worker gone, nobody had the inspiration to rise up in turn and to get their hands dirty, in order to construct the many buildings that were yet to be erected, \textbf{houses} which would be good to live in, for oneself and for others...

I once again couldn't resist rendering on the page these fully conscious images which emerged and came to light as a result of a work of reflection. There is no doubt in my mind that these images were already present in some form in the deeper layers of my being. I must have already felt the insidious reality of a \textbf{Burial} of my life's work and of my person on April 19$^{th}$ of the previous year - it suddenly appeared to me on that day, with undeniable strength and under that very name, ``The Burial". Yet, at the conscious level, I never felt offended nor even afflicted. Whatever a person, close to me or not, chose to do with their time was entirely their concern. If what had once motivated them or inspired them no longer did, that was their business, not mine. If the same shift seemed to happen, without fault, to every single one of my ex-students, it was yet again each of their personal business\marginpar{p. L9}, and I had other things to do than to start looking for an explanation, and that was that! As for the things which I had left behind, and to which a profound and ignored link continued to tie me - despite their visible state of abandon, left on deserted construction sites - I knew that they needn't fear ``the assault of time" nor the fluctuations of fashion. Even if they had not yet entered the common patrimony (something which I mistakenly believed had already happened), they would inevitably take root eventually, be it in ten years or in a hundred years, no matter...

\section{A wind of burial...}

Even though I chose to ignore the diffuse perception of a large scale Burial for years on end, this burial nonetheless obstinately returned to haunt me in different guises, less innocuous than that of a mere  disaffection with my work. I slowly came to realize, in ways I could not quite explain, that many of the constituent notions of the forgotten vision had not only fallen into disuse, they had also become, within a certain ``beau monde", the objects of condescending disdain. Such was notably the case for the crucial and unifying notion of topos, which lays at the very heart of the novel geometry - and which provides a common geometric intuition for topology, algebraic geometry, and arithmetic. The notion of topos was also pivotal to my formulation of the \'etale and $l$-adic cohomology tools, as well as to the key ideas (since then more or less forgotten, admittedly...) underlying crystalline cohomology. In fact, it was my very name which, insidiously, mysteriously and over the course of the years, had become an object of derision - becoming a synonym for rambling discourses ad infinitum (such as those I produced on the famous ``topoi", or on these ``motives" that I kept dwelling on and which nobody had ever seen...), for counting angels dancing on the head of a pin for thousands of pages on ends, and for plethoric and gigantic discussions about things which everyone already knew anyway without having to read about them somewhere...  Such was the tone being held, albeit in muted voices, by means of innuendos, and with all of the delicacy that is in order among ``high-minded people of esteemed company". 

During the reflection pursued in R\'ecoltes et Semailles, I believe I was able to point towards the deep forces at play in various characters, forces which are responsible for the airs of derision and condescension they tend to display when confronted with a work whose scope, life and breath are beyond them. I have also discovered (apart from the particular traits of my person which have influenced my work and my fate) the secret \textbf{``catalyst"} who incited\marginpar{p. L10} these forces to take the form of brazen contempt in the face of eloquent signs of an intact creativity; the Chief Funeral Officer, in sum, of this Burial muffled by derision and contempt. Strangely, this catalyst was also the person to whom I was the closest - the only one who eventually assimilated and made his own a certain vision, full of life and of intense power. But I am getting ahead of myself...

Truth be told, these ``whiffs of subtle derision" which reached me here and there did not affect me that much. They remained in a way anonymous, up until three or four years ago. I saw in them a sign of somewhat bleak times, but they did not appear to be directly targeting me, eliciting neither anxiety nor concern. What did affect me more directly were signs of a distancing away from my person which I received from several of my old friends in the mathematical world, friends to whom I continued (my departure from a mutual world notwithstanding) to feel connected through sympathy, in addition to the links created by a shared passion and a common past. Yet, here again, even though such signs pained me, I never stopped to look further into them, and the thought never occurred to me (as far as I remember) to connect the dots between these three series of signs: the abandoned construction sites (and the forgotten vision), the ``wind of derision", and the distancing of many old friends from my person. I wrote to each of these friends, and none responded. It is actually no longer a rare occurence, nowadays, for letters that I write to old friends of students about things which I hold close to my heart to remain unanswered. New times, new ways - what was there to be done? I simply stopped writing them. Yet (if you are one of them) this letter will be the exception, a word which is once again directed your way, and it will be up to you to decide whether to welcome it this time around, or to shut it off once more...

If I remember correctly, the first signs that certain old friends were distancing themselves away from me trace back to 1976. This was the year during which another ``series" of signs started to appear, which I would like to presently mention, before going back to R\'ecoltes et Semailles. To be more precise, these two series of signs appeared jointly. At the time of writing, it seems to me that they are in fact inextricably linked, that they are in essence two aspects or ``faces" of a single reality which came into being that year in the field of my own life. The aspect which I am about to address concerns a systematic ``stonewalling"\marginpar{p. L11}, muted and with no reply, directed under a ``flawless consensus"\footnote{This ``flawless consensus" is mentioned sporadically in Fatuity and Renewal, and it eventually becomes the object of a circumstantial testimony and a reflection in the following part, The Burial (1), with the ``Procession X" or ``The Funeral Service", consisting of the ``coffin-notes" (n$^o$93-96) and the note ``The Gravedigger - or the entire Congregation". 
%todo:ref
 The latter concludes this part of R\'ecoltes et Semailles, and at the same time constitutes the first culmination of the ``second breath" of the reflection.} towards some students and ex-students \textbf{post}-1970 who, through their work, their style, and their inspiration clearly bore the mark of my influence. It was perhaps on this occasion that I first perceived the ``whiff of subtle derision" which, through them, targeted a certain style and \textbf{approach} to mathematics - a style and a vision which (according to a consensus which had apparently already become universal in the mathematical establishment) \textbf{had no place in mathematics}.

This was again something which was clearly perceived at an unconscious level. During the same year, it ended up becoming visible to my conscious attention, in the wake of an aberrant scenario (regarding the impossibility of publishing a thesis which was visibly brilliant) that had happened five times over, with the burlesque obstination of a circus gag. Thinking back to these events, I realize that a certain reality was ``giving me a sign" with benevolent insistence, while I continued to play deaf: ``Hey, look this way goofy, pay some attention to what's happening right here under your nose, it concerns you I promise...!!". I slightly arose from my torpor, look over (for an instant), half-bewildered half-distracted: ``oh yes, well, it's a little strange, it sure looks like somebody's after somebody else, something must have gone wrong, and with such a perfect set I must say that's quite hard to believe!".

It was so hard to believe that I scurried to forget about both the gag and the circus. I must say that I had several other interesting occupations at hand. This didn't prevent the circus from returning to my attention in the following years - no longer in the form of gags this time around, but rather showing a certain penchant for humiliation, or for full-on punches in the face; with the caveat that we are amongst distinguished people and that the punch in the face therefore also comes in most distinguished forms, necessarily, albeit just as effective - the choice of a form was left to the discretion of the distinguished people in question...

\marginpar{p. L12}The episode which felt like a ``full-on punch in the face" (of someone else) took place in October 1981\footnote{This episode is recounted in the note ``Coffin 3 - or the slightly-too-relative jacobians" (n$^o$95), notably on pages 404-406
%todo: ref
.}. This time around, and for the first time since frequent signs of a new spirit had started reaching me, I was affected - certainly more so than I would have been had I been the one who was punched, rather than someone else whom I held in affection. This someone else was a student of mine, as well as a remarkably gifted mathematician, who had just accomplished beautiful things - but that was just a detail. What wasn't a detail, on the other hand, was that three of my students from ``before' showed direct solidarity in favor of an act which the person in question received (rightly) as a humiliation and an affront. Another two of my old students had already treated him with condescension, taking the air of well-off people chasing away a good-for-nothing\footnote{This occurrence is mentioned in passing in the note mentioned in the previous footnote.}. Yet another student was to follow suit three years later (again in the style ``punch in the face") - but of course that was something I did not yet know. What occupied me at the time was amply sufficient. It was as if my past as a mathematician, never once examined, was suddenly taunting me in the form of hideous rictus on behalf of five of my ex-students, who had gone on to become important, powerful and disdainful characters...

There had never been a better time to probe in the direction of what was suddenly calling my attention with such violence. But something inside me had decided (without ever saying it out loud...) that this past from ``before" did not really concern me after all, that there was no reason to look further into it; that if I was under the impression that it was calling me with a voice that I knew well - that of the time of contempt - there must have been a mistake. Yet, I was overcome with anxiety for days on end, maybe even weeks, without deciding to act upon it. (It was only last year, in the course of writing R\'ecoltes et Semailles and thereby returning to this episode, that I became aware of this anxiety which had been put under control as soon as it had appeared.) Instead of noticing it and probing it, I became agitated and I wrote left and right the ``letters that were in order". The parties involved even took the time to respond, naturally sending me evasive responses which did not go past the surface of anything. The waves eventually grew calmer, and everything returned to order. I never had to think back on any of it, until last year. When I did, I nonetheless noticed that there had remained a wound of sorts, or rather\marginpar{p. L13} a painful splinter which one avoids touching; a splinter which \textbf{maintains} a wound which is trying to heal itself... 

This was surely the most painful and difficult of my experiences in my life as a mathematician - when I was shown (without consenting to really \textbf{become aware} of what my eyes were seeing) ``such a cherished student or companion of yesteryear taking pleasure in discreetly crushing another cherished individual in whom he recognizes me". This impacted me even more strongly than the rather crazy discoveries which I made last year, and which (from an outsider's perspective) might seem just as incredible... It is true that this experience had brought into play several others, in the same tonalities although slightly less violent, and which as a result had been ``over-seeded".  

I am reminded that this very year 1981 had marked a drastic shift in my relationship with the only one of my ex-students which whom I had maintained regular contact following my departure, as well as the one who for the past fifteen years had taken the role of a ``preferred interlocutor" on the mathematical side of things. This was indeed the year during which the ``signs of an affectation of disdain" which had already appeared in previous years\footnote{This episode is treated in the note ``Two turning points" (n$^o$66).}
%todo: ref 
``suddenly became so brutal" that I stopped all mathematical communication with him. That was a few months before the aforementioned punch-in-the-face episode. In hindsight the coincidence seems striking, but I had never tried to bring the two events together. They were filed in separate ``cabinets"; cabinets which someone had declared, in addition, to be of no consequence - the matter was settled!

Thus also reminds me that a certain \textbf{Symposium} took place during the month of June of the same year 1981, which was memorable on a number of counts - a symposium which would have deserved to enter into History (or what remains of it...) under the indelible name of ``Perverse Symposium". I became aware of it (or rather, it dropped on me!) on May 2$^{nd}$ of last year, two weeks after the discovery (on April 19$^{th}$) of the Burial in the flesh - and I suddenly understood that I had come across \textbf{``the Apotheosis"}. The apotheosis of a burial, but also the \textbf{apotheosis} of the \textbf{scorn} targeted at what, for the more than two thousand years during which our science has existed, was the tacit and unshakeable foundation of the mathematician's code of ethic - namely, the elementary rule that one must not present as his own ideas and results which were taken from somebody else's work. In taking note presently of\marginpar{p. L14} of the remarkable temporal coincidence between two events which at first sight can seem to be of very different nature and scope, I am taken aback by the revelation of the profound and evident link there is between the \textbf{respect} of the \textbf{person}, and the respect of the elementary ethical rules of an art or a science, preventing its practice from turning into a ``free-for-all", and preventing its practitioners, known for their excellence and responsible for setting the tone, from turning into a ``maffia" without scruples. But once again I am getting ahead of myself...

\section{The journey}

I believe I have by now covered most of the context in which my ``return to mathematics", as well as, one thing leading to another, my writing of R\'ecoltes et Semailles took place. It was only in late March of last year, in the last section of `Fatuity and Renewal (``The weight of a past" (n$^o$ 50)), that I started pondering the reasons behind and the meaning of this unexpected return.
%todo: ref
Regarding the ``reasons", the most influential of those was surely the impression, at once vague and imperious , that powerful and vigorous things which I thought I had left in caring hands were actually ``in a tomb, in which they had been left in decay, away from benefits of the wind, the rain, and the sun, for the fifteen years during which they had remained out of sight".\footnote{Quote copied from the note ``The melody by the tomb - or sufficiency" (n$^o$167), page 826.}
%todo: ensure consistency of quote   
I must have understood, over time and without daring to admit it to myself before today, that I was the one that would have to finally break apart the old timbers imprisoning living things that were not made to rot away in locked coffins, but to flourish in the open air. Moreover, the false airs of self-importance and the insidious derision surrounding these upholstered and unwieldy coffins (so as to resemble the bemoaned deceased, surely...) played a role in ``eventually awakening a fighting spirit within me which had fallen into slumber over the past ten years", and in giving me the desire to throw myself into the brawl...\footnote{See ``The weight of a past" (section n$^o$50), notably p.137. (**).}
%todo: ref

These are the circumstances in which, some two years ago, what I thought at first would be a brief oversight of one of these ``construction sites" which I had left behind, a matter of a few days or a couple weeks at most, turned into a great mathematical feuilleton in $N$ volumes, inserting itself into the famous new series of ``Reflections" (``mathematical" reflections, awaiting the removal of this unnecessary qualifier). From the very moment I realized that I was in the process of writing a mathematical work destined for publication, I knew that I would also be including, in addition to a more or less standard ``mathematical" introduction, a second ``introduction"\marginpar{p. L15} of a more personal nature. I sensed that it was important for me to explain the motivations behind my ``return", which wasn't a return to a \textbf{milieu}, but only a ``return" to an intense mathematical activity and to the publication of my own mathematical works, to last for an indeterminate period. I also intended to describe the spirit with which I now approached mathematical writing, one which is in many ways different from the spirit of my earlier writing - the present spirit is closer to that of the ``travel log" accompanying a journey of discovery. And then there were naturally other things weighing on my mind, linked to the above, yet which I felt an even more pressing need to communicate. I went without saying that I was to take my time in expressing what I had to share. I viewed these things, although still diffuse, as crucial in making sense of the volumes which I was about to write, as well as of the ``Reflections" in which they would fit. I wasn't about to surreptitiously slip them in, apologizing to the busy reader in passing for wasting their precious time. If there were things in ``Pursuing Stacks" which they and others had to be aware of, they were precisely the ones which I set out to express in this introduction. If twenty or thirty pages didn't suffice in saying these things, I would write forty, or even fifty - and I didn't intend to force anyone to read me...

Thus was born R\'ecoltes et Semailles. I wrote the first pages of the introduction, meant to be completed by June 1983, during a slow period in the writing of the first volume of Pursuing Stacks. I resumed writing in February of last year, at a time when the volume had been essentially completed for several months\footnote{In the meantime, I spent a fair amount of time thinking about the ``structural surface" for a system of pseudo-lines, obtained in terms of the set of all possible ``relative positions" of a pseudo-line in relation to such a system. I also wrote ``Sketch of a Program" (``Esquisse d'un Programme" in French, which will be included in volume 3 of the Reflections.}. I intended to use this introduction as an opportunity to clarify a few things that remained somewhat blurred in my mind. But I also had no doubt that it was about to be, just like the volume which I had just written, a \textbf{journey} or \textbf{discovery}; a journey into an even richer and vaster world than the one that I was getting ready to oversee, in the volume just written and in those to follow. Over the course of days, weeks, months, without really understanding what was happening, I continued this new journey at the discovery of a certain past (which had obstinately eluded me for the past three decades...), as well as of myself and the threads linking me to that past; it was also\marginpar{p. L16}a journey at the discovery of some people to whom I had been close in the mathematical world, and whom I knew so little; and lastly, in the same stride, a journey of mathematical discovery wherein, for the first time in fifteen to twenty years\footnote{In the 1950s and 1960s, I often repressed a desire to pursue such burning and fruitful questions, as I was entirely at the mercy of countless foundational tasks which no one else would have or could have taken up in my stead, and which no one had the stamina to pursue further following my departure... }, I took the time to return to some burning questions which I had left at the time of my departure. As such, I would say that I am actually pursuing \textbf{three} intimately interlinked journeys of discoveries in the pages of R\'ecoltes et Semailles. And none of the three has reached completion by page twelve hundred and counting. The echoes with which this testimony will be met (including echoes of silence...) will be part of the ``continuation" of the journey. As to a conclusion - this is the kind of journey that never really reaches its end; not even, perhaps, on the day of our death...

And so I have come full circle, back to where I started: telling you in advance, inasmuch as it can be done, ``what R\'ecoltes et Semailles is about". But it is also true that, whether or not you had this question in mind, the previous few pages gave some sort of an answer. Perhaps it would be more interesting for me to continue in my stride and to begin \textbf{telling}, rather than ``announcing".
\begin{flushright} June 1985 \end{flushright} 

\section{The obscure side - or creation and disdain}

The previous pages were written during a ``low activity point" last month. Since then, I have finally added the finishing touch to the ``Four Operations" (the fourth part of R\'ecoltes et Semailles) - all that remains is for me to complete this letter or ``pre-letter" (which itself seems to be assuming prohibitive proportions...) and it will then be ready for typesetting and printing. I was starting to lose faith, after the almost year and a half during which I have been ``on the verge of finishing" these famous notes! When I first sat down to write this unusual ``introduction" to a work of mathematics, in February of last year (and already in June of the year before that), there were (I reckon) three things which I intended to speak about. 

First, I wanted to describe the intentions behind my decision to resume a mathematical activity, to write about the spirit in which I had written the first volume ``Pursuing Stacks" (which I had then just completed), and about the spirit\marginpar{p. L17} in which I intended to pursue an even vaster journey of exploration and mathematical discovery in the ``Reflections". From there, I would then only have to present the meticulous and well-pressed foundations of some new mathematical universe in gestation. It would read rather like a ``travel log", in which the work is carried day after day, in plain sight and as it is \textbf{actually} happening, including all the mistakes and mess-ups, the frequent look-backs as well as the sudden leaps forward - a work irresistibly being pulled forward day after day (in spite of the countless incidents and unforeseen circumstances), as if by an invisible thread - by some elusive yet nonetheless tenacious and sound vision. A work that is often fumbling, especially during the ``delicate times" in which a spring of intuition arises, barely perceptible, nameless and faceless as of yet; or during the early steps of some new journey, while still on the lookout for and at the pursuit of initial ideas and intuitions - the latter of which often prove to be elusive and escaping the meshes of language, indicating that what is actually missing is the formulation of an adequate language in which they could be captured with finesse. The task then becomes that of creating such a language, to condense it out of the intangible mist with which we are initially confronted. Thus, what was at first only intuited, before being glimpsed without being really seen or ``touched", slowly trickles out of the imponderable, extracting itself from its shadowy and hazy mantel to assume an existence in flesh and bone...

It is this part of the work which, albeit puny looking - not to say (often) harebrained - is often the most delicate and essential part of the process: it is truly there that something new becomes manifest, through intense attention, solicitude, and respect towards the fragile and infinitely delicate thing about to be born. It is the most creative part of all - that of conception and slow gestation within the warm shadows of the maternal womb, inside which the initial double gamete becomes amorphous embryo and continuously transforms itself over the course of days and months, by means of an obscure and intense process, seamless and invisible, into a new being made out of flesh and bone.

This is also the ``obscure", ``yin", or ``\textbf{feminine}" part of the work of discovery. The complementary ``clear", ``yang", or ``\textbf{masculine}" part would be closer to a work of hammer and chisel upon a piece of hardened steel (using ready-made tools whose efficacy has already been established...). Both aspects have their  individual raison d'\^etre and function, and they are inextricably tied\marginpar{p. L18} to one another by symbiosis - or rather, they are the \textbf{wife} and \textbf{husband} forming the indissoluble couple of the two original cosmic forces, whose ever-renewing embrace gives rise to the obscure creative work of conception, gestation, and birth - the birth of the \textbf{child}, of the novel thing. 

The second topic which I felt the need to write about, in this famous personal and ``philosophical" introduction to a work of mathematics, was the nature of creative work. It had been years since I first realized that the nature underlying this process was largely ignored, overshadowed by a host of clichés, repressions and ancestral fears. I only started discovering the extent to which this was a reality over the course of the days and months which I devoted to the reflection and the ``investigation" undertaken in R\'ecoltes et Semailles. Already from the ``get go" of this reflection - while writing some pages back in June 1983 - I was taken aback by a fact which, while inconspicuous at first sight, is nonetheless stupefying once one stops to think about it: namely, that this ``most creative part of all" within a work of discovery which I mentioned above \textbf{is reflected almost nowhere} in the texts and monologues which are supposed to present work of this kind (or at least present its most tangible outgrowths); such is the case in textbooks and other texts of a didactic nature, in articles and original memoirs, as well as in oral lectures, seminary presentations, etc. It is as if there existed, for what seems like millennia, tracing back to the very origins of mathematics and of other arts and sciences, a sort of ``conspiracy of silence" surrounding these \textbf{``unspeakable labors"} which precede the birth of each new idea, both big and small, and which thereby lead to a renewal of our understanding of a portion of this world in which we live, a world engaged in perpetual creation. 

To tell the truth, it seems that this most crucial aspect, or stage, of the work of discovery (as well as creative work in general) is subjected to a repression so efficient, so interiorized by the very people who come to know firsthand such a process that we could almost swear that these same people have eradicated every memory of this practice from their consciousness - akin to how a woman living in a highly restrictive puritan society might have eradicated from her living memory, in relation to each of the children who fall under her daily care, the moment of the embrace (grudgingly endured) tied to their conception, the long months of pregnancy (suffered as an impropriety), and the long hours of childbirth (experienced as a distasteful ordeal, followed by deliverance at long last).

\marginpar{p. L19}This comparison may seem outrageous, and it may indeed be so, if I were to apply it to the spirit of the mathematical milieu to which I belonged myself, as I remember it from some twenty years ago. But in the course of my reflection in R\'ecoltes et Semailles I was led to realize, especially during the past few months (while writing ``The Four Operations"), that there had been since my departure from the mathematical world a stupefying \textbf{degradation} in the spirit which nowadays rules over the milieux which I once knew, and (to what seems to be a large extent) in the mathematical world at large\footnote{This degradation is not in any way limited to the ``mathematical world". It can be detected in the entirety of the scientific sphere and, beyond it, in the contemporary world at a global scale. I begin an assessment and reflection upon this situation in the note ``The muscle and the bowel", which opens the reflection on the yin and the yang (note n$^o$106).
%todo: ref 
}. It is even possible, in view of my very particular mathematical personality and of the circumstances surrounding my departure, that the latter may have played a catalyzing role in an evolution which was already underway\footnote{I further examine this evolution in the note cited in the previous footnote. Links between this evolution and the Burial (of my person and of my work) appear and are examined in the notes ``The Funerals of the Yin (yang buries yin (4))", ``Providential circumstance - or Apotheosis", ``The disavowal (1) - or the reminder", ``The disavowal - or the metamorphosis" (n$^o$ 124, 151, 152, 153). See also the more recent notes (in ReS IV) ``The unnecessary details" (n$^o$171 (v), part (c) ``Things that don't resemble anything - or desiccation") and ``The family book" (n$^o$`73, part (c) ``The one among all - or the acquiescence").
%todo: ref
} - an evolution to which I was then blind (as much so as any other friend or colleague of mine, with the possible exception of Claude Chevalley). The aspect of this degradation which I am mostly thinking about in writing these lines (and which is just \textbf{one} aspect among many others\footnote{The aspect which is most often the focus of R\'ecoltes et Semailles, particularly in the two ``investigative" parts (ReS II or ``The robe of the Emperor of China", and ReS IV or ``The Four Operations"), is also, perhaps, the aspect that I found most ``flabbergasting", namely the degradation of the ethical code of the profession, expressed in the form of sacking, discrediting, and shameless scheming operated by some of the most prestigious and brilliant mathematicians of the time, and (to a large extent) in full view of all. For some of the other, more delicate aspects, which are in fact directly tied to this one, I refer the reader to the aforementioned note (n$^o$173 part (c)) ``Things that don't resemble anything - or desiccation".}) is the \textbf{tacit disdain}, if not outright derision, directed towards work (mathematical, in this case) which does not resemble pure hammer-and-chisel labor - disdain for the more delicate (and\marginpar{p. L20} often lesser looking) creative processes; for all that pertains to \textbf{inspiration, dream, vision} (however powerful and fruitful it might be); almost (in the extreme) for any idea, however clearly it may have been conceived and formulated: in sum, disdain for anything that hasn't been written and \textbf{published} in black on white, in the form of plain statements, classifiable and classified, ready to be incorporated into the ``databases" inscribed in the inexhaustible memories of our supercomputers. 

There has been (to borrow an expression from C.L. Siegel\footnote{This expression is cited and commented upon in the note that was just cited in the preceding footnote}) an extraordinary \textbf{``flattening"} and a \textbf{``narrowing"} of mathematical thought, as a result of the fact that an essential dimension has been stripped away: the totality of its ``obscure side", its ``feminine" side. It is true that, in accordance with an ancestral tradition, this side of the work of discovery was to remain largely hidden, and nobody (virtually) ever \textbf{spoke} about it - but, until now (to the best of my knowledge), the ability to establish a raw contact with the profound sources of dream, and to in turn nourish great ideas and great designs, had never yet been lost. It seems as if we have recently entered an \textbf{era of desiccation}, in which access to these dream sources, which are admittedly not yet dried-up, has been condemned by the unappealable verdict of general disdain and by the reprisals of derision. 

We therefore seem to be approaching the risk of eradication, within each individual, of not only the \textbf{memory} of work carried out close to the source, of a ``feminine" nature (often derided as ``muddy", ``sluggish", ``inconsistent" - or at the other extreme as ``trivial", ``child's play", ``long-winded"...), but also the loss of this very work and its outgrowths - when this work is where novel notions and visions are conceived, grow, and come to life. Such an event would lead us to an era during which the practice of our art will be reduced to arid and vain demonstrations of cerebral ``weightlifting", and to intellectual bidding wars towards the ``breaking open" of competition problems (``of proverbial difficulty") - an era of sterile, feverish, ``super-macho" hypertrophy following more than three centuries of continuous creative renewal. 

\section{Respect and fortitude}

But I digress once again, looking ahead regarding what my reflection has taught me.  My initial goal, prior even to the start of this reflection, was twofold: to present a ``declaration of intent", and (something which we just saw is closely related to the latter point) to share my thoughts on the nature of creative work. However, there was also a third goal which, although less perceptible at the conscious level, fulfilled a more\marginpar{p. L21} profound and essential need of mine. This third point was sparked by sometimes troubling ``interpellations" issued by many of my old students or colleagues, and regarding my past as a mathematician. On the surface, this need manifested itself as a desire to ``empty my bag", to say out lout some ``uncomfortable truths". Yet, at a deeper level, there was surely also the need to \textbf{become acquainted} at last with a past which I had hitherto chosen to ignore. It is as a result of this need more than anything else that R\'ecoltes et Semailles came into being. This long reflection was my ``response", day in and day out, to the inner impulse I felt towards understanding, and to the recurring interpellations that reached me from the outside world, the ``mathematical world" which I had left with no intention of return. With the exception of the first pages of ``Fatuity and Renewal", namely the first two chapters (``Work and discovery" and ``The dream and the dreamer"), and beginning with the following chapter ``The birth of fear" (p. 18) which features an unplanned ``testimony", I believe that this need to confront my past and to fully accept it was the principal force at work in the writing of R\'ecoltes et Semailles.
%todo: ref

The interpellation that had reached me from the mathematical milieu, and which returned with renewed strength throughout R\'ecoltes et Semailles (especially during the ``investigation" undertaken in parts II and IV), had taken from the get-go a self-important character, if not one of (``delicately calibrated") disdain, derision, or scorn - directed both towards myself (sometimes), and (most often) towards those who had dared take inspiration from me (without being aware of the consequences awaiting them), and who were thereby ``categorized" as being linked to me, by some tacit and implacable decree. Once again I am detecting herein the ``obvious and profound" connection between \textbf{respect} (or the absence thereof) for others; respect for the creative act and for some of its most delicate and essential fruits; and, finally, respect for the most evident rules of scientific ethic, namely those that are rooted in an elementary respect of oneself and of others, and which I am tempted to call ``\textbf{rules of decency}" in the practice of our art. All of these are but aspects of an elementary and essential ``\textbf{respect of oneself}". If I had to summarize, in a single lapidary sentence, what R\'ecoltes et Semailles has taught me about the certain world which I once called my own, and with which I identified myself for more than twenty years, I would say:\marginpar{p. L22} it is a world that has \textbf{lost the value of respect}\footnote{Once again, this formulation preserves its relevance outside of the limited milieu in which I had ample opportunities to come to this conclusion, where it seems to also summarize a certain degradation of the entirety of contemporary society. (Compare with the footnote on page L19.)
%todo: ref
Within the more self-contained context of summarizing the ``investigation" pursued in R\'ecoltes et Semailles, this formulation appears in note 2 from last April, titled ``Respect" (n$^o$179).}.

The above was something I had already strongly felt, if not outright formulated, during the preceding years. It was only further confirmed and clarified, in unexpected and sometimes stupefying ways, over the course of writing R\'ecoltes et Semailles. It is notably already apparent starting from the point at which the general reflection of a ``philosophical" nature suddenly turns into a personal testimony (in the section ``The welcome stranger" (n$^o9$, p.18) at the start of the aforementioned chapter ``The birth of fear").
%todo: ref

Yet, this observation is not to be interpreted as an acerbic or bitter recrimination, but rather (through the logic internal to the writing and through the attitude which it induces in the reader) as an \textbf{interrogation}. That is, I invite mathematicians to ask themselves: what was the nature of my own involvement in this degradation, this loss of respect which I am nowadays witnessing? Herein lies the principal interrogation which pervades and carries forth the first part of R\'ecoltes et Semailles, up until the point where it is finally resolved by a clear and unequivocal observation\footnote{See the sections ``Athletic mathematics" and ``Enough with this merry-go-round (n$^o$40, 41).}. Previously, this degradation had seemed to me as having suddenly ``materialized itself", as something that was equally unexplainable, outrageous, and unacceptable. During the reflection, I discovered that it had actually been taking its course insidiously, without anybody taking notice of its evolution either around them or within them, throughout the 1950s and 1960s, \textbf{including in my own case}. 

This humbling realization, although evident and without show, marks the first crucial turning point in the testimony, and immediately brings with it a qualitative change\footnote{Beginning the following day, the testimony deepens into a meditation on myself, a particular quality which it preserves in the following weeks, all the way to the end of this ``first breath" of R\'ecoltes et Semailles (ending with the section ``The weight of a past", n$^o$50).}. This was the first essential realization which I was to acquire about my mathematical past and about myself. This newly gained awareness of my \textbf{shared responsibility} in the general degradation (an awareness which makes itself felt more or less sharply at various points of the reflection) remains a kind of background note and a reminder throughout R\'ecoltes et Semailles. Such is the case, mostly, at the times when the reflection\marginpar{p. L23} takes the form of an investigation on the disgraces and inequities of an era. Together with the desire to understand, this curiosity which carries forward any authentic work of discovery, I believe that it is this humbling awareness (forgotten at various points along the way but always reappearing, in the places where it is least expected...) which prevented my testimony from ever turning into a compendium of sterile recriminations on the world's shortcomings, or even into a ``settlement of scores" with some of my old students or friends (or both). This absence of complacency towards myself also provided me with a sense of inner calm, or fortitude, which in turn safeguarded me from the risk of complacency towards others, including that of ``false discretion". I said all that I felt the need to say at every point of the reflection, be it about myself, about one or another of my colleagues, students, or friends, or about a milieu or an era, without ever having to jostle against my reluctance: every time such inner opposition surfaced, I needed only examine it carefully for it to disappear without a trace.

\section{``My close acquaintances" - or connivance}

The purpose of this letter isn't to list all of the ``key moments" (or ``sensitive moments") in segments of or all of  R\'ecoltes et Semailles\footnote{A short retrospective-recap of the first three parts of R\'ecoltes et Semailles can be found in the two groups of notes ``The evening fruits" (n$^o$179-182) and ``Discovering a past" (n$^o$183-186).}.
%todo: ref
I would only like to mention that there were four great steps, or ``breaths", that can be clearly distinguished in this work - akin to respiratory motions, or to the successive waves of a rising tide, vast and mute masses, at once immobile and in motion, limitless and nameless, issued from the unknown and bottomless sea that is ``me", or rather, from a sea infinitely vaster and deeper than ``I" am, and which carries and nourishes me. These ``breaths" or ``waves" materialized into the four presently written parts of R\'ecoltes et Semailles. Each wave came of its own accord and unexpectedly, and at no point could I tell where it was taking me nor when it would end. And whenever one wave ended and a new one took its place, there was a period of time during which I still believed myself to be at the end of a cycle (leading, at the very end, to the end of R\'ecoltes et Semailles!), whereas I was already being lifted and carried forward by another breath of the same vast movement. It is only in hindsight that\marginpar{p. L24} this movement becomes clearly apparent and that a \textbf{structure} unequivocally reveals itself within what has been lived as an act in motion.

Naturally, this movement did not come to an end upon the (provisory!) completion of R\'ecoltes et Semailles, and it not end with this letter either, the latter being but a ``measure" of this movement. In the same way, it wasn't born on a given day in June 1983, or February 1984, when I sat down in front of my typewriter to begin (or resume) writing the introduction to a certain mathematical work. Instead, it was born (or rather re-born) on the day that meditation appeared in my life...

But I digress once more, letting myself be carrier (and taken away...) by images and associations born in the moment, rather than diligently sticking to the thread of a planned ``message". Today's intention was to continue the narration, however briefly, of the ``discovery of the Funeral"
%todo:ref
 written last April, at a time when I thought I was already done with R\'ecoltes et Semailles two weeks earlier - to review the way in which a series of major and incredible discoveries cascaded upon me in the span of just three or four weeks, discoveries so massive and wild that it took me months to even begin ``believing the testimony of my sound perceptive capacities", and to free myself from an insidious \textbf{incredulity} in the face of evidence\footnote{I attempt to express this difficulty in the tale ``The robe of the Emperor of China" in the note of the same name (n$^o$77), and again in the note ``Duty fulfilled - or the moment of truth" (n$^o$163).
%todo: ref
}. This secret and tenacious incredulity was only dissipated last October (six months after the discovery of the ``Funeral in all its glory"),
%todo: ref
following the visit at my home of my friend and ex-student (albeit secretly) Pierre Deligne\footnote{I narrate this visit in the note cited in the previous footnote.}. For the first time, I was confronted to the Funeral not through the intermediary of \textbf{texts} going over (in nonetheless eloquent prose!) the discrediting, sacking, and massacre of a life's work, as well as the burial of a certain style and approach to doing mathematics - only this time the confrontation was direct and tangible, assuming familiar traits and a known voice, whose intonations were affable and ingenuous. The Funeral was in front of me at last, ``in flesh and bone", with the occupied and anodyne traits which I recognized, but which I saw for the first time with new eyes and a renewed attention. Here was before\marginpar{p. L25} me the person who, over the course of my reflection in the preceding months, had revealed himself to be the Chief Officer at my solemn funeral rites, at once the ``Priest in a chasuble" and the principal architect and ``beneficiary" or an unprecedented ``operation", secret inheritor of a work abandoned to derision and sacking...

This encounter takes place at the beginning of the ``third wave" of R\'ecoltes et Semailles, just as I had embarked upon a long meditation on the yin and the yang, at the pursuit of an elusive and tenacious association of ideas. At the time, this brief episode was only mentioned in passing, in the form of an echo of a few lines. It nonetheless marks an important moment, whose fruits will only make themselves known months later.

There was a second similar confrontation to the ``Funeral in flesh and bone", which happened just ten days ago and came to relaunch, ``at the last minute" once more, an investigation that kept finding new impetuses. This time, it was just a phone call with Jean-Pierre Serre\footnote{This is essentially a paraphrase of the note ``The Gravedigger - or the entire Congregation" (n$^o$97, page 417).}
%todo:ref 
. This ``jumbled" conversation served to confirm in a striking and unexpected way what I had just (a few days earlier) admitted to myself\footnote{In part (c) (``The one among all - or the compliance") of the same note (n$^o$173).}, almost against my will, concerning the role played by Serre in my Funeral and his ``secret compliance" to what was happening ``right under his nose" while he pretended not to see or feel anything. 

Here too, of course, the conversation itself was perfectly ``cool" and amicable, and it indeed seems that Serre's friendly disposition towards me is entirely sincere and genuine. The fact remains that I could clearly discern, or should I say ``touch", this compliance of his that I had just admitted to myself; undoubtedly ``secret" (as I wrote above), but most of all hastened, as I was able to observe in person beyond any doubt. A hastened and unreserved compliance, to bury what has to be buried, and to replace, wherever deemed fit and by any means, a real and undesirable paternity (which Serre knows about firsthand) by a bogus and welcome one...\footnote{This is essentially a paraphrase of the note ``The Gravedigger - or the entire Congregation" (n$^o$97, page 417).}\marginpar{p. L26} I received a striking confirmation of an intuition which had already appeared a year earlier, when I wrote\footnote{This quote is taken from the same note (see previous footnote), also page 417.}: 

\parindent=0.6cm
``Seen in this light\footnote{``In light" of the deliberate comment mentioned above, regarding the need to eliminate at all cost ``undesirable paternity ties" (or even ``intolerable" ties, to use the original expression from the note in question).}, the principal officer Deligne no longer appears as the one who created a trend reflecting the underlying forces determining his own life and actions, but rather as the \textbf{instrument} designated (because of his role as ``legitimate heir"\footnote{This role of ``heir" that Deligne takes on is both occult (in that not a single line published by Deligne could suggest that he has learned anything from me) and felt and admitted by all. There lies a typical aspect of Deligne's double-play and of his particular style, in that he was able to masterfully play with this ambiguity, cashing in the advantages of his tacit role as heir while simultaneously disavowing the deceased master and taking the direction of the large scale burial operation.}) by a highly cohesive \textbf{collective will}, and given the impossible task of erasing both my name and my personal style from contemporary mathematics." 

If Deligne thus appeared as the ``instrument" designated by a ``highly cohesive collective will" (while also being its first and principal ``beneficiary"), Serre now appears to me as being the very \textbf{incarnation} of this collective will, as well as the \textbf{guarantor} of the resulting unreserved compliance; an acquiescence to all kinds of trickeries and frauds including the ``vast" operations of shameless collective mystification and appropriation, as long as these practices contributed to the ``impossible task" targeted towards my modest and departed self, or towards any other person\footnote{I am here thinking about \textbf{Zoghman Mebkhout}, about whom I write for the first time in the Introduction 6 (``The Burial"), then later in my note ``My orphans" (n$^o 46$), as well as in the notes (written at a later time, after the discovery of the Burial) ``Failure of a teaching (2) - or creation and fatuity" and ``A sensation of injustice and powerlessness" (n$^o$s 44', 44''). I explore the iniquitous operation of concealment and appropriation of Mebkhout's pioneering work over the course of the eleven notes forming Procession VII of the Burial, ``The Colloquium - or Mebkhout's sheaves and Perversity" (n$^o$s 75-80). An investigation and a more extensive narrative about this (fourth and last) ``operation" constitutes the most substantial part of the investigation ``The four operations", under the fitting name \textbf{``The Apotheosis"} (notes n$^o$s 171 (i) through 171 ?).} who dared join forces with me and to appear, against all odds, as a ``continuator of Grothendieck".

\marginpar{p. L27}One of the paradoxical and disconcerting aspects of this Burial, among others, is that it was carried out chiefly, not to say exclusively, by people who had once been my friends or my students within a community in which I had never counted any enemies. It is mostly because of this reason, I believe, that R\'ecoltes et Semailles concerns you more than others, and that this letter is meant to serve as an \textbf{inquiry}. For if you are one of the mathematicians among my ex-students and friends, your are certainly no stranger to the Burial, be it through your acts or complicity, or even through your silence towards me regarding something which has been taking place at your doorstep. And if (by some miracle) you choose to welcome these humble words and the testimony they bring you, rather than to remain locked behind closed doors and to dismiss the unwelcome messengers, you might then learn that the burial undertaken by all and with your participation (be it active or through your tacit acquiescing) did not only affect someone else's work, the fruits and testimony of his frenetic love affair with mathematics; rather, at a deeper and more hidden level than this (unspoken) burial, there is a living and essential part of your own being, and of your original power to know, to love, and to create, which you have elected to bury with your own hands in the guise of another person.

Among all of my students, Deligne had occupied a special place, upon which I elaborate at length during the reflection\footnote{(*)See on this topic the group of seventeen notes ``My friend Pierre" (n$^o$s 60-71) in ReS II.}(*). He was, by far, the ``closest" to the vast vision which had been born and had grown within me long before we met, and he was in fact the only one (among my students as well as others) to have intimately absorbed this vision and made it his own\footnote{This ``vast vision" which Deligne successfully ``assimilated and made his own" had generated a powerful fascination within him, and it continues to fascinate him despite his own will, when an imperious force is simultaneously pushing him to destroy it, to blow up its fundamental unity and to appropriate the scattered pieces. Thus, his occult antagonism towards a renounced and ``late" master appears as the expression of a division within his being, which profoundly marked his work following my departure - a work which remained well below the rather prodigious abilities which I had known him to possess.}. And among the friends sharing with me a common passion for mathematics, it was Serre who, having taken on the role of an elder, had been the closest (by far once again) in the unique role of ``catalyst" for some of my greatest undertakings for a decade - and I owe him\marginpar{p. L28} most of the great key-ideas which inspired my mathematical thought during the 1950s and 1960s, up until my departure from the mathematical world. This special relationship that they both had with me had to do in parts with their exceptional abilities, the latter of which guaranteed their equally exceptional rise in the ranks of the mathematicians of their generation and of the ones to follow. Other than these similarities, the temperaments of Serre and Deligne appear to me to be as dissimilar as could be, and they stand at antipodal points from one another in many ways. 

In any event, if I had to name mathematicians who were in some way or another ``close" to me and to my work (and who are furthermore known ), I would have to name Serre and Deligne: the first as an elder and as a source of inspiration for my work during the crucial gestative period of my vision; and the second as the most talented of my students, for whom I was in turn (and remained, Burial or not...) his main (and secret...) source of inspiration\footnote{See on this topic the preceding footnote.}. The fact that a Burial could have been set in motion right after my departure (turning the latter into a proper ``death") and furthermore concretized into an endless procession of ``operations" both large and small serving the same end is only conceivable with the joint and solidary contribution of both of these individuals, the ex-elder and the ex-student (or even ex-``disciple"): one of them, feeling the need to destroy the \textbf{Father} (under the grotesque and derisory effigy of a plethoric and bombastic \textbf{powerpuff girl}) discretely and efficiently took charge of the operations, rallying some of my students to his cause along the way\footnote{I am here referring precisely to the five other students who had (like Deligne) chosen the cohomology of varieties as their principal theme.}; and the other granted an unconditional and unlimited ``green light" to the carrying of the (four) operations (namely, the discrediting, massacre, cutting up, and partitioning of inexhaustible remains...).

\section{The plunder}

As I alluded to earlier, I have had to surmount a considerable amount of inner resistance, or rather to resolve them through a patient, meticulous and tenacious process, so as to manage to distance myself from certain familiar images that had taken hold in my mind with great inertia and which had weakened my ability to perceive reality in a direct and nuanced way over the course of decades (as is the case for everybody else, including yourself most likely) - namely, I am referring to the picture I had of a certain mathematical world to which I continue to be linked\marginpar{p. L29} through my past and through my work. One of the most solidly anchored such images, or ready-made ideas, is that it seemed downright unconceivable that a world-renowned intellectual, or even a man recognized to be a great mathematician, could engage in fraudulent behavior of any scale (and in any capacity, let alone to let that become a matter of habit...); and it seemed just as unconceivable that, abstaining (out of habit once again) from engaging in such behavior himself, he could nonetheless welcome such operations (``at times defying any sentiment of decency") organized by someone else when, for one reason or another, they are beneficial to him.

The inertia I faced was so great that it was only two months ago, at the end of a long reflection that had taken place over the course of a whole year, that I started timidly perceiving that Serre could have had something to do with this Burial - something which now appears to me as evident, independently from the eloquent conversation which I recently had with him. There was in my mind a certain tacit ``taboo" surrounding his person, as well as all other members of the ``Bourbaki milieu" which had warmly welcomed me at the start of my career to a lesser degree. He represented the very incarnation of a certain kind of ``elegance" - an elegance that includes not only mere form but also a rigor and a scrupulous probity. 

Before my discovery of the Burial on April 19 of last year, I could never have even dreamed that one of my ex-students could have been capable of dishonesty in the exercise of their profession, whether towards me or anyone else; and such a supposition would have appeared most aberrant if applied to the most brilliant of my ex-students, the same one who was closest to me! Yet, from the moment of my departure and through the following years up to this very day, I have had ample occasion to realize the extent to which his relationship with me was divided. More than once, I had also seen him use of his power to discourage and to humiliate (as if for sports) when the occasion was fitting. Each of these events profoundly affected me (more so, surely, than I would have liked to admit at the time...). These were clear enough signs of a profound imbalance, which (as I have had ample occasion to observe) does not solely affect him, even in the very small circle of my ex-students. This imbalance, based on one's loss of respect for others, is no less flagrant nor less profound than the imbalance caused by so-called ``professional dishonesty". The fact remains\marginpar{L. 30} that the discovery of his dishonesty came to me as an utter surprise and as a shock. 

In the weeks that followed this breathtaking discovery, followed by a ``cascade" of others of a similar nature, I started to realize that a certain chicanery had already begun among certain of my ex-students\footnote{See the preceding footnote.} during the years preceding my departure. That this was the case was clearest for the most brilliant of them - the one who, after my departure, set the pace and (as I wrote earlier) ``discretely and efficiently took charge of the operations". With almost twenty years of hindsight, this swindling appears as an evidence, ``blindingly obvious". 

If I had then chosen to ignore what was happening around me, busy that I was chasing the ``white whale" in a world where ``all is for the best" (or so I thought), I own realize that I failed to assume the responsibility that was mine regarding the students who learned under my supervision about the trade that I love; a trade which goes beyond a simple savoir-faire and the development of a certain ``flair". Through my complaisance towards brilliant students, who I fancied (through a tacit decree) to treat as being ``one of a kind", absolving them from all suspicion, I have contributed my own share\footnote{This ``contribution" appears notably in the note ``One of a kind" (n$^o$ 67') as well as in the two notes ``The ascension" and ``The ambiguity" (n$^o$s 63', 63"), then once again (under a slightly different light) at the end of the note ``The eviction" (n$^o$ 169). Another type of ``contribution" appears in ``Fatuity and Renewal", in the form of attitudes of fatuity towards young mathematicians who were less visibly brilliant. This coming into awareness of my share of responsibility in the general trend of degradation culminates in the section ``Mathematics for sports" (n$^o$ 40)}\todo{check name of section} to the the hatching of (what seems like) unprecedented corruption, which I now witness to be diffusing into a world and into people whom I once held dear.

Admittedly, due to their immense inertia, it took me intense and sustained inner work to manage to separate myself from what are customarily called ``illusions" (not without a hint of regret...), and which I would rather refer to as ready-made ideas; ideas about myself, about I milieu with which I had once identified, and about people whom I had loved and who I perhaps still love. I have managed to ``separate myself" from these ideas, or rather, \textbf{to allow them to separate themselves from my person}. This process took work, but it was never a struggle; it brought me occasional bouts of sadness, among many other things, but never a feeling of regret or bitterness. Bitterness is a mechanism which allows one to elude\marginpar{p. L31} awareness of a given fact, or of the message of a lived experience; it keeps one locked in a tenacious illusion about themselves, at the cost of another ``illusion" (its negative so to speak) about the world and about others.

It is therefore without bitterness or regret that I now part with these ready-made ideas which I had once ``held dear", through force of habit and because they had ``always been there". They had become almost like a second nature to me; nonetheless, this ``second nature" was not ``me"; as such, to separate myself from it bit by bit is neither tearing nor frustrating, the way it would be for someone letting go of things which are of value to them. The ``plunder" about which I am speaking comes as a reward, as the fruit of a \textbf{labor}. It is immediately followed by a positive feeling of relief, a welcome \textbf{liberation}.

\section{Four waves in one movement}

Naturally, this letter doesn't look anything like I had expected upon starting it. I thought I would mainly be giving a little ``recap" on the Burial: here's roughly what happened; whether or not you believe it (even I struggled to believe it at first...) and whether you like it or not, this is indubitably what it is - you need only open the right periodical or book at the right page, it's all written in black and white. In fact, everything is unearthed at length in R\'ecoltes et Semailles; see ``The Four Operations", note such and such - take it or leave it! And in case you would rather abstain from reading me, others will do it in your place...

In reality, none of the above was included - even though this letter is already nearing thirty pages in length, and I only expect to go on for five or six more pages. Without even meaning to, one page leading to the next, I was led to telling you the essential things, while this ``bag" of items which I was impatient to unravel (and which itself was made clearly visible in the first few pages!) is still full! I don't even feel the urge to write about those things anymore, the need dissipated along the way. I understood that now was not the time...

Truth be told, part IV of R\'ecoltes et Semailles (the longest of all), titled ``Burial (3)" or ``The Four Operations", grew out of a ``note" initially expected to be ``a little recap" once more, in which I would sum up at a high-level the things I had learned during the surprise-investigation (\todo{translate ``et en coup de vent"}) of the preceding year, which had been pursued in part II (``The Burial (1)", or ``The robe of the Emperor of China"). I thought I was setting up to write a five or ten pages note, tops. Eventually, one thing leading to the next, the investigation was taken up again, and I was off to write nearly four hundred more pages - nearly twice the length\marginpar{p. L32} of the part which I was suppose to summarize and synthesize! As a result, the recap in question is still missing, while six hundred pages of R\'ecoltes et Semailles are devoted to the investigation around the Burial. It's a little silly, for sure. But there is always time to add a third part to the Introduction (which is no longer at the stage where ten or twenty pages will make a difference anyways), before handing my notes to a printer.

The five parts of R\'ecoltes et Semailles (the first of which is not yet finished, and probably won't be for another few months) represent an alternation between (three) ``meditation" waves and (two) ``investigation" waves. This in some sense reflects, in shortened format, my life during the past nine years, which also consisted of an alternation between ``waves" issued from the two passions that sustain me today, namely my passion for meditation and my passion for mathematics. In fact, the two parts (or ``waves") of R\'ecoltes et Semailles which I have given the cookie-cutter name ``investigation" are precisely those which have directly emerged from my ties to my past as a mathematician - they were engendered by the mathematical passion within me and by the ego-attachments that took roots within it.

The first wave, ``Fatuity and Renewal", consists of a first encounter with my past as a mathematician, leading to a meditation on my present, whose rooting in my past I freshly unraveled. Without the slightest amount of advance planning, this part lays out the ``base tone" to persist throughout the rest of R\'ecoltes et Semailles: it serves as a providential and indispensable inner preparation for me to undertake the discovery of the ``Burial in all its glory" in what closely follows, during the second wave ``The Burial (1) - or the robe of the Emperor of China". More than a simple ``investigation", this part really retraces the process of discovery day after day, its impact on my being, and the efforts I have made to face what suddenly befell me without warning and to situate the unbelievable relative to my lived experience, so as to eventually be able to articulate what had happened in the context of the familiar. This movement leads to a first provisory conclusion, in the note ``The Gravedigger - or the whole Congregation" (n$^o$97), the first essay in which I attempt to find an explanation and a meaning to something which, for years and now more acutely than ever, presented a formidable challenge to all common sense! 

This same second movement leads into a ``sickness episode"\footnote{This episode is covered in the two notes ``The incident - or mind and body" and ``The trap - or ease and exhaustion" (n$^o$98, 99), opening ``Procession XI" titled ``The (living) dead".}\marginpar{p. L33} which forced me into absolute rest, putting an end to all intellectual activity for over three months. This happened at a time when I again thought I had almost brought R\'ecoltes et Semailles to an end (modulo some last ``housekeeping" tasks...). Upon resuming a normal activity around the end of September of last year, as I was getting ready to bring the last touch to the notes which I had left unattended, I thought that I would only need to add two or three final notes, including one regarding the ``health-incident" which I had just gone through. Instead, one week after the other and one month after the other, there came a thousand more pages - more than twice what had already been written - and this time around, it was clear to me that I was still not done\footnote{``Still not done", if only because a part V is yet to come, which is not complete at the time of writing.}! In fact, this long interruption, during which I nearly lost contact with a warm (nearly burning!) substance, practically forced me to come back to this substance with fresh eyes, in order not to stupidly ``wrap-up" the last part of a ``program" with which I had lost living contact.

This is how the third wave in the vast movement that is R\'ecoltes et Semailles was born - a long ``meditation wave" on the theme of the yin and the yang, the ``shadow" and ``light" sides present in the dynamic of things and in human existence. This meditation grew out of a desire to reach a more in-depth understanding of the profound forces at work in the Burial, yet it acquires from the get-go its own autonomy and unity, orienting itself towards the universal as well as the intimately personal. It is during this meditation that I discover the (admittedly obvious, once the question has been posed) fact that, in my spontaneous approach to discovery, be it in mathematics or elsewhere, the ``base tone" is ``yin", ``feminine"; I also come to realize that, unlike what happens in most cases, I have remained truthful to my original nature\footnote{This ``truthfulness to my original nature" was in no way total. For a long time, it concerned only my mathematical work, while in every other respect, and notably in my relationships, I followed the general motion, valuing and giving primacy to those traits within me perceived as ``manly", while at the same time repressing ``feminine" traits. I write about it in some details in the group of notes ``Story of a lifetime: a cycle in three movements" (n$^o$107-110), which practically serves as an opening to ``The key to the Yin and the Yang".}, never bending it or correcting it in an attempt to conform to the dominating values put forward by the environing milieux. At first, this discovery appears\marginpar{p. L34} as a mere curiosity. Later, it is revealed to be an essential key to understanding the Burial. Furthermore - and this is something which appears to be even farther reaching - I can now see the following thing clearly and without any doubt: that if, with my entirely un-extraordinary intellectual abilities, I was able to birth vast, powerful, and fertile work and vision, it is chiefly thanks to this very fidelity to my nature, this absence of any concern within me for conforming to the norms, which allows me to abandon myself with complete trust to the quest for original knowledge, without in any way cutting away from or amputating what gives this quest its strength, finesse, and indivisible nature,

Nonetheless, creativity and its sources are not at the center of attention in the meditation ``The Burial (2) - or the Key to the Yin and the Yang"; rather, the dominating theme is the ``conflict", the state of creative block, of dispersion of creative energy caused by the confrontation, within one's psyche, between (most frequently hidden) antagonistic forces. Aspects of a \textbf{violence}, which seemed furthermore (in appearance) ``gratuitous", ``for sports", had disconcerted me more than once during the Burial, and they brought back within me a host of similar lived experiences. The experience of this violence has been in my life the ``hard and irreducible kernel of conflict". I had never before faced heads on the dreadful mystery that is the very existence and universality of this violence in human life in general, and within myself in particular. It is this mystery that is at the center of attention throughout the second half (the ``yin" side, or ``decline") of the meditation on the yin and the yang. It is during this part of the mediation that a more profound vision emerges progressively regarding the meaning of the Burial and the forces which are expressed within it. This is also the part of R\'ecoltes et Semailles which seems to have been the most fruitful in terms of my self-understanding, as it put me in contact with delicate questions and situations, enabling me to actually feel their ``delicate" character, when up until last year this character had been eluded. 

At the term of this endless ``digression" on the yin and the yang, I was still left, more or less, with the ``two or three notes" that I still had to write (in addition to one or two more, at most, one of which already had a name: ``The four operations"...) before finishing R\'ecoltes et Semailles. The rest of the story is known: the ``few last notes" turned into the longest part of R\'ecoltes et Semailles, spanning over nearly\marginpar{p. L35} five hundred pages: this is therefore the ``fourth wave" of the movement. It is also the the third and last part of the Burial, and I named it ``The Four Operations", a name which is also used to designate the group of notes constituting the heart of this fourth breath of the reflection (``The four operations (on the remains)". Herein lies the ``investigation" segment of R\'ecoltes et Semailles in the strictest sense - with the caveat nonetheless that the investigation is not limited purely to the ``technical", or ``detective" aspect, and that it is instead carried out in accordance with the desire to know and to understand, as is everything else in R\'ecoltes et Semailles. The tone is noticeably ``stronger" than in the first part of the Burial, in which I was still pinching myself, trying to convince myself that I was not dreaming! Regardless, the facts established over the course of the pages often emerge at the perfect time, serving as vivid illustration for many of the things which had until then only been occasionally touched upon in passing, and which as a result find an incarnation into precise and striking examples. It is also in this part that mathematical digressions take a more prominent place, stimulated by a renewed contact (which grew out of the needs of the investigation) with a substance of which I had lost sight for over fifteen years. At the other end of the spectrum, you will also find a timely narrative regarding the misadventures of my friend Zoghman Mebkhout (to whom this part is dedicated) at the hands of a high-flying\todo{check that this is correct} and merciless ``mafia" of which he was entirely unaware upon launching into the (passionating, and innocent-looking) subject of the cohomology of varieties of all kinds. For a succinct guiding thread through the intricate maze of notes, sub-notes, and sub-sub-notes... of this ``investigation" part, I invite you to refer back to the table of contents (notes 167' through $176_7$), as well as to the first note of the bunch, ``The detective - or life through rose-colored glasses" (n$^o$167'). I should nonetheless signal that this note, dating from April 22, was later somewhat ``overcome by recent events", in the sense that through several twists and turns this investigation which I then believed to be (practically) complete ended up continuing\todo{look up ``\`a brin de zinc"} for two more months.

This fourth breath lasted for four consecutive months, from mid-February through the end of June. It is mostly in this part of the reflection that a concrete and tangible contact with the reality of the Burial is established, day after day and page after page, through a meticulous and obstinate ``piece-work"\todo{look up the right translation to ``travail sur pi\`eces"}. It is also there that I manage to ``familiarize" myself with the Burial to some extent, putting to the side the visceral reactions of denial which it had prompted within me (and continues to prompt) and which constituted an obstacle to a true coming into awareness. This long reflection takes as a starting point a retrospective on Deligne's visit (about which I have already spoken in this\marginpar{p. L36}), and it ends with a ``last minute" reflection upon my relationship with Serre and upon Serre's role in the Burial\footnote{Among parts c, d, and e of the note ``The family album" (n$^o$173), the last one dates from June 18 (which was exactly ten days ago). Only one other note or note segment corresponds to a later date (namely, ``Five theses for a single massacre - or filial piety", n$^o 176_7$, dating from the next day, June 19). You will note that in this fourth part of R\'ecoltes et Semailles, or ``investigation segment", unlike elsewhere in the reflection, the notes are organized in logical rather than chronological order. Thus, the last two notes of the Burial (constituting the final ``De Profundis") date from April 7, two and a half months earlier than the note I just mentioned. I should nonetheless signal that aside from the ``investigation" proper of the Burial (3) (notes n$^o$s 167'-$176_7$), constituting the ``fifth movement" of the Funereal ceremony (``The Key to the Yin and the Yang" being the second), the notes appear in chronological order, modulo some rare exceptions.}. Until last month, I believe that the most serious shortcoming of my understanding of the Burial was to have tacitly ``excluded" Serre from the list of culprits, as a result of the ``taboo" which I mentioned earlier - as such, this ``last minute" reflection appears to me as the most important thing to have come out of this ``fourth breath" of R\'ecoltes et Semailles in contribution to a more substantial understanding of the Burial and of the forces that are expressed within it.

\section{Movement and structure}

I think I am now done enumerating the most important things which I wanted to tell you concerning R\'ecoltes et Semailles, so as to let you know ``what it's all about". I have surely said enough for you to be able to determine if \textbf{you} consider that the subsequent letter of (over) a thousand pages ``is relevant to you" or isn't - and from there decide whether or not you will keep on reading. In case your answer is a ``yes", I thought I would add some explanations (notably of a practical nature) regarding the \textbf{structure} of R\'ecoltes et Semailles.

This structure is the reflection and expression of a certain \textbf{spirit} which I have tried to ``convey" in the preceding pages. In comparison to my previous publications, the key novelty in both R\'ecoltes et Semailles as well as ``Pursuing Stacks" is \textbf{spontaneity}. Granted, you will find guiding threads and wide-ranging interrogations providing coherence and unity to the reflection as a whole. Nonetheless, the reflection is taken up day by day, without any pre-established ``program" or ``plan", never setting out ahead of time ``what is to be demonstrated". My purpose is not to prove,\marginpar{p. L37}but rather to \textbf{discover}, to probe further ahead into an unknown substance, to condense what is as of yet only dimly sensed, suspected, or glimpsed. I can truly say, without any exaggeration, that in the course of this work, not a single day or night of reflection passed within the realm of the ``planned", as much in terms of the ideas, images, and associations that presented themselves to me at the time when I sat down to write, and in so doing to persistently pursue a tenacious ``thread", or to take up a new one that had just appeared. Each time, what ends up appearing during the reflection is different from what I would have predicted had I ventured to describe in advance what I expected to discover ahead. Most of the time, the reflection ends up taking me down entirely unforeseen paths, eventually leading to new and equally unexpected landscapes. And even when the reflection follows a more or less expected trajectory, the sequence of images that I encounter over time differ from the picture I had at the outset as much as a real landscape, with its interplay of fresh shadows and warm light, its delicate features ever-changing with the trekker's every step, its countless sounds and nameless scents carries around by a breeze which makes the weeds dance and the forest sing... - as much as such a lively, elusive landscape differs from a postal card, however pretty and well-done - however ``accurate" it may be.

It is this reflection carried out in one go, over the course of one day or one night, which constitutes the indivisible unit, the living and individual cell of sorts, underlying the entirety of the reflection (R\'ecoltes et Semailles in this case). The reflection as a whole is to each of these units (or ``notes"\footnote{}, constituting\marginpar{p. L38} a melody...) as the body of a living organism is to each of its individual cells, the latter of which each fulfill a unique place and function in their infinite diversity. Nonetheless, it can sometimes happen that within a reflection issued from a single outpouring, we are able to perceive in retrospect important divisions, bringing to light several uniting themes or messages, each of which thereafter receives its own name and thus acquires an identity and autonomy. At other times, a reflection which had been cut short for some reason (fortuitously most of the time) is spontaneously continued into the next day and the one after that; or yet a reflection carried over the course of two or more consecutive days appears to use in retrospect as if it had been written in a single stretch - as if the necessity of sleep had forced us to mark some sort of pause (in a sense ``physiological"), indicated by a lapidary indication of the date (or of several such) separating consecutive paragraphs of the same ``note", with the latter carved out with its own name. 

\marginpar{p. L39}Thus, each of the notes in R\'ecoltes et Semailles has its own individuality, and carries a face and a function distinguishing it from every other. I have attempted to express the particularity of each note by a \textbf{name}, supposed to reconstitute or evoke the essential, or at least something essential regarding what that note ``has to say". I identify each note by its name before all else, and that is how I choose to refer to a given note every time that it is relevant. 

The name of a note often occurred to me spontaneously, before I even paused to think of one. I see such an unprompted appearance as a sign that the note at hand is nearly complete - that it will have said what it had to say by the time I complete the paragraph in progress... Equally as often, the name appears, just as spontaneously, as I am re-reading the notes from the preceding day or the day before that in preparation for the continuation of my reflection. The name may be slightly modified during the days or week following the creation of the new note, or it might even be enriched by a second name which had not occurred to me in the first place. Several notes carry such a double name, shining two different and sometimes complementary lights on their message. The first time a double name occurred to me was at the beginning of ``Fatuity and Renewal", through ``Encounter with Claude Chevalley - or freedom and kind sentiments" (n$^o$11).

I only had a name in mind prior to starting a note at two occasions - and both times, the name was shaken up by the turn of events!

It is only with the hindsight of weeks, and sometimes months, that an \textbf{overall movement} becomes visible, together with a \textbf{structure} underlying the collection of notes written day after day. I have attempted to delineate these superstructures through various groupings and sub-groupings of the notes, each with its own name, granting it a separate existence as well as a function or message; akin to division of a single body into its organs and limbs (to continue the earlier analogy), as well as the decomposition of a limb into its parts. Thus, within the ``Whole" of R\'ecoltes et Semailles are included the five ``parts" mentioned earlier, each with its own particular structure: Fatuity and Renewal consists of eight ``chapters" I through VIII\footnote{}, while the three parts constituting\marginpar{p. L40} the Burial (which themselves became their own entities over time...) consist of one long and solemn Procession involving twelve ``Cort\`ges" I through XII. The last of them, \textbf{``The Funereal Ceremony"} (as it is named) towards which the preceding eleven Cort\`eges progressed (surely without suspecting it...), takes truly gigantic proportions, matching the scope of the lifework which is the subject of these solemn Funerals: it occupies the near-totality of RS III (The Burial (2)) and the totality of RS IV (The Burial (3)), spanning nearly 800 pages and 150 notes (even though said ceremony was only expected to fill two notes!). Under the skilled direction (and well-known modesty...) of the great officiating priest himself, the ceremony takes place in nine ``acts", or separate liturgical acts, beginning with \textbf{The Funereal Ceremony} (naturally) and ending (as is fit) with the final \textbf{De Profundis}. Two of these ``acts", namely \textbf{``The Key to the Yin and the Yang"} and \textbf{``The Four Operations"}, each constitute (by far) the largest section of the part of R\'ecoltes et Semailles (III or IV) to which they belong, and as such the latter bear their names.

Throughout R\'ecoltes et Semailles, I took good care of the table of contents (as if it were the apple of my eyes!), ceaselessly restructuring it so as to account for the ever-renewed influx of unexpected notes\footnote{}, in the hope that it would reflect as finely as possible the overall movement of the reflection and the delicate structure emerging therein. It is in parts III and mostly IV (just mentioned above), ``The Key [...]" and ``The Four Operations", that this structure is the most complex and imbricated. 

In order to preserve the spontaneous character of the text, and to render the unexpected aspects of the reflection as they were truly experienced, I decided not to append a name to the notes which only appeared after the fact. As such, I recommend\marginpar{p. L41} that you refer back to the table of contents after reading each note so as to discover its name, and to also get a chance to appreciate at a glance how it inserts itself in the reflection thus far, or how it relates to what is yet to come. Otherwise, you run the risk of losing yourself in a seemingly indigestible and heteroclite, not to say cumbersome\footnote{}, collection of (sometimes strangely numbered) notes; so as to resemble a lost traveler in a foreign city (which sprouted in a bizarre fashion following the whims of the generations over the course of centuries...), with no guide nor even a map to help them orient themselves. In the manuscript destined to publication, I plan on including throughout the text the names of the ``chapters" as well as those of other groupings of notes and sections, except for the notes (or sections) themselves. But even then, the occasional recourse to the table of contents seems vital so as not to get lost in an aggregation of hundreds of notes, one following the next for over a thousand pages...

\section{Spontaneity and rigor}

Spontaneity and rigor constitute the ``shadow" and ``light" side of one indivisible quality. It is only through their marriage that this particular quality can be born in a text or a person - it may be approximately described as the ``quality of truth". Although spontaneity has been played down (if not downright absent) from my past publications, I do not think that its recent blossoming within me has affected my rigor. Rather, the presence of rigor's yin companion gives it a new dimension and a renewed fecundity.

Rigor has to monitor itself, so as to prevent the careful ``sifting" of the multitude of occurrences of the field of consciousness, constantly separating what is significant or essential from what is only fortuitous or accessory, from hardening into bouts of censorship and complacency. Curiosity alone, that thirst for knowledge within us, is capable of stimulating such an effortless alertness and vivacity in the face of the immense and ubiquitous inertia of so-called ``natural downward slopes" consisting of ready-made ideas expressing our fears and our conditioning. 

The same rigor and alert attention may be directed towards spontaneity and its outgrowths, so as to distinguish once again between these entirely natural ``downward slopes"\marginpar{p. L42} and that which has truly emerged from the depths of one's being, issued from the original impulse towards knowledge and action which invites us to dive into the world.

In the context of writing, rigor manifests itself in the form of a constant need to capture things as finely and faithfully as possible, using language, thoughts, feelings, perceptions, pictures, or intuitions... So that one is not content merely expressing oneself vaguely or approximatively when the thing at hand has a clearly defined shape; nor is it satisfactory to use an ostensibly precise term (and thereby further deforming the truth) in order to express something which remains immersed in the mist of what has been felt but not yet grasped. Only when we attempt to capture the thing as it is in the moment can we hope to witness its true nature, perhaps even in broad daylight if it is meant to be seen in this way and if our desire persuades it to let go of its shadowy veils and enveloping mist. Our role is not to pretend being able to describe and pin down that which we still ignore; rather, it is to humbly and passionately endeavor to understand the mysteries and unknown entities that surround us from all sides.

That is, the role of writing is not to present the results of research, but rather to report on the process of research itself - to document the work of love and the fruits of this work with Mother Nature, the Unknown, who tirelessly calls us to know more about her inexhaustible Body, venturing wherever the mysterious winds of desire carry us.

In the documentation of this process, retracing one's steps so as to add nuance, precisions, supplements and sometimes corrections to the written ``first jet" constitute an essential part of the very act of discovery. These backtracks form an essential part of the text and give it its full meaning. This is the reason why the ``notes" (or ``annotations") placed at the end of Fatuity and Renewal, to which I refer throughout the 50 ``sections" constituting the ``first jet" of the text, form an inseparable and essential component of the latter. I warmly recommend that you refer to them every now and then, at least whenever you are done reading a given section which makes one or several references to these ``notes". The same applies to the footnotes in the other parts of R\'ecoltes et Semailles, or the references made by a given ``note" (constituting a part of the ``main text") to a later note, which takes on the role of a ``flashback", or an annotation, relative to the first note. The present advice, together with my advice not to forget about the table of contents in the course of your reading, are the main reading recommendations which I can think of. 

\marginpar{p. L43}One last practical point remains to mention before I finally conclude (somewhat prosaically) this letter. There was a bit of a ``stampede" at times so as to ensure that the various parts of R\'ecoltes et Semailles issued from the copying Service at the University would be ready (if possible) in time for the summer holidays. As a result of this rush, there remains a whole page of last minute footnotes which ``didn't make the cut" to be added to part 2 (The Burial (1) - or The Robe of the Emperor of China). These mostly consist of rectifications of some material errors which only recently became apparent, during the writing of The Four Operations. One of these footnotes is more important than the others, and as such I would like to mention it here. It pertains to an annotation to the note ``The victim - or the two silences" (n$^o$78, page 304). In this note, I endeavor, among other things, to collect my (admittedly subjective) impressions regarding the way in which my friend Zoghman Mebkhout was then ``interiorizing" the iniquitous plundering of which he was the victim. Zoghman expressed that he felt that the note was unjust towards him, in that I apparently put him ``in the same bag" as his plunderers. One thing is for sure, namely that this note, which doesn't pretend to go beyond the recollection of impressions linked to a specific ``moment", focuses on a single aspect of the situation, while leaving unsaid (and surely as obvious) other equally real aspects (which are perhaps less conducive to debate). Nonetheless, my reflection surrounding this delicate topic had significantly deepened by the time I wrote the note ``Roots and Solitude" (n$^o$171), about a year later. The latter note was not the subject of further comments from Zoghman. Other elements of my reflection on the same topic can be found in the two notes ``Three milestones - or innocence" and ``The bygone pages" (n$^o$s 171 (x) and (xii)). The aforementioned three notes are part of ``The Apotheosis", the segment of The Four Operations centered around the operation of appropriation and hijacking to which Zoghman Mebkhout's work was subjected.

All that remains is for me to wish you a pleasant read - and to look forward to reading you in turn!

\begin{flushright}
Alexandre Grothendieck
\end{flushright}

\section*{Epilogue in postscript - or context and preliminaries to the debate}
\addcontentsline{toc}{section}{\textbf{Epilogue in postscript - or context and preliminaries to the debate}}


\begin{flushright}
February 1986
\end{flushright} 

\section{The bottle spectrograph}

\marginpar{p. L44}It has now been seven packed months since this Letter was written, and nearly four months since it was sent, along with the accompanying ``tome" - and a handwritten dedication accompanying each\footnote{With a few exceptions, especially including colleagues whom I do not know personally, and who only received parts 0 through 4 of the preprint, in recognition of their active participation in my Burial.}. Like ``message in a bottle", or rather, like a whole host of vagrant bottles, my message circulated and reached the outermost corners of the mathematical microcosm that I once knew. Through the direct and indirect echoes that are flowing back to me over the course of days, weeks, and months, I have now unexpectedly become the witness of a vast radiography of the mathematical world, which was probed by a sort of tentacular spectrograph, with each of my innocuous ``bottles" serving as a separate traveling tentacle. As such (noblesse oblige!), and due to no lack in current occupations, I am now faced with the task of deciphering said radio and producing a report of what I have read therein, to the best of my abilities. This will constitute a sixth (and last, I promise!) part of R\'ecoltes et Semailles. It will thus come to crown, God willing, ``the great sociological work of my old days". For now, I shall restrict myself to some initial commentaries.

In response to my modest and artisanal flotilla, the tone which seems to prevail is by far a half-sneering, half-snarling tone along the lines ``here comes Grothendieck, who is falling prey to paranoia in his old age", or ``he is decidedly taking himself quite seriously" - and voila! And yet, I only received a single letter taking this tone\footnote{This letter came from one of my ex-students, who is furthermore a fellow entombed.}, along with two others adopting an attitude of effaced derision and self-satisfaction\footnote{Coming from two of my old colleagues from the Bourbaki group, one of whom being the elder who had once welcomed me with warmth and benevolence at my beginnings.}. By and large, most of the mathematicians I contacted, including many of my old students, responded with their silence\footnote{Among the 131 letters mathematicians to whom I sent a copy, only 53 have thus far responded in some way or another, if only to acknowledge receipt. Among those are six of my ex-students - I have yet to hear from the remaining eight.} - a silence that speaks volumes.

\marginpar{p. L45}I have nonetheless already received voluminous feedback. The majority of the letters are framed in a tone of polite embarrassment, intended to appear as amicable, as if stemming from a desire to act with decorum. Twice or thrice, I could sense the warmth of an authentic sentiment hiding behind this embarrassment, the latter of which was muted as a result. Most often, when the embarrassment is not expressed in the form of protestations (regarding oneself, or on behalf of another), it comes out through compliments - never in my whole life have I received so many! The likes of ``great mathematician", ``superb writing" (in terms of creativity and ``all that"...), ``indisputable writer", among others. For good measure, I also received a perceptive (and by no means ironic) compliment regarding the richness of my inner life. Needless to say that in all of these letters, my correspondents kept off the heart of the matter, and wholly refrained from involving themselves personally; rather, the tone assumed was that of someone who has been ``solicited to share their opinion" (paraphrasing the language of one of the letters), on a somewhat indecorous story, which is moreover hypothetical, not to say imaginary, and which in any case \textbf{does not personally concern them}. When they do consent to touching one of these questions, they only do so with their fingertips, keeping it as far away as possible - be it by providing me with good advice, or prudent conditionals, or by means of commonplace sayings when one is at a loss for something to say, etc... Some even tacitly suggested that some unusual things may have indeed happened - all the while taking great care to remain completely vague as to what it is they are referencing...

I have also received some truly warm echoes on behalf of fifteen or sixteen of my friends old and new. Some shared an emotion with me, without trying to hide it or suppress it. These echoes, as well as other equally warm ones coming from outside of the mathematical world, will have been my reward for a long and solitary endeavor, produced not only for myself, but for all.

And among the hundred something colleagues who received my letter, only three have responded in the full sense of the term, with personal involvement rather than through a detached commentary on the century's ebbs and flows. I also received another such echo from a non-mathematical correspondent. These were true \textbf{responses} to my message - and they constituted the best of rewards.

\section{Three feet in a single dish}

\marginpar{p.L46}Many of the mathematicians among my colleagues and friends have expressed the hope that R\'ecoltes et Semailles would open a large \textbf{debate} in the mathematical world, concerning the state of affairs in the milieu, the mathematicians' ethical code, as well as the meaning and finality of their work. For now, the least I could say is that such a debate does not seem to be underway. It presently appears (to use the obvious pun) that the debate around a Burial may well have been replaced by the burial of a debate!

Whether or not the majority remains silent and apathetic, the fact remains that a debate has been started. It is unlikely that it will ever take the scale of a true public debate, or even (God forbid!) the pomp and rigor of an ``official" debate. Several people have swiftly gone past it, closing the door to their inner self before even acquainting themselves with the issue, strongly attached to the timeless and immutable consensus that ``all is for the best in the best of worlds" (in this case, the mathematical world). Perhaps a reckoning will eventually come from \textbf{without}, progressively, through ``witnesses" who, in their quality of outsiders, will not be tied down by groupthink, and as such will not perceive themselves (including in their inner core) to be personally targeted.

In almost all of the echoes that I received, I noticed a pervading confusion regarding the above two questions: \textbf{what} is the ``debate" (tacitly) suggested by R\'ecoltes et Semailles about; and who is in a position to recognize the issue and speak up about it, or to formulate an informed opinion about it.In this regard, I would like to offer \textbf{three ``orienting points"}. This will not prevent those who hold their confusion dear to continue holding onto it. As to those who would like to know what it is I am talking about, this will perhaps help them not get too distracted by the noises coming from every direction (including from the most well-meaning individuals...).

a) Certain sincere friends of mine are assuring me that ``everything with eventually be sorted out" (where the ``everything", I suppose, refers to things that have been accidentally damaged...); that I need only make my comeback, ``impose myself through new results", speak at conferences, etc - and the others would take care of the rest. They may magnanimously say that ``people acted rather unfairly toward good old Grothendieck after all", then discretely adjust their behavior with varying degrees of conviction\footnote{(*) I have already alluded to several discrete signs of such behavior, indicating that people had taken note of the fact that the lion had come out of his den...}(*);\marginpar{p. L47}. They may even paternalistically tap his shoulder and throw a few ``great mathematician" at him, intent to calm down an individual who remains respectable when all is said and done, but who alas seems to be upset and is making undesirable waves as a result.  

Unlike what said friends are suggesting, ``letting off steam" or causing steam to be let off is not what this is about. I am neither in need of compliments nor of sincere admirers, and I do not need ``allies" for ``my" cause, nor for any other cause. This is not about me - I am doing wonderfully well - nor is it about my work, which speaks for itself - even if it lands on deaf ears. If this debate regards my person and my work, among other things, it is in a \textbf{revealing} role more than anything else, through the reality of the (most revealing) Burial.

If I had to name ``someone" who in my eyes inspires a feeling of alarm, disquietude, and urgency, I would not point to myself, nor to any of my ``co-burried". Rather, I would point to a collective being, at once elusive and very much tangible, about whom we often speak but refrain from ever examining - namely, \textbf{``the mathematical community"}.

Over the course of the past weeks, I have come to see it as a person in flesh and blood, whose body is stricken by severe gangrene. Food of the highest quality and the most select of dishes turn into poison when fed to it, only serving to further propagate and entrench the disease. Yet, an irresistible bulimia pushes it to binge on ever more food, as if this were a way to even out the disease which it avoids facing at all costs. Nothing one could say could get to it - even the simplest of words have lost their meaning. They have stopped being communication vehicles, and they now only act as triggers of fear and denial...

b) Most of my colleagues and old friends, even when acting in good faith, only risk voicing an opinion by surrounding it by cautious conditionals such as ``if it were true that.... then this would indeed be inadmissible" - after which they may soundly go back to sleep. And here I am thinking I had been clear...

With seven months of hindsight, I am able to confirm that the \textbf{quasi-totality of the facts} reported and commented upon in R\'ecoltes et Semailles \textbf{are uncontroversially correct}. I will come back to the rare exceptions later, and they will be signaled as such, each in due time. As for all of the remaining fact, following the writing of the preliminary version of\marginpar{p. L48} R\'ecoltes et Semailles, a careful confrontation with some of those principally concerned (namely, Pierre Deligne, Jean-Pierre Serre, and Luc Illusie) allowed me to eliminate technical mistakes, so as to arrive to an unambiguous agreement regarding the material facts proper\footnote{(*) I happily extend my gratitude to all three of them for the good faith which they have demonstrated in this occasion; I hereby recognize their complete good faith in answering questions about material facts.}(*).

Thus, the debate does not revolve around the accuracy of the facts at hand, which is not in doubt; rather, it concerns the question of \textbf{whether the practices and attitudes illustrated by these facts should or should not be considered to be admissible and ``normal"}.

I am referring to practices which I qualify (perhaps wrongly...) in my testimony as scandalous; as breaches of trust, abuses of power, and glaring acts of dishonesty, often displayed with brazenness and iniquitousness. The unimaginable thing that I had yet to discover was that, upon learning about these facts (which would have been unthinkable just fifteen years ago), a large majority of my mathematical colleagues, including some of my ex-students and friends, thought these practices to be normal and perfectly honorable. 

c) Another approach to which several of my colleagues and old friends resort to maintain a confusion is by saying a version of the following: ``sorry, but we are not qualified to address these matters - stop asking us to take stock of a situation which (providentially...) goes above our head...".

Quite to the contrary, I assert that there is no need for ``qualifications" (and for that I am sorry in turn!) in order to become aware of the principal facts - not even the need to know one's multiplication table, or Pythagoras' theorem; nor to have read ``Le Cid" or La Fontaine's Fables. A normally developed ten year old child would be just as qualified as the specialists of highest repute (if not more so...)\footnote{(**) Of course, R\'ecoltes et Semailles is not addressed to said ten year old child - had it been so, I would have chosen a language which he would have found more familiar.}(**).

Allow me to illustrate this claim by an example, the ``first to come to mind" from the Burial\footnote{(***) This is the first ``great operation" pertaining to the Burial which I have discovered, on a fated April 19$^{th}$ 1984, the same day that the title ``The Burial" occurred to me. See on this subject the two notes written on the same day, ``Memories from a dream - or the birth of motives", and ``The Burial - or the New Father" (ReS III, n$^o$51, 52). The complete reference for the book about to be mentioned may also be found there.}(***). There is no need to understand the ins\marginpar{p. L49} and outs of the multifaceted and delicate notion of ``motive", nor to have even received one's elementary school certificate, in order to take stock of the following facts, and to emit a judgement thereupon.

\begin{itemize}

\item 1) Between 1963 and 1969, I introduced the notion of ``motive" and developed surrounding it a ``philosophy" and a ``theory" which remained partially conjectural. Rightly or wrongly (it doesn't matter for the present purposes), I consider the theory of motives to be the most profound contribution that I have made to the mathematics of my time. Today, nobody questions the importance and the depth of this ``motivic yoga" (and this after ten years of quasi-total silence surrounding this subject, following my departure from the mathematical world).

\item 2) In the first and only book (published in 1981) devoted for the most part to the theory of motives (whose name, which I first introduced, appears in the title), the only passage that could indicate to the reader that my humble self had anything to do with a theory that could resemble the one extensively developed in this book can be found on page 261. This passage (two and a half lines long) consists in explaining to the reader that the theory developed in the book has nothing to do with that of a so-called Grothendieck (a theory which is mentioned there for the first and last time, with no further reference or details).

\item 3) There is a famous conjecture, called the ``Hodge conjecture" (no need to know what exactly it is about), whose validity would imply that the so-called ``other" theory of motives developed in this brilliant volume is identical to (a very special case of) the theory which I had developed, out in the open for everyone to see, nearly twenty years earlier.
\end{itemize}

To these, I could also add 4) that the most prestigious of the four co-signatories of the book was once my student, and that he learned from me the brilliant ideas which he hereby presents as if he had just thought of them this very instant\footnote{(*) I am not implying that this book is devoid from ideas, or even beautiful ideas, which are attributed to this author or to the other co-authors. Rather, I am highlighting that the chief concern of the book, and the conceptual context which gives it meaning, including the delicate theory of $X$-categories (wrongly called ``Tannakian categories") which lies at the heart of the book, are issued from my own work.}(*), 5) that the aforementioned two circumstances are a matter of public knowledge among the informed public, but that it would be vain to try to find a written acknowledgment in the literature that said brilliant author could\marginpar{p. L50} have learned anything from me\footnote{(*) An exception should be made for a line included in a handwritten report to Serre from 1977, which will be addressed in due time.}(*), and 6) that I was the one who formulated the delicate arithmetic question which (according to the principal author's personal explanation) constitutes the core of the book (without my name ever being mentioned surrounding it) during the 1960s, in the wake of the ``yoga of motives", after which the author learned it from me; I could follow-up with a 7), an 8) and so forth (and in fact I do so in due time).

The above shall suffice to make my point, which is the following: in order to take stock of the above facts and to emit a judgement thereupon, there is no need for any particular ``competencies" - \textbf{the heart of the matter is not ``taking place" at that level}. The faculty which is herein called for, other than a sound mind (possessed in principle by each and every one), is what I will call \textbf{a sentiment of decency}.

The book in question is already one of the most cited works of the mathematical literature, and its ``principal author" one of the most prestigious mathematicians of the era. That being said, the thing that is by far the most remarkable in my eyes, in this whole story, is that \textbf{not a single one of the countless readers of this book}, including those who know firsthand about the state of affairs, being my ex-students or friends \textbf{noticed that anything was out of the ordinary}. Or at least, not a single one, up to the very day where I am writing these lines, has come to me to express reserve of any kind\footnote{(**) All in all, only two colleagues (including Zoghman Mebkhout) have expresses such ``reserves" to me. Neither of them may be considered to be among the targeted ``readers" of this book. They opened it out of curiosity, so as to see things for themselves...}(**) regarding this prestigious book.

As to those, among my colleagues and old friends, who have never held this book in their hands, and fall back on this fact to plead their incompetence, I would like to tell them the following: there is no need to be a ``specialist" in order to ask to see the volume at the nearest mathematical library, skim through it, and thereby see for yourself what is not contested by anyone...

\section{Gangrene - or the spirit of our times (1)}

This ``operation motives" is but one of four ``great operations" of the same type, which in turn fit into a swarm of other events of lesser scale but of similar spirit. It is neither the most ``obnoxious" of the collective mystifications which\marginpar{p. L51} come to form the ``tableau des moeurs" of an era, nor the most iniquitous. It consisted only in pillaging the wealthy man's herd in his absence (or demise...), rather than coming to strangle (in a general climate of indifference) the poor man's lamb, for sports and right before his eyes. Even within the mathematical language now commonly in use, certain apparently anodyne names of books, notions, and theorems cited left and right are themselves indicative of a mystification and an imposture\footnote{Here, I am mostly referring to the bizarre acronym ``SGA 4$\frac{1}{2}$" (how convenient to have access to fractions!), a double imposture all by itself (and one of the most cited acronyms of contemporary mathematical literature), as well as to the names ``Verdier duality", ``Grothendieck-Deligne conjecture", and ``Tannakian categories" (for the latter, Tannaka himself is not to be blamed, as he was never consulted...). I will come back to this topic in more depth at a later time.}, thereby serving as witnesses to the disgrace of an era.

If I have ever made a positive contribution to the ``mathematical community", it was by shedding light on a number of disreputable facts which were lurking in the shadows. These were furthermore facts which everybody interacted with on a near-daily basis in some capacity. Yet, how many among us have ever taken the time to stop, take a look, and smell the air surrounding it?

Some of those who have been faced with the arrogance or dishonesty of certain individuals (sometimes one and the same) may have filed away the incident as a strike of bad luck, or as something solely targeted at them. In comparing their experience with my testimony, perhaps they will sense that this ``bad luck" is just a name that they have given to the \textbf{spirit of our times}, which weighs on them just as much as it weighs on everyone else. And (who knows!) perhaps this realization will encourage them to become involved in a debate, which concerns them just as much as it concerns me.

But if the ``dirty laundry" which I am ``spreading out in a public place" induces only joyless snickers from some and polite embarrassment from others, amidst general indifference, what was once a confusing situation would become very clear. (At least for those who still care to see things with their own eyes.) The traditional consensus of good faith and decency\footnote{In mentioning these ``consensus of good faith and decency", I am not suggesting that they were never violated. But even when they were violated, the act was treated as a ``violation", and the consensus did not become any less accepted thereafter.} in the relationships\marginpar{p. L52} between mathematicians and between a mathematician and his art would have become a thing of the past, ``outdated". The following standard would have become understood and quasi-official, without the need for some international association of mathematicians to solemnly proclaim it: from now on, \textbf{it's a free for all}, without any further reserve or limitations, for the ``co-option brotherhood" consisting of the powerful individuals of the mathematical world. Everything is fair game, including the blurring of ideas' origins intended to lead astray the apathetic reader who will readily believe anything, the trafficking of authorship, blank-citations between associates and silent treatments for those condemned to silence, cronyism and falsification of all kinds, all the way to the most heavy-handed plagiarism in full-view - \textbf{yes, and amen to it all}, with the benediction, openly voiced or through silent agreement (if not by actively and hastily participating) of all of the ``household names" and all of the bosses, big and small, on the mathematical public square. Yes, and amen to the \textbf{``new style"} that's all the rage! What was once an art has now become, through (quasi-) unanimous agreement, , 
under the paternal gaze of the leaders.

There was a time when the exercising of power in the mathematical world was constricted by unanimous and intangibles consensus, indicative of a collective feeling of \textbf{decency}. These consensus and this collective feeling would thereafter have become obsolete and outdated, unworthy surely of the glorious era of computers, space shuttles and neutron bombs.

The following principle would become set in stone and taken for granted: the use of power, for the members of the dominating brotherhood, is \textbf{discretionary}.

\section{Honorable amend - or the spirit of our times (2)}

In the Letter, I believe I have been sufficiently clear regarding the spirit in which I have written R\'ecoltes et Semailles; in particular, regarding the fact that I did not in any way aim to act as a historian. R\'ecoltes et Semailles is first and foremost a testimony produced in good faith about my lived experience and my reflection thereupon. The testimony and reflection are available to all, including historians, who may use it as one primary source among others. It will be the historians' task to critically examine this primary source, in conformity with the standards of rigor of their art.

It is naturally in order to distinguish between facts in the strict sense (\textbf{``raw facts"}, or ``material facts") and the ``evaluation" or \marginpar{p. L53}\textbf{``interpretation"} of these facts giving them a \textbf{meaning}, the latter possibly differing from one observer (or co-actor) to the next. Roughly, one could say that the ``testimony" aspect of R\'ecoltes et Semailles concerns the facts, while the ``reflection" aspect concerns their interpretation, i.e. the process through which I have assigned meaning to them. Among the ``facts" constituting this testimony, I am including the ``psychological facts", notably the feelings, associations, and images of all kinds, dating from a more or less distant past or occurring at the time of writing, of which my testimony is the reflection.

There are three kinds of \textbf{sources} for the facts which I describe or relate in R\'ecoltes et Semailles. Some of the facts are obtained from my \textbf{memory}, more or less precise or hazy from one instance to the next, and sometimes distorted. I can vouch for my intention to be truthful at the time of writing these facts, but I cannot guarantee the absence of mistakes. Quite to the contrary, I have been able at times to locate a number of mistakes regarding details, and I signal each one in time in subsequent footnotes. Secondly, there are \textbf{written documents}, notably letters and mostly scientific publications proper, to which I refer with suitable precision. Finally, there are \textbf{testimonies from third parties}. Sometimes, the latter come as complements to my own memories, allowing me to rekindle them, to make them more precise, and sometimes to correct them. In some rare occasions (to which In will soon return), the testimony brings in entirely new information with respect to my understanding at the time. When I choose to echo a given testimony, I do not guarantee that I have been able to thoroughly verify its exactness and well-foundedness, only that it inserted itself sufficiently convincingly in the rich thread of facts which I knew about firsthand, leading me to believe (rightly or wrongly...) that this testimony was indeed mostly true.

I believe that the attentive reader will have no trouble ``sorting out" at any time what constitutes a retelling of facts as opposed to an interpretation thereof, and (in the first case) to discern which of the three sources I described is relevant.

\begin{center} * \ \ \ \ * \\ *  \end{center}

\marginpar{p. L54}The testimony from a third party which I just alluded to, and which I have been echoing without having been able to ``thoroughly verify its validity", is \textbf{Zoghman Mebkhout}'s testimony regarding the vast operation of discrediting surrounding his work. Among the ``material facts" which I enumerate in R\'ecoltes et Semailles, the only facts which are currently subject to controversy or which, according to my present judgement, are in need of rectification, are some of the facts supported only by Mebkhout's testimony. Before ending this post-scriptum, I would like to offer some critical commentaries regarding the version of the ``Mebkhout affair" presented in the preliminary printing of R\'ecoltes et Semailles. More in-depth commentaries and reflections will be included, each in its place, in the printed edition constituting the definitive text of R\'ecoltes et Semailles. 

The ``Mebkhout version" for which I have been the interpreter seems to essentially boil down to the following two theses:
\begin{itemize}

\item 1. Between 1972 and 1979, Mebkhout was the only one\footnote{An exception should be made for Kashiwara's constructibility theorem in 1975, whose importance is in no way contested. But according to Mebkhout's version, this was Kashiwara's one and only contribution to the emerging theory. This (inexact) version was corroborated by the absence of other publications by Kashiwara, in which he would likely have at least alluded to some of the key ideas.} to develop, amidst a general atmosphere of indifference and drawing inspiration from my work, the ``philosophy of $\mathcal{D}$-modules", viewed as a new theory of ``cohomology coefficients" in my sense of the term.

\item 2. There was unanimous consensus, both in France and abroad, to retract his name and his role in this new theory once its scope started being recognized.

\end{itemize}
This version was extensively documented, on the one hand through Mebkhout's perfectly convincing publications, and on the other hand through several publications by other authors (notably, the \textbf{Actes} of the June 1981 Luminy Colloquium), wherein the retraction claim no longer raises any doubts. Finally, the more in-depth details which Mebkhout shared with me at a later date (and which I echo in the part ``The Burial (3) - or the Four Operations"), while not being directly verifiable, were entirely aligned with a certain general atmosphere whose reality was no longer in doubt in my eyes.

\marginpar{p. L55}I have just been made aware of several new facts\footnote{I am grateful to Pierre Schapira and to Christian Houzel for drawing my attention to these facts, and to my tendencious treatment of the Kashiwara-Mebkhout dispute.}, which indicate that the aforementioned point 1) deserves to be strongly nuanced. The isolation under which Mebkhout operated\footnote{This isolation was the result first and foremost of the indifference of my ex-students for Mebkhout's ideas and for his work, the price he had to pay for taking inspiration from an ``ancient figure" destined to oblivion by unanimous consensus...} was real, but it was only a relative isolation. In France, \textbf{J. P. Ramis} was concurrently producing work in the same subject (works which Mebkhout never mentioned to me); secondly and most importantly, it appears that some of the important ideas which Mebkhout developed and carried out, and for which he claims authorship, could be due to Kashiwara\footnote{The most important of these ideas is that of the so called ``Riemann-Hilbert correspondence" (to employ the trending jargon) for $\mathcal{D}$-modules. The relevant conjecture was proven by Mebkhout, as well as (according to Schapira's correction) Kashiwara (even though Mebkhout assured me that his proof was the only one to have been published). The question of which of the two proofs appeared first remains unclear in my eyes, and I do not intend to spend the rest of my days trying to arrive at an answer...

As for the sister-statement in terms of $\mathcal{D}^{\infty}$-modules, there seems to be no doubt that the authorship for both the idea and the proof belongs to Mebkhout.}. This renders some of the episodes of the Kashiwara-Mebkhout dispute implausible or doubtful in the form in which they are reported in the Mebkhout version, for which I took on the role of (overly) faithful interpreter.

There is no doubt about the fact that, in terms of ``full-fledged work", as well as regarding the conception of some of the ideas which he was able to carry out to their end, Mebkhout was one of the principal pioneers of the new theory of $\mathcal{D}$-modules, if not \textbf{the} principal pioneer; in any case, he was the only one to have invested himself body and soul into this task, whose true scope was still eluding him, just as it was eluding everyone else. It also remains true that the retraction operation that took place surrounding his work, culminating in the Luminy Colloquium, is in my eyes one of the greatest disgraces of the century in the mathematical world. But it would be wrong to pretend (as I once did in good faith) that Mebkhout was the only one working on the subject. On the other hand, he was the only one with enough honesty and courage to clearly spell out the importance of my ideas and my contributions in his work and in the burgeoning of this new theory.

This post-scriptum is not the place to write about this affair in more details. I shall do the latter in due time, including commentaries attempting to shed light on the psychological context of the ``Mebkhout version". If the ``contentious Mebkhout-Kashiware dispute" is of interest to me, it is only to the extent that it sheds light on the general atmosphere of an era. In my opinion, the ``Mebkhout version" also belongs, along with its deformations and due to the forces that led to its formation, to the list of less contestable materials which I am adding to the ``folder of an era", as an eloquent ``sign of our times".

\marginpar{p. L56}All that remains is for me to make an honorable amend for my flimsiness in presenting the Mebkhout-Kashiwara dispute in a way that only took into account the testimony and documentation provided by Mebkhout, as if this version was beyond any doubt; and for doing so even when this version presented a third party as ridiculous, or even odious, providing me with all the more reasons to exercise caution. For my flimsiness and lack of sane caution, I would like to hereby extend my most sincere excuses to M. Kashiwara.

%\end{document}

