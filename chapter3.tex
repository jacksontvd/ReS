\begin{comment}
\documentclass{book}
\usepackage{master}
\newcommand{\rec}{$\text{R\'ecoltes et Semailles}$}
\newcommand{\no}{n$^\circ$}
\hfuzz = 100pt
\begin{document}
\setcounter{chapter}{2}
\end{comment}

\chapter{A letter}

May 1985

\section{The one-thousand page letter}

\marginpar{p. L1}The text which I am hereby sending you, of which a limited number of copies were typed and printed by my university, is neither an off-print, nor a preprint. Its title, R\'ecoltes et Semailles, makes this much clear. I am sending it to you the way I would send a long letter - including the personal dimension, indeed. If I have decided to send it to you, rather than await for you to learn about it some day (if your curiosity leads you to it) in the form of some publicly available volume in a library (if there even exists an editor crazy enough to engage in such an adventure...), it is because I am addressing this letter to you more than to others. I have thought of you more than once in the course of writing it - I must say that I have now been writing this letter for more than a year, and devoting all my energy to the task. It is a gift I am making you, and I took great care in the process to give out what I had best to offer (at any given moment). I do not know whether or not you will welcome this gift - until your response (or absence thereof) brings me the answer. 

At the same time as I am sending you R\'ecoltes et Semailles, I am also sending it to all of the colleagues, friends, and (ex-)students of the mathematical world with whom I was close at one time or another, as well as to those who appear in my reflection in some form, both named and unnamed. There is a chance that you yourself appear in what follows, and if you make the effort to read it not only with your eyes and head but also with your heart, you will surely recognize yourself even in places where you are not explicitly named. I am also sending R\'ecoltes et Semailles to a handful of other friends, both inside and outside the scientific community.

This ``letter of introduction" which you are currently reading, which announces and introduces a ``one-thousand page letter" (to begin with...), will also serve as a Foreword. The latter has not yet been written at the time of writing these words. Additionally, R\'ecoltes et Semailles consists of five parts (including an introduction ``with drawers"). I am hereby sending you parts I (Fatuity and Renewal), II (The Burial (1) - or the Robe of The Emperor of China), and IV (The Burial (3) - or the Four Operations)\footnote{I am singling out colleagues who appear in my reflection in some way, but who I do not know personally. To those I am only sending ``The Four Operations" (which particularly concerns them), as well as ``booklet O" which consists of the present letter together with the Introduction to R\'ecoltes et Semailles (as well as the detailed table of contents for the first four parts).}. These are the parts which seemed to concern you in particular. Part III (The Burial (2) - or the Key of the Yin and Yang) is the most personal segment of my testimony, and at the same time it is the part which, more so than\marginpar{p. L2} the others, appears to me to hold ``universal" value, beyond the particular circumstances that surrounded its creation. I refer to that part in various places in part IV (The Four Operations), which can nonetheless be read independently, and even (to a large extent) independently of the three parts which precede it\footnote{(*) More generally, you will notice that each ``section" (in Fatuity and Renewal) and each ``note" (in any of the following three parts of R\'ecoltes et Semailles) has its own unity and autonomy. Each can be read independently of the rest, the way one can find interest and stimulation in simply observing a hand, a foot, a finger, a toe, or any part, large or small, of the human body, without forgetting that it is part of a Whole, and that it is only with respect to that Whole (which remains unspoken) that the part takes on its full meaning.}(*) If reading what I have sent you prompts you to respond (as is my wish), and if it makes you want to read the missing part as well, do let me know. It will be my pleasure to send it to you, as long as your response makes it clear that your interest goes beyond superficial curiosity.

\section{Birth of R\'ecoltes et Semailles (a lightning fast retrospective)}

In this pre-letter, I would like to tell you in the span of a few pages (if at all possible) what R\'ecoltes et Semailles is about - to do so in more details than can manage the subtitle: ``Reflections and testimony about the past of a mathematician" (my past, as you will have guessed...). R\'ecoltes et Semailles contains many things, and many will see them as several different things: a \textbf{voyage} of discovery through the past; a \textbf{meditation} on existence; a \textbf{painting of mores} of a milieu and an era (or a painting of the insidious and unstoppable transition from one era to the next...); an \textbf{investigation} (almost police-like at times, and elsewhere approaching the style of a cloak-and-dagger novel taking place in the underbelly of the mathematical megapolis...); a vast \textbf{mathematical divagation} (which will lose more than one reader...); a practical treaty in applied psychoanalysis (or, alternatively, a book of \textbf{``psychoanalytic fiction"}); a eulogy of \textbf{self-knowledge}; \textbf{``My confessions"}; a private \textbf{diary}; a psychological study of the processes of \textbf{discovery} and \textbf{creation}; an \textbf{indictment} (unforgiving, as it must...), or even a \textbf{settling of accounts} in the mathematical ``beau monde" (in which no punches are pulled...). If there is one thing I can guarantee, it is that I never once was bored in the process of writing it, and that from it I have learned and seen a great deal. If you find the time to read it among your other important duties,\marginpar{p. L3} I doubt that you will be bored. Unless you force yourself to read it, who knows...

As such, this work is not only meant to be read by mathematicians. It is true that certain parts of it will cater to mathematicians more than to others. In this pre-letter to R\'ecoltes et Semailles, I would like to summarize and highlight what it is, then, which concerns you in particular as a mathematician. The most natural way to go about this is to simply tell you how, one thing leading to another, I progressively got around to writing the four or five ``tomes" mentioned above.

As you know, I left the ``grand monde" of mathematics in the year 1970, following an issue of military funding at my home institution (the IHES). After a few years devoted to anti-military and ecological activism, in the style of a ``cultural revolution" - about which you probably received echoes here and there - I practically disappeared from the public sphere, and settled at a remote provincial university. Rumor has it that I am spending my days herding sheep and digging wells. The truth is that, in addition to several other occupations, I am bravely continuing to give university lecture, like everybody else (teaching was my first source of income, and it remains so to this day). It even just so happened that I would sometimes spend a few days, or even a few weeks or a few months doing mathematics again - I have filled several boxes with my doodles, which I am probably the only person in a position to decrypt. However - at least at first sight - these projects revolved around very different topics than the ones I used to work on. Between the years 1955 and 1970, my theme of predilection had been cohomology, more specifically the cohomology of varieties of all kinds (and of algebraic varieties in particular). I considered that I had done enough work in that direction for others to carry on without my help, and decided that if I were to continue doing mathematics, I might as well change things up...

In 1976, a new passion appeared in my life, one about which I felt as strongly as I once felt about mathematics, and which is closely related to the latter. It was a passion for what I call ``meditation" (things need to be named after all). This name, like any other, risks leading to countless misunderstandings. As for mathematics, the process of meditation is one of discovery. I express myself on this subject at various points in R\'ecoltes et Semailles. It visibly held enough in store to keep me occupied for the rest of my life. In fact, I have more than once gotten to thinking that mathematics was now a matter of the past, and that it was time for me to orient myself towards more serious matters - time to ``meditate".

\marginpar{p. L4}I nonetheless ended up facing the evidence (four years ago) that my passion for mathematics was still very much alive. In fact, to my own surprise, and despite my long-standing conviction (for almost fifteen years) that I would never publish a single new line of mathematics in my lifetime, I found myself suddenly engaged in the writing of a mathematical project which seemed never-ending and would require producing volume after volume; and while I was at it, I might as well just write out all that I had to say about mathematics in an (infinite?) series of books which would be called ``Mathematical Reflections", and that would be that.

This project began two years ago, during the spring of 1983. I was then already too busy writing (volume 1 of) ``Pursuing Stacks" (``\`A la Poursuite des Champs" in French), which also was to constitute volume 1 of the (mathematical) ``Reflections", to stop and reflect on what was happening to me. Nine months later, as is fit, this first volume was virtually complete, and all that was left for me to do was to write the introduction, re-read everything, add annotations - and off it could be sent to be printed... 

The volume in question is still not finished to this day - it hasn't moved by an inch for the past year and a half. The introduction that I had left to write grew to take over twelve hundred pages (typewritten), and when all will be said and done I estimate it will be some fourteen hundred pages long. You will have guessed that this ``introduction" is nothing but R\'ecoltes et Semailles. Last I checked, it was supposed to constitute volumes 1 and 2, as well as part of volume 3 of the much-feted ``series" in the works. The latter had to undergo a name change and is now called ``Reflections" (i.e. not necessarily mathematical). The remainder of volume 3 will consist mostly of mathematical writings, ones which I deem more pressing than Pursuing Stacks. The latter can wait until next year for me to come around to adding annotations, an index, and, of course, an introduction...

End of the first Act!

\section{The death of the boss - or abandoned construction sites}

I sense that it is time for me to provide some explanations as to why I so abruptly left a world in which I had apparently felt at ease for more than twenty years of my life; why I had the strange idea of ``coming back" (like a ghost...) when everybody seemed to have been doing just fine without me for the past fifteen years; finally, as to why the introduction to a mathematical work of six or seven hundred pages grew in turn to reach the length of twelve (or fourteen) hundred pages. As I cut to the chase, I will doubtlessly sadden you (sorry!), perhaps even upset you. For you, as I once did, surely prefer to see through ``rose-colored glasses" the milieu to which you belong, the one in which you have found your place, your name and so on. I know what this is like... And what follows might cause some teeth grinding...

\marginpar{p. L5}I mention the episode of my departure at various places in R\'ecoltes et Semailles, without lingering on it. This ``departure" serves rather as an important rupture in my life as a mathematician - it is in relation to that ``reference point" that the events of my life as a mathematician arrange themselves, taking place on either side of a ``before" and an ``after". A very strong \textbf{shock} was needed to uproot me from a milieu in which I was firmly entrenched, and from a clearly delineated ``trajectory". This shock came in the form of a confrontation, within a milieu with which I strongly identified myself, with a certain kind of corruption\footnote{I am referring to the open collaboration, ``establishments" at the head, of scientists from all of the world's countries with military institutions, as a convenient source of funding, prestige, and power. This question is barely scratched in passing, once or twice, in R\'ecoltes et Semailles, such as for instance in the note ``Respect" from April 2$^{nd}$ of last year (n$^o$ 179, pages 1221-1223).
%todo: ref
} which I had chosen to ignore up to that point (by simply abstaining from participating in it). In hindsight, I eventually realized that beyond that singular event, there was a deeper force at work within me, signaling an intense \textbf{need for an inner renewal}. Such a renewal could not be accomplished nor pursued in the lukewarm atmosphere that is the scientific vacuum of an institution of high-standing. Behind me lay twenty years of intense mathematical creativity and of mathematical devotion beyond measure - and, at the same time, twenty years of spiritual stagnation, ``in a silo"... Without realizing it, I was suffocating - what I needed was some fresh air! My providential ``departure" marked the sudden end of a long stagnation period, and it also marked the first step taken towards an equalization of the deep forces at play within my being, which were folded and screwed in a state of intense disequilibrium, frozen in place... This departure was, truly, a \textbf{new start} - the first step in a new journey...

As I said earlier, my passion for mathematics nonetheless remains alive. In recent times, it has found expression in the form of reflections that have remained sporadic, and going in directions which are different from those that I had been working on ``before". As to the \textbf{body of work} which I was leaving behind, what I had produced ``before", including both published texts and, perhaps even more importantly, material which hadn't yet reached the stage of writing and publication - it could almost appear as if it had effectively become detached from my person - and it seemed so to me. Up until last year, with the beginning of R\'ecoltes et Semailles, I never thought of ever ``weighing in" on the scattered echoes that reached me here and there.
%todo: think about use of ``weighing in" for ``poser"
I knew that all that I had done in mathematics, and in particular what I had produced during my \marginpar{p. L6}``geometric" period between 1955 and 1970, were things that \textbf{had} to be done - and that the things which I had seen or glimpsed were things that \textbf{had} to appear, that \textbf{must} be brought to light. Additionally, I knew that the work which I had done, as well as the work which was done under my direction, was work well done, and that I had applied myself to it entirely. I had devoted all of my strength and love to it, and (or so it seemed to me) it could henceforth carry on autonomously - as a living and vigorous body which no longer needed to rely on my parental care. On this front, I left with a perfectly clear conscience. There was no doubt in my mind that the written and unwritten things which I was leaving behind were in good hands, put under the care of others who would make sure that they would deploy themselves, grow and multiply following the intrinsic nature of living and vigorous things.

During these fifteen years of intense mathematical work, a vast \textbf{unifying vision} had hatched, matured, and grown within me, taking the form of a handful of very simple \textbf{id\'ees-force}. It was the vision of an ``arithmetic geometry", a synthesis of topology, geometry (both algebraic and analytic), and arithmetic, a first embryo of which I had found in the Weil conjectures. This vision was my principal source of inspiration during those years, a period which I mostly remember as the one during which I managed to formulate the key ideas of this novel geometry, and to develop some of its main tools. Over time, this vision and these ``id\'ees-force" became second nature to me. (And this feeling of ``second nature" persists in me to this day, despite having ceased all contact with these ideas for nearly fifteen years!) I found them to be so simple, so obvious, that it was natural for ``everyone else" to internalize them and to make them their own over time, at the same time as I went through these motions myself. It is only recently, during the past few months, that I realized that neither this vision, nor the ``id\'ees-force" which had been my constant guides, could be found in writing in any existing publication, save perhaps for tacit appearances between the lines. Most importantly, I also realized that this vision which I thought I had imparted to others, and the ``id\'ees forces" which carry it, remained ignored by all to this day, twenty years after having reached their full maturity. I alone, the worker and servant to these things which I had the privilege of discovering, remain the sole vessel in which they have remained alive.

Some tool or other which I have crafted will find itself used in various places to ``break open" a problem reputed for being difficult, the way one would break open a safe. The tool is visibly solid. Yet, I am aware of a  ``force" it has other than that of a crowbar. The tool is part of a Whole, just as a limb\marginpar{p. L7} is part of a body - it is part of a Whole from which it is issued, which gives it its full meaning and infuses it with strength and life energy. Granted, you can use a bone (if it is big enough) to break open a skull. But such is not its true function, its ``raison d'\^etre". Yet, I am witnessing these tools being dispersed, grabbed by one person or another, like bones being carefully butchered and cleaned, after being torn from a body - a living body that they are pretending to ignore...

What I am hereby spelling out in carefully chosen terms, at the term of a long reflection, I probably first noticed progressively and vaguely, over the course of successive years. It first occurred to me at the level of the unformulated which does not yet seek to take the form of a thought or conscious image, nor that of clearly articulated speech. I had decided that the past, after all, no longer concerned me. The echoes that occasionally reached me, although filtered, were nonetheless eloquent. I had considered myself a worker among others, busying myself on five or six ``construction sites"\footnote{I speak about these deserted ``construction sites", and I eventually list them, in the series of notes ``The deserted construction sites" (n$^o$176 through 178) written three months ago. A year prior, before the discovery of the Burial, I had already touched on this, in the first note in which I resume contact with my previous work and its recent course, titled ``My orphans" (n$^o$46).} in full swing - a more experienced worker perhaps, the senior who for many years was the only person working on these sites, waiting for a welcome succession; senior, perhaps, but not fundamentally different from the others. And yet, upon his departure, it was as if a masonry enterprise had gone bankrupt, following the unexpected passing of the boss: from one day to the next, so to speak, the construction sites were deserted. The ``workers" were gone, each of them carrying some small gewgaw which they thought may be of use at a later time. The cash register was gone, and as such there was no longer any reason for them to tire themselves out at work...

The above is once again a formulation that is the result of a slow process of decantation, the end product of a reflection and an investigation that took place over the course of more than a year. Yet, this was surely something I already felt ``on some level" in the first few years following my departure. Putting aside Deligne's work on the absolute values of Frobenius eigenvalues (the ``million dollar question", from what I have recently gathered...) - whenever I happened to come across a close acquaintance from yesteryear, someone with whom I had worked on the same construction sites, and asked them ``so... ?", I was always met with the same eloquent gesture, arms in the air\marginpar{p. L8} as if asking for grace... Visibly, they were all too busy working on other things which were more important than the projects which were close to my heart - and, just as visibly, while they were all chugging along with an occupied and important air, nothing much was actually being done. The essential feature of mathematical work had disappeared - the presence of a \textbf{unity} which gave meaning to each partial task, and also, I believe, the presence of a \textbf{warmth}. What remained was a scattered collection of tasks unattached to a whole, with each worker hiding their little bounty away in a corner, or scrambling for a way to bring it to fruition. 

I couldn't help but to feel sorrow over the fact that everything seemed to have stopped in its tracks; I no longer heard news about motives, topoi, the six functor formalism, De Rham and Hodge coefficients, nor about the ``mysterious functor" which was supposed to unite under one umbrella De Rham and $l$-adic coefficients for all prime numbers, nor about crystals (except to learn that they remained at a standstill), nor about the ``standard conjectures" and other conjectures which I had formulated and which, evidently, represented crucial questions. Even the vast foundational work begun in the El\'ements de G\'eom\'etrie Alg\'ebrique (with the unflagging help of Dieudonn\'e), which one needed only push along a set track, was left behind: everybody was content to simply settle between the four walls and amidst the furniture that someone else had patiently assembled, built, and polished. With the worker gone, nobody had the inspiration to rise up in turn and to get their hands dirty, in order to construct the many buildings that were yet to be erected, \textbf{houses} which would be good to live in, for oneself and for others...

I once again couldn't resist rendering on the page these fully conscious images which emerged and came to light as a result of a work of reflection. There is no doubt in my mind that these images were already present in some form in the deeper layers of my being. I must have already felt the insidious reality of a \textbf{Burial} of my life's work and of my person on April 19$^{th}$ of the previous year - it suddenly appeared to me on that day, with undeniable strength and under that very name, ``The Burial". Yet, at the conscious level, I never felt offended nor even afflicted. Whatever a person, close to me or not, chose to do with their time was entirely their concern. If what had once motivated them or inspired them no longer did, that was their business, not mine. If the same shift seemed to happen, without fault, to every single one of my ex-students, it was yet again each of their personal business\marginpar{p. L9}, and I had other things to do than to start looking for an explanation, and that was that! As for the things which I had left behind, and to which a profound and ignored link continued to tie me - despite their visible state of abandon, left on deserted construction sites - I knew that they needn't fear ``the assault of time" nor the fluctuations of fashion. Even if they had not yet entered the common patrimony (something which I mistakenly believed had already happened), they would inevitably take root eventually, be it in ten years or in a hundred years, no matter...

\section{A wind of burial...}

Even though I chose to ignore the diffuse perception of a large scale Burial for years on end, this burial nonetheless obstinately returned to haunt me in different guises, less innocuous than that of a mere  disaffection with my work. I slowly came to realize, in ways I could not quite explain, that many of the constituent notions of the forgotten vision had not only fallen into disuse, they had also become, within a certain ``beau monde", the objects of condescending disdain. Such was notably the case for the crucial and unifying notion of topos, which lays at the very heart of the novel geometry - and which provides a common geometric intuition for topology, algebraic geometry, and arithmetic. The notion of topos was also pivotal to my formulation of the \'etale and $l$-adic cohomology tools, as well as to the key ideas (since then more or less forgotten, admittedly...) underlying crystalline cohomology. In fact, it was my very name which, insidiously, mysteriously and over the course of the years, had become an object of derision - becoming a synonym for rambling discourses ad infinitum (such as those I produced on the famous ``topoi", or on these ``motives" that I kept dwelling on and which nobody had ever seen...), for counting angels dancing on the head of a pin for thousands of pages on ends, and for plethoric and gigantic discussions about things which everyone already knew anyway without having to read about them somewhere...  Such was the tone being held, albeit in muted voices, by means of innuendos, and with all of the delicacy that is in order among ``high-minded people of esteemed company". 

During the reflection pursued in R\'ecoltes et Semailles, I believe I was able to point towards the deep forces at play in various characters, forces which are responsible for the airs of derision and condescension they tend to display when confronted with a work whose scope, life and breath are beyond them. I have also discovered (apart from the particular traits of my person which have influenced my work and my fate) the secret \textbf{``catalyst"} who incited\marginpar{p. L10} these forces to take the form of brazen contempt in the face of eloquent signs of an intact creativity; the Chief Funeral Officer, in sum, of this Burial muffled by derision and contempt. Strangely, this catalyst was also the person to whom I was the closest - the only one who eventually assimilated and made his own a certain vision, full of life and of intense power. But I am getting ahead of myself...

Truth be told, these ``whiffs of subtle derision" which reached me here and there did not affect me that much. They remained in a way anonymous, up until three or four years ago. I saw in them a sign of somewhat bleak times, but they did not appear to be directly targeting me, eliciting neither anxiety nor concern. What did affect me more directly were signs of a distancing away from my person which I received from several of my old friends in the mathematical world, friends to whom I continued (my departure from a mutual world notwithstanding) to feel connected through sympathy, in addition to the links created by a shared passion and a common past. Yet, here again, even though such signs pained me, I never stopped to look further into them, and the thought never occurred to me (as far as I remember) to connect the dots between these three series of signs: the abandoned construction sites (and the forgotten vision), the ``wind of derision", and the distancing of many old friends from my person. I wrote to each of these friends, and none responded. It is actually no longer a rare occurence, nowadays, for letters that I write to old friends of students about things which I hold close to my heart to remain unanswered. New times, new ways - what was there to be done? I simply stopped writing them. Yet (if you are one of them) this letter will be the exception, a word which is once again directed your way, and it will be up to you to decide whether to welcome it this time around, or to shut it off once more...

If I remember correctly, the first signs that certain old friends were distancing themselves away from me trace back to 1976. This was the year during which another ``series" of signs started to appear, which I would like to presently mention, before going back to R\'ecoltes et Semailles. To be more precise, these two series of signs appeared jointly. At the time of writing, it seems to me that they are in fact inextricably linked, that they are in essence two aspects or ``faces" of a single reality which came into being that year in the field of my own life. The aspect which I am about to address concerns a systematic ``stonewalling"\marginpar{p. L11}, muted and with no reply, directed under a ``flawless consensus"\footnote{This ``flawless consensus" is mentioned sporadically in Fatuity and Renewal, and it eventually becomes the object of a circumstantial testimony and a reflection in the following part, The Burial (1), with the ``Procession X" or ``The Funeral Service", consisting of the ``coffin-notes" (n$^o$93-96) and the note ``The Gravedigger - or the entire Congregation". 
%todo:ref
 The latter concludes this part of R\'ecoltes et Semailles, and at the same time constitutes the first culmination of the ``second breath" of the reflection.} towards some students and ex-students \textbf{post}-1970 who, through their work, their style, and their inspiration clearly bore the mark of my influence. It was perhaps on this occasion that I first perceived the ``whiff of subtle derision" which, through them, targeted a certain style and \textbf{approach} to mathematics - a style and a vision which (according to a consensus which had apparently already become universal in the mathematical establishment) \textbf{had no place in mathematics}.

This was again something which was clearly perceived at an unconscious level. During the same year, it ended up becoming visible to my conscious attention, in the wake of an aberrant scenario (regarding the impossibility of publishing a thesis which was visibly brilliant) that had happened five times over, with the burlesque obstination of a circus gag. Thinking back to these events, I realize that a certain reality was ``giving me a sign" with benevolent insistence, while I continued to play deaf: ``Hey, look this way goofy, pay some attention to what's happening right here under your nose, it concerns you I promise...!!". I slightly arose from my torpor, look over (for an instant), half-bewildered half-distracted: ``oh yes, well, it's a little strange, it sure looks like somebody's after somebody else, something must have gone wrong, and with such a perfect set I must say that's quite hard to believe!".

It was so hard to believe that I scurried to forget about both the gag and the circus. I must say that I had several other interesting occupations at hand. This didn't prevent the circus from returning to my attention in the following years - no longer in the form of gags this time around, but rather showing a certain penchant for humiliation, or for full-on punches in the face; with the caveat that we are amongst distinguished people and that the punch in the face therefore also comes in most distinguished forms, necessarily, albeit just as effective - the choice of a form was left to the discretion of the distinguished people in question...

\marginpar{p. L12}The episode which felt like a ``full-on punch in the face" (of someone else) took place in October 1981\footnote{This episode is recounted in the note ``Coffin 3 - or the slightly-too-relative jacobians" (n$^o$95), notably on pages 404-406
%todo: ref
.}. This time around, and for the first time since frequent signs of a new spirit had started reaching me, I was affected - certainly more so than I would have been had I been the one who was punched, rather than someone else whom I held in affection. This someone else was a student of mine, as well as a remarkably gifted mathematician, who had just accomplished beautiful things - but that was just a detail. What wasn't a detail, on the other hand, was that three of my students from ``before' showed direct solidarity in favor of an act which the person in question received (rightly) as a humiliation and an affront. Another two of my old students had already treated him with condescension, taking the air of well-off people chasing away a good-for-nothing\footnote{This occurrence is mentioned in passing in the note mentioned in the previous footnote.}. Yet another student was to follow suit three years later (again in the style ``punch in the face") - but of course that was something I did not yet know. What occupied me at the time was amply sufficient. It was as if my past as a mathematician, never once examined, was suddenly taunting me in the form of hideous rictus on behalf of five of my ex-students, who had gone on to become important, powerful and disdainful characters...

There had never been a better time to probe in the direction of what was suddenly calling my attention with such violence. But something inside me had decided (without ever saying it out loud...) that this past from ``before" did not really concern me after all, that there was no reason to look further into it; that if I was under the impression that it was calling me with a voice that I knew well - that of the time of contempt - there must have been a mistake. Yet, I was overcome with anxiety for days on end, maybe even weeks, without deciding to act upon it. (It was only last year, in the course of writing R\'ecoltes et Semailles and thereby returning to this episode, that I became aware of this anxiety which had been put under control as soon as it had appeared.) Instead of noticing it and probing it, I became agitated and I wrote left and right the ``letters that were in order". The parties involved even took the time to respond, naturally sending me evasive responses which did not go past the surface of anything. The waves eventually grew calmer, and everything returned to order. I never had to think back on any of it, until last year. When I did, I nonetheless noticed that there had remained a wound of sorts, or rather\marginpar{p. L13} a painful splinter which one avoids touching; a splinter which \textbf{maintains} a wound which is trying to heal itself... 

This was surely the most painful and difficult of my experiences in my life as a mathematician - when I was shown (without consenting to really \textbf{become aware} of what my eyes were seeing) ``such a cherished student or companion of yesteryear taking pleasure in discreetly crushing another cherished individual in whom he recognizes me". This impacted me even more strongly than the rather crazy discoveries which I made last year, and which (from an outsider's perspective) might seem just as incredible... It is true that this experience had brought into play several others, in the same tonalities although slightly less violent, and which as a result had been ``over-seeded".  

I am reminded that this very year 1981 had marked a drastic shift in my relationship with the only one of my ex-students which whom I had maintained regular contact following my departure, as well as the one who for the past fifteen years had taken the role of a ``preferred interlocutor" on the mathematical side of things. This was indeed the year during which the ``signs of an affectation of disdain" which had already appeared in previous years\footnote{This episode is treated in the note ``Two turning points" (n$^o$66).}
%todo: ref 
``suddenly became so brutal" that I stopped all mathematical communication with him. That was a few months before the aforementioned punch-in-the-face episode. In hindsight the coincidence seems striking, but I had never tried to bring the two events together. They were filed in separate ``cabinets"; cabinets which someone had declared, in addition, to be of no consequence - the matter was settled!

Thus also reminds me that a certain \textbf{Symposium} took place during the month of June of the same year 1981, which was memorable on a number of counts - a symposium which would have deserved to enter into History (or what remains of it...) under the indelible name of ``Perverse Symposium". I became aware of it (or rather, it dropped on me!) on May 2$^{nd}$ of last year, two weeks after the discovery (on April 19$^{th}$) of the Burial in the flesh - and I suddenly understood that I had come across \textbf{``the Apotheosis"}. The apotheosis of a burial, but also the \textbf{apotheosis} of the \textbf{scorn} targeted at what, for the more than two thousand years during which our science has existed, was the tacit and unshakeable foundation of the mathematician's code of ethic - namely, the elementary rule that one must not present as his own ideas and results which were taken from somebody else's work. In taking note presently of\marginpar{p. L14} of the remarkable temporal coincidence between two events which at first sight can seem to be of very different nature and scope, I am taken aback by the revelation of the profound and evident link there is between the \textbf{respect} of the \textbf{person}, and the respect of the elementary ethical rules of an art or a science, preventing its practice from turning into a ``free-for-all", and preventing its practitioners, known for their excellence and responsible for setting the tone, from turning into a ``maffia" without scruples. But once again I am getting ahead of myself...

\section{The journey}

I believe I have by now covered most of the context in which my ``return to mathematics", as well as, one thing leading to another, my writing of R\'ecoltes et Semailles took place. It was only in late March of last year, in the last section of `Fatuity and Renewal (``The weight of a past" (n$^o$ 50)), that I started pondering the reasons behind and the meaning of this unexpected return.
%todo: ref
Regarding the ``reasons", the most influential of those was surely the impression, at once vague and imperious , that powerful and vigorous things which I thought I had left in caring hands were actually ``in a tomb, in which they had been left in decay, away from benefits of the wind, the rain, and the sun, for the fifteen years during which they had remained out of sight".\footnote{Quote copied from the note ``The melody by the tomb - or sufficiency" (n$^o$167), page 826.}
%todo: ensure consistency of quote   
I must have understood, over time and without daring to admit it to myself before today, that I was the one that would have to finally break apart the old timbers imprisoning living things that were not made to rot away in locked coffins, but to flourish in the open air. Moreover, the false airs of self-importance and the insidious derision surrounding these upholstered and unwieldy coffins (so as to resemble the bemoaned deceased, surely...) played a role in ``eventually awakening a fighting spirit within me which had fallen into slumber over the past ten years", and in giving me the desire to throw myself into the brawl...\footnote{See ``The weight of a past" (section n$^o$50), notably p.137. (**).}
%todo: ref

These are the circumstances in which, some two years ago, what I thought at first would be a brief oversight of one of these ``construction sites" which I had left behind, a matter of a few days or a couple weeks at most, turned into a great mathematical feuilleton in $N$ volumes, inserting itself into the famous new series of ``Reflections" (``mathematical" reflections, awaiting the removal of this unnecessary qualifier). From the very moment I realized that I was in the process of writing a mathematical work destined for publication, I knew that I would also be including, in addition to a more or less standard ``mathematical" introduction, a second ``introduction"\marginpar{p. L15} of a more personal nature. I sensed that it was important for me to explain the motivations behind my ``return", which wasn't a return to a \textbf{milieu}, but only a ``return" to an intense mathematical activity and to the publication of my own mathematical works, to last for an indeterminate period. I also intended to describe the spirit with which I now approached mathematical writing, one which is in many ways different from the spirit of my earlier writing - the present spirit is closer to that of the ``travel log" accompanying a journey of discovery. And then there were naturally other things weighing on my mind, linked to the above, yet which I felt an even more pressing need to communicate. I went without saying that I was to take my time in expressing what I had to share. I viewed these things, although still diffuse, as crucial in making sense of the volumes which I was about to write, as well as of the ``Reflections" in which they would fit. I wasn't about to surreptitiously slip them in, apologizing to the busy reader in passing for wasting their precious time. If there were things in ``Pursuing Stacks" which they and others had to be aware of, they were precisely the ones which I set out to express in this introduction. If twenty or thirty pages didn't suffice in saying these things, I would write forty, or even fifty - and I didn't intend to force anyone to read me...

Thus was born R\'ecoltes et Semailles. I wrote the first pages of the introduction, meant to be completed by June 1983, during a slow period in the writing of the first volume of Pursuing Stacks. I resumed writing in February of last year, at a time when the volume had been essentially completed for several months\footnote{In the meantime, I spent a fair amount of time thinking about the ``structural surface" for a system of pseudo-lines, obtained in terms of the set of all possible ``relative positions" of a pseudo-line in relation to such a system. I also wrote ``Sketch of a Program" (``Esquisse d'un Programme" in French, which will be included in volume 3 of the Reflections.}. I intended to use this introduction as an opportunity to clarify a few things that remained somewhat blurred in my mind. But I also had no doubt that it was about to be, just like the volume which I had just written, a \textbf{journey} or \textbf{discovery}; a journey into an even richer and vaster world than the one that I was getting ready to oversee, in the volume just written and in those to follow. Over the course of days, weeks, months, without really understanding what was happening, I continued this new journey at the discovery of a certain past (which had obstinately eluded me for the past three decades...), as well as of myself and the threads linking me to that past; it was also\marginpar{p. L16}a journey at the discovery of some people to whom I had been close in the mathematical world, and whom I knew so little; and lastly, in the same stride, a journey of mathematical discovery wherein, for the first time in fifteen to twenty years\footnote{In the 1950s and 1960s, I often repressed a desire to pursue such burning and fruitful questions, as I was entirely at the mercy of countless foundational tasks which no one else would have or could have taken up in my stead, and which no one had the stamina to pursue further following my departure... }, I took the time to return to some burning questions which I had left at the time of my departure. As such, I would say that I am actually pursuing \textbf{three} intimately interlinked journeys of discoveries in the pages of R\'ecoltes et Semailles. And none of the three has reached completion by page twelve hundred and counting. The echoes with which this testimony will be met (including echoes of silence...) will be part of the ``continuation" of the journey. As to a conclusion - this is the kind of journey that never really reaches its end; not even, perhaps, on the day of our death...

And so I have come full circle, back to where I started: telling you in advance, inasmuch as it can be done, ``what R\'ecoltes et Semailles is about". But it is also true that, whether or not you had this question in mind, the previous few pages gave some sort of an answer. Perhaps it would be more interesting for me to continue in my stride and to begin \textbf{telling}, rather than ``announcing".
\begin{flushright} June 1985 \end{flushright} 

\section{The obscure side - or creation and disdain}

The previous pages were written during a ``low activity point" last month. Since then, I have finally added the finishing touch to the ``Four Operations" (the fourth part of R\'ecoltes et Semailles) - all that remains is for me to complete this letter or ``pre-letter" (which itself seems to be assuming prohibitive proportions...) and it will then be ready for typesetting and printing. I was starting to lose faith, after the almost year and a half during which I have been ``on the verge of finishing" these famous notes! When I first sat down to write this unusual ``introduction" to a work of mathematics, in February of last year (and already in June of the year before that), there were (I reckon) three things which I intended to speak about. 

First, I wanted to describe the intentions behind my decision to resume a mathematical activity, to write about the spirit in which I had written the first volume ``Pursuing Stacks" (which I had then just completed), and about the spirit\marginpar{p. L17} in which I intended to pursue an even vaster journey of exploration and mathematical discovery in the ``Reflections". From there, I would then only have to present the meticulous and well-pressed foundations of some new mathematical universe in gestation. It would read rather like a ``travel log", in which the work is carried day after day, in plain sight and as it is \textbf{actually} happening, including all the mistakes and mess-ups, the frequent look-backs as well as the sudden leaps forward - a work irresistibly being pulled forward day after day (in spite of the countless incidents and unforeseen circumstances), as if by an invisible thread - by some elusive yet nonetheless tenacious and sound vision. A work that is often fumbling, especially during the ``delicate times" in which a spring of intuition arises, barely perceptible, nameless and faceless as of yet; or during the early steps of some new journey, while still on the lookout for and at the pursuit of initial ideas and intuitions - the latter of which often prove to be elusive and escaping the meshes of language, indicating that what is actually missing is the formulation of an adequate language in which they could be captured with finesse. The task then becomes that of creating such a language, to condense it out of the intangible mist with which we are initially confronted. Thus, what was at first only intuited, before being glimpsed without being really seen or ``touched", slowly trickles out of the imponderable, extracting itself from its shadowy and hazy mantel to assume an existence in flesh and bone...

It is this part of the work which, albeit puny looking - not to say (often) harebrained - is often the most delicate and essential part of the process: it is truly there that something new becomes manifest, through intense attention, solicitude, and respect towards the fragile and infinitely delicate thing about to be born. It is the most creative part of all - that of conception and slow gestation within the warm shadows of the maternal womb, inside which the initial double gamete becomes amorphous embryo and continuously transforms itself over the course of days and months, by means of an obscure and intense process, seamless and invisible, into a new being made out of flesh and bone.

This is also the ``obscure", ``yin", or ``\textbf{feminine}" part of the work of discovery. The complementary ``clear", ``yang", or ``\textbf{masculine}" part would be closer to a work of hammer and chisel upon a piece of hardened steel (using ready-made tools whose efficacy has already been established...). Both aspects have their  individual raison d'\^etre and function, and they are inextricably tied\marginpar{p. L18} to one another by symbiosis - or rather, they are the \textbf{wife} and \textbf{husband} forming the indissoluble couple of the two original cosmic forces, whose ever-renewing embrace gives rise to the obscure creative work of conception, gestation, and birth - the birth of the \textbf{child}, of the novel thing. 

The second topic which I felt the need to write about, in this famous personal and
``philosophical" introduction to a work of mathematics, was the nature of creative work.
It had been years since I first realized that the nature underlying this process was
largely ignored, overshadowed by a host of clich\'es, repressions and ancestral fears. I only started discovering the extent to which this was a reality over the course of the days and months which I devoted to the reflection and the ``investigation" undertaken in R\'ecoltes et Semailles. Already from the ``get go" of this reflection - while writing some pages back in June 1983 - I was taken aback by a fact which, while inconspicuous at first sight, is nonetheless stupefying once one stops to think about it: namely, that this ``most creative part of all" within a work of discovery which I mentioned above \textbf{is reflected almost nowhere} in the texts and monologues which are supposed to present work of this kind (or at least present its most tangible outgrowths); such is the case in textbooks and other texts of a didactic nature, in articles and original memoirs, as well as in oral lectures, seminary presentations, etc. It is as if there existed, for what seems like millennia, tracing back to the very origins of mathematics and of other arts and sciences, a sort of ``conspiracy of silence" surrounding these \textbf{``unspeakable labors"} which precede the birth of each new idea, both big and small, and which thereby lead to a renewal of our understanding of a portion of this world in which we live, a world engaged in perpetual creation. 

To tell the truth, it seems that this most crucial aspect, or stage, of the work of discovery (as well as creative work in general) is subjected to a repression so efficient, so interiorized by the very people who come to know firsthand such a process that we could almost swear that these same people have eradicated every memory of this practice from their consciousness - akin to how a woman living in a highly restrictive puritan society might have eradicated from her living memory, in relation to each of the children who fall under her daily care, the moment of the embrace (grudgingly endured) tied to their conception, the long months of pregnancy (suffered as an impropriety), and the long hours of childbirth (experienced as a distasteful ordeal, followed by deliverance at long last).

\marginpar{p. L19}This comparison may seem outrageous, and it may indeed be so, if I were to apply it to the spirit of the mathematical milieu to which I belonged myself, as I remember it from some twenty years ago. But in the course of my reflection in R\'ecoltes et Semailles I was led to realize, especially during the past few months (while writing ``The Four Operations"), that there had been since my departure from the mathematical world a stupefying \textbf{degradation} in the spirit which nowadays rules over the milieux which I once knew, and (to what seems to be a large extent) in the mathematical world at large\footnote{This degradation is not in any way limited to the ``mathematical world". It can be detected in the entirety of the scientific sphere and, beyond it, in the contemporary world at a global scale. I begin an assessment and reflection upon this situation in the note ``The muscle and the bowel", which opens the reflection on the yin and the yang (note n$^o$106).
%todo: ref 
}. It is even possible, in view of my very particular mathematical personality and of the circumstances surrounding my departure, that the latter may have played a catalyzing role in an evolution which was already underway\footnote{I further examine this evolution in the note cited in the previous footnote. Links between this evolution and the Burial (of my person and of my work) appear and are examined in the notes ``The Funerals of the Yin (yang buries yin (4))", ``Providential circumstance - or Apotheosis", ``The disavowal (1) - or the reminder", ``The disavowal - or the metamorphosis" (n$^o$ 124, 151, 152, 153). See also the more recent notes (in ReS IV) ``The unnecessary details" (n$^o$171 (v), part (c) ``Things that don't resemble anything - or desiccation") and ``The family book" (n$^o$`73, part (c) ``The one among all - or the acquiescence").
%todo: ref
} - an evolution to which I was then blind (as much so as any other friend or colleague of mine, with the possible exception of Claude Chevalley). The aspect of this degradation which I am mostly thinking about in writing these lines (and which is just \textbf{one} aspect among many others\footnote{The aspect which is most often the focus of R\'ecoltes et Semailles, particularly in the two ``investigative" parts (ReS II or ``The robe of the Emperor of China", and ReS IV or ``The Four Operations"), is also, perhaps, the aspect that I found most ``flabbergasting", namely the degradation of the ethical code of the profession, expressed in the form of sacking, discrediting, and shameless scheming operated by some of the most prestigious and brilliant mathematicians of the time, and (to a large extent) in full view of all. For some of the other, more delicate aspects, which are in fact directly tied to this one, I refer the reader to the aforementioned note (n$^o$173 part (c)) ``Things that don't resemble anything - or desiccation".}) is the \textbf{tacit disdain}, if not outright derision, directed towards work (mathematical, in this case) which does not resemble pure hammer-and-chisel labor - disdain for the more delicate (and\marginpar{p. L20} often lesser looking) creative processes; for all that pertains to \textbf{inspiration, dream, vision} (however powerful and fruitful it might be); almost (in the extreme) for any idea, however clearly it may have been conceived and formulated: in sum, disdain for anything that hasn't been written and \textbf{published} in black on white, in the form of plain statements, classifiable and classified, ready to be incorporated into the ``databases" inscribed in the inexhaustible memories of our supercomputers. 

There has been (to borrow an expression from C.L. Siegel\footnote{This expression is cited and commented upon in the note that was just cited in the preceding footnote}) an extraordinary \textbf{``flattening"} and a \textbf{``narrowing"} of mathematical thought, as a result of the fact that an essential dimension has been stripped away: the totality of its ``obscure side", its ``feminine" side. It is true that, in accordance with an ancestral tradition, this side of the work of discovery was to remain largely hidden, and nobody (virtually) ever \textbf{spoke} about it - but, until now (to the best of my knowledge), the ability to establish a raw contact with the profound sources of dream, and to in turn nourish great ideas and great designs, had never yet been lost. It seems as if we have recently entered an \textbf{era of desiccation}, in which access to these dream sources, which are admittedly not yet dried-up, has been condemned by the unappealable verdict of general disdain and by the reprisals of derision. 

We therefore seem to be approaching the risk of eradication, within each individual, of not only the \textbf{memory} of work carried out close to the source, of a ``feminine" nature (often derided as ``muddy", ``sluggish", ``inconsistent" - or at the other extreme as ``trivial", ``child's play", ``long-winded"...), but also the loss of this very work and its outgrowths - when this work is where novel notions and visions are conceived, grow, and come to life. Such an event would lead us to an era during which the practice of our art will be reduced to arid and vain demonstrations of cerebral ``weightlifting", and to intellectual bidding wars towards the ``breaking open" of competition problems (``of proverbial difficulty") - an era of sterile, feverish, ``super-macho" hypertrophy following more than three centuries of continuous creative renewal. 

\section{Respect and fortitude}

But I digress once again, looking ahead regarding what my reflection has taught me.  My initial goal, prior even to the start of this reflection, was twofold: to present a ``declaration of intent", and (something which we just saw is closely related to the latter point) to share my thoughts on the nature of creative work. However, there was also a third goal which, although less perceptible at the conscious level, fulfilled a more\marginpar{p. L21} profound and essential need of mine. This third point was sparked by sometimes troubling ``interpellations" issued by many of my old students or colleagues, and regarding my past as a mathematician. On the surface, this need manifested itself as a desire to ``empty my bag", to say out lout some ``uncomfortable truths". Yet, at a deeper level, there was surely also the need to \textbf{become acquainted} at last with a past which I had hitherto chosen to ignore. It is as a result of this need more than anything else that R\'ecoltes et Semailles came into being. This long reflection was my ``response", day in and day out, to the inner impulse I felt towards understanding, and to the recurring interpellations that reached me from the outside world, the ``mathematical world" which I had left with no intention of return. With the exception of the first pages of ``Fatuity and Renewal", namely the first two chapters (``Work and discovery" and ``The dream and the dreamer"), and beginning with the following chapter ``The birth of fear" (p. 18) which features an unplanned ``testimony", I believe that this need to confront my past and to fully accept it was the principal force at work in the writing of R\'ecoltes et Semailles.
%todo: ref

The interpellation that had reached me from the mathematical milieu, and which returned with renewed strength throughout R\'ecoltes et Semailles (especially during the ``investigation" undertaken in parts II and IV), had taken from the get-go a self-important character, if not one of (``delicately calibrated") disdain, derision, or scorn - directed both towards myself (sometimes), and (most often) towards those who had dared take inspiration from me (without being aware of the consequences awaiting them), and who were thereby ``categorized" as being linked to me, by some tacit and implacable decree. Once again I am detecting herein the ``obvious and profound" connection between \textbf{respect} (or the absence thereof) for others; respect for the creative act and for some of its most delicate and essential fruits; and, finally, respect for the most evident rules of scientific ethic, namely those that are rooted in an elementary respect of oneself and of others, and which I am tempted to call ``\textbf{rules of decency}" in the practice of our art. All of these are but aspects of an elementary and essential ``\textbf{respect of oneself}". If I had to summarize, in a single lapidary sentence, what R\'ecoltes et Semailles has taught me about the certain world which I once called my own, and with which I identified myself for more than twenty years, I would say:\marginpar{p. L22} it is a world that has \textbf{lost the value of respect}\footnote{Once again, this formulation preserves its relevance outside of the limited milieu in which I had ample opportunities to come to this conclusion, where it seems to also summarize a certain degradation of the entirety of contemporary society. (Compare with the footnote on page L19.)
%todo: ref
Within the more self-contained context of summarizing the ``investigation" pursued in R\'ecoltes et Semailles, this formulation appears in note 2 from last April, titled ``Respect" (n$^o$179).}.

The above was something I had already strongly felt, if not outright formulated, during the preceding years. It was only further confirmed and clarified, in unexpected and sometimes stupefying ways, over the course of writing R\'ecoltes et Semailles. It is notably already apparent starting from the point at which the general reflection of a ``philosophical" nature suddenly turns into a personal testimony (in the section ``The welcome stranger" (n$^o9$, p.18) at the start of the aforementioned chapter ``The birth of fear").
%todo: ref

Yet, this observation is not to be interpreted as an acerbic or bitter recrimination, but rather (through the logic internal to the writing and through the attitude which it induces in the reader) as an \textbf{interrogation}. That is, I invite mathematicians to ask themselves: what was the nature of my own involvement in this degradation, this loss of respect which I am nowadays witnessing? Herein lies the principal interrogation which pervades and carries forth the first part of R\'ecoltes et Semailles, up until the point where it is finally resolved by a clear and unequivocal observation\footnote{See the sections ``Athletic mathematics" and ``Enough with this merry-go-round (n$^o$40, 41).}. Previously, this degradation had seemed to me as having suddenly ``materialized itself", as something that was equally unexplainable, outrageous, and unacceptable. During the reflection, I discovered that it had actually been taking its course insidiously, without anybody taking notice of its evolution either around them or within them, throughout the 1950s and 1960s, \textbf{including in my own case}. 

This humbling realization, although evident and without show, marks the first crucial turning point in the testimony, and immediately brings with it a qualitative change\footnote{Beginning the following day, the testimony deepens into a meditation on myself, a particular quality which it preserves in the following weeks, all the way to the end of this ``first breath" of R\'ecoltes et Semailles (ending with the section ``The weight of a past", n$^o$50).}. This was the first essential realization which I was to acquire about my mathematical past and about myself. This newly gained awareness of my \textbf{shared responsibility} in the general degradation (an awareness which makes itself felt more or less sharply at various points of the reflection) remains a kind of background note and a reminder throughout R\'ecoltes et Semailles. Such is the case, mostly, at the times when the reflection\marginpar{p. L23} takes the form of an investigation on the disgraces and inequities of an era. Together with the desire to understand, this curiosity which carries forward any authentic work of discovery, I believe that it is this humbling awareness (forgotten at various points along the way but always reappearing, in the places where it is least expected...) which prevented my testimony from ever turning into a compendium of sterile recriminations on the world's shortcomings, or even into a ``settlement of scores" with some of my old students or friends (or both). This absence of complacency towards myself also provided me with a sense of inner calm, or fortitude, which in turn safeguarded me from the risk of complacency towards others, including that of ``false discretion". I said all that I felt the need to say at every point of the reflection, be it about myself, about one or another of my colleagues, students, or friends, or about a milieu or an era, without ever having to jostle against my reluctance: every time such inner opposition surfaced, I needed only examine it carefully for it to disappear without a trace.

\section{``My close acquaintances" - or connivance}

The purpose of this letter isn't to list all of the ``key moments" (or ``sensitive moments") in segments of or all of  R\'ecoltes et Semailles\footnote{A short retrospective-recap of the first three parts of R\'ecoltes et Semailles can be found in the two groups of notes ``The evening fruits" (n$^o$179-182) and ``Discovering a past" (n$^o$183-186).}.
%todo: ref
I would only like to mention that there were four great steps, or ``breaths", that can be clearly distinguished in this work - akin to respiratory motions, or to the successive waves of a rising tide, vast and mute masses, at once immobile and in motion, limitless and nameless, issued from the unknown and bottomless sea that is ``me", or rather, from a sea infinitely vaster and deeper than ``I" am, and which carries and nourishes me. These ``breaths" or ``waves" materialized into the four presently written parts of R\'ecoltes et Semailles. Each wave came of its own accord and unexpectedly, and at no point could I tell where it was taking me nor when it would end. And whenever one wave ended and a new one took its place, there was a period of time during which I still believed myself to be at the end of a cycle (leading, at the very end, to the end of R\'ecoltes et Semailles!), whereas I was already being lifted and carried forward by another breath of the same vast movement. It is only in hindsight that\marginpar{p. L24} this movement becomes clearly apparent and that a \textbf{structure} unequivocally reveals itself within what has been lived as an act in motion.

Naturally, this movement did not come to an end upon the (provisory!) completion of R\'ecoltes et Semailles, and it not end with this letter either, the latter being but a ``measure" of this movement. In the same way, it wasn't born on a given day in June 1983, or February 1984, when I sat down in front of my typewriter to begin (or resume) writing the introduction to a certain mathematical work. Instead, it was born (or rather re-born) on the day that meditation appeared in my life...

But I digress once more, letting myself be carrier (and taken away...) by images and associations born in the moment, rather than diligently sticking to the thread of a planned ``message". Today's intention was to continue the narration, however briefly, of the ``discovery of the Funeral"
%todo:ref
 written last April, at a time when I thought I was already done with R\'ecoltes et Semailles two weeks earlier - to review the way in which a series of major and incredible discoveries cascaded upon me in the span of just three or four weeks, discoveries so massive and wild that it took me months to even begin ``believing the testimony of my sound perceptive capacities", and to free myself from an insidious \textbf{incredulity} in the face of evidence\footnote{I attempt to express this difficulty in the tale ``The robe of the Emperor of China" in the note of the same name (n$^o$77), and again in the note ``Duty fulfilled - or the moment of truth" (n$^o$163).
%todo: ref
}. This secret and tenacious incredulity was only dissipated last October (six months after the discovery of the ``Funeral in all its glory"),
%todo: ref
following the visit at my home of my friend and ex-student (albeit secretly) Pierre Deligne\footnote{I narrate this visit in the note cited in the previous footnote.}. For the first time, I was confronted to the Funeral not through the intermediary of \textbf{texts} going over (in nonetheless eloquent prose!) the discrediting, sacking, and massacre of a life's work, as well as the burial of a certain style and approach to doing mathematics - only this time the confrontation was direct and tangible, assuming familiar traits and a known voice, whose intonations were affable and ingenuous. The Funeral was in front of me at last, ``in flesh and bone", with the occupied and anodyne traits which I recognized, but which I saw for the first time with new eyes and a renewed attention. Here was before\marginpar{p. L25} me the person who, over the course of my reflection in the preceding months, had revealed himself to be the Chief Officer at my solemn funeral rites, at once the ``Priest in a chasuble" and the principal architect and ``beneficiary" or an unprecedented ``operation", secret inheritor of a work abandoned to derision and sacking...

This encounter takes place at the beginning of the ``third wave" of R\'ecoltes et Semailles, just as I had embarked upon a long meditation on the yin and the yang, at the pursuit of an elusive and tenacious association of ideas. At the time, this brief episode was only mentioned in passing, in the form of an echo of a few lines. It nonetheless marks an important moment, whose fruits will only make themselves known months later.

There was a second similar confrontation to the ``Funeral in flesh and bone", which happened just ten days ago and came to relaunch, ``at the last minute" once more, an investigation that kept finding new impetuses. This time, it was just a phone call with Jean-Pierre Serre\footnote{This is essentially a paraphrase of the note ``The Gravedigger - or the entire Congregation" (n$^o$97, page 417).}
%todo:ref 
. This ``jumbled" conversation served to confirm in a striking and unexpected way what I had just (a few days earlier) admitted to myself\footnote{In part (c) (``The one among all - or the compliance") of the same note (n$^o$173).}, almost against my will, concerning the role played by Serre in my Funeral and his ``secret compliance" to what was happening ``right under his nose" while he pretended not to see or feel anything. 

Here too, of course, the conversation itself was perfectly ``cool" and amicable, and it indeed seems that Serre's friendly disposition towards me is entirely sincere and genuine. The fact remains that I could clearly discern, or should I say ``touch", this compliance of his that I had just admitted to myself; undoubtedly ``secret" (as I wrote above), but most of all hastened, as I was able to observe in person beyond any doubt. A hastened and unreserved compliance, to bury what has to be buried, and to replace, wherever deemed fit and by any means, a real and undesirable paternity (which Serre knows about firsthand) by a bogus and welcome one...\footnote{This is essentially a paraphrase of the note ``The Gravedigger - or the entire Congregation" (n$^o$97, page 417).}\marginpar{p. L26} I received a striking confirmation of an intuition which had already appeared a year earlier, when I wrote\footnote{This quote is taken from the same note (see previous footnote), also page 417.}: \\

\parindent=0.6cm
``Seen in this light\footnote{``In light'' of this deliberate comment, which was just
mentioned, regarding the suppression at all costs of ``undesirable affiliations'' (or even
``intolerable'', to use the language employed in the aforementioned footnote).}, the
principal officer Deligne no longer appears as the one who created a trend reflecting the
underlying forces determining his own life and actions, but rather as the
\textbf{instrument} designated (because of his role as ``legitimate heir"\footnote{This
role of ``heir'' in Deligne was at once occult and (when not a single line published by
Deligne could yet reveal that he had learned anything from me) at the same time
clearly felt and recognized by all. This was one of the typical aspects of Deligne's double
entendre and his particular ``style''. Namely that he was able to expertly play on this
ambiguity, while cashing in on the benefits of his role as heir. All the while disavowing
the deceased master and taking charge of his large scale funeral.}) by a highly cohesive \textbf{collective will}, and given the impossible task of erasing both my name and my personal style from contemporary mathematics." \\

If Deligne thus appeared as the ``instrument" designated (as well as being both the first
and principal ``beneficiary") by a ``highly cohesive collective will", Serre now seems to
be the very \textbf{incarnation} of this collective will, and as the \textbf{guarantor} of
the resulting unreserved compliance; an acquiescence to all kinds of trickeries and frauds
including the ``vast" operations of shameless collective mystification and appropriation,
as long as these practices contributed to the ``impossible task" targeted towards my
modest and departed self, or towards any other person\footnote{I am here thinking about
\textbf{Zoghman Mebkhout}, who is first mentioned in the Introduction, 6 (``The
Funeral''), and again later in the note ``My orphans'' (no. 46), as well as in the notes
(written at a later time, following the discovery of the ``Funeral'') ``A failed
instruction (2) - or creation and fatuity'' and ``a feeling of injustice and powerlessness''
(n$^o$ $44'$ and ${44'}'$). 
I unravel the unfair operation and appropriation of Mebkhout's pioneering work.
Over the course of the $11$ notes constituting the Funeral
Procession VII, the ``Colloquium - or Mebkhout's sheaves and Perversity''
(n$^o$ 75-80)
An investigation, as well as a more detailed narrative regarding this 
(fourth and last) ``operation'' form the most substantial part of the investigation
``The four operations'', under the fitting name ``\textbf{the Apotheosis}'' (notes n$^o$
171 (I) and 171.).} who dared join forces with me and to appear, against all odds, as a ``continuator of Grothendieck".

\marginpar{p. L27}One thing that I find most paradoxical and disconcerting among many others at the Funeral,
is that the latter would have been the product of those who had been my friends and
students more than anyone else, not to say exclusively so, in a world where I never
thought I had any enemies.
It is, I believe, in this sense that 
R\'ecoltes et Semailles
concerns you more than anyone else, and the letter I am currently writing you is
\textbf{an interpolation}.
For if you are a mathematician, and if you were once my student or friend, 
then you will be no stranger to the Funeral, be it through actions or through complicity,
or even simply by your silence in our relationship regarding something that has been
happening right in front of you. 
And if you were to (extraordinarily) welcome my humble words and the testimony which they
bring to you, rather than to remain hidden behind closed doors, and to send away the
unwelcome messenger, then you would perhaps learn that what has been buried by all, and
with your participation, either actively or through acquiescence, was not only the life
work of somebody else, but 
a living testimony of a loving relationship with mathematics; but also at a deeper and
more sacred level than this burial (which must remain nameless\ldots), it is an essential
part of your own being, of your original capacity for knowledge, love, and creation, that
you have been burying with your own hands in the guise of somebody else. 

Among all of my students, Deligne had occupied a very special place, about which I will write
extensively during the reflection.\footnote{See especially, regarding this topic, the
collection of seventeen notes in ``My friend Pierre'' (n$^0$ 60-71) in RS II.} 
He was by far the ``closest'', as well as the only one (student or not)
to have intimately assimilated and made his\footnote{This ``vast vision'' which Deligne
has indeed ``intimately assimilated and made his'' had induced a powerful fascination in
him, and continues to fascinate him even though an imperious force pressures him to
destroy it at the same time, to break apart its defining unity and to take possession of
the scattered pieces. Thus his occult antagonism with respect to a disavowed 
``teacher'' is nothing about the expression of a 
division within his being which has made a profound mark on his work following my
departure - work which has remained well below the early prodigious nature which I once
witnessed in him.} the vast vision which was born and had flourished within me long before we met. 
And among all of the friends who share with me a common passion for mathematics, it was
Serre who, while at the same time playing the role of an elder figure, had been the
closest to me (by far), and he played 
the unique role of ``detonator'' in my work for a decade in some of my greatest
undertakings. As well as 
\marginpar{p. L28}
most of the key ideas which inspired my mathematical thought during the 50's and 60's, all
the way to the time of my departure.
The particular relationships which I shared with both of these individuals 
is not unrelated to their exceptional abilities. One which guaranteed them an equally
exceptional ascendancy over other mathematicians of their generation, and of the
generations that followed. Apart from these common factors, the temperament
and the manners of Serre and Deligne
seem to me to be as dissimilar as can be, and they appear in many ways to be
antipodal to one another.

In any event, if I had to name mathematicians who were in some way or another ``close'' to
me and to my work (and, furthermore, known as such) it would be Serre and Deligne: one as
an elder, and a source of inspiration in my work during a crucial period of gestation of a
vision; the other as the most gifted of my students, for whom I in turn
became (and have remained, Funeral or not\ldots) his principal (and secret\ldots)
source of inspiration.\footnote{For more on this, see the preceding footnote.}
If a Funeral gathered momentum in the aftermath of my departure (turning into a (proper)
``death''), and later concretized into an endless procession of ``operations'', both large
and small, put at the service of a common end, it could only have been the result of the 
joint and tightly solidary participation of both of them, the ex-elder and the ex-student
(or ex-``disciple''): one taking charge of the 
discreet and efficient direction of the operations, while rallying some of my students to his cause,\footnote{I am speaking
precisely about the five other students who have chosen, as their principal theme of
study
(just like Deligne), that of the cohomology of varieties.}
satisfying his need to torment the \textbf{Father}
(under the grotesque and derisory effigy of an unwieldy and 
bombastic \textbf{super woman}); and the other giving the ``green light'' without reserve,
unconditionally, and without boundaries, to the pursuit of the (four) operations
(discrediting, pillaging, dismembering, and sharing of inexhaustible remains\ldots).

\section{The plunder}

As I alluded to earlier, I had to overcome considerable inner friction, 
or rather let them heal through a patient, meticulous and tenacious process, so as to
separate myself from certain familiar images which were firmly set within me
with considerable inertia, which had taken 
in me for decades (just like they do in everyone else, including you probably)
the form of a direct and nuanced perception of reality - 
in this case that of a certain certain mathematical milieu
to which I continue to be 
\marginpar{p. L29}
linked, through my past and through my work. 
One of the most strongly anchored of these images, or ready made ideas, is that it appears
inconceivable for a scholar of international renown, 
or even a person recognized as a great mathematician to participate
(even if only exceptionally, let alone customarily\ldots); or if he were to abstain 
(out of habit again) to take part himself, that he could nonetheless welcome such
operations (``defying any sentiment of decency at times'') put together by somebody else,
and from which, for some reason or another, he can profit. 

This inertia of the mind was so strong in me, that it was only about two months ago, at
the end of a long reflection that had taken place over the course of a year, that I could
finally glimpse timidly at the fact that Serre might have had something to do with this
Funeral - something which now appears to me as obvious, independently of the eloquent
conversation that I recently had with him. 
As for the other members of the Bourbaki milieu, who welcomed me with kindness in my early
days, there was a tacet taboo surrounding Serre's person.
He represented the very incarnation of a kind of ``elegance'' - an elegance that includes
not only form, but also rigor and a scrupulous probity.

Before discovering the Funeral on April 19 of last year, I would never have thought, not
even in my dreams, that someone who was once my student 
would be capable of dishonesty in the exercise of his profession, 
be it towards me or anyone else.
And it would have been towards the most brilliant of them all, the one who also was
closest to me, that such a supposition would have seemed the most aberrant!
Yet from the very moment of my departure and throughout the following years, up to this
very day, I have had ample opportunities to realize the extent to which his relationship
with me was strained. More than once, I also saw his use (as if for the sheer pleasure of
it, one might think) of the power to discourage and humiliate at times when the situation
was pertinent. Each time I was profoundly affected by these episodes (more so, probably,
than I would have cared to admit at the time\ldots). These were the rather telling signs
of a profound imbalance, which (as I had ample occasions to witness or to observe)
was not limited to him only, 
even in the small circle of those who had been my students. 
Such an imbalance, through the loss of respect for the other, is no less flagrant or
profound than that imbalance which we come to call ``professional dishonesty''. 
The fact remains that 
\marginpar{p. L30}
the discovery of such dishonesty came to me as a total shock and
surprise. 

In the weeks that followed this spectacular revelation, 
one of a ``cascade'' of others along the same vein, I slowly became aware of the fact that
a certain connivance among some of my old students\footnote{See preceding footnote.}
had really begun in the years preceding my departure. This was most striking in the case
of the most brilliant among them - the one who, after my departure, set the pace and (as I
mentioned earlier) ``took the discreet and efficient direction of the operations''.
With almost twenty years of hindsight, this connivance appears to me as evidence,
``blindingly obvious''. If I had then chosen to ignore what was happening, 
occupied with the pursuit of the ``white whale'' in a world where ``all is orderly and
beautiful'' (as I liked to believe it was), I now realize that I failed to assume the
responsibility that was incumbent upon me, regarding students who were learning about a
profession which I love under my supervision; a profession that consists of more than
simple savoir faire more than the development of a certain ``flair''. 
Through my complacency with regards to brilliant students, whom I took (following tacit
decree) to treating as ``equals'' and beyond any suspicion, I myself contributed\footnote{
This ``contribution'' notably appears in the note ``the free body''
(n$^o$ $67'$), as well as in the two notes ``The ascension'' and ``The ambiguity'' ($n^o$
$63'$, $63''$), and once more (under a slightly different lens) at the end of the note
``The eviction'' (n$^o$ 169). Another kind of contribution appears in 
``fatuity and renewal'' via attitudes of fatuity towards less brilliant young
mathematicians. This coming to terms with a shared responsibility in the general
degradation culminates in the section ``Mathematics as sport'' (n$^o$ 40).}to the
eruption of the (unprecedented, or so it seems to me) corruption 
which I now see spreading in a world and among beings who were once dear to me.

Admittedly, due to their immense inertia, it took a long and intense bout of work to
separate myself from what are usually called ``illusions'' (a word used not without some
amount of regret\ldots), and which I would rather call preconceived ideas; about myself,  
about a milieu with which I once identified, and about people who I have loved and who
perhaps still love - to separate myself from these ideas, or rather to \textbf{let them
decouple from my person}.
This process took work, but it was never a struggle - it brought me, among many other
valuable things, occasional moments of sadness, but not a single moment of regret, nor of
bitterness. 
Bitterness is one of the ways by which we tend to avoid 
\marginpar{p. L31}
knowledge, or the 
content of a lived experience; and in so doing to maintain a certain tenacious illusion
about oneself, at the cost of another ``illusion'' (its negative in a sense)
about the world, and about others.

It is without bitterness and without regret that I now 
see these preconceived ideas which were once ``dear'' to me - 
out of habit and because they
had ``always been there" - 
decouple from my person like cumbersome or even smothering weights. 
They had become second-nature, but this ``second nature'' was not ``me''.
To let go of these ideas one by one was neither harrowing nor frustrating, as it would
have felt had they been of value to me.
This ``stripping away'' of sorts, comes as the reward and fruit of a 
\textbf{labor}. 
It is indicated by an immediate and gratifying feeling of relief, a welcome 
\textbf{liberation}.

\section{Four waves in one movement}



% \marginpar{p. L30}

% \end{document}
