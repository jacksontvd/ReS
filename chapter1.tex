\begin{comment}
\documentclass{book}
\usepackage{master}
\newcommand{\rec}{$\text{R\'ecoltes et Semailles}$}
\newcommand{\no}{n$^\circ$}
\hfuzz = 100pt
\begin{document}
\end{comment}

\chapter{By way of a foreword}
\label{chapter:1}

\marginpar{p. A1}
January 30 1986

In order for R\'ecoltes et Semailles to be ready to be sent to the publisher, all that was left to write was the foreword.
And I swear that I went into it with all the goodwill in the world to write something suitable - something \textbf{reasonable} this time. No more than three or four carefully phrased pages that would introduce this huge volume of over a thousand
pages. Something which  would ``grab'' the attention of the jaded reader, 
perhaps suggesting to him that in these frightening ``over a thousand pages'', he might be able to find
things of interest (or things which concern him, who knows?). It is not
really my style to pander to others, but I was ready to make an exception for once!
The ``publisher crazy enough to give it a shot'' (i.e. to publish this visibly unpublishable
monster) had to make ends meet one way or another. 

But then, it didn't happen. Even though I tried my best. And not only over the course of an afternoon, as I had
originally planned. Tomorrow will mark the three weeks point since I started writing the foreword - since the sheets began
accumulating.
What came out, to be sure, isn't what one could reasonably call a ``foreword''.
I guess it is yet another miss! Blame it on my old age - I have never been much of a salesman. Even when
it comes to pleasing (oneself or one's friends\ldots).

What came out instead is a sort of long ``Walk'', with commentary, through my life's work as a mathematician. 
A Walk intended mostly for the ``layman'' - he who ``never understood anything about
math'', as well as for myself, having never indulged in such a Walk. 
Step by step, I found myself uncovering and writing things that had previously remained
unspoken. As if by chance, these are also the things which I find most essential, both in
my work process and in my body of work. 
They are things which are not technical in nature. It will be up to you, the reader, to decide whether
or not I have succeeded in my naive enterprise to ``get the message across'' - an enterprise
which is surely a bit mad. 
My satisfaction and pleasure will come from making you feel these things. Many of my wise colleagues no longer feel them, 
possibly because they have become too wise
and too prestigious - a situation which often leads to losing touch with the simplest and most essential things in life. 

During this ``Walk through a life's work'', 
I also speak about my life,
as well as, here and there, what 
R\'ecoltes et Semailles as a whole
is about.
I go into the latter in more detail in the ``Letter'' 
\marginpar{p. A2}
(dating from last May) which
follows the ``Walk''. 
This Letter was addressed to my
previous students and to my ``old friends'' in the mathematical community. 
But even the Letter is not technical in nature. 
It can be read by anyone
interested in learning,
in the form of a ``heartfelt'' narrative, about the various reasons that led me to writing 
R\'ecoltes et Semailles. Even more than the Walk, the Letter will provide a 
glimpse of a particular prevailing atmosphere in the ``prestigious mathematical world'',
as well as a preview of my peculiar writing style (the same could be said of the Walk) and of the
spirit that is expressed through this style - a spirit which is not to everybody's taste. 

In the Walk and throughout R\'ecoltes et Semailles, I speak about the act of 
\textbf{doing mathematics}.
It is an activity for which I have first-hand experience and which I know very well. Most of the
things I say about it can surely be said of any kind of creative work, or of any work involving
discovery - it is at the very least true of ``intellectual''
work, meaning work which involves ``using one's brain'' and writing. 
All such work is characterized by the formation and deepening of an
\textbf{understanding} of the things which are being explored. 
To take an example of such work from the 
opposite extreme of the spectrum, romantic passion also qualifies as an act of discovery. 
It opens us to a so-called ``carnal" understanding which also renews itself, evolves, and deepens over time. 
Both of these impulses - 
that which, say, animates a mathematician at work and
that which moves a lover 
- are much closer in nature than we generally assume or would readily admit. 
I hope that the pages of 
R\'ecoltes et Semailles 
will contribute to making you feel this impulse in your work and in your daily life. 

Most of the Walk focuses on mathematical work itself. 
I mostly remain silent on the \textbf{context} in which this work takes place, and
on the \textbf{motivations} at play outside of mathematical work itself. 
This comes at the risk of painting a flattering but deformed image of myself, of mathematicians, or of ``scientists'' in general - think ``grand and noble passion'' without any further qualifications.
In accordance, that is, with the great ``Myth of Science" 
(with a capital S, thank you very mich!).
A heroic myth, ``promethean'' in nature, to which countless writers and thinkers have succumbed 
(and continue to succumb). 
Only historians, if that, manage to sometimes resist this tantalizing myth. 
The truth is, among the things which motivate these ``scientists'',
sometimes to the point of inciting them to devote themselves entirely to their work, ambition and vanity
play just as important and universal a role as they do in any other profession. 
This phenomenon 
\marginpar{p. A3}
is manifested in
more or less coarse or subtle ways depending on the person. I am no exception to this -
the reading of my testimony will hopefully settle this fact beyond doubt. 

It is true also that even the most intense ambition is
powerless when it comes to discovering or proving a novel mathematical statement - just as it cannot (for instance) ``make one hard'' (in the proper sense of the term).
Whether man or woman, what ``makes one hard'' is not ambition, nor is it the desire to shine or to
display power (of a sexual nature in this case) - quite to the contrary!
It is the acute perception of something potent, at once very real and very delicate. 
One could call it ``beauty'', though this is only one of
this thing's thousand faces. To be clear, having ambition doesn't necessarily prevent one from seeing the beauty within a being or a thing. But it is \textbf{not} ambition which makes us
sense it\ldots 

The person who first discovered and mastered fire was somebody just like you and me -
not at all the kind of being we would conventionally refer to as ``hero'', ``demi-god'', and so on.  
Surely, just like you and me, he found himself in the grip of anxiety and became acquainted with the time-worn fixture of vanity which makes one forget about anxiety's bite.
But at the instant at which he ``knew'' fire, there was neither fear nor vanity. 
Such is the truth lying at the core the heroic myth.
The myth turns insipid, it becomes a fixture, when it is used
to conceal another aspect of things which is just as real and essential. 

My aim in R\'ecoltes et Semailles has been to address both of the above aspects of the heroic myth - 
the impulse towards understanding on the one hand, fear and its vain antidotes on the other. 
I believe I ``understand'', or at least \textbf{know} the impulse and its origins. (Or
perhaps I will one day discover the extent to which I was deluded).  
However, I am well aware that I have not yet gotten to the bottom of the great enigma that is fear and vanity, 
and the insidious creativity blocks that result from them - 
who knows if I will even reach the conclusion of this mystery in the years I have
left\ldots

While writing
R\'ecoltes et Semailles, 
two images emerged when I attempted to represent these two aspects of the human journey:
the image of the \textbf{child} (aka the \textbf{laborer}), and that of the
\textbf{Boss}.
In the Walk which we are about to undertake, we will be dealing mostly with the ``child''.
The same child features in the subtitle ``\textbf{The child and the Mother}''. 
The motivation behind this name 
will hopefully become clear over the course of the walk.

\marginpar{p. A4}
In the remainder of the reflection, however, the Boss will take the lead. 
And he shall live up to his name! It would be more accurate 
to speak of multiple Bosses of competing enterprises rather than of a \textbf{single}
Boss.
But it is also true that all Bosses share the same essential features.
And once Bosses have been mentioned, it stands to reason that there will also be ``villains''.
In part I of the reflection (which is called ``Complacency and Renewal'' and will follow the present
introductory section, ``Prelude in four Parts"), I chiefly play the role of ``the villain''. In the following
three parts, it is mostly the ``others" who do. 
Each in turn!

Thus, in addition to philosophical reflections and (unapologetic) ``confessions'', 
there will be ``vitriolic portraits'' (to reuse the expression of a colleague of mine who found himself mistreated\ldots), not to mention large-scale and well-oiled ``operations''.
Robert Jaulin\footnote{Robert Jaulin is an old friend of mine. From what I understand, his position within the establishment of the ethnology world (as a ``white wolf") mirrors the one I occupy within the ``high society'' of mathematics.} assured me (half-jokingly) that in 
R\'ecoltes et Semailles I was writing the ``ethnology
of the mathematical world'' (or maybe the sociology, I don't quite remember).
It is flattering of course to learn
that one has been (unknowingly) doing scholarly things!
It is true that during the ``investigation'' segment of the reflection, I mention in passing (and against my will\ldots) a good chunk of the mathematical establishment,
as well as a number of my colleagues and friends of a more modest status.
Over the past few months, since I started distributing preliminary versions of
R\'ecoltes et Semailles, 
this observation was ``brought up'' again. My testimony had the effect of a rock landing in a pond. 
There were returns of every kind (except for boredom\ldots). 
Yet, almost every time, the response was far from the one expected. 
There was also a lot of silence, which spoke volumes. 
Visibly, I had (and still have) a lot to learn about what happens in people's minds, be it
my previous students or some of my more or less highly situated colleagues - or should I say: I still have a lot to learn about ``the sociology of the mathematical milieu''!
To all those who contributed to the great sociological project of my old days, I would like
to express my sincere gratitude.

I was of course particularly touched by responses with warm tonalities. There were also some rare
colleagues who opened up about a (previously unexpressed) feeling or sentiment regarding a state of crisis, or 
\marginpar{p. A5}
degradation, affecting the mathematical milieu to which they feel that they belong. 

Outside of this milieu, I would like to acknowledge 
Sylvie and Catherine Chevalley\footnote{Sylvie and Catherine Chevalley are the widow and
daughter, respectively, of Claude Chevalley, the colleague and friend to whom the central part of 
R\'ecoltes et Semailles is devoted (ReS III, ``The Key of the Yin and the Yang").
I write about him and the role that he played in my
journey at several points in the reflection.}, Robert Jaulin, St\'ephane Deligeorge, Christian Bourgois for being among the very first people to respond warmly, or even affectionately, to my testimony. 
If R\'ecoltes et Semailles achieves a wider diffusion than that of the initial 
printing (which was addressed to a very limited social circle), it will be mostly thanks to
them - thanks, mostly, to their contagious 
conviction that what I strived to grasp and verbalize had to be said, and that my message could find an audience beyond my colleagues (who are often sullen,
if not belligerent, and strictly opposed to questioning their position\ldots).
Case in point, Christian Bourgois decided without hesitation to run the risk of publishing the unpublishable, and St\'ephane Deligeorge to honor me by placing my 
indigestible testimony alongside works of Newton, Cuvier, and Arago (I could not have asked for
better company!).
To each of them, for their repeated expressions of sympathy and trust offered at an
especially ``sensitive" moment, I hereby gladly express all my gratitude. 

And so here we are, about to begin a Walk through my life's work, which shall serve as a primer to
a journey through my life. 
A long journey, to be sure, spanning over a thousand densely packed pages. I spent a lifetime undergoing this journey, without the latter ever 
coming to an end, and it took me more than a year to rediscover it, 
one page at a time.
Words were sometimes difficult to come by, as I sought to convey an experience which continues to evade my comprehension - just as ripe and thick grapes stacked in a press sometimes seem to be evading the force applied upon them\ldots
But even at times when the words seemed to come flooding, nothing was left to chance. 
Each word was gauged in the moment, or after the fact, so as to obtain a carefully calibrated end result. As such, this reflection/testimony/journey is not meant to be read hurriedly, in a day or in a
month, by a reader who is in a rush to reach the final word. There is \textbf{no} 
\marginpar{p. A6}
``final word'', no
``conclusion'' in 
R\'ecoltes et Semailles, no more than there are any such things in my life or in yours. 
What you will find instead is a wine, aged over the course of a lifetime in the depths of my being.
The last glass you will drink will be no better nor worse than the first or the
hundredth. 
They are all ``the same'', just as they are all different. And if the first glass is spoiled,
so is the rest of the barrel; better to drink fresh water (if such can be found),
than to drink bad wine. 

But a good wine ought not to be drunk in haste, nor should it be drunk
expeditiously. 

%\end{document}

% \marginpar{p. A1}
% R\'ecoltes et Semailles
% p .
% c ,
% q "
% pa (
