\begin{comment}
\documentclass{book}
\usepackage{master}
\newcommand{\rec}{$\text{R\'ecoltes et Semailles}$}
\newcommand{\no}{n$^\circ$}
\hfuzz = 100pt
\begin{document}
\end{comment}

\chapter{By way of a foreword}

\marginpar{p. A1}
January 30 1986

All that was left to write was the foreword, in order for
R\'ecoltes et Semailles to be given to the publisher.
And I swear that I went into it with all the will in the world to write something suitable. 
Something \textbf{reasonable} this time. No more than three or four pages, carefully phrased, that would introduce this huge tome of more than one thousand
pages. Something which ``grabs'' the attention of the jaded reader, 
which perhaps suggests that in these frightening ``more than a thousand pages'', there
could be things of interest to him (or things which concern him, who knows?). It is not
really my style to pander. But I was ready to make an exception for once!
The ``publisher crazy enough to give it a shot'' (to publish this visibly unpublishable
monster) had to make ends meet one way or another. 

But then, it didn't come. And yet I tried my best. And not only for an afternoon, as I
originally planned. Tomorrow will mark three weeks since I started, since the sheets began
accumulating.
What came, for sure, isn't what one could decently call a ``foreword''.
It is yet another miss! Blame it on my old age - I have never been a salesman. Even when
it comes to pleasing (oneself or friends\ldots).

What came instead is a sort of long ``Walk'' with commentary, through my work as a mathematician. 
A Walk intended mostly for the ``layman'' - he who ``never understood anything about
math''. And for myself as well, having never indulged in such a Walk. 
Step by step, I found myself unearthing and saying things that had previously remained
unspoken. As if by chance, these are also things which I feel are most essential, both in
my practice and its outcome. 
They are things which are not technical in nature. It will be up to you to decide whether
or not I succeeded in my naive enterprise to ``get the message through'' - an exterprise
which surely is also a bit mad. 
My satisfaction and my pleasure will come from making you feel these things. 
Things that many of my wise colleagues do not feel anymore. Maybe they have become too wise
and too prestigious.
This often leads to losing touch with the simplest and most essential things. 

During this ``Walk through a body of work'' 
I also speak of my life.
As well as, here and there, what 
R\'ecoltes et Semailles 
is about.
I mention this in more detail in the ``Letter'' 
\marginpar{p. A2}
(dated from May of last year) which
follows the ``Walk''. 
This Letter was directed towards my
previous students and to my ``old friends'' in the mathematical community. 
But even the Letter is not technical in nature. 
It can be read by any 
reader interested in learning 
through a ``heartfelt'' narrative, the odds and ends that led me to writing 
R\'ecoltes et Semailles. Even more than the Walk, the Letter will provide a 
preview to a particular atmosphere in the ``prestigious mathematical world''.
And also (just as in the Walk) of my writing style, as peculiar as it may seem, and of the
spirit that is expressed through this style - a spirit which is not universally
appreciated. 

In the Walk and throughout R\'ecoltes et Semailles, I speak of the activity of 
\textbf{doing mathematics}.
It is an activity for which I have first-hand experience and know very well. Most of the
things I say anbout it can surely be said of any kind of creative work, or work involving
discovery. 
In any case it is true of all ``intellectual''
work, that which is done using the ``brain'' and in writing. 
All such work proceeds through the the outbreak and development of an
\textbf{understanding} of the things which are being probed. 
But to take an example at the 
opposite extreme, romantic passion is also an activity of discovery. 
It opens us to understanding of a ``physical'' nature which also renews itself, develops, and
deepends over time. 
Both of these impulses - 
that which, say, livens the mathematician at work and
that of the lover 
- are much close in nature than we generally assume or we readily admit. 
I hope that the pages of 
R\'ecoltes et Semailles 
will make you feel this impulse in your work and in your daily life. 

Most of the Walk focuses on mathematical work itself. 
I remain mostly silent concerning the \textbf{context} in which this work takes place, and
concerning the \textbf{motivations} at play outside of mathematical work itself. 
This risks giving me, or the mathematician, or the ``scientist'' in general 
a flattering but deformed image. 
In the style of ``grand and noble passion'' without any form of rectification.
In accordance with the great ``Myth'' of Science 
(with a capital S, if you will!).
The heroic myth, ``promethean'', to which writers and thinkers have succumbed 
(and continue to succumb). 
Only historians, maybe, manage to sometimes resist this tantalizing myth. 
The truth is, within the motivation of these ``scientists'',
which sometimes lead them to devote themselves entirely to their work, ambition and vanity
play a role just as important and universal as they do in any other profession. 
This phenomenon 
\marginpar{p. A3}
appears in
blunt or subtle ways depending on the person - and I am no exception to this pattern.
The reading of my testimony will hopefully leave no doubt about this fact. 

It is true also that even the most intense ambitions are
powerless at discovering or proving a novel mathematical statement - just as they are
powerless (for instance) to ``make one hard'' (in the proper sense of the term).
Whether man or woman, what ``makes one hard'' is not ambition, nor the desire to shine, to
exhibit power, of a sexual nature in this case - quite the contrary!
It is the acute perception of something strong, at once very real and very delicate. 
One could call it ``beauty'', thought this is one of
a thousand faces of this thing. Being ambitious doesn't prevent one from 
sensing the beauty of a being or a thing. But it is \textbf{not} ambition which makes us
feel it\ldots 

The person that first discovered and mastered fire was somebody just like you and me. 
Not at all what we refer to as ``hero'', or ``demi-god'', and so on.  
Surely, just like you and me, he has encountered the grip of anxiety as well as the 
time-worn remedy of vanity which alleviates the grip.
At the instant at which he ``knew'' fire, there was no fear nor vanity. 
Such is the truth in the heroic myth. 
The myth becomes insipid when it is used to
to disguise another aspect of things which is just as real and essential. 

My aim in R\'ecoltes et Semailles has been to address both aspects of the myth - that of
the impulse towards understanding, and that of fear and its vain antidotes. 
I believe I ``understand'', or at least \textbf{know} this impulse and its origin (or
perhaps one day I will discover to what extent I was deluded). 
But concerning fear and vanity, as well as the resulting insidious creativity blocks, 
I know thta I have yet to thoroughly uncover this great enigma. 
And who knows if I will ever reach the conclusion of this mystery in the year I have
left\ldots

As I was writing 
R\'ecoltes et Semailles 
two images emerged in order to represent the two aspects of the human journey:
that of the \textbf{child} (aka the \textbf{worker}), and that of the
\textbf{boss}.
In the Work which we are about to undertake, we will be dealing mostly with the ``child''.
It is him also that is featured in the subtitle ``\textbf{The Child and the Mother}''. 
The motivation for this name 
will hopefully become clear over the course of this work.

\marginpar{p. A4}
In the remainder of this reflection, however, it is the Boss who takes the lead. 
He is living up to his name! It would be more accurate 
to speak of multiple bosses of competing enterprises rather than of a \textbf{singular}
boss.
But it is also true that all bosses essentially resemble one another.
And once we mention bosses it is implied that we will also have to deal with ``villains''.
In part I of the reflection (named ``Fatuity and Renewal'', which follows the present
introductory section). I mostly take on the role of the ``villain''. In the following
three parts it is mostly the ``others''. 
Chacun son Tour!
That is to say that, in addition to philosophical reflections and ``confessions''
(not contrite), there will be ``vitriolic portraits'' (to use the expression of one of my
colleagues who found himself tormented). 
Not to mention large-scale, well-oiled ``operations''.
Robert Jaulin\footnote{Rober Jaulin is an old friend of mine. From what I understand, his position
with respect to the establishment of the ethnological milieu mirrors mine with respect to
the ``high society'' of mathematics (as white wolves).} assured me (half jokingly) that in 
R\'ecoltes et Semailles I was making the ``ethnology
of the mathematical community'' (or maybe the sociology I do not quite remember).
It is flattering of course to learn
that one has been (unknowingly) doing scholarly things!
It is true that during the ``investigation'' segment of the reflection, I saw in passing,
in the pages I was writing, a good chunk of the mathematical establishment
without counting a number of my colleagues and friends of more modest status.
Over the past few months, since I have been sending preliminary versions of
R\'ecoltes et Semailles 
this has been ``brought up'' again. My testimony arrived like a tome landing in a pond. 
There were responses of every kind (except for boredom\ldots). 
Yet almost every time the response was far from what I expected. 
There was also a lot of silence, which speaks volumes. 
Visibly, I had (and still have) a lot to learn about what happens in people's minds, among
my previous students and other colleagues - excuse me I meant about ``the sociology of the
mathematical milieu''!
To all those that contributed to the great sociological work of my old days, I would like
to express sincere recognition.

Of course, I was particularly sensitive to warm responses. There were also some rare
colleagues who conveyed a sentiment (thus far unexpressed) of crisis, or of 
\marginpar{p. A5}
degradation of
the inner workings of the mathematical milieu with which they identify themselves. 

Outside of this milieu, among the very first to respond positively to my testimony,
I would like to recognize 
Sylvie et Catherine Chevalley\footnote{Sylvie et Catherine Chevalley are the widow and
daughter of Claude Chevalley, the colleague and friend to whom the central part of 
R\'ecoltes et Semailles is devoted (ReS III, the key of the Yin and the Yang).
At multiple times in the reflection I speak of him and of the role he played in my
journey.} Robert Jaulin, St\'ephane Deligeorge, Christian Bourgois. 
If R\'ecoltes et Semailles achieves a wider diffusion than that of the initial 
printing (addressed to a very limited social circle of people), it is mostly thanks to
them. Thanks mostly to them communicating their 
conviction that what I strived to seize and say had to be said. 
And that it could have an audience outside of my colleagues (who are often sullen,
sometimes even belligerent, and strictly opposed to question their position\ldots).
Indeed Christian Bourgois did not hesitate to risk publishing the unpublishable, and
St\'ephane Deligeorge did not hesitate to place my 
indigestible testimony alongside works of Newton, Cuivier, and Arago (I could not ask for
better company).
To each of them, for their repeated expressions of sympathy and trust, intervening at an
especially sensitive moment, I happily extend all my gratitude. 

And here we are at the beginning of a Walk through a life's work, serving as a prelude to
a journey through a lifetime. 
A long journey, over a thousand pages long, each of which is
densely packed. I spent a lifetime undergoing this journey without ever 
exhausting it, and it then took me more than a year to rediscover it, 
one page at a time.
Words were sometimes hard to come by, as they were 
intended to convey an experience which evaded comprehension - just as ripe grapes stacked
in a press occasionally seem to evade the force upon them\ldots
But even in those moments when words come flooding, it is not by happenstance. 
Each word has been carefully 
weighed in passing, or after the fact.
Thus, this reflection/testimony/journey is not meant to be read hastily, in a day or in a
month, by a reader rushed to reach the final word. There is \textbf{no} 
\marginpar{p. A6}
``final word'', no
``conclusion'' in 
R\'ecoltes et Semailles, no more than there are any such things in my life or in yours. 
There is only a wine, aged over the course of a lifetime, at the core of my being.
The last glass which you will be drinking will be no better nor worse than the first or the
hundredth. 
They are all ``the same'', and they are all different. And if the first glass is spoiled,
so is the rest of the barrel; it is better to drink fresh water (if such can be found),
than to drink bad wine. 

But a good wine ought not to be drunk in haste, nor 
expeditiously. 

% \end{document}
