\begin{comment}
\documentclass{book}
\usepackage{master}
\usepackage{changepage}
\newcommand{\rec}{$\text{R\'ecoltes et Semailles}$}
\newcommand{\no}{n$^\circ$}
\hfuzz = 100pt

% NOTE and SUBNOTE FORMATTING
\usepackage{titlesec}
\usepackage[dotinlabels]{titletoc}

% define `note'
\titleclass{\nnote}{straight}[\section]
\newcounter{nnote}
\renewcommand{\thennote}{\thesection.\arabic{nnote}}
\titlespacing*{\nnote}{0pt}{3.5ex plus 1ex minus .2ex}{1ex plus .2ex}
\titleformat{\nnote}[runin]{\bfseries }{\bfseries Note }{0pt}{}[]

% define `subnote'
\titleclass{\subnote}{straight}[\section]
\newcounter{subnote}
\renewcommand{\thesubnote}{\thesection.\arabic{subnote}}
\titlespacing*{\subnote}{0pt}{3.5ex plus 1ex minus .2ex}{1ex plus .2ex}
\titleformat{\subnote}[runin]{\bfseries }{\bfseries Note }{0pt}{}[]

% the first optional arg sets the size of the indentation in the TOC
\titlecontents{nnote}[9em]
{}{Note }{}{\titlerule*[1pc]{.}\contentspage}

% the first optional arg sets the size of the indentation in the TOC
\titlecontents{subnote}[11em]
{}{Note }{}{\titlerule*[1pc]{.}\contentspage}

\newtheorem{remark}{Remark}
\theoremstyle{nonumberbreak}
\newtheorem{remarknn}{Remark}

\begin{document}
% print table of contents with notes and subnotes
 \setcounter{tocdepth}{5}
 \setcounter{secnumdepth}{5}
 \startcontents[chapters]
 \printcontents[chapters]{}{0}{}

\setcounter{chapter}{14}
\end{comment}

\chapter{C) High Society}

\section{VII The Colloquium - or Mebkhout's sheaves and Perversity}

\subsection{The Iniquity or a feeling of return}

\nnote{75}

\subsection{The colloquium}

\nnote{75'}

\subsection{The prestidigitator}

\nnote{!75''}

\subsection{Perversity}

\nnote{76}

\subsection{Thumbs up!}

\nnote{77}

\subsection{The robe of the emperor of China}

\nnote{77'}

\subsection{Encounters from beyond the grave}

\nnote{78}

\subnote{78$_1$}

\subsection{The victim - or the two silences}

\nnote{78'}

\subnote{78$_1$'}

\subnote{78$_2$'}

\subsection{The Boss}

\nnote{!78''}

\subsection{My friends}

\nnote{79}

\subsection{The tome and high society - or moon and green cheese...}

\nnote{80}

\section{VIII The Student - a.k.a. the Boss}

\subsection{Thesis on credit and risk-proof insurance}

\nnote{81}

\subnote{81$_1$}

\subnote{81$_2$}

\subnote{81$_3$}

\subsection{The right references}

\nnote{82}

\subsection{The jest, or the ``weight complexes"}

\nnote{83}

\section{IX My students}

\subsection{Silence}

\nnote{84}

\subnote{84$_1$}

\subsection{Solidarity}

\nnote{85}

\subsection{Mystification}

\nnote{!85'}

\subnote{85$_1$}

\subnote{85$_2$}

\subsection{The defunct}

\nnote{86}

\subsection{The massacre}

\nnote{87}

\subnote{87$_1$} (May 31) This closing talk, probably one of the most interesting and substantial, along with  the opening talk, were visibly not lost on everyone, as I now realize upon learning about MacPherson's paper ``Chern classes for singular algebraic varieties" (Chern classes\marginpar{p. 362} for singular algebraic varieties, Annals of Math. (2) 100, 1974, p. 423-432)\todo{cite} (submitted in April 1973). There, I found under the name of ``Deligne-Grothendieck conjecture" one of the main conjectures which I had introduced in said talk in the context of schemes. The conjecture was reformulated by MacPherson in the transcendental context of algebraic varieties over the complex numbers, where the Chow ring is replaced by the homology group. Deligne had learned about this conjecture\footnote{(*) In a slightly different form admittedly, see the rest of the note dated May 31.}(*) during my 1966 talk, the same year that he joined the seminar and started familiarizing himself with the language of schemes and cohomological methods (see the note ``One of a kind", n$^o$ 67'). It is nonetheless kind to have included me in the name of this conjecture - a few years later this would have been out of the question...

(June 6) I would like to use this occasion to explicitly write down the conjecture which I had announced in the context of schemes, while also probably hinting at the obvious analogue in the complex analytic (or even rigid analytic) context. I viewed it as a ``Riemann-Roch"-type theorem, albeit with discrete coefficients instead of coherent coefficients. (Zoghman Mebkhout also told me that his viewpoint on $\mathcal{D}$-modules should enable one to consider both Riemann-Roch theorems as contained in a single crystalline Riemann-Roch theorem, which in zero characteristic would constitute the natural synthesis of the two Riemann-Roch theorems that I have introduced in mathematics, the first in 1957 and the second in 1966). Start by fixing a coefficient ring $\Lambda$ (not necessarily commutative, but noetherian for simplicity and furthermore with torsion prime to the characteristic of the schemes under considerations, to meet the needs of \'etale cohomology...). Given a scheme $X$, write 
$$ \text{K}_{\cdot}(X, \Lambda) $$
to denote the Grothendieck group associated to constructible \'etale sheaves of $\Lambda$-modules. This group is functorial in $X$ with respect to the functors $\textbf{R}f_!$ when restricting our attention to separated scheme morphisms of finite type. For regular $X$, I claimed that there exists a canonical group homomorphism, playing the role of the ``Chern character"\marginpar{p. 363} in the coherent Riemann-Roch theorem,
\begin{equation}\label{chern1.1} 
\text{ch}_X: \text{K}_{\cdot}(X, \Lambda) \to \text{A}(X) \otimes_{\mathbb{Z}} \text{K}_{\cdot}(\Lambda),
\end{equation}
where $\text{A}(X)$ is the Chow ring of $X$ and $\text{K}_{\cdot}(\Lambda)$ is the Grothendieck group associated to $\Lambda$-modules of finite type. This homomorphism was supposed to be completely determined by the presence of a ``discrete Riemann-Roch formula" for \textbf{proper} morphisms between regular schemes $f: X \to Y$, whose form is analogous to the Riemann-Roch formula in the coherent context, except that the Todd ``multiplier" is replaced by the total relative Chern class:
$$ \text{ch}_Y(f_!(x)) = f_*(\text{ch}(x)c(f)), $$
where $c(f)$ denotes the total Chern class of $f$. It isn't hard to see that, in a context where one has access to a resolution of singularities theorem in the strong sense of Hironaka's, this Riemann-Roch formula does indeed uniquely determine the $\text{ch}_X$'s. 

Of course, we are supposing that we are working in a context in which there is a notion of Chow ring. (I am not aware of any attempt to develop a theory of Chow rings for regular schemes that are not of finite type over a field.) Otherwise, we could also work with the graded ring associated to the usual ``Grothendieck ring" $K^0(X)$ in the coherent context, equipped with the usual filtration (see SGA 6); or we could replace $\text{A}(X)$ by the even $\ell$-adic cohomology ring, given by the direct sum $\bigoplus_i \text{H}^{2i}(X, \underline{\mathbb{Z}}_\ell(i)$. This comes with the added baggage of an artificial parameter $\ell$ and produces coarser, ``purely numerical" formulas, whereas the Chow ring has the added charm of having a continuous structure which is destroyed upon passing to cohomology.

Already in the case where $X$ is a smooth algebraic curve over an algebraically closed field, computing $\text{ch}_X$ involves studying delicate Artin-Serre-Swan-type local invariants. This hints at the depth of the general conjecture, whose pursuit would involve understanding the analogues of these invariants in higher dimensions. 

\textbf{Remark.}  Writing $\text{K}^{\cdot}(X, \Lambda)$ to denote the ``Grothendieck ring" associated to constructible complexes of \'etale sheaves of $\Lambda$-modules of finite Tor-dimension (which acts on $\text{K}_{\cdot}(X, \Lambda)$ when $\Lambda$ is commutative...), we also expect to have a homomorphism
\begin{equation}\label{chern1.2} 
\text{ch}_X: \text{K}^{\cdot}(X, \Lambda) \to \text{A}(X) \otimes_{\mathbb{Z}} \text{K}^\cdot(\Lambda), 
\end{equation}
giving rise (mutatis mutandis) to the same Riemann-Roch formula.

\marginpar{p. 364}Now, let $\text{Cons}(X)$ denote the ring of constructible integral-valued functions on $X$. We can define more or less tautologically canonical homomorphisms:
\begin{equation}\label{chern2.1} 
\text{K}_{\cdot}(X, \Lambda) \to \text{Cons}(X) \otimes_{\mathbb{Z}} \text{K}_\cdot(\Lambda), \text{ and} 
\end{equation}
\begin{equation}\label{chern2.2} 
\text{K}^{\cdot}(X, \Lambda) \to \text{Cons}(X) \otimes_{\mathbb{Z}} \text{K}^\cdot(\Lambda). 
\end{equation}
If we restrict our attention to schemes in \textbf{zero characteristic}, then (using Euler-Poincar\'e characteristics with proper support) we see that the group $\text{Cons}(X)$ is a \underline{covariant} functor with respect to morphisms of finite-type between noetherian schemes (in addition to being contravariant as a ring-functor, and this independently of the characteristic), compatibly with the above tautological morphisms. (This corresponds to the ``well-known" fact, which I don't recall being proven in the oral seminar SGA 5, that in \textbf{zero characteristic}, a locally constant sheaf of $\Lambda$-modules $F$ over an algebraic scheme\todo{algebraic space?} $X$ has image $d_\chi(X)$ under the map 
$$ f_!: \text{K}^{\cdot}(X, \Lambda) \to \text{K}^{\cdot}(e, \Lambda) \simeq K^\cdot(\Lambda), $$
where $d$ is the rank of $F$, $e = \Spec k$, and $k$ is the algebraically closed base field...)\todo{figure out what $d_\chi(X)$ means} This suggests that the Chern homomorphisms \ref{chern1.1} and \ref{chern1.2} should be deducible from the tautological homomorphisms \ref{chern2.1} and \ref{chern2.2} upon composition with a ``universal" Chern homomorphism (independent of the choice of coefficient ring $\Lambda$)
$$ \text{ch}_X: \text{Cons}(X) \to \text{A}(X), $$
in such a way that the two versions of the Riemann-Roch formula ``with $\Lambda$-coefficients" appear as formally enclosed in a RR formula at the level of constructible functions, with the latter always taking the same form.

When working with schemes over a fixed base field (this time in arbitrary characteristic), or more generally over a fixed \textbf{regular} base scheme $S$ (such as for instance $S = \Spec \mathbb{Z}$), the form of the Riemann-Roch formula closest to the traditional notation (in the coherent context, familiar since 1957) can be obtained by introducing the product
\begin{equation}\label{relchern} 
\text{ch}_X(x)c(X/S) = c_{X/S}(x)
\end{equation}
(where $x$ is in either $\text{K}_\cdot(X, \Lambda)$ or in $\text{K}^\cdot(X, \Lambda)$), which could be called the \marginpar{p. 365}\textbf{Chern class of $x$ relative to the base $S$}. When $x$ is the unit of $\text{K}^\cdot(X, \Lambda)$, i.e. the class of the constant sheaf with value $\Lambda$, we recover the image of the relative total Chern class of $X$ with respect to $S$ under the canonical homomorphism $\text{A}(X) \to \text{A}(X) \otimes \text{K}^\cdot(\Lambda)$. With this notation in place, the RR formula becomes equivalent to the fact that the formation of these relative Chern classes
\begin{equation}\label{chern3.1}
 c_{X/S}: \text{K}_\cdot(X, \Lambda) \to \text{A}(X) \otimes \text{K}(\Lambda) 
 \end{equation}
for fixed $S$ and varying regular scheme $X$ of finite type over $S$ is functorial with respect to proper morphisms, and likewise for the variant \ref{chern1.2}\todo{make sure the right equation is being referenced}. In zero characteristic, this can be reduced to the functoriality (with respect to proper morphisms) of the corresponding map
\begin{equation}\label{abschern} 
c_{X/S}: \text{Const}(X) \to \text{A}(X). 
\end{equation}

It is under this form that the existence and uniqueness of an absolute ``Chern class" \ref{abschern} in the case $S = \Spec \mathbb{C}$ is conjectured in the work of MacPherson, the relevant conditions being (here as in the general case in zero characteristic) (a) functoriality of \ref{abschern} with respect to proper morphisms and (b) the identity $c_{X/S}(1) = c(X/S)$ (in this case, the ``absolute" total Chern class). The form of the conjecture presented and proven by MacPherson differs from my initial conjecture in two ways. The first is a ``negative", namely that he is not working in the Chow ring, but rather in the integral cohomology ring, or more precisely the integral homology group, defined by transcendental methods. The other is a ``positive" - and this is possibly where Deligne contributed to my initial conjecture (unless this contribution is due to MacPhersson\footnote{(*) (March 1985) That was indeed the case, see note n$^o$164 referenced in the previous footnote.}(*)). Namely, the observation is that in order to prove existence and uniqueness for \ref{abschern}, we don't need to restrict ourselves to regular schemes $X$, as long as we replace $\text{A}(X)$ by the integral homology group. As such, it is probable that the same holds in the general case, if we write $\text{A}(X)$ (or better $\text{A}(X)$) to denote the \textbf{Chow group} (which is no longer a ring in general) of a noetherian scheme $X$. Said differently: while the heuristic definition of the invariants $\text{ch}_X(x)$ (for $x$ in either $\text{K}_\cdot(X, \Lambda)$ or in $\text{K}^\cdot(X, \Lambda)$) uses in an essential way the hypothesis that the ambient scheme is regular, upon multiplying by the ``multiplier" $c(X/S)$ (for $X$ of finite type over a fixed regular scheme $S$), the product obtained in \cite{relchern} seems to still make sense regardless of any regularity hypothesis on\marginpar{p. 366}X, as an element of the tensor product 
$$ \text{A}_\cdot(X) \otimes K_\cdot(\Lambda) \text{  or } \text{A}_\cdot(X) \otimes K^\cdot(\Lambda),  $$
where $\text{A}_\cdot(X)$ denotes the Chow group of $X$. The spirit of MacPherson's proof (which does not use resolution of singularities) seems to suggest that it it possible to exhibit a ``constructive" and explicit construction of the homomorphism \ref{chern3.1}, by ``making do" with the singularities of $X$ as they are, as well as with the singularities of the sheaf of coefficients $F$ (whose class is $x$), so as to ``collect" a cycle on $X$ with coefficients in $\text{K}_\cdot(\Lambda)$. This would fit in the circle of ideas which I had introduced in 1957 with the coherent Riemann-Roch theorem, where I notably computed self-intersections, without quite ``moving around" the cycle under consideration. An initial obvious step (obtained by immersing $X$ in an $S$-scheme) would be to reduce to the case where $X$ is a closed subscheme of a regular $S$-scheme\todo{make sure this is what Grothendieck meant}.

The idea that it should be possible to develop a \textbf{singular} (coherent) Riemann-Roch theorem was already familiar to me, although I couldn't say for how long, but I never seriously put it to the test. It was in part this idea (other than the analogy with the ``cohomology, homology, cap-product" formalism) which had led me to systematically introduce $\text{K}_\cdot(X)$, $\text{K}^\cdot(X)$, $\text{A}_\cdot(X)$, and $\text{A}^\cdot(X)$  in SGA 6 (in 1966/67), instead of choosing to work only with $\text{K}^\cdot(X)$. I can't remember if I had thought about something along those lines in the SGA 5 seminar in 1966, or if I made mention of it in my talk. As my handwritten notes have disappeared (perhaps while moving?), I may never know...

(June 7) In reading through MacPherson's article, I was stricken by the fact that the word ``Riemann-Roch" is never used - this is also the reason why I did not immediately recognize the conjecture which I had made in the SGA 5 seminar in 1966, the latter having always been (and still is) a ``Riemann-Roch"-type theorem in my view. It seems that at the time of writing his article, MacPherson did not notice this evident filiation. I am guessing that the reason behind this is that Deligne, who circulated this conjecture in the form he liked best after my departure, took care to ``erase", insofar as possible, the evident filiation with the Riemann-Roch-Grothendieck theorem. I think I understand his motivation behind this. On the one hand, this weakens the link between the conjecture and myself, making more plausible\marginpar{p. 367}  the name currently under circulation, ``Deligne-Grothendieck conjecture". (N.B. I ignore where this conjecture is currently circulating in the scheme context, and is so, I would be curious to know under which name). But the deeper reason seems to be his obsession with denying and destructing, to the extent possible, the fundamental unity of my work and mathematical vision\footnote{(*) Compare with the comment made in the note ``The remains" (n$^o$88) regarding the profound significance of the SGA 4$\frac{1}{2}$ operation, similarly aiming to break out into an amorphous collection of ``technical digressions" the profound unity of my work on \'etale cohomology via the ``violent insertion" of the outlandish text SGA 4$\frac{1}{2}$ between the two indissoluble parts SGA 4 and SGA 5 in which this work is carried out.}(*). This is a striking example of the way in which a fixed idea entirely foreign to any mathematical motivation can obscure (if not downright seal) what I have called the ``sane mathematical instinct" of a mathematician whose abilities are nonetheless exceptional. Such a mathematical instinct would not fail to perceive the analogy between the two statements, one ``continuous" and the other ``discrete", of a ``single" Riemann-Roch theorem, an analogy which I had of course spelled out during my talk. As I indicated yesterday, this filiation will probably be confirmed in the near future by a formal statement (conjectured by Zoghman Mebkhout), at least in the complex analytic context, enabling us to deduce both theorems from a common result. Clearly, given Deligne's ``grave-digging" attitude towards the Riemann-Roch theorem\footnote{(**) This attitude towards the Riemann-Roch-Grothendieck theorem are manifested particularly clearly in the ``Funeral Rite"; see the note ``The Funeral Rite (1) - or the compliments", n$^o$104.

}(**), he was not positioned to discover the common statement connecting them in the analytic context, nor to think to look for an analogous statement in the general context of schemes. In the same way, this attitude prevented him from unearthing the fruitful viewpoint on $\mathcal{D}$-modules in studying the cohomology theory of algebraic varieties, which followed too naturally from a circle of ideas that needed to be buried; neither was he able to recognize Mebkhout's fertile work for years on end, a work which had been successful where he had failed. 

\subnote{87$_2$} (May 31) That was also the year when I gave my Bourbaki talk on the rationality of $L$-functions, heuristically using Verdier's result\todo{figure out a citation} (mostly the expected form of local terms in the case at hand) thirteen years before Illusie proved it upon Deligne's request. I seem to recall that the super-general\marginpar{p. 368} formula Verdier showed me, which came as a surprise, was proven using the ``six operations" formalism in a few lines - it was the kind of formula for which writing it was (nearly) proving it! If there was any ``difficulty", it could only have been regarding the verification of one or two compatibility conditions\footnote{(*) (June 6) It furthermore seems that the initial proof of the Lefschetz-Verdier formula via the biduality theorem (now promoted to ``Deligne's theorem") depended on a hypothesis regarding the resolution of singularities, which Deligne was able to circumvent in the case of schemes of finite type over a field. This constituted a good opportunity to turn the situation to their advantage, by giving the impression that SGA 5 is relying to the (sic) seminar SGA 4$\frac{1}{2}$ which ``precedes" it (and which was indeed published earlier!).}(*). Furthermore, both Illusie and Deligne knew full well that the proofs I had given during the seminar regarding various explicit trace formulas \textbf{were complete}, and that they did not depend in any way upon Verdier's general formula, the latter having only played the role of a ``catalyst" inciting me to formulate and prove trace formulas in the most general possible settings. Both showed a patent lack of good faith at this occasion. In the case of Deligne, this was already clear to me while writing the note ``Clean slate" (n$^o$67) - but it was probably not so clear to an uninformed reader, or to an informed reader who renounced the use of their sane faculties.

(June 6) As for Illusie, he fully played along in trying to muddy the waters so as to project the appearance of a hyper-technical oral seminar which did not even provide complete proofs of all the results, and notably of the trace formulas. These were nonetheless proven (for the first time) in 1965/66, and this was also the place where Deligne had the privilege to learn about them and about all of the delicate machinery that accompanies them\footnote{(**) In the second paragraph of the Introduction to the volume titled SGA 5, Illusie presents the three expos\'es III, III B, and XII, on the Lefschetz formula in \'etale cohomology, as the ``heart of the seminar", even though it is also said in the introduction to expos\'e III B that (contrary to reality) ``this expos\'e does not correspond to an oral talk from the seminar", while in the introductions to both expos\'es III and III B he tries to give the impression that they rely on SGA 4$\frac{1}{2}$, with expos\'e III being presented as ``conjectural"!! In reality, the totality of the seminar SGA 5 was technically independent of the expos\'e III (Lefschetz-Verdier formula), which acted as heuristic motivation, and expos\'e III B was nothing but the ``hole" (expos\'e XI) created by Bucur's moving, which turned into a welcomed excuse for this additional dismemberment.

In order to add to the impression of a seminar of ``technical digressions" (whispered to him by his friend Deligne), Illusie took care to remove the introductory expos\'e in which I had brushed a preliminary outline of the principal great themes which were to be developed during the seminar - an outline in which trace formulas only account for a small (which is of particular importance due to their arithmetic implications, towards the Weil conjectures). For an overview of these ``great themes", see the sub-note n$^o 87_5$ later.}(**).

\marginpar{p. 369}This reminds me that I naturally had taken the time to prove the Lefschetz-Verdier formula during the seminar - this was the least I could do, and it provided a particularly striking application of the local-global duality formalism which I was setting out to develop. A question recently occurred to me as to why on earth Deligne and Illusie had chosen to cover the theorem of their good friend Verdier, whose name the result carried, similarly to the derived and triangulated categories which he never took the time to develop in full (or at least to make such developments available to the public), when there were ten or so other expos\'es begging to be written by one of my students, so that they were not short of choices when it came to naming the technical ``obstacle" to the publication of SGA 5. Therein lies a \textbf{challenge} of sorts in the absurdity displayed (or in the collective cynicism exhibited by my ex-cohomology students, whom I consider to have been unified in this operation-massacre) which reminds me of the ``weight-complex" brilliantly invented by Verdier in the preceding year (see the note with the same name, n$^o$83), as well as (in the iniquitous register) the choice by Deligne to call ``perverse" the sheaves which should have been called ``Mebkhout sheaves" (see the note ``Perversity", n$^o$76). I see all of these inventions as acts of dominance and contempts towards the mathematical community as a whole - as well as a \textbf{bet} which had visibly been won until the unexpected reappearance of the deceased, almost appearing as the only person awake amidst a community in slumber...

\subnote{87$_3$} (June 5) In light of this summary of the massacre, one can fully appreciate the following declaration of Illusie, written on line 2 of his introduction to the volume titled SGA 5:

\begin{quote}
``Compared to the primitive version, the only important changes regard expos\'e II [generic K\"unneth formulas], which is not reproduced, as well as expos\'e III [Lefschetz-Verdier formula], which has been entirely rewritten and augmented by an appendix numbered III B\footnote{(*) With this appendix presented as belonging to the ``heart of the seminar"! (see preceding footnote.)}(*). Except for some detail\marginpar{p. 370} modifications and added footnotes, the other expos\'es were left \textbf{tel quel}" (emphasis mine).
\end{quote}

\subnote{87$_4$} (June 6)


\subsection{The remains}

\nnote{88}

\subsection{... and the body}

\nnote{89}

\subsection{The heir}

\nnote{90}

\subsection{The co-heirs}

\nnote{91}

\subnote{91$_1$}

\subnote{91$_2$}

\subnote{91$_3$}

\subnote{91$_4$}

\subsection{... and the chainsaw}

\nnote{92}







%\end{document}
