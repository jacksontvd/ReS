\begin{comment}
\documentclass{book}
\usepackage{master}
\usepackage{changepage}
\newcommand{\rec}{$\text{R\'ecoltes et Semailles}$}
\newcommand{\no}{n$^\circ$}
\hfuzz = 100pt

% NOTE and SUBNOTE FORMATTING
\usepackage{titlesec}
\usepackage[dotinlabels]{titletoc}

% define `note'
\titleclass{\nnote}{straight}[\section]
\newcounter{nnote}
\renewcommand{\thennote}{\thesection.\arabic{nnote}}
\titlespacing*{\nnote}{0pt}{3.5ex plus 1ex minus .2ex}{1ex plus .2ex}
\titleformat{\nnote}[runin]{\bfseries }{\bfseries Note }{0pt}{}[]

% define `subnote'
\titleclass{\subnote}{straight}[\section]
\newcounter{subnote}
\renewcommand{\thesubnote}{\thesection.\arabic{subnote}}
\titlespacing*{\subnote}{0pt}{3.5ex plus 1ex minus .2ex}{1ex plus .2ex}
\titleformat{\subnote}[runin]{\bfseries }{\bfseries Note }{0pt}{}[]

% the first optional arg sets the size of the indentation in the TOC
\titlecontents{nnote}[9em]
{}{Note }{}{\titlerule*[1pc]{.}\contentspage}

% the first optional arg sets the size of the indentation in the TOC
\titlecontents{subnote}[11em]
{}{Note }{}{\titlerule*[1pc]{.}\contentspage}

\newtheorem{remark}{Remark}
\theoremstyle{nonumberbreak}
\newtheorem{remarknn}{Remark}

\begin{document}
% print table of contents with notes and subnotes
 \setcounter{tocdepth}{5}
 \setcounter{secnumdepth}{5}
 \startcontents[chapters]
 \printcontents[chapters]{}{0}{}

\setcounter{chapter}{14}
\end{comment}

\chapter{C) High Society}

\section{VII The Colloquium - or Mebkhout's sheaves and Perversity}

\subsection{The Iniquity or a feeling of return}

\nnote{75}

\subsection{The colloquium}

\nnote{75'}

\subsection{The prestidigitator}

\nnote{!75''}

\subsection{Perversity}

\nnote{76}

\subsection{Thumbs up!}

\nnote{77}

\subsection{The robe of the emperor of China}

\nnote{77'}

\subsection{Encounters from beyond the grave}

\nnote{78}

\subnote{78$_1$}

\subsection{The victim - or the two silences}

\nnote{78'}

\subnote{78$_1$'}

\subnote{78$_2$'}

\subsection{The Boss}

\nnote{!78''}

\subsection{My friends}

\nnote{79}

\subsection{The tome and high society - or moon and green cheese...}

\nnote{80}

\section{VIII The Student - a.k.a. the Boss}

\subsection{Thesis on credit and risk-proof insurance}

\nnote{81}

\subnote{81$_1$}

\subnote{81$_2$}

\subnote{81$_3$}

\subsection{The right references}

\nnote{82}

\subsection{The jest, or the ``weight complexes"}

\nnote{83}

\section{IX My students}

\subsection{Silence}

\nnote{84}

\subnote{84$_1$}

\subsection{Solidarity}

\nnote{85}

\subsection{Mystification}

\nnote{!85'}

\subnote{85$_1$}

\subnote{85$_2$}

\subsection{The defunct}

\nnote{86}

\subsection{The massacre}

\nnote{87}

\subnote{87$_1$} (May 31) This closing talk, probably one of the most interesting and substantial, along with  the opening talk, were visibly not lost on everyone, as I now realize upon learning about MacPherson's paper ``Chern classes for singular algebraic varieties" (Chern classes\marginpar{p. 362} for singular algebraic varieties, Annals of Math. (2) 100, 1974, p. 423-432)\todo{cite} (submitted in April 1973). There, I found under the name of ``Deligne-Grothendieck conjecture" one of the main conjectures which I had introduced in said talk in the context of schemes. The conjecture was reformulated by MacPherson in the transcendental context of algebraic varieties over the complex numbers, where the Chow ring is replaced by the homology group. Deligne had learned about this conjecture\footnote{(*) In a slightly different form admittedly, see the rest of the note dated May 31.}(*) during my 1966 talk, the same year that he joined the seminar and started familiarizing himself with the language of schemes and cohomological methods (see the note ``One of a kind", n$^o$ 67'). It is nonetheless kind to have included me in the name of this conjecture - a few years later this would have been out of the question...

(June 6) I would like to use this occasion to explicitly write down the conjecture which I had announced in the context of schemes, while also probably hinting at the obvious analogue in the complex analytic (or even rigid analytic) context. I viewed it as a ``Riemann-Roch"-type theorem, albeit with discrete coefficients instead of coherent coefficients. (Zoghman Mebkhout also told me that his viewpoint on $\mathcal{D}$-modules should enable one to consider both Riemann-Roch theorems as contained in a single crystalline Riemann-Roch theorem, which in zero characteristic would constitute the natural synthesis of the two Riemann-Roch theorems that I have introduced in mathematics, the first in 1957 and the second in 1966). Start by fixing a coefficient ring $\Lambda$ (not necessarily commutative, but noetherian for simplicity and furthermore with torsion prime to the characteristic of the schemes under considerations, to meet the needs of \'etale cohomology...). Given a scheme $X$, write 
$$ \text{K}_{\cdot}(X, \Lambda) $$
to denote the Grothendieck group associated to constructible \'etale sheaves of $\Lambda$-modules. This group is functorial in $X$ with respect to the functors $\textbf{R}f_!$ when restricting our attention to separated scheme morphisms of finite type. For regular $X$, I claimed that there exists a canonical group homomorphism, playing the role of the ``Chern character"\marginpar{p. 363} in the coherent Riemann-Roch theorem,
\begin{equation}\label{chern1.1} 
\text{ch}_X: \text{K}_{\cdot}(X, \Lambda) \to \text{A}(X) \otimes_{\mathbb{Z}} \text{K}_{\cdot}(\Lambda),
\end{equation}
where $\text{A}(X)$ is the Chow ring of $X$ and $\text{K}_{\cdot}(\Lambda)$ is the Grothendieck group associated to $\Lambda$-modules of finite type. This homomorphism was supposed to be completely determined by the presence of a ``discrete Riemann-Roch formula" for \textbf{proper} morphisms between regular schemes $f: X \to Y$, whose form is analogous to the Riemann-Roch formula in the coherent context, except that the Todd ``multiplier" is replaced by the total relative Chern class:
$$ \text{ch}_Y(f_!(x)) = f_*(\text{ch}(x)c(f)), $$
where $c(f)$ denotes the total Chern class of $f$. It isn't hard to see that, in a context where one has access to a resolution of singularities theorem in the strong sense of Hironaka's, this Riemann-Roch formula does indeed uniquely determine the $\text{ch}_X$'s. 

Of course, we are supposing that we are working in a context in which there is a notion of Chow ring. (I am not aware of any attempt to develop a theory of Chow rings for regular schemes that are not of finite type over a field.) Otherwise, we could also work with the graded ring associated to the usual ``Grothendieck ring" $K^0(X)$ in the coherent context, equipped with the usual filtration (see SGA 6); or we could replace $\text{A}(X)$ by the even $\ell$-adic cohomology ring, given by the direct sum $\bigoplus_i \text{H}^{2i}(X, \underline{\mathbb{Z}}_\ell(i)$. This comes with the added baggage of an artificial parameter $\ell$ and produces coarser, ``purely numerical" formulas, whereas the Chow ring has the added charm of having a continuous structure which is destroyed upon passing to cohomology.

Already in the case where $X$ is a smooth algebraic curve over an algebraically closed field, computing $\text{ch}_X$ involves studying delicate Artin-Serre-Swan-type local invariants. This hints at the depth of the general conjecture, whose pursuit would involve understanding the analogues of these invariants in higher dimensions. 

\textbf{Remark.}  Writing $\text{K}^{\cdot}(X, \Lambda)$ to denote the ``Grothendieck ring" associated to constructible complexes of \'etale sheaves of $\Lambda$-modules of finite Tor-dimension (which acts on $\text{K}_{\cdot}(X, \Lambda)$ when $\Lambda$ is commutative...), we also expect to have a homomorphism
\begin{equation}\label{chern1.2} 
\text{ch}_X: \text{K}^{\cdot}(X, \Lambda) \to \text{A}(X) \otimes_{\mathbb{Z}} \text{K}^\cdot(\Lambda), 
\end{equation}
giving rise (mutatis mutandis) to the same Riemann-Roch formula.

\marginpar{p. 364}Now, let $\text{Cons}(X)$ denote the ring of constructible integral-valued functions on $X$. We can define more or less tautologically canonical homomorphisms:
\begin{equation}\label{chern2.1} 
\text{K}_{\cdot}(X, \Lambda) \to \text{Cons}(X) \otimes_{\mathbb{Z}} \text{K}_\cdot(\Lambda), \text{ and} 
\end{equation}
\begin{equation}\label{chern2.2} 
\text{K}^{\cdot}(X, \Lambda) \to \text{Cons}(X) \otimes_{\mathbb{Z}} \text{K}^\cdot(\Lambda). 
\end{equation}
If we restrict our attention to schemes in \textbf{zero characteristic}, then (using Euler-Poincar\'e characteristics with proper support) we see that the group $\text{Cons}(X)$ is a \underline{covariant} functor with respect to morphisms of finite-type between noetherian schemes (in addition to being contravariant as a ring-functor, and this independently of the characteristic), compatibly with the above tautological morphisms. (This corresponds to the ``well-known" fact, which I don't recall being proven in the oral seminar SGA 5, that in \textbf{zero characteristic}, a locally constant sheaf of $\Lambda$-modules $F$ over an algebraic scheme\todo{algebraic space?} $X$ has image $d_\chi(X)$ under the map 
$$ f_!: \text{K}^{\cdot}(X, \Lambda) \to \text{K}^{\cdot}(e, \Lambda) \simeq K^\cdot(\Lambda), $$
where $d$ is the rank of $F$, $e = \Spec k$, and $k$ is the algebraically closed base field...)\todo{figure out what $d_\chi(X)$ means} This suggests that the Chern homomorphisms \ref{chern1.1} and \ref{chern1.2} should be deducible from the tautological homomorphisms \ref{chern2.1} and \ref{chern2.2} upon composition with a ``universal" Chern homomorphism (independent of the choice of coefficient ring $\Lambda$)
$$ \text{ch}_X: \text{Cons}(X) \to \text{A}(X), $$
in such a way that the two versions of the Riemann-Roch formula ``with $\Lambda$-coefficients" appear as formally enclosed in a RR formula at the level of constructible functions, with the latter always taking the same form.

When working with schemes over a fixed base field (this time in arbitrary characteristic), or more generally over a fixed \textbf{regular} base scheme $S$ (such as for instance $S = \Spec \mathbb{Z}$), the form of the Riemann-Roch formula closest to the traditional notation (in the coherent context, familiar since 1957) can be obtained by introducing the product
\begin{equation}\label{relchern} 
\text{ch}_X(x)c(X/S) = c_{X/S}(x)
\end{equation}
(where $x$ is in either $\text{K}_\cdot(X, \Lambda)$ or in $\text{K}^\cdot(X, \Lambda)$), which could be called the \marginpar{p. 365}\textbf{Chern class of $x$ relative to the base $S$}. When $x$ is the unit of $\text{K}^\cdot(X, \Lambda)$, i.e. the class of the constant sheaf with value $\Lambda$, we recover the image of the relative total Chern class of $X$ with respect to $S$ under the canonical homomorphism $\text{A}(X) \to \text{A}(X) \otimes \text{K}^\cdot(\Lambda)$. With this notation in place, the RR formula becomes equivalent to the fact that the formation of these relative Chern classes
\begin{equation}\label{chern3.1}
 c_{X/S}: \text{K}_\cdot(X, \Lambda) \to \text{A}(X) \otimes \text{K}(\Lambda) 
 \end{equation}
for fixed $S$ and varying regular scheme $X$ of finite type over $S$ is functorial with respect to proper morphisms, and likewise for the variant \ref{chern1.2}\todo{make sure the right equation is being referenced}. In zero characteristic, this can be reduced to the functoriality (with respect to proper morphisms) of the corresponding map
\begin{equation}\label{abschern} 
c_{X/S}: \text{Const}(X) \to \text{A}(X). 
\end{equation}

It is under this form that the existence and uniqueness of an absolute ``Chern class" \ref{abschern} in the case $S = \Spec \mathbb{C}$ is conjectured in the work of MacPherson, the relevant conditions being (here as in the general case in zero characteristic) (a) functoriality of \ref{abschern} with respect to proper morphisms and (b) the identity $c_{X/S}(1) = c(X/S)$ (in this case, the ``absolute" total Chern class). The form of the conjecture presented and proven by MacPherson differs from my initial conjecture in two ways. The first is a ``negative", namely that he is not working in the Chow ring, but rather in the integral cohomology ring, or more precisely the integral homology group, defined by transcendental methods. The other is a ``positive" - and this is possibly where Deligne contributed to my initial conjecture (unless this contribution is due to MacPhersson\footnote{(*) (March 1985) That was indeed the case, see note n$^o$164 referenced in the previous footnote.}(*)). Namely, the observation is that in order to prove existence and uniqueness for \ref{abschern}, we don't need to restrict ourselves to regular schemes $X$, as long as we replace $\text{A}(X)$ by the integral homology group. As such, it is probable that the same holds in the general case, if we write $\text{A}(X)$ (or better $\text{A}(X)$) to denote the \textbf{Chow group} (which is no longer a ring in general) of a noetherian scheme $X$. Said differently: while the heuristic definition of the invariants $\text{ch}_X(x)$ (for $x$ in either $\text{K}_\cdot(X, \Lambda)$ or in $\text{K}^\cdot(X, \Lambda)$) uses in an essential way the hypothesis that the ambient scheme is regular, upon multiplying by the ``multiplier" $c(X/S)$ (for $X$ of finite type over a fixed regular scheme $S$), the product obtained in \cite{relchern} seems to still make sense regardless of any regularity hypothesis on\marginpar{p. 366}X, as an element of the tensor product 
$$ \text{A}_\cdot(X) \otimes K_\cdot(\Lambda) \text{  or } \text{A}_\cdot(X) \otimes K^\cdot(\Lambda),  $$
where $\text{A}_\cdot(X)$ denotes the Chow group of $X$. The spirit of MacPherson's proof (which does not use resolution of singularities) seems to suggest that it it possible to exhibit a ``constructive" and explicit construction of the homomorphism \ref{chern3.1}, by ``making do" with the singularities of $X$ as they are, as well as with the singularities of the sheaf of coefficients $F$ (whose class is $x$), so as to ``collect" a cycle on $X$ with coefficients in $\text{K}_\cdot(\Lambda)$. This would fit in the circle of ideas which I had introduced in 1957 with the coherent Riemann-Roch theorem, where I notably computed self-intersections, without quite ``moving around" the cycle under consideration. An initial obvious step (obtained by immersing $X$ in an $S$-scheme) would be to reduce to the case where $X$ is a closed subscheme of a regular $S$-scheme\todo{make sure this is what Grothendieck meant}.

The idea that it should be possible to develop a \textbf{singular} (coherent) Riemann-Roch theorem was already familiar to me, although I couldn't say for how long, but I never seriously put it to the test. It was in part this idea (other than the analogy with the ``cohomology, homology, cap-product" formalism) which had led me to systematically introduce $\text{K}_\cdot(X)$, $\text{K}^\cdot(X)$, $\text{A}_\cdot(X)$, and $\text{A}^\cdot(X)$  in SGA 6 (in 1966/67), instead of choosing to work only with $\text{K}^\cdot(X)$. I can't remember if I had thought about something along those lines in the SGA 5 seminar in 1966, or if I made mention of it in my talk. As my handwritten notes have disappeared (perhaps while moving?), I may never know...

(June 7) In reading through MacPherson's article, I was stricken by the fact that the word ``Riemann-Roch" is never used - this is also the reason why I did not immediately recognize the conjecture which I had made in the SGA 5 seminar in 1966, the latter having always been (and still is) a ``Riemann-Roch"-type theorem in my view. It seems that at the time of writing his article, MacPherson did not notice this evident filiation. I am guessing that the reason behind this is that Deligne, who circulated this conjecture in the form he liked best after my departure, took care to ``erase", insofar as possible, the evident filiation with the Riemann-Roch-Grothendieck theorem. I think I understand his motivation behind this. On the one hand, this weakens the link between the conjecture and myself, making more plausible\marginpar{p. 367}  the name currently under circulation, ``Deligne-Grothendieck conjecture". (N.B. I ignore where this conjecture is currently circulating in the scheme context, and is so, I would be curious to know under which name). But the deeper reason seems to be his obsession with denying and destructing, to the extent possible, the fundamental unity of my work and mathematical vision\footnote{(*) Compare with the comment made in the note ``The remains" (n$^o$88) regarding the profound significance of the SGA 4$\frac{1}{2}$ operation, similarly aiming to break out into an amorphous collection of ``technical digressions" the profound unity of my work on \'etale cohomology via the ``violent insertion" of the outlandish text SGA 4$\frac{1}{2}$ between the two indissoluble parts SGA 4 and SGA 5 in which this work is carried out.}(*). This is a striking example of the way in which a fixed idea entirely foreign to any mathematical motivation can obscure (if not downright seal) what I have called the ``sane mathematical instinct" of a mathematician whose abilities are nonetheless exceptional. Such a mathematical instinct would not fail to perceive the analogy between the two statements, one ``continuous" and the other ``discrete", of a ``single" Riemann-Roch theorem, an analogy which I had of course spelled out during my talk. As I indicated yesterday, this filiation will probably be confirmed in the near future by a formal statement (conjectured by Zoghman Mebkhout), at least in the complex analytic context, enabling us to deduce both theorems from a common result. Clearly, given Deligne's ``grave-digging" attitude towards the Riemann-Roch theorem\footnote{(**) This attitude towards the Riemann-Roch-Grothendieck theorem are manifested particularly clearly in the ``Funeral Rite"; see the note ``The Funeral Rite (1) - or the compliments", n$^o$104.

}(**), he was not positioned to discover the common statement connecting them in the analytic context, nor to think to look for an analogous statement in the general context of schemes. In the same way, this attitude prevented him from unearthing the fruitful viewpoint on $\mathcal{D}$-modules in studying the cohomology theory of algebraic varieties, which followed too naturally from a circle of ideas that needed to be buried; neither was he able to recognize Mebkhout's fertile work for years on end, a work which had been successful where he had failed. 

\subnote{87$_2$} (May 31) That was also the year when I gave my Bourbaki talk on the rationality of $L$-functions, heuristically using Verdier's result\todo{figure out a citation} (mostly the expected form of local terms in the case at hand) thirteen years before Illusie proved it upon Deligne's request. I seem to recall that the super-general\marginpar{p. 368} formula Verdier showed me, which came as a surprise, was proven using the ``six operations" formalism in a few lines - it was the kind of formula for which writing it was (nearly) proving it! If there was any ``difficulty", it could only have been regarding the verification of one or two compatibility conditions\footnote{(*) (June 6) It furthermore seems that the initial proof of the Lefschetz-Verdier formula via the biduality theorem (now promoted to ``Deligne's theorem") depended on a hypothesis regarding the resolution of singularities, which Deligne was able to circumvent in the case of schemes of finite type over a field. This constituted a good opportunity to turn the situation to their advantage, by giving the impression that SGA 5 is relying to the (sic) seminar SGA 4$\frac{1}{2}$ which ``precedes" it (and which was indeed published earlier!).}(*). Furthermore, both Illusie and Deligne knew full well that the proofs I had given during the seminar regarding various explicit trace formulas \textbf{were complete}, and that they did not depend in any way upon Verdier's general formula, the latter having only played the role of a ``catalyst" inciting me to formulate and prove trace formulas in the most general possible settings. Both showed a patent lack of good faith at this occasion. In the case of Deligne, this was already clear to me while writing the note ``Clean slate" (n$^o$67) - but it was probably not so clear to an uninformed reader, or to an informed reader who renounced the use of their sane faculties.

(June 6) As for Illusie, he fully played along in trying to muddy the waters so as to project the appearance of a hyper-technical oral seminar which did not even provide complete proofs of all the results, and notably of the trace formulas. These were nonetheless proven (for the first time) in 1965/66, and this was also the place where Deligne had the privilege to learn about them and about all of the delicate machinery that accompanies them\footnote{(**) In the second paragraph of the Introduction to the volume titled SGA 5, Illusie presents the three expos\'es III, III B, and XII, on the Lefschetz formula in \'etale cohomology, as the ``heart of the seminar", even though it is also said in the introduction to expos\'e III B that (contrary to reality) ``this expos\'e does not correspond to an oral talk from the seminar", while in the introductions to both expos\'es III and III B he tries to give the impression that they rely on SGA 4$\frac{1}{2}$, with expos\'e III being presented as ``conjectural"!! In reality, the totality of the seminar SGA 5 was technically independent of the expos\'e III (Lefschetz-Verdier formula), which acted as heuristic motivation, and expos\'e III B was nothing but the ``hole" (expos\'e XI) created by Bucur's moving, which turned into a welcomed excuse for this additional dismemberment.

In order to add to the impression of a seminar of ``technical digressions" (whispered to him by his friend Deligne), Illusie took care to remove the introductory expos\'e in which I had brushed a preliminary outline of the principal great themes which were to be developed during the seminar - an outline in which trace formulas only account for a small (which is of particular importance due to their arithmetic implications, towards the Weil conjectures). For an overview of these ``great themes", see the sub-note n$^o 87_5$ later.}(**).

\marginpar{p. 369}This reminds me that I naturally had taken the time to prove the Lefschetz-Verdier formula during the seminar - this was the least I could do, and it provided a particularly striking application of the local-global duality formalism which I was setting out to develop. A question recently occurred to me as to why on earth Deligne and Illusie had chosen to cover the theorem of their good friend Verdier, whose name the result carried, similarly to the derived and triangulated categories which he never took the time to develop in full (or at least to make such developments available to the public), when there were ten or so other expos\'es begging to be written by one of my students, so that they were not short of choices when it came to naming the technical ``obstacle" to the publication of SGA 5. Therein lies a \textbf{challenge} of sorts in the absurdity displayed (or in the collective cynicism exhibited by my ex-cohomology students, whom I consider to have been unified in this operation-massacre) which reminds me of the ``weight-complex" brilliantly invented by Verdier in the preceding year (see the note with the same name, n$^o$83), as well as (in the iniquitous register) the choice by Deligne to call ``perverse" the sheaves which should have been called ``Mebkhout sheaves" (see the note ``Perversity", n$^o$76). I see all of these inventions as acts of dominance and contempts towards the mathematical community as a whole - as well as a \textbf{bet} which had visibly been won until the unexpected reappearance of the deceased, almost appearing as the only person awake amidst a community in slumber...

\subnote{87$_3$} (June 5) In light of this summary of the massacre, one can fully appreciate the following declaration of Illusie, written on line 2 of his introduction to the volume titled SGA 5:

\begin{quote}
``Compared to the primitive version, the only important changes regard expos\'e II [generic K\"unneth formulas], which is not reproduced, as well as expos\'e III [Lefschetz-Verdier formula], which has been entirely rewritten and augmented by an appendix numbered III B\footnote{(*) With this appendix presented as belonging to the ``heart of the seminar"! (see preceding footnote.)}(*). Except for some detail\marginpar{p. 370} modifications and added footnotes, the other expos\'es were left \textbf{tel quel}" (emphasis mine).
\end{quote}

Here as elsewhere, Illusie complaisantly echoes one of his ineffable friend's charades, namely that the existence of SGA 4$\frac{1}{2}$ ``will enable the publication of SGA 5 \textbf{tel quel} in the near future" (see the note ``Clean slate", n$^o$67). Illusie tries his best throughout his expos\'es and introductions to add credit to this imposture (the fact that SGA 5, where he and his friend first learned about their trade, depends on the pirate-volume SGA 4$\frac{1}{2}$ which consists of bits and pieces gleaned or pillaged over the course of the twelve following years) by generously sprinkling references to SGA 4$\frac{1}{2}$ at every corner of the volume...

The final word comes from Deligne (as is appropriate), who wrote to me a month ago (on May 3rd) in response to a laconic request for information (for more on this topic see the beginning of the note ``The Funeral Rites", n$^o$70):

\begin{quote}
``In summary, the fact that seven years had elapsed since you last did mathematics [?!] at the time when SGA 4$\frac{1}{2}$ appeared corresponds [?] to the long delay to which the edition of SGA 5 was subjected, \textbf{as it was too incomplete to be usefully published tel quel.}

In the hope that you will approve of these explanations."
\end{quote}

If I did not ``approve" of these explanations, they will have at least edified me...

\subnote{87$_4$} (June 6) Now might be a good time to list the principal themes that were developed during the oral seminar, of which the published text only paints a disjointed picture. 

I) Local aspects of the theory of duality, whose key technical ingredient is (as in the coherent case) the theorem of biduality (together with a theorem of ``cohomological purity"). I am under the impression that the geometric meaning of the latter theorem as a form of local Poincar\'e duality has been entirely forgotten by my ex-students\footnote{(*) Upon verification, its turns out that this geometric interpretation was at least preserved in Illusie's write-up.}(*), even though I had explained it during the oral seminar.

II) Trace formulas, including ``non-commutative" trace formulas which were subtler than the usual trace formulas (where both sides are integers, or more generally elements of the coefficient ring - such as $\mathbb{Z}/n\mathbb{Z}$, $\mathbb{Z}_l$, or even $\mathbb{Q}_l$) in that they take values in the \marginpar{p. 371}group algebra of a finite group acting on the scheme under consideration, with coefficients in a suitable ring (such as the ones listed in the previous parenthesis). This generalization came naturally from the fact that even in the usual Lefschetz-type formulas for ``twised" sheaves of coefficients, one was led to replace the initial scheme with a Galois covering (possibly ramified) so as to ``untwist" the coefficients, while keeping track of a Galois group action. In this way, ``Nielsen-Wecken"-type formulas are naturally introduced into the schematic context.

III) Euler-Poincar\'e formulas. This consisted of a detailed study of an ``absolute" formula for algebraic curves using Serre-Swan modules (generalizing the case of tamely ramified coefficients which gave rise to the more naive Ogg-Chafar\'evitch-Grothendieck formula), as well as new and profound conjectures regarding a ``discrete" Riemann-Roch formula, one of which reappeared seven years later in hybrid form under the name of ``Deligne-Grothendieck conjecture" and was proven by MacPherson via transcendental methods (see note n$^{o}87_1$).

The comments which I had inevitably made regarding the profound connections between these two themes (Lefschetz formulas and Euler-Poincar\'e formulas) have also disappeared without a trace. (As was usual, I had left all of my handwritten notes to the sic volunteer writers, so that I do not have access to any written trace of the oral seminar, even though I once naturally had a complete if at times succinct set of handwritten notes.)

IV) Detailed formalism of the homology and cohomology classes associated to a cycle, following naturally from the general duality formalism and from the key idea of working with cohomology ``with supports" in the cycle under consideration, using the theorems of cohomological purity.

V) Finiteness theorems (including generic finiteness theorems) and generic K\"unneth theorems for cohomology with arbitrary support.

The seminar also developed a technique for passing from torsion coefficients to $\ell$-adic coefficients (expos\'es V and VI). This was the most technical part of the seminar, with the latter generally worked with torsion coefficients, with the possibility to ``take the limit" in order to deduce the corresponding $l$-adic results. This viewpoint was a temporary\marginpar{p. 372} trade-off, awaiting for Jouanolou's thesis (unpublished to this day) which was to develop the appropriate formalism to work directly in the $\ell$-adic settings.

In this list of the main ``themes", I am not including the computations for certain classical schemes and the cohomological theory of Chern classes, which Illusie highlights in his introduction as ``one of the most interesting themes" of the seminar. As the program was already packed, I did not think it was necessary to spend time on these computations and this construction during the oral seminar, given that one needed only follow, almost verbatim, the arguments which I had produced ten years earlier in the context of Chow rings for the purposes of the Riemann-Roch theorem. It was clear on the other hand that these ideas needed to be included in the written seminar, so as to provide a useful reference to the user of \'etale cohomology.  Jouanolou had taken charge of this work (expos\'e VIII), and instead of seeing it as a service done for the mathematical community as well as an opportunity to learn essential basic techniques for his own use, he must have viewed it as a chore, since the write-up lagged for years\footnote{(*) (June 12) After reading the expos\'e in question, I convinced myself of a perfect complicity between Jouanolou and my other cohomology students.}(*). The same probably held true for his thesis, which was to forever remain a phantom reference like Verdier's... The ``taking the limit" section should not be included as one of the ``main themes" of the seminar either, in that it does not correspond to a particular geometric idea. Rather, it reflects a technical complication particular to the context of \'etale cohomology (distinguishing it from the transcendental contexts), namely the fact that the main theorems in \'etale cohomology pertain in the first place to \textbf{torsion} coefficients (prime to the residual characteristics), and that in order to obtain a theory corresponding to rings of coefficients in zero characteristic (as needed for the Weil conjectures), one needs to take the limit over the rings of coefficients $\mathbb{Z}/{\ell^n\mathbb{Z}}$ in order to obtain ``$\ell$-adic" results.

Having described all of this, the only main theme of the oral seminar which seems to appear in complete form in the published text is theme I. Themes IV and V have downright disappeared; they have been incorporated into SGA 4$\frac{1}{2}$ with the added benefit of being able to make numerous references to the latter, so as to give the impression that SGA 5 depends on a text of Deligne framed as an earlier work. Themes II and III appear in the published volume in a mutilated form, always maintaining the same counterfeited appearance of a dependency on the text SGA 4$\frac{1}{2}$ (which in reality was issued in its entirety from the mother-seminars SGA 4 and SGA 5).

\subsection{The remains}

\nnote{88} \marginpar{p. 373}(May 16) The seminars SGA 4 and SGA 5 (which in my eyes constitute a \textbf{unique} ``seminar") taken together develop from scratch the \textbf{language} of topoi, as a  powerful instrument of synthesis and discovery, as well as the fully sharpened \textbf{tool} of perfect efficacy that is \'etale cohomology - whose essential formal properties were henceforth better understood than even the cohomology theory of ordinary spaces\footnote{(*) And that is so even if we restrict ourselves to the spaces closest to ``varieties", such as spaces admitting a triangulation.}(*). Among the projects that I carried out in full, this work represents the most profound and innovative contribution that I have made to mathematics. At the same time - and without intending to, in the sense that things simply followed their self-evident natural course at every moment - this work represents the vastest technical ``tour de force" that I have accomplished in my mathematical career\footnote{(**) Some of the difficult or unexpected results were obtained by others (Artin, Verdier, Giraud, Deligne), and certain parts of my work were carried out in collaboration with others. This doesn't affect (at least in my mind) how strongly I feel about the position of this work in the totality of my mathematical work. I believe I come back to this point in more details in an appendix to the ``Esquisse Th\'ematique", dotting the i's where it had visibly become necessary.}(**). In my eyes, these two seminars are inextricably linked. They represent in their unity both the \textbf{vision} and its \textbf{tools} - namely, topoi and the complete formalism of \'etale cohomology.

Even though the vision remains rejected to this day, the tools have in their twenty years of existence brought about a profound renewal of algebraic geometry in its most fascinating aspect - namely, its ``arithmetic" aspect, apprehended by means of an intuition as well as a conceptual and technical toolkit of a ``geometric" nature. 

It was surely not the intention to suggest that his cohomological ``digest" was \textbf{anterior} to SGA 5 which motivated Deligne to deceivingly call it SGA 4$\frac{1}{2}$ - after all, he might as well have called it SGA 3$\frac{1}{2}$! I see the ``SGA 4$\frac{1}{2}$ operation" as an attempt to frame the work when all of his own work is issued (that very work from which he cannot detach himself!) - a work whose unity and depth is clearly visible in SGA 4 and (the true) SGA 5, as a \textbf{divided} entity (just as he himself is divided...), \textbf{cut in half} by the violent insertion of a foreign and disdainful text; the latter pretends to be the living heart, the \marginpar{p. 374}quintessence of a school of thought and of a vision in which it in truth played no part\footnote{(*) This school of thought had reached full maturity, both in terms of key ideas and essential results, before Deligne entered the stage as a young man wishing to learn algebraic geometry and cohomological methods in my contact, between 1965 and 1969.

(May 30) See on this topic the note ``One of a kind", n$^o$67'.}(*), while the surrounding two ``quarters" appear as vaguely grotesque appendices of sorts, a hodgepodge of ``digressions" and ``technical complements" to a work framed as central and essential, written by Deligne, with my name being graciously included (prior to its final burial) in the list of ``collaborators"\footnote{(**) See the notes ``Green light", ``The reversal", n$^o$s 68, 68'.}(**).

As chance would have it, these ``remains left at their mercy", this ``unfortunate seminar" always brushed to the side by its ``writers" and left in the hands of my cohomology students after my departure, was \textbf{not just any} part of my masterwork! It was neither SGA 1 and SGA 2 (where I quietly developed the tools which I was yet to discover would constitute the two indispensable technical auxiliaries to the ``take-off" of my main work to come), nor SGA 3 (where my contribution consisted chiefly in a series of scales and arpeggios - sometimes difficult - meant to refine the theory of schemes ``in all directions"), nor SGA 6 (in which I systematically developed my ten-year old ideas around a Riemann-Roch theorem and the formalism of intersection theory), nor SGA 7 (which, through the logic internal to the reflection, follows from the wielding of a single central tool, namely the mastery of cohomology). It truly was the \textbf{masterpiece} of my work, whose write-up had remained incomplete (by their own fault...), which I had left, at least in part, in the hand of my cohomology students. They decided to destroy this masterpiece of my work, appropriating the pieces while forgetting the unity which gives them their meaning and beauty, as well as their creative virtue (90).

It is also not a coincidence if, equipped with a motley of tools whilst denying the spirit and vision that brought them into being, they were unable to discern the innovative work which was being reborn in front of their indifferent and disdainful gaze. Neither is it an accident that after six years, when the new tool was at last accessed by Deligne, there was a unanimous agreement to bury the person who had created it in solitude - namely, Zoghman Mebkhout,\marginpar{p. 375} the posthumous student of the disavowed master! Finally, it is not by chance that following Deligne's initial impetus (which led him over the span of a few years toward the start of a new version of Hodge theory and the proof of the Weil conjectures), and despite his astounding abilities as well as my cohomology students' great means, I nowadays witness a ``morose stagnation" in their domain, despite its prodigious richness and the multitude of things yet to be done. This is not surprising in light of the fact that for nearly fifteen years the main source of inspiration as well as some of the ``great open problems"\footnote{(*) This ``main source of inspiration" is of course the ``yoga of motives". It has been active in Deligne alone, who kept it for his only ``benefit" in a restricted form lacking a large part of its strength, in that he rejected some of the essential aspects of this yoga. Among the ``great open problems" inspired by this yoga, many of which were ignored or discretely discredited, I can state right away (however much of an outsider I may currently be) the standard conjectures, as well as the development of the ``six functor formalism" for all of the usual kinds of coefficients, more or less close to ``motives" themselves (the latter playing the role of ``universal" coefficients, giving rise to every other). Compare with my comments on this topic in the note ``My orphans", n$^o$46.}(*), which have been present in the background and relevant at every step of the way, have been carefully circumvented and hidden away, in line with the treatment of the messengers of the person to be buried.

\subsection{... and the body}

\nnote{89} (May 17) I view the philosophy and viewpoint which lived within me and which I thought I had communicated as a breathing, healthy, and harmonious body, moved by the power of renewal of all living things: the power to conceive and to engender. This body has now turned into \textbf{remains}, divided between several people - one of the limbs was duly embalmed and now features as somebody's trophy; another has been skinned and now serves as a club or boomerang for somebody else; yet another, who knows, may have been used for some home cooking (why not!) - everything else may as well be left to rot in a landfill...

Such is the scene that was eventually revealed to me, admittedly presented in colorful terms, but which I nonetheless believe accurately express a certain reality. The club may occasionally be used to fracture a few skulls\footnote{(*)}(*) - but none of these disparate pieces, be they trophies, clubs, or homemade soups, will ever inherit the simple and obvious power of the living body: that of the loving embrace at the origin of new beings...

(May 18) This picture of the living body whose ``remains" were scattered to the four winds must have formed within me during this past week.\marginpar{p. 376} The quirky form in which it materialized itself through my typewriter should not be taken as an indication that this image is in any way a (slightly morbid) \textbf{invention} or as a burlesque improvisation spontaneously generated by the needs of a speech. The image is expressing a \textbf{reality} which was profoundly felt at the time when it took on a material form through my writing. I must have already grappled with this reality in bits and pieces during the fourteen years that elapsed since my ``departure", or perhaps even earlier than that. These fragments of information were at first registered at an informal level, while my distracted attention was absorbed in something else; these pieces were nonetheless all congruent, and they must have assembled into a coherent image at some deeper level, even though I was too busy to pay attention to it. This image grew considerably richer and more precise over the course of the reflection which has been taking place since late March, beginning six or seven weeks ago. More precisely, the disparate pieces of information slowly assembled into \textbf{another} image which appeared at the more superficial level of the examining and probing mind through a process which may appear to be independent from the presence of the first image, lodged in a deeper layer. This conscious process culminated six days ago in my sudden vision of the ``slaughter" that took place - I could sense the ``wind", the ``odor" of \textbf{violence} for what is I believe the first time in the entire reflection\footnote{(*)}(*). This was also the time when the awareness of a living and harmonious body which had been ``slaughtered" must have started rising to the surface of my consciousness; whilst the deeper and diffuse image began surfacing as well, adding to the image in the making a carnal dimension which thought alone cannot produce.

This ``carnal" aspect manifested itself once more in a dream yesterday night - and it is under the impulse of this dream that I am choosing to return to the lines I wrote yesterday. In this dream, I had rather deep wounds at several locations of my body. The most glaring were cuts in my lips and inside my mouth, which were bleeding profusely as I rinsed my (blood-reddened) mouth abundantly in front of a mirror. There were also cuts in my stomach, bleeding just as profusely. One of them was particularly severe, and blood was flowing\marginpar{p. 377} as if out of an artery (the Dreamer did not bother with anatomical accuracy). The thought even occurred to me that I may be done for if I kept bleeding at this rate; I applied my hand to the wound and huddled so as to stop the bleeding. The maneuver was successful: the bleeding slowed down until the blood clotted and formed a large crust. Later, I carefully lifted the crust to find that a delicate cicatrization had already begun. I also had a cut at one of my fingers, which was already wrapped in a large bandage...

I do not intend to launch into a more careful and detailed description of this dream, nor to thoroughly examine it (here or elsewhere). What the dream ``as is" already reveals to me with striking power is that the ``body" about which I was speaking yesterday as something detached from my being, like a child I would have conceived and who later left off to trace his or her own path in the world, had in fact been an intimate piece of my person: it is \textbf{my} body, in flesh and blood, endowed with a life force allowing it to survive and recover from profound wounds. And my body is in turn, doubtlessly, the thing in the world to which I am most profoundly and inextricably linked...

The Dreamer disagreed with the image I had of the ``slaughter" and of the sharing of the remains. The image I had in mind depicted a set of intentions and dispositions in \textbf{others} which I had sharply perceived, rather than the way in which I myself experienced this aggression, this mutilation to which I was tied through the thing I held close. The Dreamer allowed me to realize the extent to which I was still tied to this mutilation. This aligns with what I perceived (less strongly) in the note ``What goes around comes around - or spilling the beans" (n$^o$73), where I try to articulate the feeling of a ``profound link between the conceiver of something and the thing itself" which had appeared during that day's reflection. Prior to the reflection of April 30$^{th}$ (just three weeks ago), and throughout my entire life, I had pretended to ignore this link, or at least to minimize it, thereby following the path pre-determined by the norms of the time. To become preoccupied with a work that is no longer in one's hands, and especially to wonder whether one's name remains somewhat attached to them, is perceived as a form of pettiness and narrowness - all the while deeming it perfectly natural for someone to be profoundly affected by the fact that a child one raised (and believes to have loved) suddenly chooses to repudiate the name he or she was given at birth.

\subsection{The heir}

\nnote{90}\marginpar{p. 378} (May 18)

\subsection{The co-heirs}

\nnote{91}

\subnote{91$_1$}

\subnote{91$_2$}

\subnote{91$_3$}

\subnote{91$_4$}

\subsection{... and the chainsaw}

\nnote{92}







%\end{document}


